\chapter{邓巴}
 
    自己投下的炸弹落到哪儿去了,约塞连已经一点也不在乎了。

    可他并没有邓巴干得那么过分。邓巴飞过那个村庄几百码后才把炸弹扔下去。如果有证据能表明他是故意这样干的,他就得上军事法庭。邓巴甚至没对约塞连讲一声,就洗手不再执行飞行命令了。

    他在医院里跌的那一跤不是使他开了窍,就是把他摔糊涂了。到底是哪种情况,就很难说了。

    邓巴很少放声大笑了,而且似乎一天天消瘦下去。对级别比他高的军官,甚至对丹比少校,他都敢挑衅般地大吼大叫。即使在牧师面前,他也是那样地粗暴无礼,满嘴污言秽语。牧师现在很怕邓巴,他似乎也在一天天消瘦下去。他对温特格林的朝拜以失败而告终,他只不过是再次进入了一座空空如也的圣殿而已。温特格林太忙了,没有工夫接见牧师。他的一个傲慢的助手把一个偷来的齐波牌打火机赠送给牧师,居高临下地通知他说,温特格林正忙于战争事务,无暇过问空勤人员飞行次数之类的小事情。现在,既然奥尔已经失踪,牧师就更加为邓巴担心,为约塞连想得也更多了。牧师独自住在一顶宽敞的大帐篷里。每到晚上,他就觉得这顶帐篷活像坟墓的拱顶,严严实实地把他封在阴森孤寂之中。他简直弄不懂,约塞连为什么会宁愿自己一个人住而不愿跟别人合住一顶帐篷。

    约塞连再次担任了领航轰炸手,给他做驾驶员的是麦克沃特。

    这也算是一种安慰,尽管他仍然像以往一样丝毫得不到保护。想反击是办不到的。他坐在机头里的座位上,却连麦克沃特和他的副驾驶员都看不到。他能看见的只有阿费。阿费那张圆脸上粗俗愚蠢的神态真叫他烦透了。在空中,有时怒气和失望一起向他袭来,折磨得他难以忍受,真恨不得自己再次降到僚机上,去操纵机舱里一挺压满子弹的机关枪,而不是守着这么一只他压根不需要的高精度轰炸瞄准器。如果真能那样,他就可以怀着满腔仇恨,双手紧握着一挺五十口径的重型机关枪,对着所有压迫他虐待他的混蛋狂扫乱射;对着高射炮火的黑烟;对着地面上的德国高射炮手,这些家伙他甚至看不见,而且,即使他来得及朝他们开火,他的机枪火力也伤害不着他们;对着长机上的哈弗迈耶和阿普尔比,这两个天不怕地不怕的家伙执行第三次轰炸博洛尼亚的任务时,带队一直俯冲到二百五十门高射炮的火力网之中,结果一发炮弹打掉了奥尔飞机上的一个引擎,使奥尔正赶在一场短暂的雷暴雨来临之前栽进了热那亚和斯培西亚之间的大海里。

    实际上,他就是手中握着那挺重型机关枪,也干不了什么事,最多不过装上子弹,打几个连发试试火力罢了。对他来说,机关枪和轰炸瞄准器同样没有什么用处。他可以用它猛烈扫射前来攻击的德国战斗机,但现在已经没有德国战斗机了。他甚至不能够掉转枪口对准驾驶员那惊慌失措的面孔,比方说赫普尔和多布斯,命令他们老老实实地返航。有一回他就是这么命令基德-桑普森返航的。执行第一次轰炸阿维尼翁的可怕任务时,他与多布斯和赫普尔一起坐在僚机里,跟在哈弗迈耶和阿普尔比的长机后面飞过高空。

 


    突然,他意识到自己处在一种糟糕透顶的困境之中,当时他真想像对待基德-桑普森那样命令多布斯和赫普尔返航。是多布斯和赫普尔吗?是赫普尔和多布斯吗?他们俩是什么人呢?没长胡子的娃娃叫赫普尔,神经紧张的疯子叫多布斯。这两个傻乎乎的新手,竟敢凭着他们那蹩脚的技术和迟钝的大脑,驾着一架用一两英寸厚的合金制成的飞机在两英里高的稀薄空气中穿行,而且居然保住了性命,这真是荒谬绝伦、疯狂透顶。多布斯当时在飞机里就发起疯来。他身体仍然坐在副驾驶员的位置上,手却伸过去从赫普尔那里一把夺过操纵器猛地一推,飞机立刻杀气腾腾地朝着轰炸目标俯冲下去,一下子钻到他们刚刚逃离的高射炮火力网里面去了。

    约塞连吓得浑身冰凉,对讲耳机的插头也给震掉了。接下来他记得的就是另一个新来的无线电通讯员兼机枪手,名叫斯诺登,躺在机舱的后部快要咽气了。是不是多布斯送了他的命,这无法肯定,反正当约塞连重新插上对讲耳机的插头时,多布斯正在内部对讲机里呼救,叫人赶快到前舱去救救轰炸手。几乎与此同时,斯诺登插进来呜咽着说:“救救我吧,救救我吧。我冷啊,我冷啊。”约塞连慢慢地爬出机头,爬上炸弹舱的舱顶,一步一挪地退到机尾舱——路过急救药箱时他却忘了拿,只好又返回去取——去抢救斯诺登,结果却找错了伤口。在斯诺登的大腿外侧有一个橄榄球那么大的西瓜形状的窟窿,大张着口子,血肉淋漓,一缕缕一丝丝浸透鲜血的肌肉组织在里面奇怪地颤动着,仿佛它们本身是有生命的瞎眼动物似的。这个裸露着的椭圆形伤口几乎有一英尺长。一看到它,约塞连又是震惊又是怜悯,不禁呻吟起来,还差一点吐了出来。那个矮小瘦弱的尾舱机枪手昏死在斯诺登身旁的地上,他的脸色白得像一块手帕,约塞连只好强忍住嫌恶扑过去先救他。

    是的,从长远来看,和麦克沃特一起飞行要安全得多。可是,和麦克沃特一起飞行也可以说是一点都不安全的,因为麦克沃特太喜欢飞行了。奥尔失踪后,卡思卡特上校从机组补充人员中挑选了一名轰炸手给他们,他们带着这个新手完成飞行训练返航时,约塞连坐在机头里,麦克沃特驾驶着飞机冒冒失失地从离地几英寸的地方轰鸣而过。轰炸训练场设在皮亚诺萨岛的另一头。从那儿经过岛中部的群山往回飞时,麦克沃特把机腹紧贴着山脊,让飞机懒洋洋、慢悠悠地飘行着。突然间,他非但不保持高度,反而开足两个引擎,猛地把飞机向一侧倾斜过去。更叫约塞连吃惊的是,麦克沃特快活地摆动着机翼,让飞机顺着斜坡飞快地冲下去。飞机时而飞腾,时而下跌,发出刺耳的隆隆巨响,轻快地掠过绵延起伏的山峦,就像一只吓傻了的海鸥在汹涌的浊浪之中穿行。约塞连吓得呆若木鸡。那个新来的轰炸手故作镇定地坐在他身旁,着魔般地咧嘴傻笑着,一个劲地吹口哨。约塞连真想伸出手去在这个白痴的脸上扇一巴掌。就在这时,飞机钻进了遍布巨石的丘陵地带,一排排树枝劈里啪啦地从他眼前和头顶擦过,随即在他的身后模模糊糊地一闪即逝。约塞连给震得东倒西晃。谁也没有权利拿自己的性命冒这么可怕的危险。

 


    “朝上飞,朝上飞,朝上飞!”他冲着麦克沃特狂叫着。他简直恨死这家伙了。可麦克沃特正对着内部对讲机快快活活地唱着呢,也许根本没有听见他的话。约塞连不禁怒火中烧,恨得眼泪都快掉下来了。他扑向爬行通道,顶着引力和惯性的强大拉力,费劲地朝主舱爬去。他一口气爬进驾驶舱,站在麦克沃特的驾驶员座位后面直打哆嗦。他四下里望着,急于找到一把手枪,一把零点四五口径的灰色自动手枪。他要拿着这手枪朝麦克沃特的后脑勺猛砸下去。可是驾驶舱里没有枪,也没有猎刀,更没有别的可以让他拿来砸过去或者戳过去的武器。约塞连双手一把揪住麦克沃特的飞行服领子,猛力摇晃着,大声叫他朝上飞,朝上飞。陆地仍然继续从飞机的左右两侧飞快地闪过去。麦克沃特转脸看着约塞连,快活地哈哈大笑,好像约塞连正在分享他的快乐似的。约塞连伸出双手掐住麦克沃特袒露的脖颈,猛地一用劲,麦克沃特顿时僵住了。

    “朝上飞。”约塞连咬着牙,用低沉、威胁的口吻不容置辩地命令他。“否则我就掐死你。”

    麦克沃特紧张而又小心地扳回操纵杆,让飞机逐渐爬升。约塞连掐着麦克沃特脖子的双手瘫软下来,滑下他的肩头,无力地晃动着。他的火气全消了。他感到难为情。麦克沃特转过身来时,他觉得很难过,那双手竟然是他的,他真恨不得有个地方把它们埋藏起来。他的手上毫无感觉。

    麦克沃特深沉地凝视着他,目光里没有一丝友好的神情。“伙计,”他冷冷地说,“你的情况很不好。你该回家了。”

    “他们不让我回家,”约塞连躲避着他的目光回答道,说完便悄悄地离开了。

    从驾驶舱里爬下来后,约塞连一屁股坐到地上。他又愧又悔,耷拉着脑袋,浑身大汗淋漓。

    麦克沃特直接把飞机开回基地。约塞连拿不准麦克沃特会不会跑到指挥部的帐篷里去找皮尔查德和雷恩,要求他们以后再也不要派约塞连到他的飞机上去。他自己以前就曾偷偷摸摸地去找过他们,要求不跟多布斯、赫普尔或者奥尔,还有阿费,一起执行飞行任务,不过没有成功。他以前从来没有见过麦克沃特这么生气。

    麦克沃特不论在什么时候什么地方都是一副轻松愉快的样子。约塞连担心自己是不是又失去了一个朋友。

    但是,他从飞机上下来时,麦克沃特却向他眨眨眼睛叫他放心。在乘吉普车返回中队的路上,麦克沃特兴致勃勃地跟那个新来的什么话都相信的飞行员及轰炸手开着玩笑,却没有跟约塞连说一句话。直到他们四个人交还降落伞后分了手,他和约塞连肩并肩往他们自己的那排帐篷走去时,麦克沃特那张长着稀疏雀斑的苏格兰-爱尔兰人的棕褐色脸上才突然绽开了笑容。他用指关节开玩笑地戳了戳约塞连的肋骨,好像是要打他一拳似的。

    “你这个混蛋,”他笑道,“在天上时你真的想掐死我吗?”

    约塞连后悔地笑着摇了摇头。“不,我想我不至于。”

    “我真没想到你会受不了。唉!你为什么不去找个人谈谈?”

    “我跟每个人都谈了。你他妈的怎么了?你难道没听见我谈吗?”

    “恐怕我从来没有真正相信过你说的那些话。”

    “难道你没害怕过吗?”

    “也许我应该害怕。”

    “甚至执行飞行任务的时候也没害怕?”

    “恐怕我没有多少头脑,不知道害怕。”麦克沃特不好意思地笑笑。

    “已经有那么多杀死我的办法啦,”约塞连发议论道,“你还要再找出一种来。”

    麦克沃特又笑了。“嘿,我敢打赌,我贴着你的帐篷飞过去时,把你吓了个半死,对不对?”

    “把我吓死了。这我告诉过你了。”

    “我还以为你不过是向我抱怨飞机的噪音呢。”麦克沃特耸耸肩表示让步。“噢,好吧,真他妈的,”他叫道,“我想我只好不这么干了。”

    但是,麦克沃特是不可救药的。他虽然不再贴着约塞连的帐篷飞行,却一有机会就驾着飞机在海滩上低空盘旋,如同一串震耳欲聋的落地雷那样掠过水面上的浮筏和海滩上僻静的沙坑,约塞连常常躺在海滩上抚摸达克特护士,或者跟内特利、邓巴和亨格利-乔打红桃纸牌戏、扑克牌戏或平纳克尔牌戏。约塞连和达克特护士几乎每天下午都没事,他们双双跑到沙滩上,坐到一堆窄窄的齐肩高的沙丘后面,沙丘把他们跟海滩上赤身裸体游泳的军官和士兵分隔了开来。内特利、邓巴和亨格利-乔常常去那儿,麦克沃特偶尔也参加进去,还有阿费也常去。他总是鼓鼓囊囊地穿着全套军装,到了那儿以后,除了鞋帽,从来不肯脱一件衣服,当然也从来不肯游泳,其他的男人都穿着游泳裤头,这是出于对达克特护士,也是出于对克拉默护士的尊重。克拉默护士每次都陪着达克特护士和约塞连到海滩上去,独自一人高傲地坐在离他们十码以外的地方。只有阿费提起过那些一丝不挂的男人,他们或者在远处的海滩上晒日光浴,或者从一个漆成白色的大浮筏上跳水潜泳。那个大浮笺架设在沙堤外面的几只空油桶上,随着海浪上下颠簸着。克拉默护士生约塞连的气,又对达克特护士失望,所以总是一个人单独坐着。
 


    苏-安-达克特护士有许多约塞连十分欣赏的迷人之处,其中之一就是瞧不起阿费。约塞连喜欢她的另一个原因是她长着两条白嫩的长腿和一个丰满富于弹性的屁股。约塞连常常感情一激动就过分粗鲁地搂抱她。每逢这时,他就忘掉了她腰以上的身体部分过于纤细,过于单薄了。他喜欢在薄暮中和她一块躺在沙滩上时她那种懒散柔顺的卧姿。有她在身旁,他感到欣慰和镇静。他有一种强烈的欲望,那就是一直抚摸着她的胴体,一直跟她保持着肉体的接触。她的大腿白皙光滑。当他跟内特利、邓巴和亨格利-乔玩牌时,他喜欢用手指松松地握住她的脚脖子,用手指甲轻轻地、怜爱地抚弄她腿上那长满绒毛的皮肤,或者心不在焉地、感觉愉快地、几乎无意识地伸手顺着她那贝壳般的脊梁骨向上摸去。她天天穿着一件三点式泳装,泳装的上半截刚好能遮住她那垂着长长xx头的娇小Rx房。约塞连经常毫无拘束地把手伸到她泳装背后的松紧带下面,以满足自己的占有欲望。达克特护士自豪地表现出一种对他的依恋感。约塞连很喜欢她这种沉静的、心满意足的反应。亨格利-乔也很想上下摸一摸达克特护士,可是不止一次地被约塞连恶狠狠的目光给吓回去了。达克特护士跟亨格利-乔眉来眼去,只不过是为了挑起他的欲火。每回约塞连用胳膊肘或者拳头猛戳她一下,叫她老实点时,她那双浅褐色的圆眼睛里就闪烁出恶作剧的光芒来。

    这几个男人往沙滩上铺一条毛巾、汗衫或者毯子什么的,就在上面打起了纸牌。达克特护士则倚在旁边的一个沙丘上,洗着一副多余的牌。有时她不洗这牌,而是坐在那里眯缝着眼睛对着一面小镜子左顾右盼,没完没了地往她那卷曲的淡红色睫毛上涂睫毛油。

    她傻乎乎地认为,这样会使它们越长越长。偶尔她洗牌时会故意作弊,或者搞点别的鬼名堂。他们打了好一会才发现,只好气恼地把牌统统扔下,一起扑上前去捶她的胳膊和大腿,用脏话骂她,警告她不许再这么胡闹,她却得意极了,满脸通红地哈哈大笑起来,当他们正绞尽脑汁想着如何出牌时,她会在旁边唠唠叨叨地乱出主意,于是他们又用拳头使劲捶她的胳膊和大腿,叫她闭嘴,这时她就会高兴得面颊泛起淡淡的红晕。达克特护士特别喜欢招人注意。

    当约塞连或者其他人盯着她看时,她会快活地垂下留着栗色前刘海的脑袋。每当她想到有许多一丝不挂的小伙子和男人就在沙丘另一侧不远的地方闲荡时,心中就不由得生出一种温暖的、企望快乐的奇怪感觉。她只要随便找个借口伸长脖子或者站起身来,就能够看见那边三四十个裸体男人在阳光下溜达或是打球。对她自己来说,她的身体既熟悉又普通,她怎么也弄不明白,男人们为什么能从她的肉体上得到令他们神魂颠倒的狂喜,为什么能对她的肉体产生出那么强烈的欲念,为什么仅仅摸摸她,揿揿她,捏捏她,拧拧她,触触她,就能给他们带来那么大的乐趣,她不理解约塞连的情欲,但她愿意相信他说的话。
 


    晚上,当约塞连性欲冲动时,他就拿着两条毯子把达克特护士带到海滩上。他喜欢穿着大部分衣服跟她做爱,他觉得这比跟罗马那些情欲旺盛的裸体妓女做爱更有乐趣。夜里他俩常常一块到海滩上去,不过不是去做爱,而是搂抱着躺在毯子底下瑟瑟发抖,互相为对方抵御着清新湿润的寒气。墨汁般漆黑的夜晚越来越冷,星星闪烁着一层寒光渐渐隐去。那个浮筏在阴冷的月光下左右摇摆,似乎正在渐渐漂去。天气明显地变冷了,别的军官这才开始动手装炉子。每天都有人到约塞连的帐篷里来对奥尔的手艺发出一番赞叹。达克特护士兴奋得发狂,因为约塞连和她呆在一起时手从来不离开她的身体。不过,白天附近有人能看见他俩时,她不允许他把手伸到她的游泳裤里,即使旁边只有克拉默护士一个人时也不行。

    克拉默护士总是独自坐在沙丘的另一侧,责备地翘着鼻子,装着什么都没有看见。

    达克特护士本来是克拉默护士最好的朋友,可是由于她和约塞连发生了那种关系,克拉默护士便不再跟她说话了。不过,看在她们曾经是最好的朋友的分上,达克特护士走到哪儿她仍然跟到哪儿。她对约塞连以及他所有的那些朋友都不满意。当他们站起来和达克特护士去游泳时,她也站起来去游泳。不过,即使在水里她仍然和他们保持着十码的距离,仍然对他们保持着沉默的、冷冰冰的态度。他们笑着泼溅水花时,她也笑着泼溅水花;他们潜水时,她也潜水;他们游到沙堤上休息时,她也游到沙堤上休息。最后,他们上岸时,她也上岸,用她自己的浴巾把臂膀擦干,回到远处她自己的那块地方坐下来,腰板挺得直直的,一圈阳光映照在她的亚麻色头发上,就像一个光环。如果达克特护士表示出悔恨和歉意的话,克拉默护士准备重新开口跟她讲话。可是,达克特护士偏偏愿意保持现在这种局面。很久以来,她一直想痛骂克拉默护士一通,以便叫她闭上她那张嘴。

    达克特护士觉得约塞连棒极了,并且已经开始设法改造他了。

    她非常喜欢看他用一只胳膊搂着她、脸朝下趴着打盹的模样,或是看着他悲伤地凝视着平静柔缓的海浪。那一排排的浪花不断地拍击着海岸,像快活的小狗似的蹦跳到沙滩上一两英尺远的地方,又急急忙忙地退了回去。他沉默不语的时候她也很安静。她知道自己没有惹他厌烦。他打瞌睡或者想心思时,她就仔仔细细地涂手指甲。午后的徐徐暖风轻轻吹拂在海滩上。她非常喜欢打量他那又宽又长、肌肉强健的后背和后背上那光滑油亮的古铜色皮肤。她喜欢突然把他的整个耳朵咬在嘴里,同时用手顺着他的前胸往下抚摸,从而一下子撩拨起他的欲火。她喜欢挑逗得他心急火燎、坐立不安,一直拖到天黑才满足他的要求。完事以后,她爱慕地吻着他。

    她给他带来了多么巨大的幸福啊。

    有达克特护士陪着,约塞连从来不感到孤寂。达克特护士切切实实地懂得如何保持沉默,而且不算过分地任性。广阔无垠的海洋时时萦绕在约塞连的心头,折磨得他痛苦不堪。达克特护士擦拭指甲的时候,他悲伤地怀念起死在水底下的所有人来。他们肯定已经超过一百万了吧。他们在哪儿呢?是什么样的虫子吃掉了他们的肉呢?他想象着他们在水中无能为力的样子,想象着他们被迫大口大口往肚里灌水的可怕情景。约塞连目送着远处穿梭往返的小渔船和军用汽艇,觉得它们显得那么虚幻,每回它们往远处什么地方驶去时,上面的人看上去那么渺小,简直不像有血有肉的真人。他望着厄尔巴岛的石崖,眼睛不由自主地向空中寻找着一片萝卜形的絮状白云。克莱文杰就是在这么一片白云中消失的。他凝视着意大利雾茫茫的地平线,心中思念起奥尔来。克莱文杰和奥尔。他们到哪里去了?有一天黎明时分,约塞连站在防波堤上,看到一捆圆木随着潮水朝他漂移过来,等到离他近了,这捆圆木出乎意料地变成了一个溺死者泡得肿胀的脸,这是他这辈子见到的第一个死人。他渴望生活,急切地伸出手去牢牢抓住达克特护士的肉体不放。他心惊胆战地仔细打量着每一件漂浮物,寻找着有关克莱文杰和奥尔的某种令人毛骨悚然的迹象,做好准备迎接任何令人震惊的恐怖情景。但是,麦克沃特给他带来的震惊却是他始料不及的。

    有一天,麦克沃特驾着飞机疾风般穿过远处的寂静,突然出现在海滩的上空。飞机朝着海岸线恶狠狠地直冲过去,轰隆轰隆地吼叫着掠过海面上起伏不定的浮筏。此时,亚麻色头发、面容苍白的基德-桑普森正站在浮筏上,他那裸露着的胸部肋骨根根突出,甚至在很远的地方也看得一清二楚。就在飞机飞过他头顶的一瞬间,他笨拙地跳起身去摸飞机。也就在这时,一阵狂风卷过,不知是由于这阵风作怪,还是由于麦克沃特小小的判断失误,反正一闪而过的飞机飞得稍微低了一点,一个螺旋桨把他的身体一劈两半。
 


    接下来发生的事情甚至当时不在场的人也记得清清楚楚,透过震撼人心压倒一切的飞机轰鸣声,人们只听到最短暂最轻微的“嚓”的一声,随即就看见基德-桑普森两条苍白干瘦的腿不知怎么地仍有几根筋与那齐刷刷截断的血肉模糊的臀部相连接着。这两条腿在浮筏上一动不动地站立了一两秒钟才摇摇晃晃地向后翻倒在水里,发出一声微弱的溅水花的声响。基德-桑普森的身体在水里翻了个个儿,露在水面上的只剩下他那奇形怪状的脚趾和灰白色的脚掌。

    海滩上乱成一团。克拉默护士突然不知从哪儿冒了出来,伏在约塞连的胸脯上歇斯底里地哭泣着。约塞连用一只胳膊搂住她的肩膀抚慰着她;另一只胳膊则搀着达克特护士,她也正倚在他的身上,瘦削的长脸惨白惨白的,浑身战栗,抽抽搭搭地哭泣着。

    海滩上,人人都在狂叫乱窜,男人像女人那样尖叫着。他们惊慌失措地四处寻找着自己的东西,匆匆忙忙俯下身偷眼望着每一个缓缓涌上沙滩的齐膝深的浪头,好象海浪会把某个血淋淋的、令人恶心的可怕器官,比方肝或肺之类,直接冲到他们的面前。那些在水里的人全都奋力往外逃去。慌忙之中,他们竟忘了游泳,只知道哀嚎着涉水往海滩奔,粘糊糊的海水像刺骨的寒风那样揪住他们,拦着不让他们逃跑。基德-桑普森的鲜血溅得到处都是。许多人发现自己的四肢或躯干上溅有血迹。他们恐怖而嫌恶地后退着,好像要竭力甩掉自己那可憎的皮肤似的。人人都在没头没脑地乱窜。

    他们时不时地回头瞥上一眼,目光中充满着痛苦和惊恐。他们钻进幽深阴暗的树林,树叶沙沙作响,虚弱的喘息声和叫喊声此起彼伏。约塞连发狂地拖着两个跌跌撞撞的女人往回跑,连拉带拽地催促她们快点走,接着又跑回去骂骂咧咧地扶起亨格利-乔,后者踩到了他拖在身后的毯子或者照相机壳上,脸朝下摔了一跤,扑倒在一滩稀泥上。

    中队里人人都已经知道这件事了。穿着军服的人们也都在那里狂叫乱窜,不过也有人一动不动地肃然站立着,好像扎了根似的,比方奈特中士和丹尼卡医生。这两个人目光严肃地伸长脖子仰望着麦克沃待那架闯了祸的飞机,看着它孤零零地在空中慢慢盘旋上升。

    “谁在飞机上?”约塞连一瘸一拐、上气不接下气地跑上前,忧郁的眼睛里闪动着焦虑和痛苦的泪光,急切不安地冲着丹尼卡医生喊道。

    “麦克沃特,”奈特中士说,“他正带着两个新来的驾驶员进行飞行训练。丹尼卡医生也在上面。”

    “我正在这里呢,”丹尼卡医生焦虑不安地迅速看了奈特中士一眼,用一种奇怪而困惑的声调争辩道。

    “他为什么不降落?”约塞连绝望地叫道,“他为什么一个劲地往上飞?”

    “他大概不敢降落,”奈特中士回答说,“他知道自己闯下了什么祸。”

    麦克沃特越飞越高。飞机发出嗡嗡的声响,机头朝上,平稳缓慢地呈椭圆形地螺旋上升,而后朝南边远处的海面上飞去,接着又折回头,在小飞机场上空盘旋一圈之后,便往北飞越远处黄褐色的丘陵地带,不一会,飞机就上升到五千英尺以上的高空,引擎的声音低得近似耳语声。一顶白色的降落伞突然噗的一下在空中张开。

    几分钟之后,第二顶降落伞又张开了,像第一顶一样一直朝着简易机场的空处飘落下去。地面上毫无动静。飞机继续往南飞了三十来秒钟。它依然保持着方才那种飞行方式,不过这种方式现在人们已经很熟悉了,毫无意外之处。麦克沃特扬起一侧机翼,让飞机优雅地倾斜盘旋着,然后转了一个弯朝下冲去。

    “又有两个人完了,”奈特中士说,“麦克沃特和丹尼卡医生。”

    “我就在这儿呢,奈特中士,”丹尼卡医生可怜巴巴地对他说,“我没在飞机上。”

    “他们为什么不跳伞?”奈特中士自言自语地大声询问道,“他们为什么不跳伞?”

    “这样做毫无意义,”丹尼卡医生咬着嘴唇说,“这样做根本毫无意义。”

    但是,约塞连突然间明白了麦克沃特为什么不跳伞。他跟着麦克沃特的飞机狂奔着从中队营地的一头追到另一头,恳求地挥动着双臂冲他大声呼喊,快降落吧,麦克沃特,快降落吧。然而,似乎没有人听见,当然不用说麦克沃特了。麦克沃特又转了一个弯,摆动了一下机翼向地面致敬,啊,老天爷,他下决心了,飞机猛然朝着一座大山撞去。约塞连只觉得一阵窒息,喉咙里不由自主地发出一声悲叹。

    基德-桑普森和麦克沃特的死弄得卡思卡特上校心烦意乱。

    他决定把飞行任务提高到六十五次
