\chapter{米洛}
 
    对米洛来说,四月一直是他最喜欢的一个月份。丁香花总在四月里盛开,结在藤蔓上的水果也在这时成熟。人的心跳会比以前加快,减弱了的胃口也会重新恢复起来。四月里,曾有一道色彩更为艳丽的彩虹在那只周身发光的鸽子的身上闪烁。四月是春天,而一到春天米洛-明德宾德的脑筋一下子就转到了柑橘上面。

    “柑橘?”

    “是的,长官。”

    “我的士兵会喜欢柑橘的,”那位指挥驻扎撒丁岛的四个B26型飞机中队的上校承认说。

    “他们吃多少都不成问题,只要你能从伙食费里弄到钱来付帐。”米洛向他保证。

    “卡萨巴甜瓜弄得到吗?”

    “在大马士革便宜极了。”

    “我特别爱吃卡萨巴甜瓜。我一向都爱吃得不得了。”

    “只要每个中队借给我一架飞机就成,各队只要出一架,那你想吃多少卡萨巴甜瓜就有多少,只要你付得起钱。”

    “我们是从辛迪加联合体中购买吗?”

    “人人都在联合体里有股份。”

    “这真令人吃惊,简直太令人吃惊了。你是怎么办到的?”

    “集团购买力能使得一切都大不一样。比如说,想来点裹了面包屑的炸小牛排也成。”

    “我可不大爱吃裹了面包屑的炸小牛排,”那位驻扎科西嘉北部的B25型机群指挥官嘀嘀咕咕地说,他仍然心存疑虑。

    “裹了面包屑的炸小牛排可是很有营养的噢。”米洛非常诚恳地忠告他。“它含有蛋黄和面包屑。小羊排也很有营养。”

    “哈,小羊排!”这位B25指挥官立即作出响应。“是上好的小羊排吗?”

    “是最好的,”米洛说,“黑市上卖的最好的。”

    “是小羊羔的排骨?”

    “是你从未见过的、穿着最漂亮的粉红色小纸尿裤的小羊羔。

    在葡萄牙,这种小羊排卖得非常便宜。”

    “我可不能派一架飞机去葡萄牙。我没这个权力。”

    “只要你借飞机给我,我就能办到。再派一名飞行员驾驶就行了。别忘了——这能使你讨得德里德尔将军的欢心。”

    “德里德尔将军会再来我们食堂吃饭?”

    “会吃得像头猪似的,只要你用我的纯黄油煎上一些最新鲜的鸡蛋,然后拿给他吃,他就会这样。你还会有柑橘、卡萨巴甜瓜、白兰瓜、多佛的纯鳎鱼片、烘烤冰淇淋、鸟蛤和贻贝等。”

    “人人都有份吗?”

    米洛说:“这是整件事中最妙的部分。”

    “这事我一点也不喜欢,”这位不肯合作的战斗机指挥官咆哮道,他也不喜欢米洛这个人。

    “北边部队里有个战斗机指挥官不肯合作,他跟我过不去,”米洛对德里德尔将军抱怨道,“往往一个人就会把整个事给毁了,这一来你就再也吃不上用我的纯黄油煎出来的新鲜鸡蛋了。”

    于是,德里德尔将军便把这位不肯合作的战斗机指挥官调到所罗门群岛去了,让他在那里挖坟墓,后来又换了一个患有滑囊炎的老头子上校来接替他。这老头特别爱吃荔枝,他又将米洛介绍给了驻扎在陆地上的一位指挥B17型机群的将军,此人尤其爱吃波兰香肠。

    “在克拉科夫,用花生可以换到波兰香肠,”米洛告诉他说。

    “啊,波兰香肠,”将军怀旧地感叹道,“要知道,只要能买到一大截波兰香肠,我什么都愿意拿出来。什么都愿意。”

    “你什么都不必拿出来。只要给我一架飞机,每个食堂一架,外加一名叫干啥就干啥的驾驶员。还有,在第一次订货时,你得付上一小笔现金作为定金。”

    “可是克拉科夫远在敌后几百英里,你怎么去那里弄香肠?”

    “在日内瓦有一个波兰香肠国际交易市场。我只要将花生空运到瑞士,以市场上的公开价格将其换成波兰香肠。他们将把花生运到克拉科夫,我呢,则把波兰香肠运回来给你。你要多少波兰香肠,就可以通过辛迪加联合体买到多少。你还能买到柑橘,只不过上面稍微染了点人造颜色。还有马耳他的鸡蛋和西西里的苏格兰威士忌。当你通过辛迪加联合体买这些东西时,你等于是自己付钱给自己,因为你将在里面拥有一份股份。所以,你实际上是不花一个子儿就买到了所有的东西。这不是挺有意义吗?”

    “你简直是个天才。你究竟是怎样想出这个主意来的?”

    “我叫米洛-明德宾德,今年二十六岁。”

    米洛-明德宾德的飞机从各处飞了回来,驱逐机、轰炸机,还有运输机接连不断地涌进卡思卡特上校的机场,开飞机的飞行员都是些叫干啥就干啥的人。这些飞机的机身上都装饰有各个飞行中队的象征图案,其色彩艳丽夺目。每一个图案都代表着一种值得称赞的理想,如勇敢、力量、正义、真理、自由、博爱、荣誉和爱国主义等等。飞机归米洛调遣后,机械师立即用乳白色的油漆刷了两遍,将这些图案涂掉,取而代之的是将事先刻好的标志用耀眼的紫色喷在飞机上。那标志是:M&M果蔬产品联合公司。在这个名称里,“M&M”代表米洛和明德宾德。米洛坦白地透露,之所以要将连接符号“&”插在中间,是为了消除这样一个印象:这个辛迪加联合体实际上是在一个人的操纵下。在米洛的调遣下,一架架飞机分别从意大利、北非和英国的机场,以及设在利比里亚、阿森松岛、开罗,还有卡拉奇等地的空运指挥站飞来。那些驱逐机有些被拿来做了交易,以多换几架运输机,有些则留着用来应付紧急托运事宜和运送一些小包裹。他还从地面部队弄来了一些卡车和坦克,用它们来搞短途运输。凡参与的单位人人都有股份,个个吃得发福,两片油光光的嘴唇间整天叼着根牙签,懒洋洋地到处逛游。米洛独自掌管着所有的正在日益扩大的经营业务。由于他全神贯注地投入该项工作,一条条水獭皮似的褐色皱纹渐渐地爬满了他那张操劳过度的脸,永远也休想消除掉。这一来,他看上去既清醒理智,又满腹狐疑,整天不是为这,就是为那而头疼。除约塞连之外,人人都认为米洛是个笨蛋,一则是因为他主动要求去干事务长的工作,二则是因为他干这差事干得太卖力。约塞连也认为米洛是个笨蛋,但同时他也知道米洛是个天才。

    有一天,米洛飞往英国去采购一批土耳其芝麻糖,然后领着四架德国飞机从马达加斯加飞了回来。那些德国飞机上装满了甘薯、甘蓝、芥菜和乔治亚黑斑豌豆等蔬菜。米洛从飞机上走了下来。他刚一踏上地面就呆住了,因为他发现有一小队宪兵正等在那里,准备俘获德国驾驶员,并还要没收他们的飞机。没收!仅仅这两个字就使他又气又恨。只见他暴跳如雷地来回走个不停,一根非难的手指犹如一柄利剑,在卡思卡特上校、科恩中校和那位统领着宪兵、脸上带有战场上留下的疤痕、手上端着冲锋枪的可怜上尉那三张满含愧疚的脸前舞个不休,嘴里还在不住地严辞痛斥着他们。

    “这是在俄国吗?”米洛以怀疑的口吻声嘶力竭地斥责着他们。

    “没收?”他尖叫着,好像不相信自己的耳朵似的。“美国政府从什么时候起开始执行没收私人财产的政策了?你们真不要脸!你们竟会生出这么一个可怕念头,一个个都不要脸极了。”

    “可是,米洛,”丹比少校胆怯地打断了他,“我们毕竟是在同德国人打仗呀。这些可全都是德国飞机。”

    “它们根本不是!”米洛愤怒地反驳道,“这些飞机都属于咱们的辛迪加联合体,大伙人人都有股份。没收?你们怎么能自己没收自己的私有财产?没收,亏你们想得出!我这一辈子还从来没有听说过这么卑鄙的事呢。”

    米洛果然没说错,因为等他们再细看时,他的那些机械师早已将德国飞机机翼、机尾和机身上原有的“十”形纳粹符号用乳白色的油漆给涂掉了,而且还涂了两遍,然后又用模板在这些地方印上了“M&M果蔬产品联合公司”的字样。就这样,米洛当着他们的面将他的辛迪加组织变成了一个国际性卡特尔。

 


    如今,米洛的庞大的空中商船队充斥着整个天空。一架又一架的飞机源源不断地从各地涌来,从挪威、丹麦、法国、德国、奥地利、意大利、南斯拉夫、罗马尼亚、保加利亚、瑞典、芬兰、波兰等地方涌来。实际上,这些飞机欧洲的什么地方都去,唯独不去俄国,因为米洛拒绝同俄国做生意。当他找过的那些人都同“M&M果蔬产品联合公司”签了约以后,米洛又创办了一个集体所有的附属公司,取名为“M&M花色糕点公司”。他又弄来了一些飞机,并从伙食费中拨出更多的公款来做这项生意。他经营的糕点有英伦三岛的烤饼和松饼,有哥本哈根的梅干和丹麦奶酪,还有从巴黎、尼姆斯和格勒诺布尔弄来的奶酪饼、奶油卷、奶油千层饼、花色小蛋糕,另有柏林的水果蛋糕、稞麦面包、姜汁面包、维也纳的杏仁果酱饼、巧克力饼和分别从匈牙利和安卡拉搞来的包馅卷饼和果仁蛋糕。每天早上米洛都要往欧洲和北非派遣飞机,飞机上拖着两条长长的红色广告标牌,上面用大大的方体字写着当天的特色商品:“注意:

    有圆腿肉,79¢……鳍鱼,21¢。”他还将两条这样的牌子租给了佩特牛奶公司、盖恩斯狗食公司以及诺克泽默公司,大大提高了辛迪加联合体的现金收入。为了体现自己有愿意为公众服务的公民意识,他还常常在空中广告里留出一些位置,免费为佩克姆将军做公益宣传广告,如“要讲究整洁”,“欲速则不达”,还有“能同做祈祷的家庭是永不离散的家庭”。在柏林,阿克西斯-萨利和霍-霍爵士这两位大名鼎鼎的广播员每天都要主持宣传性的广播节目,而米洛居然花钱买到了这些节目前的广告插播权,以促进他的业务活动。就这样,他的生意在各前线战场都做得很红火。

    米洛的飞机成了人们司空见惯的东西。它们享有在各处随便通行的自由。有一天米洛同美军当局签订了一份合同,由他负责去轰炸德军在奥尔维那托守卫的一座公路桥,同时又同德军当局签订了由他来守护该大桥的合同,用高射炮火来对付他自己策划的攻击。为美军轰炸桥梁,米洛可得到轰炸的全部成本费用外加百分之六的酬金,为德军守护大桥的协议款项也是如此,只不过还附加了一条,即他每击落一架美军飞机,德方将付给他一千美元奖金。
 


    米洛强调指出,这些交易的圆满成功标志着私有企业的重大胜利,因为两国的军队都是社会化的团体。这两个合同一经签订,无论是炸桥还是守桥,似乎都无需让辛迪加联合体破费一文,因为双方的政府有的是现成的人力和物力来从事这些事情,更何况双方都非常情愿将其投入进去。结果,米洛通过他的双边谋划实现了巨额利润,而他所做的仅仅是签了两次名而已。

    米洛的这个安排对双方都是很公平的。一方面,由于米洛有在各处随意通行的自由,因此他的飞机就可以悄悄潜入德军阵地进行偷袭,而不会惊动德军的高射炮火;而另一方面,由于米洛知道袭击行动,因此他有充分的时间向德军的高射炮手发出警告,待美军飞机一进入他们的炮火射程,就准确地向它们开火。除了约塞连帐篷里的那个死人以外,没有一个人不认为这是一个绝妙的策划。

    当天,那家伙刚飞到目标上空就被击中,丧了命。

    “我可没杀他!”米洛感情激动地一再重复着这句话,以此来回答约塞连那怒不可遏的非难。“告诉你,我那天根本没在场。你难道认为那天咱们的飞机飞来的时候,我就呆在那边的地面上朝它们开火?”

    “但这整个事情都是你一手策划的,不是吗?”约塞连大叫着回敬他。此时他们正站在黑缎子般的黑暗之中,这黑暗同时也笼罩着那条穿过寂静的停车场直通露天影院的小路。

    “我什么也没策划,”米洛气冲冲地回答说,一边激动地使劲吸气,将他那咝咝有声、毫无血色的鼻子挤成了一团。“不管有没有我的插手,德国人总归占着大桥,而我们则要去炸了它。我只不过发现了一个极好的机会,可以让我们从这一任务中捞到一把。这有什么大不了的?”

    “有什么大不了的?米洛,躺在我帐篷里的那个人在这次任务中丢了命,而他连背包都没来得及打开呢。”

    “可我又没杀他。”

    “你为此而得到了一千美元的外快。”

    “可他不是我杀的。我说过,我根本不在场。我当时在巴塞罗那,在那里购买橄榄油和去皮剔骨的沙丁鱼。我有定货单,它可以为我作证。我也没得到那一千美元。这一千美元都入了咱们联合体的帐,每个人都有份,连你也有,”米洛万般诚恳地向约塞连倾诉道,“瞧,约塞连,不管那个混帐的温特格林说过些什么,反正这场战争不是我发起的。我只不过是尽量以做买卖的方式来对待它。这难道有什么不对吗?要知道,用一架中型轰炸机另加上面的机组人员来换一千美元,这不能说是坏价钱。如果我能说服德国人,要他们每击落一架飞机就付给我一千美元,那我为什么不能拿这笔钱呢?”

    “因为你在同敌人做交易,这就是全部理由。难道你就不明白,我们是在打仗?有人正在死亡。看在基督的分上,你朝你的周围看看吧!”
 


    米洛已极不耐烦,但他仍克制着自己。“德国人并不是我们的敌人,他声明道,“哦,我知道你想说什么。不错,我们是在同他们打仗。不过德国人也是咱们辛迪加联合体里声誉很好的成员。作为我们的股东,我有责任保护他们的权利。也许是他们挑起了战争,也许他们的确杀了成千上万的人,可他们付起帐来却比我所知道的我们的一些盟国痛快得多。我得维护我同德国人订的合同的严肃性,你明白吗?你就不能从我的角度来看待这个问题?”

    “不能!”约塞连厉声回绝道。

    米洛被狠狠刺了一下,觉得感情受到了极大的伤害,他也并不想设法掩饰这一事实。那是一个闷热的月夜,空中到处飞有小虫、飞蛾和蚊子。米洛突然伸出一只胳臂,指向那边的露天影院,只见那里的放映机正在工作,平射出一道银白色的光芒,映得灰尘清晰可见,似一柄利剑,在黑暗中划出一道圆锥形的光痕,将一层薄膜似的荧光覆盖在观众的身上。那里的观众一个个都斜倚在椅子上,像受了催眠似地软瘫无力,大家的脸都朝上抬着,正对着那面白色银幕。此时,只见米洛的双眼里噙着泪水,显得无比真诚,脸上透着朴实和清白,并因渗出的亮晶晶的汗水和所搽的避蚊油而闪闪发光。

    “你瞧瞧他们,”他大声说,因感情激动而有些透不过气来。“他们是我的朋友,我的同胞,我的战友。任何人都不会拥有比他们这么一群人更好的伙伴了。难道你认为我会做出一桩伤害他们的事情吗?除非是万不得已。我现在的烦心事还不够多吗?你没看见?

    为了那些堆积在埃及各个码头上的大批棉花,我已经头疼死了。”

    米洛的说话声音断断续续的,突然,他像个溺水者一样,一把抓住了约塞连的衬衣前襟。他的眼睛像一对褐色毛虫一样,醒目地眨动个不歇。“约塞连,我该拿这么些棉花怎么办呀?这都是你的错,让我买下这么多的棉花。”

    那些棉花在埃及的码头上堆积如山,却没有一个买主。米洛从前做梦也没想到尼罗河流域的土地竟会这么肥沃,也没想到他买下的这批农作物会找不到市场。他的辛迪加联合体的各个食堂都帮不上他的忙。不仅如此,食堂成员还纷纷起来造反,毫不妥协地反对米洛要按人头硬性摊派给每人一份埃及棉花的建议。连他最忠实的朋友德国人在这次危机中也不肯帮他的忙。他们宁愿使用棉花的代用品。米洛的食堂甚至都不肯让他将棉花堆在那里。他只好租用仓库,其费用是直线上升,导致了他的现金储备彻底枯竭。从那次奥尔维那托战斗行动中所赚到的利润渐渐被耗光了。他开始不断写信回家去要钱,这些钱是他在生意兴隆的时候寄回去的,但不久这笔钱也几乎要用完了。仍有一包一包的棉花接连不断地被运到亚历山大港的码头。每次,只要米洛在国际市场上以亏本价脱手一批棉花,那些狡猾的埃及掮客就在地中海东部各地将其统统吃进,然后再以合同规定的原价卖给米洛。这一来,米洛就变得越来越穷了。
 


    “M&M果蔬产品联合公司”眼看就要垮台。米洛无时无刻不在咒骂自己,恨自己大贪婪,太愚蠢,不该买下埃及的所有棉花。然而,不管怎么样合同就是合同,非得信守不行。于是,一天晚上,在吃了一顿丰盛的晚餐之后,米洛的所有战斗机和轰炸机一起起飞,在基地上空编好队形,随后便开始向自己的空军大队投起炸弹来了。原来米洛又同德国人弄了一个合同,这一次他得轰炸自己大队的全部装备和设施。米洛的飞机分成几路协同袭击,轰炸了机场的油料库、弹药库、修理库,还有停在棒糖形停机坪上的B25轰炸机。他的机组人员总算对起落跑道和各个食堂手下留了情,因为这样一来他们干完活之后便可以安全着陆,而且在上床睡觉之前还可以享用到一顿热气腾腾的快餐。他们轰炸时机上的着陆灯一直亮着,因为地面上根本没人向他们开火还击。他们轰炸了四个中队、军官俱乐部和大队的指挥大楼。官兵们纷纷逃出各自的帐篷,个个惊恐万状,都不知道往哪个方向逃窜是好。不一会,受伤者躺得到处都是,尖叫声不绝于耳。连续几颗杀伤弹在军官俱乐部的院子里爆炸开来,使得这座木头建筑的一侧墙壁上留下了累累弹痕,也弹穿了那排站在吧台前的中尉和上尉们的腹背。他们痛苦万状地先是弯曲了身子,然后倒了下去。剩下的那些军官都给吓得魂不附体,纷纷朝那两个出口处逃窜,但他们又不敢出去,于是只好全都鬼哭狼嚎着挤在门口,就像一道厚实的人肉堤坝。

    卡思卡特上校又是爬又是挤,好不容易才从乱成一团、茫然失措的人群中钻出来,独自站在了门外。他瞪大双眼朝天上一看,不禁大惊失色。只见米洛的飞机像气球一样从容不迫地掠过花朵盛开的树梢,朝他们逼过来。机上的投弹舱的门敞开着,机翼上的风门片也向下垂着;那些巨大的着陆灯一直亮着,好似一对对暴眼,闪烁着强烈、炫目而又可怕的光芒。这番景象犹如一种神灵的启示,他以往从未目睹过。卡思卡特上校像被什么击中了一样,惊愕地叫了一声,接着便向前猛冲,几乎是呜咽着一头扑进自己的吉普车。他的脚找到了油门踏板和车子的发火装置,随后便以这辆摇摇摆摆的汽车所能达到的最快速度朝着机场疾驶而去。他那双松软无力的手因紧紧地握着方向盘而变得毫无血色。间或他还乱摁一阵子喇叭,似想故意折磨它一样。一次,他碰到了一群人,一个个只穿内衣,惊恐万状地低着脸,一边将瘦弱的胳臂当成不堪一击的盾牌紧紧抱着脑袋,一边疯了似的没命地朝小山上狂奔。为了避让这帮人,他来了一个急转弯,只听轮胎发出了一阵刺耳的尖叫声,差点没送掉他的小命。公路两旁,黄色、桔红色和红色的火焰在熊熊燃烧。帐篷和树木也在火中燃烧,而米洛的飞机还在不断地盘旋,不停地闪烁着的白色着陆灯仍旧亮着,投弹舱的门也还敞开着。吉普车开到机场指挥塔时,卡思卡特上校猛拉了一下刹车,车子几乎给弄翻掉。没等车子停稳,他就不顾危险地一跃跳下了汽车,飞快地冲上一段楼梯进到塔内。塔里有三个人正在忙着摆弄仪器,指挥着天上的飞机。他猛地冲上前去,一把推开其中的两人,伸手夺过那只镀镍的麦克风,两眼冒着怒火,那张结实的脸由于紧张而扭曲得变了形。他使着蛮劲紧紧地抓着麦克风,开始声嘶力竭地对着话筒狂叫。

    “米洛,你这个狗杂种!你疯了吗?你他妈究竟要干什么?下来!快给我下来!”

    “别这么大喊大叫,行吗?”米洛答道,这会儿米洛正在指挥塔里,就站在他的旁边,手里也拿着一个话筒。“我就在这儿。”米洛不满地瞟了他一眼,又回身去忙自己的事了。“很好,弟兄们,你们干得很好,”他赞不绝口地冲着手里的麦克风说,“不过我瞧见还有一个给养棚立着呢。那可不行,珀维斯,我以前跟你说过,别干这种差劲事。现在你马上给我飞回去,再去加把劲。这次你可要慢慢地向它靠拢……要慢慢地。要知道‘欲速则不达’,珀维斯。‘欲速则不达’,如果这话我以前曾对你说过,那么我肯定我对你说过已不下一百次了。记住,‘欲速则不达’。”

    这时他头顶上方的喇叭高声响了起来。“米洛,我是阿尔文-布朗。我的炸弹已经扔完了。现在我该干什么?”

    “扫射,”米洛说。

    “扫射?”阿尔文-布朗大吃一惊。

    “没法子,”米洛无可奈何地告诉他说,“合同上是这样规定的。”

    “哦,那么好吧,”阿尔文-布朗默认道,“既然这样,我就扫射吧。”

    这一次米洛做得太过分了。他竟然轰炸自己方面的人员和飞机,这事甚至连最冷漠的旁观者都感到无法容忍,看来,他的未日来临了。许许多多的政府高官蜂拥而至,对此事进行调查。各家的报纸都用醒目的大标题向米洛发起猛烈抨击。国会议员们个个义愤填膺,都声若洪钟地谴责他的凶残暴行,扬言要惩罚他。有孩子在部队服役的母亲们纷纷组织了起来,组成了若干个颇具战斗力的团体,要求给孩子们报仇。大队里没有一个人肯站出来为米洛说句话。无论他走到哪里,所有正派的人都觉得受到了他的侮辱。米洛陷进了墙倒众人推的困境,最后他只好向大伙公开了他的帐本,透露了他所赚得的巨额利润。至于他摧毁的人员及财产,他可以用这笔钱来向政府进行赔偿,而且还有多余,足以让他将埃及的棉花生意继续做下去。当然,这笔钱是人人有份的。然而,这整桩买卖妙就妙在根本没有任何必要向政府进行赔偿。

    “在一个民主政体中,政府即是人民,”米洛解释说,“我们是人民,不是吗?所以我们完全可以将这笔钱留着,而让那些中间经手人统统见鬼去。老实说,我倒情愿政府彻底撤手,别管战争的事,把整个战场留给私人企业去经营。如果我们欠了政府什么就赔什么,那我们只会怂恿政府加紧控制,阻碍其他的私营单位轰炸它们自己的人员和飞机。我们就会使它们丧失经营积极性。”
 


    当然,米洛是对的,因为除了少数几人之外,大队里所有的人不久就都同意了米洛的观点。那几个忿忿不平且不识相的家伙中就有丹尼卡医生。他整天气冲冲的,动辄跟人吵架,嘴里还总是嘀嘀咕咕说些讨厌的含沙射影的话,说这整桩投机买卖是件不道德的事。为平息他的怒气,米洛以辛迪加联合体的名义送给了他一张在花园用的铝架轻便折叠椅。这样,每当一级准尉怀特-哈尔福特一跨进他的帐篷,丹尼卡医生就可以很方便地将椅子折叠起来,拿到帐篷外面去;等一级准尉怀特-哈尔福特一走,他就可以立即将椅子重新拿回帐篷。在米洛进行轰炸的那天,丹尼卡医生像丧失了理智一样。他不朝掩蔽处跑,反而留在户外履行他的职责。他像只诡秘狡猾的蜥蜴似的趴在地上,冒着横飞的弹片、猛烈的扫射和无数的燃烧弹在伤员之间爬动着,给他们扎止血带,打吗啡针,上夹板以及磺胺药。他沉着脸,满脸的悲哀,除非说话不可,否则绝不开口。从每个伤员那发青的伤处,他看到了自己将来有一天腐烂时的可怕预兆。他不停地工作着,丝毫也不怜惜自己的身体,把自己弄得筋疲力尽。这个长夜总算熬了过去,第二天,他使劲抽着鼻子,终于顶不住了,于是又抱怨不休地跑进医务室的帐篷,要格斯和韦斯给他量体温,然后又拿了块芥未硬膏和一只喷雾器。

    那天夜晚,丹尼卡医生带着阴郁、深沉而又无法表露的沉痛心情护理着每一个呻吟的伤员。在大队执行轰炸阿维尼翁的任务的那天,他在机场也流露出同样的沉痛表情。当时,约塞连赤身裸体,丧魂落魄地从他的飞机的舷梯上朝下走了几级,一言不发,只是朝机舱里指了指。他那赤裸着的脚后跟、脚趾头、膝盖、手臂和手指上到处都沾满了斯诺登的鲜血。机舱里,那位年轻的无线电通讯员兼炮手全身僵硬地卧在那里,眼看就要死了,而他的旁边则躺着更年轻的尾炮手,每次只要一睁眼看到垂死的斯诺登,就立即又昏死过去。

    人们把斯诺登抬出飞机,用担架抬着送进了一辆救护车。这时丹尼卡医生将一条毯子披在了约塞连的肩上,那动作简直轻柔极了,然后领着约塞连上了他的吉普车。在麦克沃特的帮助下,他们三人默默地驱车来到中队的医务室帐篷。麦克沃特和丹尼卡医生将约塞连引进帐篷,让他在一张椅子上坐了下来,然后用冰冷的脱脂湿棉球把斯诺登溅在他身上的血全部擦洗干净。丹尼卡医生给他服了一片药,接着又给他打了一针,这些东西让他整整睡了十二个小时。当约塞连醒来后又去见他时,丹尼卡医生又给他服了药片并又给他打了一针,这使他又足足睡了十二个小时。等约塞连再次醒来去见医生时,医生准备再给他吃药打针。

    “你到底还要给我吃多少药,打多少针?”约塞连问他。

    “直到你感觉好些了为止。”

    “我现在就感觉好些了。”

    丹尼卡医生那被太阳晒成棕黄色的憔悴的额头因惊讶而皱了起来。“那你为什么还不穿上衣裳呢?你为什么要像这样赤身裸体地到处乱跑?”

    “我再也不想穿制服了。”

    丹尼卡医生接受了他的这一解释,将手上的注射器收了起来。

    “你肯定感觉良好?”

    “我感觉很好。只是你给我吃了那么多的药,打了那么多的针,我感觉自己有点呆呆的。”

    在那天余下的时间里约塞连就这么一丝不挂地到处走动。第二天上午九、十点钟的时候,米洛到处找他,最后发现他坐在距那小巧的军人公墓后方不远的一棵树上,身上仍旧是精赤条条的。斯诺登即将被安葬在这里。米洛是按平时规定着装的——下着草绿色军裤,上身穿一件干净的草绿色衬衫,打着领带,衣领上那道标志中尉军衔的银杠杠闪闪发亮。他头上还戴着一顶有硬皮帽檐的军帽。

    “我一直在到处找你,”米洛仰起头,以责怪的口吻朝着树上的约塞连喊道。

    “你应该到这棵树上来找我,”约塞连答道,“我整整一个上午都在这上面。”

    “下来,尝尝这个,告诉我好不好吃。这很重要。”

    约塞连摇了摇头。他赤身裸体地坐在最低的那很大树枝上,两手紧紧地抓住它上方的一根树枝,以让身体保持平衡。他拒绝动弹,米洛没办法,只好张开双臂,极不情愿地抱住树干,开始向上爬去。他笨手笨脚地爬着,一边大声呼哧呼哧地喘着粗气。待他爬到一定高度,足以让他将一条腿钩在树枝上停下来喘口气时,他身上的衣服已被挤压得不像样了。他头上的军帽也歪了,随时都有掉下来的危险。当帽子往下滑的时候,米洛赶紧一把将它抓住。豆粒般的汗珠像晶莹剔透的珍珠一样,在他的唇须上闪闪发光,而他眼睛下的汗珠则像鼓起来的混浊的水泡一样。约塞连冷眼瞅着他。米洛小心翼翼地将身体翻转半圈,这样他就可以面对着约塞连了。他把包在一团软软的、圆圆的棕色物体上的薄纸揭开,然后将其递给约塞连。

    “请尝一尝,再告诉我味道怎么样。我想把这东西拿给大伙吃。”

    “这是什么?”约塞连问,一边咬了一大口。

    “裹了一层巧克力的棉花。”

    约塞连恶心得直作呕,那一大口巧克力糖衣棉花不偏不斜正好吐在米洛的脸上。“给,快把它拿走!”他一边往外喷棉花,一边生气他说,“天哪!难道你疯了?你他妈的连棉花籽都没弄掉。”

    “别说得那么绝好不好?”米洛恳求说,“不至于那么糟吧。真的那么难吃?”

    “比难吃还糟。”

    “可我必须让食堂把这东西给大伙当饭吃。”

    “他们谁都不会咽得下去。”

    “他们一定得咽下去,”米洛带着一脸专横的庄重神情,以命令的口气说道。他边说边松开一只胳臂,理直气壮地在空中挥了挥一根手指,可没料到自己差点摔下去跌断脖子。

    “你往这边挪过来点,”约塞连对他说,“这样会安全得多,并且还能看到周围的一切。”

    米洛双手抓住头顶上方的树枝,带着十二分小心开始一点一点地往旁边挪动。他的脸因紧张而绷得紧紧的。当他发现自己终于平安无事地坐在了约塞连身边时,不禁长长地松了口气。他亲切地抚摸着那棵树。“这棵树多好哇,”他以一种树的主人的感激口气赞叹地说。

    “这就是生命之树,”约塞连回答说,一边晃动着他的脚趾头。

    “也是识别善恶之树。”

    米洛眯起眼睛仔细打量树皮和树枝。“不是,它不是的,”他答道,“这是棵栗树。我应该能看得出来。我也卖栗子。”

    “你爱怎么叫就怎么叫吧。”

    他俩坐在树上,有好几秒钟谁也没开口,腿从树上垂下,双手几乎伸得笔直,抓着头顶上的树枝。他俩一个除穿着一双绉胶底鞋外,全身上下一丝不挂,而另一个却齐齐整整地穿着全套草绿色粗呢毛料军装,连领带都系得紧紧的。米洛胆怯地透过眼角仔细地打量着约塞连,很识相地犹豫着不开口。

    “我想问你件事。”他终于开口了。“你什么衣服也不穿,当然我一点也不想干涉你,我只不过好奇罢了。你为什么不穿制服?”

    “我不想穿。”

    米洛像麻雀啄食那样飞快地连连点头。“我明白了,我明白了,”他忙不迭地说,但脸上却现出一片迷茫。“我完全理解。我听阿普尔比和布莱克上尉说你疯了,我只想弄个清楚。”出于礼貌,他又犹豫了一会,斟酌着下一句问话。“你真的以后再也不穿制服了?”

    “我可没这么想。”

    米洛忙又使劲点头,装出他仍能明白的模样,接着就默不作声地坐在那里,神情严肃而又烦恼不安地陷入了深思。一只头顶红冠的鸟儿,扇动着有力的黑色翅膀,擦过那摇曳不停的灌木丛,从他们的下面飞过。树荫里的约塞连和米洛由一层层斜斜的薄薄的绿叶挡着,四周则是围了其他的灰色栗树和一棵银色的云杉。太阳高高地悬挂在他俩头顶上那片蔚蓝色的辽阔天空上,在这一片蓝色中低低地浮动着几小团蓬松的白云,好似缀成一串的珍珠。空气中一丝风也没有,他们周围的树叶一动不动地低垂着。那树荫好像是由羽毛覆盖而成。除了米洛,一切似乎都是在静止的状态之中。只见米洛突然直起腰,压低嗓子叫了一声,手激动地指着一个方向。

    “快看!”他惊呼道,“快看那边!那里正在举行葬礼。那像是一片公墓,对吗?”

    约塞连用平淡的语气慢吞吞地答道:“他们正在安葬一个小伙子,就是那天轰炸阿维尼翁时被打死在我机上的那位。就是斯诺登。”

    “他是怎么死的?”米洛问,因害怕连声音都变了调。

    “被打死的。”

    “那太可怕了,”米洛悲叹道,一对褐色大眼睛里充满了泪水。

    “多可怜的小伙子。这实在太可怕了。”他使劲咬住他那颤动不已的下嘴唇,随后又颇带感情地抬高嗓门继续说,“可如果这些食堂都不肯购买我的棉花,那事情会变得更糟糕。约塞连,这些人都是怎么了?难道他们不明白,这辛迪加联合体可是他们自己的呀。难道他们不知道?他们人人都有一份啊。”

    “连我帐篷里的那个死人也有一份吗?”约塞连挖苦地问。

    “他当然也有,”米洛十分大方地向他保证道,“中队里的每一个人都有一份。”

    “他还没来得及到我们中队就给打死了。”

    米洛熟练地做了一个表示痛苦的怪相,然后将脸转开。“我希望你不要老是拿你帐篷里的那个死人来找我的茬,”他用愠怒的语气恳求道,“我跟你说过,那人被打死同我一点关系也没有。我看到了这个垄断埃及棉花市场的大好机会,结果给咱们大伙惹来了麻烦,这难道是我的错?难道我应该有未卜先知的本领,事先就知道会出现棉花供应过剩?那时我连供应过剩是怎么回事都不知道。垄断市场的机会是不常有的,我遇到这样的机会能一把抓住就够精明的了。”米洛本想发出一声呜咽,可他忍住了,因为这时他看到六个身穿制服的抬灵柩的人把一口简陋的棺材从救护车上抬了下来,轻轻放在那条狭长的裂口——那口新挖的墓穴——旁边。“可现在我连一个子儿的棉花也卖不出去。”

    面对这一套不足道的葬礼游戏,以及米洛那副如丧考妣似的悲痛欲绝的样子,约塞连根本就无动于衷。随军牧师的声音从很远的地方轻轻传来,那单调的声音含混不清,几乎一句话也听不出,就像一种虚无的喃喃低语。约塞连从那个骨瘦如柴的高高身影辨认出梅杰少校,还相信自己也认出那个正在用手帕擦额头的人是丹比少校。丹比少校自那次与德里德尔将军冲突过后就从没停止过发抖。几排士兵围着这三个军官,站成一个弧形,像一根根木桩子似的直挺挺地立在那里。四个闲着无事、身穿条子工作服的掘墓人,身体倚着铲子,带着一脸的冷漠,站在那一大堆难看的紫铜色的松土旁。在约塞连盯着他们看的时候,牧师抬眼朝约塞连送去了祝福的目光,痛苦似地用手指揉了揉眼睛,然后又用探究的目光注视着约塞连这个方向,接着低下了头,结束约塞连视之为葬礼高xdx潮的最后程序。那四个穿工作服的人用吊索将棺材吊起来,慢慢放进墓穴。这时米洛的身体猛烈地颤动了一下。

    “我不能再看下去啦,”他极度痛苦地转过脸去叫道,“我可不能光坐在这里,眼睁睁地看着这种场面,而与此同时那些食堂却在让我的辛迪加联合体死亡。”他简直在咬牙切齿,满脸悲哀和忿恨地直摇头。“要是他们真有那么一点忠心的话,他们就会买我的棉花,直到他们发觉亏了本,而一旦这样,他们就会接连不断地买我的棉花,直到他们赔了更大的本。这样,他们就会去放火,将他们的内衣内裤以及夏季制服统统烧掉,好为棉花创造较大的销路。可他们连一下忙都不肯帮。约塞连,你就试试吧,帮我把这团剩下的巧克力糖衣棉花吃下去。也许这会儿味道会很好的。”

    约塞连推开了他的手。“得了吧,米洛。人是不能吃棉花的。”

    米洛狡猾地堆起了一副笑脸。“这并不真的是棉花,”他哄骗道,“我刚才是开玩笑的。这其实是棉花糖,是美味的棉花糖。你再尝尝看。”

    “你在撒谎。”

    “我从不撒谎!”米洛带着一种自豪的庄重神情反驳说。

    “你此时就在撒谎。”

    “我只在必要的时候才撒谎,”米洛为自己辩解道,同时将目光移开了一会,一面怪可爱地眨动着他的眼睫毛,“这东西比棉花糖要好,真的。它是用真正的棉花做成的。约塞连,你得帮着我让大伙将这东西吃下去。埃及棉花可是世界上最最好的棉花呀。”

    “可它不能被消化,”约塞连强调说,“它会让大伙生病,这你不明白吗?要是你不信我的话,你自己干吗不试试靠吃棉花过日子呢?”

    “我试过了,”米洛沮丧地承认道,“它使我很不舒服。”

    墓地里一片黄色,是那种夹着青色的干草颜色,就像烧熟的卷心菜。过了一会,牧师朝后退了几步,那一小群围成半圆形、穿着米色制服的人像漂浮在水面上的碎片一样,开始缓缓散开。这些人不急不慢、不声不响地朝着各自沿高低不平的土路停放着的车辆飘了过去,牧师、梅杰少校和丹比少校不在这些人当中,他们自成一队,郁郁寡欢地朝着他们各自的吉普车走去,彼此间保持着几英尺的距离,好像素不相识似的。

    “一切都结束了,”约塞连说。

    “一切都完了,”米洛丧气地赞同道,“一点希望也没有了。这都是因为我让他们自作决定的结果。这倒给了我一个教训:下一次我要是再干类似的事情,我一定要先明确纪律。”

    “你干吗不把棉花卖给政府?”约塞连漫不经心地建议道,眼睛则盯着那四个穿条子工作服的人,他们正在将一铲铲紫铜色的泥土扔回到墓穴里去。

    米洛断然否定了约塞连的想法。“这可是个原则问题,”他以决然的口气解释说,“政府无权做生意,而我也是世界上最不愿让政府卷入我的生意的人。不过政府的职责就是做生意。”他突然灵机一动,想起了什么,于是得意洋洋地继续说道,“这话是卡尔文-柯立芝说的,卡尔文-柯立芝当过总统,所以他的话是不会错的。我弄到了那么多的埃及棉花,可没人肯要,政府有责任把它们统统买下来,这样我就可以有大赚头了,不是吗?”米洛的脸突然又阴沉下来,情绪一下子一落千丈,变得焦虑不安。“可我怎样才能让政府买下我的棉花呢?”

    “行贿嘛。”

    “行贿!”米洛勃然大怒,差点儿再次失去平衡,跌断自己的脖子。“你真可耻!”他厉声呵斥道,从他那翕动不已的鼻孔和一本正经的双唇里喷出的气息,如同正直的火焰,上下翻动着,直冲他上唇那抹铁锈色的小胡子。“行贿犯法,这你是知道的。可是做生意赚钱是不犯法的,对吧?所以,对我来说,为赚点正当的利润而去贿赂某人,这不能算犯法,不是吗?不算,当然不算犯法!”他又一次陷入了沉思,脸上挂着逆来顺受和近乎可怜的苦恼表情。“可我又怎么知道该贿赂谁呢?”

    “哦,这你不用担心,”约塞连窃笑了一下,用平淡的语调安慰他说。此时吉普车和救护车发动引擎的声音打破了使人昏昏欲睡的寂静,排在后面的车辆也开始倒着开走了。“只要你行贿的数目大,他们会来找你的。有一点务必要做到,那就是你一切都得说在明处。要让每一个人都明明白白地知道你想干什么,肯为此而出多大的价钱。假如你第一次行事时表现出一副心中有鬼或问心有愧的样子,那你就要倒霉了。”

    “我希望你能和我一起去办这事,”米洛说,“和那些受贿的人呆在一起我感到很不安全。这些家伙比一帮骗子好不了多少。”

    “你不会有事的。”约塞连很有把握地向他担保。“要是你碰到了麻烦,那你就让每一个人都知道,为了美国的安全,需要有一个强大的埃及棉花投机企业。”

    “确实需要,”米洛神情庄重地对他说,“有了强大的埃及棉花投机企业就意味着有了一个更强大的美国。”

    “这是当然的啦。要是这招不灵,那你可以列出数字,说明有多少美国家庭得依赖该企业的存在来谋取收入。”

    “确实有许许多多的美国家庭得靠它来取得收入。”

    “你明白了?”约塞连说,“这些你比我更在行。你几乎让这事听起来像真的一样。”

    “本来就是这么回事嘛,”米洛大声他说,脸上重又明显地挂上了他原来的那副傲慢神气。

    “我正是这个意思。你就带着这种深信不疑的信念去干吧。”

    “你真的不愿和我一道去?”

    约塞连摇了摇头。

    米洛急不可耐地想行动了。他将那团剩下的巧克力糖衣棉花塞进了他的衬衣口袋,然后战战兢兢、一点一点地顺着树枝向后挪着,一直挪到那光滑的灰色树干。接着,他张开双臂笨拙地抱住树身,开始向下滑去,可他穿的皮底鞋的鞋边老是打滑,因此有好几次他险些跌卞去,将自己摔伤。滑了一半的时候,他突然改变了主意,又重新爬了上去。他的唇须上沾满了树皮的碎屑,那张紧张的脸因用劲而涨得通红。

    “我希望你把制服穿起来,不要像这样一丝不挂地到处乱跑。”

    在他重新爬下树匆匆离去之前,他忧郁地向约塞连吐露了自己的担忧。“你这样有可能会带出一股风气,这一来我的那些该死的棉花就永远也脱不了手了。”
