\chapter{丹尼卡太太}
 
    卡思卡特上校得知丹尼卡医生也死在麦克沃特的飞机上后,便把飞行任务增加到了七十次。

    中队里第一个发现丹尼卡医生死了的是陶塞军士。事故发生前,机场指挥塔台上的那个人就告诉过他,麦克沃特起飞前填写的飞行员日志上面有丹尼卡医生的名字。陶塞军士抹去一颗泪珠,从中队的花名册上勾掉了丹尼卡医生的名字。随后,他站起身,嘴唇依然颤抖着,步履沉重地硬撑着走出门去,把这个不幸的消息告诉洛斯和韦斯。经过传达室和医务室帐篷之间时,他看见在落日的余晖里,丹尼卡医生耷拉着脑袋坐在自己的凳子上。他小心翼翼地从这位瘦小的令人感到阴森可怕的航空军医身旁绕过去,没有跟他说一句话。陶塞军士的心情非常沉重。眼下他手上有两个死人——

    个是约塞连帐篷里的死人马德,这家伙甚至根本没到那帐篷去过;另一个就是中队里刚刚死去的丹尼卡医生,此人毫无疑问仍然在中队里,而且,种种迹象表明,这个人的问题对他的行政勤务工作来说将会更加棘手。

    格斯和韦斯带着惊奇而淡漠的神情听陶塞军士讲完这件事,没有向任何人说一句表示他们悲痛心情的话。大约一小时后,丹尼卡医生走进来要求量体温和测血压,这是这一天里他第三次提出这种要求。他平时的体温就比一般人低,只有九十六点八度,可这次测量出的体温又比他平日的体温低半度。丹尼卡医生不由得惊慌起来。更叫他恼火的是,他手底下的这两个士兵木头人似的呆呆地死盯住他。

    “真他妈的该死。”他内心极为恼怒,不过还是很有礼貌地劝诫他们俩。“你们两个人到底怎么了?一个人如果一直体温偏低,散步时鼻子又不通气的话,那就不正常了。”丹尼卡医生闷闷不乐自怜自爱地吸了吸鼻子,忧心忡忡地走到帐篷的另一边拿了些阿司匹林和磺胺药片吃下去,接着又往喉咙里喷了点弱蛋白银。他那张愁眉不展的面孔显得虚弱、凄惨,就像一只孤燕。他有节奏地揉搓着两只臂膀的外侧。“瞧瞧,我现在身体冰凉冰凉的,你们真的没对我隐瞒什么事情吗?”

    “你已经死了,长官,”他手底下这两个士兵中的一个解释道。

    丹尼卡医生猛地抬起头来,愤愤地望着他们,疑惑不解地问:

    “你说什么?”

    “你已经死了,长官,”另一个士兵重复道,“也许这就是你总是感到身体冰凉的原因。”

    “不错,长官。你大概死了很久了,我们原先不过没觉察出来罢了。”

    “你们俩究竟在胡说些什么?”丹尼卡医生尖叫起来。他本能地感到某种不可避免的灾难正在向他逼近,一时间竟愣住了。

    “这是真的,长官,”其中一个士兵说,“记录表明,你为了统计飞行时间,上了麦克沃特的飞机。而且,你没有跳伞降落,所以飞机坠毁时你肯定牺牲了。”

    “是啊,长官,”另一个士兵说,“你居然还有体温,你应该高兴才对。”

    丹尼卡医生顿时头晕目眩。“你们俩都疯了吗?”他质问道,“我要把这个犯上事件原原本本地报告给陶塞军士。”

    “就是陶塞军士告诉我们这件事的,”不知是格斯还是韦斯说,“陆军部已经准备通知你的妻子了。”

    丹尼卡医生大叫一声,冲出医务室帐篷去找陶塞军士提出抗议。陶塞军士厌恶地侧身躲开他,并且劝告他在军方就他的遗体安排作出某种决定之前尽量少露面。

    “唉,我想他真的死了,”他手底下的一个士兵恭恭敬敬地低声叹息道,“我会怀念他的。他是个很了不起的家伙,不是吗?”

 


    “是啊,他当然是,”另一个士兵悲伤他说,“不过这个小王八蛋死了,我还是很高兴的。天天给他测量血压,我都快烦死了。”

    得知丹尼卡医生的死讯后,丹尼卡医生的妻子丹尼卡太太非常难过。当她收到陆军部通知他丈夫阵亡消息的电报时,她悲痛欲绝,尖厉的恸哭声刺破了斯塔腾岛宁静的夜空。女人们前去安慰他,她们的丈夫也登门吊唁,心里却盼望着她赶快搬到别处去,免得他们不得不三天两头地向她表示同情。几乎整整一个星期,这可怜的女人完全心神错乱。随后,她慢慢地恢复了勇气和力量,开始为自己和孩子们多钟的前途作通盘打算。就在她渐渐听天由命地接受了丈夫的死亡时,邮递员前来按了一下门铃,带来了一个晴天霹雳——封有她丈夫亲笔签名的海外来信。信中再三嘱咐她不要理会任何有关他的坏消息。这封信把丹尼卡太太惊得目瞪口呆。

    信封上的日期已经无法辨认,信上的字迹从头到尾歪歪扭扭、潦潦草草,不过字体倒像是她丈夫的。而且,字里行间流露出的那种忧郁凄凉自怜自爱的情绪虽然比往常更消沉,但却是她熟悉的。丹尼卡太太大喜过望,心中如释重负,一边纵情大哭,一边无数次地吻着那封皱巴巴脏兮兮的缩印邮递信笺。她匆匆忙忙写了一封充满感激之情的短信给她的丈夫,催促他快点来信告诉她详情。她又赶快给陆军部拍了一份电报,指出他们的错误。陆军部生气地回复说,他们没有犯任何错误,她肯定是受骗上当了,那封信肯定是她丈夫所在中队的某个虐待狂和精神病患者伪造的。她写给丈夫的信被原封不动地退了回来,信封上盖着阵亡两个字。

    冷酷的现实又一次使丹尼卡太太失去了丈夫,不过,这一回她的悲痛多多少少减轻了几分,因为她收到了一份来自华盛顿的通知,那上面说,她是她丈夫一万美元美国军人保险金的唯一受益人,这笔钱她随时可以领取。她意识到自己和孩子眼下不会挨饿了,脸上不禁露出一个无所畏惧的微笑。她的悲痛从此出现转折。

 


    就在第二天,退伍军人管理局来函通知她,由于她丈夫的牺牲,她今后有权终生享受抚恤金,此外还可以得到一笔二百五十美元的丧葬费。来函内附着一张二百五十美元的政府支票。毫无疑问,她的前途一天天光明起来。同一星期,社会保障总署来函通知她说,根据一九三五年《老年和鳏寡保险法令》的条例,她和由她抚养的十八岁以内未成年儿女都可以按月领取补助费,此外她还可以领取二百五十美元的丧葬费。她以上述政府公丞作为丈夫的死亡证明,申请兑付丹尼卡医生名下的三张保险金额均为五万美元的人寿保险单。她的申请很快得到认可,各项手续迅速办理完毕。每天都给她带来出乎意料的新财富。她得到一把保险箱的钥匙,在保险箱里找到了第四张面值五万美元的人寿保险单,以及一万八千美元的现金,这笔钱从来没有交纳过所得税,而且永远也不必交了。丈夫生前所属的某个兄弟互助会的分会向她提供了一块墓地。

    另一个他生前参加过的兄弟互助组织给她寄来了二百五十美元的丧葬费。他县里的医学协会也给了她二百五十美元的丧葬费。

    她最亲密的女友们的丈夫开始和她调情。事情发展成这种结局,丹尼卡太太开心极了。她甚至把头发都染了。她那笔惊人的财富仍在不断增加,她不得不天天提醒自己,没有丈夫来和自己分享这笔源源而来的巨款,她手头的这几十万美元等于一钱不值。使她感到惊奇的是,有这么多互不相干的组织都愿意帮助安葬丹尼卡医生。而此时,皮亚诺萨岛上的丹尼卡医生却为了不被埋入地下而苦苦挣扎。他终日垂头丧气惶恐不安,想不通他的太太为什么不回他写的那封信。

 


    他发现中队里人人见了他都避之不及。大伙用下流恶毒的语言咒骂他这个死人,因为正是他的死惹恼了卡思卡特上校,这才又一次增加了战斗飞行任务的次数。有关他阵亡的证明材料像虫卵一样剧增,而且彼此互为佐证,无可争议地判定了他的死亡,他领不到军饷,也得不到陆军消费合作社的配给供应,只好靠陶塞军士和米洛的施舍勉强度日,这两个人也都知道他已经死了。卡思卡特上校拒绝接见他,科恩中校则叫丹比少校捎过话来,丹尼卡医生要是胆敢在大队部露面的话,他就要叫人当场把他火化掉。丹比少校还私下里告诉他,邓巴中队里有一名姓斯塔布斯的航空军医,他长着一头浓密的头发和一个松弛下垂的下巴,是个邋邋遢遢不修边幅的人,他存心跟上级作对,极其巧妙地使那些完成了六十次战斗飞行任务的空勤人员全都留在了地面上,结果弄得大队里人心浮动,敌对不满情绪甚嚣尘上。大队部愤怒地斥责了他的这种做法,命令那些给弄得莫名其妙的飞行员、领航员、轰炸手和机枪手重返岗位执行战斗任务。队里的士气迅速低落下去,邓巴也遭到了监视。由于这个缘故,大队部对所有的航空军医都非常敌视。所以,丹尼卡医生阵亡以后,大队部十分高兴,不打算请求上级再派一名军医来。

    在这种情况下,就连牧师也没有办法让丹尼卡医生起死回生。

    丹尼卡医生起初惊慌失措,后来就只好听天由命了。他的模样越来越像一只病恹恹的老鼠,眼睛下面的眼袋变得又瘪又黑。他在阴影里徒劳无益地徘徊着,活像一个无处不在的幽灵。甚至当他在树林里找到弗卢姆上尉请求帮助时,后者也赶快躲得远远的。格斯和韦斯无情地把他从医务室帐篷里赶了出去,甚至连一只体温表也没让他带走。只是到了这个时候,他才真正意识到,自己实质上已经死了,如果他还想救活自己的话,那就得赶快采取行动。

    他没有别的办法,只有向妻子求援。他潦潦草草写就一封感情真挚的信,恳求妻子提请陆军部注意他目前的困境,催促她立刻给他的大队指挥官卡思卡特上校写信,以便证实——无论她听到了什么别的谣传——的确是他,她的丈夫丹尼卡医生,而不是什么死尸和骗子,在向她恳求。丹尼卡太太收到了这封潦草得几乎无法辨认的信,信中流露出的一片深切情感强烈地震撼了她的心灵。她悔恨交加,深感不安,打算马上照丈夫的话办,可就在这一天,她接下来拆开的第二封信就是她丈夫的大队指挥官卡思卡特上校寄来的。信是这样开头的:

    亲爱的丹尼卡太太/先生/小姐/先生和太太:

    您的丈夫/儿子/父亲或兄弟在战斗中牺牲或负伤或失踪,对此,语言无法表达我个人所感受到的深切悲痛。

    丹尼卡太太带着孩子们搬到密执安州的兰辛去了,连信件转递地址都没有留下
