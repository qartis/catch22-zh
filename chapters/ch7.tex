\chapter{麦克沃特}
 
    通常,与约塞连搭档的飞行员是麦克沃特。每天清晨,麦克沃特总是穿了洁净的大红睡衣裤,在自己的帐篷外面刮胡子。约塞连身边有不少莫名其妙、令人啼笑皆非的怪人,麦克沃特就是其中一个。在所有参战官兵当中,麦克沃特兴许是最古怪的一个,因为他神志十分正常,可对战争依旧无动于衷。他腿短肩宽,年纪很轻,常面带笑容,口里总不停地哼唧欢快的流行曲调。每次玩二十一点或是打扑克牌时,总要把牌摔得劈啪响,结果,摔得亨格利-乔心烦意乱、浑身不爽,亨格利便厉声责骂,让他别再这样摔牌。

    “你这婊子养的,你是存心折磨我,”亨格利-乔便会大声怒骂,一旁的约塞连则会用一手拦住他,让他消气镇静。“他是故意跟我作对,因为他喜欢听我歇斯底里地喊叫——你这狗杂种!”

    麦克沃特很感抱歉地皱了皱雀斑点点但长得挺漂亮的鼻子,发誓以后再不摔牌,但总是过后便忘。麦克沃特穿的是大红睡衣裤和室内软拖鞋,睡觉时盖的是新熨烫过的印花被单——极似米洛从那个嬉皮笑脸、嗜爱甜食的小偷处取回的那半条被单。当初,去取那半条被单时,米洛向约塞连借了些去核枣,结果,一颗没用。麦克沃特对米洛印象极深,原因是,米洛总是把七分钱买的鸡蛋以五分钱的价格卖出去,这实在是让给养军士斯纳克下士觉得有趣。不过,麦克沃特对米洛的印象,从来就没有米洛对约塞连从丹尼卡医生手上得来的那张肝病证明的印象深刻。

    “这是什么?”米洛惊讶地叫道,他发现了那只大大的瓦楞纸板箱,里边装满了一包包干果、一听听果汁和甜点心,两名意大利劳工——是德-科弗利少校诱拐来替他在厨房干活的——正准备搬了这箱子去约塞连帐篷。

    “这是约塞连上尉,长官,”斯纳克下士很是神气活现地笑了笑,说道。斯纳克下士一向自认为很有知识,觉着自己领先时代二十年。他实在很讨厌给大伙儿煮饭。“他有丹尼卡医生出具的证明,不管他想要什么水果和果汁,他都可以享用。”

    “这是怎么回事儿?”约塞连大叫道,这当儿,米洛脸色煞白,又摇晃了起来。

    “上尉,这是米洛-明德宾德中尉,”斯纳克下士嘲讽地眨了眨眼,说道,“是新来的一位飞行员。这一次你住院期间,他当上了司务长。”

    当天傍晚,米洛交给麦克沃特半条床单,麦克沃特大叫道:“这是什么?”

    “就是今天上午从你帐篷里偷走的那半条床单,”米洛兴致勃勃且又沾沾自喜地给他做了解释,赭色的鬓须急速地抽搐着。“我敢说,你甚至还不知道床单让人给偷去了呢。”

    “怎么竟会有人要偷半条床单?”约塞连问。

    米洛紧张不安了。“这你是不会懂的,”他抗辩道。

    米洛为何如此迫不及待地花钱,想从丹尼卡医生那儿买一张简捷的证明,对此,约塞连始终弄不明白。丹尼卡医生在证明书上写道:“请把约塞连所要的全部干果和果汁给他。他说他的肝脏有病。”

    “像这样的证明,”米洛沮丧地咕哝道,“足以葬送天底下任何一位司务长的前程。”米洛来到约塞连的帐篷,就是想再看一看那张证明。他跟在那一盒发给约塞连的食物的后面,穿过中队营地,活像在给什么人送葬似的。“你要多少,我都得给你。嗨,这证明可没说你必须一人独吃。”

    “没那么说,倒是桩好事,”约塞连告诉他说,“因为我向来就不吃这东西。我的肝脏不好。”

    “哦,对了,我把这给忘了,”米洛很是恭敬,放低了嗓音说道,“情况糟吗?”

    “糟糕得很呢,”约塞连快乐地答道。

    “是这样,”米洛说,“这话怎么讲?”

    “就是说,情况不可能比这会儿再好了……”

    “我想我还是听不明白。”

    “……再好的话,那就更糟了。现在你明白了?”

    “是的,我现在明白了。不过,我想我还是不懂你的意思。”

    “算啦,你就别为这事费神了。让我自个儿来烦心吧。你知道,我其实没什么肝病,只是有了些症状而已,是加涅特-弗莱沙克综合症。”

    “是这么回事儿,”米洛说,“那什么是加涅特-弗莱沙克综合症?”

    “就是肝病。”

    “我明白了,”米洛说着,便不耐烦地摩挲起自己的两道浓黑的眉毛,露出了苦涩的神情,仿佛在煎熬什么令人浑身不自在的痛楚。“既然如此,”他最后接着说,“我想你的确得好好留心自己的饮食,是不是?”

    “是得好好留心,”约塞连跟他说,“有益的加涅特-弗莱沙克综合症,是不怎么容易得到的,而我呢,又不想把自身的这种症状给毁了,所以,我从来就不吃什么水果。”

    “这下我可真明白了,”米洛说,“水果有损你的肝脏?”

    “不,水果对我的肝脏很有好处。所以,我绝对不吃。”

    “那你要了水果做什么?”米洛越搞越糊涂,可他不罢休,费了好大的劲,才把憋了老半天不说的这句问话吐了出来。“你把水果卖了?”

    “我送人。”

    “送给谁?”米洛叫道,惊愕得连嗓音都变了样。

    “谁要就送谁。”约塞连高声回敬了一句。

    米洛很忧戚地发出一声长长的哀叹,摇晃着后退了几步,苍白的脸上突然冒出一颗颗汗珠。他心不在焉地硬拽着那两撇丧气的八字须,浑身直打战。

    “我送了不少给邓巴,”约塞连接着又说。

    “邓巴?”米洛机械地重复了一遍。

    “没错。邓巴要多少水果,就能吃多少,可这对他压根就没一点好处。那盒子我就放在帐篷外面,谁想要,就自个儿来取。阿费来这儿拿些李子,因为他说,食堂里的李子从来就不够他吃。你什么时候有空,应该查一查这事,因为阿费老在这里闲荡实在不是什么趣事。什么时候盒子里的水果不多了,我就让斯纳克下士重新给我添满。内特利每次去罗马,总要带足了水果。他爱上了那儿的一个妓女。那个妓女很讨厌我,不过,对他也没有丝毫的兴趣。她有个小妹妹,从来就没让他俩单独上过床。他们住的是一幢公寓楼,合住的房客有一对老头老太,还有一群别的女孩——个个长有两条肥壮迷人的大腿,总是戏谑不止。内特利每次上那儿,总给她们捎带一整盒水果。”

    “是卖给她们?”

    “不,是送给她们。”

    米洛蹩起了额头。“喔,我想他倒是挺慷慨的,”他漠然地说。

    “没错,的确挺慷慨,”约塞连赞同道。

    “而且我敢保证,这绝对合法,”米洛说,“因为一旦食物从我这儿到了你手里,便是你的了。我猜想,这些人境况那么恶劣,能弄到水果,一定高兴得很。”

    “是的,确实很高兴,”约塞连深信不疑地对他说,“那两个姑娘把水果全拿到黑市上去卖,再用挣到的钱,去买俗艳的人造珠宝饰物和廉价香水。”

    米洛振作了起来。“人造珠宝饰物!”他惊叫道,“我怎么不知道?买廉价香水她们得花多少钱?”

    “那老头卖了自己的一份水果,去买纯威士忌酒和色情图片。

    他是个色鬼。”

    “色鬼?”

    “倒不是你所想的那样。”

    “色情图片在罗马是不是很有市场?”米洛问。

    “情况并非像你想的那样。就说阿费吧。你认识他,从来就不会怀疑他,是不是?”

    “难道他也是个色鬼?”

    “不是。他是个领航员。你认识阿德瓦克上尉,是不是?这家伙人挺不错,你到中队的第一天,他就跑来见你,说:‘我叫阿德瓦克,干的是领航。’当时,他嘴里叼了个烟斗,好像还问了你上过哪所大学。你是不是认识他?”

    米洛压根就没理会。“让我跟你合伙干吧,”他冷不丁地恳求道。

    约塞连拒绝了他的恳求,即使他毫不怀疑,一旦他凭丹尼卡医生的证明,从食堂申请领取了一卡车一卡车水果,那么,这些水果就归他们所有,他们爱怎么处理就怎么处理。米洛很是丧气,不过,从那以后,除一桩事以外,他什么秘密都跟约塞连说,因为他敏锐地感悟出,凡是不窃取自己所爱国家的财产者,绝不会偷盗他人的财物。对约塞连,米洛毫无保留,有秘密便讲,但关于山上那些洞——从士麦那运回一飞机无花果后,听约塞连说,刑事调查部的一名工作人员住进了医院,他便开始把钱埋在了洞里——的位置,他始终没吐半个字。米洛极易受骗,结果,便自告奋勇当上了司务长,不过,在他,这实在是神圣的职责。

    “食堂里的李子不够吃,我竟连这还不知道呢,”上任后的第一天,米洛承认道,“我想这是因为我对一切还相当不熟悉。我会跟厨师长提这事的。”

    约塞连机警地注视着他。“什么厨师长?”他问道,“你哪来的厨师长?”

    “斯纳克下士,”米洛解释道,很有些歉疚地把目光移向了别处。“他是我唯一的厨师,其实,也就是厨师长,虽然我希望让他负责行政勤务。依我的感觉,斯纳克下士似乎过于锋芒毕露了。在他看来,当一名给养军士实在只是一种摆设而已。他老是抱怨说,自己是被迫糟蹋才华。可压根就没人让他非做这事不可!顺便问一下,你是否知道他当初为什么被降为列兵,至今还只是个下士?”

    “知道,”约塞连说,“他在中队的食物里下过毒。”

    米洛听罢,脸色再次刷白。“他做什么?”

    “他把数百块军用肥皂捣碎成泥,羼入白薯中,只是想证明大家的口味很平庸,不辨优劣。中队的全体官兵都病了。飞行任务被迫取消。”

    “啊!”米洛惊呼道,颇有些异议。“他一定发觉自己铸成了大错,是不是?”

    “恰好相反,”约塞连纠正道,“他觉得这事他做得对极了。我们每个人都吃了满满一盘,还一个劲地嚷着要他再给添满。我们都知道自己病了,但万万没想到是中了毒。”

    米洛惊愕地倒吸了两口气,模样极似一只棕色的粗毛野兔。

    “既然如此,我就非得让他去负责行政勤务不可了。我可不希望在我主管期间出这种事。你知道,”他颇严肃他说出了真心活,“我想做的,就是要让中队的弟兄们一日三餐吃上全世界最好的饭菜。这才是司务长应尽的职责,你说对不?假如他连这最起码的目标都达不到,那么,他就不配做一名司务长。你同意吗?”

    约塞连缓缓地转过身,深表怀疑地直视着米洛。在他眼前的,是一张单纯、诚实的脸,绝不会做出任何奸诈狡猾或是不择手段的勾当;是一张正直、坦诚的脸,嵌一对斜视的浓眉大眼,长一头赭发和两撇丧气的红棕色八字须。米洛的鼻子极长,且瘦尖,鼻孔始终是湿滴滴的,不时哧哧地吸鼻子,鼻尖右歪得厉害,总与身体其余部位的面向相悖。这是刚正不阿者的脸:他绝不可能有意识地违背作为其正直品性依赖的道德准则,如同他不可能把自己变成令人厌恶的可鄙小人一样。这些道德准则之中,有一条即是,只要实际情况允许,无论要价多少,也算不得是罪孽。米洛时时会表现出极大的义愤。当听说刑事调查部的一名工作人员正在这一带找他时,他简直气愤到了极点。
 


    “他找的不是你,”约塞连说,想让他消气。“是住院的一个人,哪家伙检查信件时,老是签上华盛顿-欧文的名字。”

    “我可从来没有在什么信件上签华盛顿-欧文的名字,”米洛声言道。

    “那当然。”

    “不过,这只是个骗局,目的是想让我承认自己一直在黑市上捞钱。”米洛狠拽了自己那一撮凌乱的变了色的八字须。“我讨厌那种家伙。总是鬼头鬼脑地四处打探我们这些人的秘密。假如政府想做些什么好事,它干吗不追查前一等兵温特格林?他眼里可从来没有什么规章制度,老是跟我砍价。”

    米洛的八字须之所以触楣头,是因为左右两撇向来是不相称的,就跟他的那对斜眼一样,永远无法同时看着同一样东西。较之大多数人,米洛眼见的东西要多些,但没一样他是看得真切的。当获知刑事调查部那名工作人员的消息时,他的反应极其激动,但相比之下,在听约塞连说,卡思卡特上校已经把飞行次数增加到五十五次之后,他倒是颇显得沉着勇敢。

    “这可是在打仗,”他说,“所以,规定的飞行次数,我们必须完成,发牢骚是毫无用处的。假如上校说我们必须飞五十五次,我们就得不折不扣地飞满五十五次。”

    “哦,我可不必飞那么多次,”约塞连发誓说,“我要去见梅杰少校。”

    “你能行吗?梅杰少校向来不见任何人。”

    “那我就回医院去。”

    “可你出院才十天,”米洛提醒他说,语调里颇有些责备的成份。“你总不能一遇到什么不如意的事儿就往医院跑吧。不能这样,最好还是完成规定的飞行次数。这可是我们的职责。”

    米洛办事相当固执死板,且顾虑重重。因此,就在麦克沃特的床单被窃那天,他怎么也不愿从食堂借用一袋去核枣子,因为食堂的食品依然都是政府的财产。
 


    “不过我可以向你借,”他给约塞连解释道,“因为所有这些水果,一旦你凭丹尼卡医生的证明从我这里领到手,就都归你了。你想怎么处理就怎么处理,甚至可以不送人,高价出售。难道你不想跟我合伙干?”

    “不想。”

    米洛只得作罢。“那就借我一袋去核枣,”他恳求道,“我会还你的。我向你保证,而且会多给你一些分外的东西。”

    米洛言而有信。回来见约塞连时,把那袋去核枣原封未动地还给了他,此外,还交给他麦克沃特那条黄色床单的四分之一。而且,米洛把那个毗牙咧嘴、喜吃甜食的小偷——从麦克沃特帐篷里窃得床单的便是他——也一起带了回来。这块床单,现在就归约塞连所有了。这床单到他手上的当儿,他正打着盹儿,不过、他自己不明白究竟是怎么回事。麦克沃特也同样糊里糊涂。

    “这是什么东西?”麦克沃特大声叫道,直盯着撕下来的半条床单,很是困惑不解。

    “这就是今天上午你帐篷失窃的那条床单的一半,”米洛解释说,“我敢打赌,你连床单被人偷了还不知道哩。”

    “干吗要偷半条床单?”约塞连问。

    米洛慌了神儿。“你不明白,”他抗辩道,“小偷偷走的是整条床单。我就用你投资的那袋去核枣,把它给换了回来。所以,床单的四分之一就归你了。你的投资,收获可不小啊,尤其是因为你收回了给我的每一颗去核枣。”接着,米洛又对麦克沃特说,“另外半条床单就归你,因为这整条床单本来就是你的。我实在搞不明白,你究竟埋怨些啥。要不是约塞连上尉和我为了你插手此事,你恐怕连床单的一角都甭想拿到。”

    “谁埋怨啦?”麦克沃特大声嚷道,“我只不过是想看看,该怎么处理这半条床单。”

    “你用半条床单可做不少东西哩。”米洛向他断言。“床单的另外四分之一,我自己留下了,作为对自己积极进取、工作一丝不苟的奖励。你知道,这可不是为我自己,而是为了辛迪加联合体。你那半条床单或许可以在这里派上用处。你可以把它留存在辛迪加联合体,看着它生利。”

    “什么辛迪加联合体?”

    “就是有朝一日我想成立的那个联合体,这样一来,我就可以给弟兄们供应你们理该得到的美味可口的食品。”

    “你想成立辛迪加联合体?”

    “没错,是这样。说确切一点,就是一个市场。你可知道什么是市场?”

    “就是买东西的地方,对吗?”

    “还有卖东西,”米洛纠正道。

    “还有卖东西。”

    “我一辈子都想要个市场。有了市场,你就可以做许多事儿。

    但,你首先得有个市场。”

    “你想要一个市场?”

    “而且人人都有一股。”

    约塞连还是困惑不解,因为这是生意经,再说,生意经方面总有不少东西令他费解。

    “让我再给你解释解释。”米洛主动提议,但尽管如此,还是愈发不耐烦,继而颇感恼怒。他猛地竖起大拇指,直指站在他一旁的那个喜甜食的小偷——还一个劲地龄牙咧嘴地笑呢。“我知道,枣子和床单之间,他更喜欢枣子。正因为他对英语一窍不通,所以,在处理这件事的过程中,我自始至终说的是英语。”

    “你干吗不在他头上狠打一下,再把床单夺过来呢?”约塞连问道。

    米洛极严肃地紧抿了双唇,摇摇头。“那样的话,就太不公平了,”他严厉地责备道,“暴力是错误的,两个错误绝对不会变成正确。相比之下,我的方法可高明多了。当我把枣子递给他,再又伸手取床单时,他很可能以为我是在主动跟他做交易。”

    “那你究竟是在干什么?”

    “说真的,当时我确实是主动在跟他做交易,但既然他不懂英语,我就随时都可以否认这一点。”

    “要是他生了气,一定得要那些枣子呢?”

    “嗨,我们只要在他头上狠打一下,拿了枣子便走不就得啦。”

    米洛答得极干脆。他看看约塞连,又看看麦克沃特,然后,看看麦克沃特,再又看看约塞连。“我实在不明白,大伙儿发什么牢骚。我们这会儿的日子比以前可要强多了。没有谁活得不滋润的,只有这小偷除外,不过,也用不着替他操心,因为他连我们的语言都说不来,活该有这么个下场。你明白了吧?”

    然而,米洛在马耳他买鸡蛋,七分钱一只,可他在皮亚诺萨出售时,却是五分钱一只,最终还赚了钱。这到底是怎么一回事,约塞连终究还是没有弄明白
