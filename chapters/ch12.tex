\chapter{博洛尼亚}
 
    其实,那场博洛尼亚大恐慌完全是由奈特中士一手造成的,与布莱克上尉毫无关系。奈特中士一听说要去轰炸博洛尼亚,就悄悄溜下卡车,又取来了两件防弹衣。这一来,其余的人也跟着效仿,一个个铁板着脸跑回降落伞室,没等抢完余下的防弹衣,便已溃军似地慌乱成一团了。

    “嗨,这是怎么回事儿?”基德-桑普森很不安地问道,“博洛尼亚还不至于那么危险吧?”

    内特利恍惚地坐在卡车铺板上,双手捂住那张年轻但阴沉的脸,没答话。

    造成这一局面的,是奈特中士,以及无数次折磨人的任务延期。就在命令下达后的头天上午,大伙正在登机,突然来了一辆吉普车,通知他们说,博洛尼亚正在下雨,轰炸任务延期执行。待他们返回中队驻地,皮亚诺萨亦下起了雨。那天,回到驻地后,他们全都木然地凝视着情报室遮篷下那张地图上的轰炸路线,脑子昏昏欲睡,始终是一个念头:这次他们是无论如何没有了退路。那条横钉在意大利大陆上的细长的红缎带,便是醒目的证据:驻守意大利的地面部队被牵制在目标以南四十二英里的地方,根本就没法往前进逼一步。因此,他们是无论如何也攻不下博洛尼亚城的。而屯扎皮亚诺萨岛的空军官兵却是万难躲开这次去轰炸博洛尼亚的飞行任务的。他们陷入了困境。

    他们的唯一希望,便是雨不停地下,但这希望实在是乌有的,因为他们全部清楚,雨终究是要停的。皮亚诺萨停了雨,博洛尼亚便下雨;博洛尼亚停雨,皮亚诺萨便又下雨。假如两地都没了雨,那么,便会出现一些莫名其妙的奇怪现象,诸如流行性腹泻的传播,或是轰炸路线的移动。最初的六天里,他们被召集了四次,听取下达简令,随后又给打发回驻地。一次,他们起飞了,正在编队飞行,突然,指挥塔命令他们降落。雨下的时间越长,他们就越遭罪;他们越是遭罪,也就越要祈求雨不停地下。晚上,大伙通宵仰望天空,满天的星斗让他们深感哀戚。白昼,他们就一天到晚盯着意大利地图上的那条轰炸路线。地图很大,挂在一只摇晃不稳的黑报架上,随风飘动,天一下雨,黑报架便住里拖,置于情报室遮篷底下。轰炸路线是一条细长的红缎带,用来标明布于意大利大陆各处的盟军地面部队的最前沿阵地。

    亨格利-乔与赫普尔的猫拳斗后的次日上午,皮亚诺萨和博洛尼亚都停了雨。机场的起降跑道干了起来,但要硬结,还得等上整整二十四小时。天空依旧是万里无云。郁结在每个兵士心中的怨怼都已化作了仇恨。最先,他们痛恨意大利大陆上的步兵,因为他们没能进占博洛尼亚。之后,他们开始憎恨起那条轰炸路线来了。他们死死盯着地图上的那条红缎带,一盯便是好几个小时,切齿地恨它,因为它不愿上移,将博洛尼亚城包围起来。待到夜幕降临,他们便聚在黑暗中,凭了手电,继续阴森森地注视着那条轰炸路线,心里在默默地哀求,仿佛他们这样郁郁不乐地集体祈祷,可以产生相当的威力,于是,便有了希望,让红缎带上移。

 


    “我实在不敢相信会有这等事,”克莱文杰对约塞连惊叫道,声音忽高忽低,既表示异议,又深感疑惑。“这完全是愚昧迷信,是彻彻底底的倒退。他们混淆了因果关系。这和手碰木头或交叉食指和中指一样毫无意义。难道他们真的相信,假如有人半夜蹑手蹑脚地走到地图前,把轰炸路线移到博洛尼亚上面,我们明天就不必再去执行那次轰炸任务了?你能想象得出?很可能只有我们两个人才是有理智的。”

    至午夜,约塞连用手碰了木头,又交叉了食指和中指,于是,便轻手轻脚地溜出帐篷,把那条轰炸路线上移,盖住了博洛尼亚。

    次日一清早,科洛尼下士鬼鬼祟祟地钻进布莱克上尉的帐篷,手伸进蚊帐,摸到湿漉漉的肩胛,轻轻摇动,直摇到布莱克上尉睁开了双眼。

    “你摇醒我干什么?”布莱克上尉埋怨道。

    “他们占领了博洛尼亚,上尉,”科洛尼说,“我觉得你大概想知道这个消息。这次任务取消了吗?”

    布莱克上尉猛地挺起了身,极有条理地在那两条瘦成皮包骨的细长大腿上挠起了痒痒。不一会儿,他穿上衣服,不及修面,便走出帐篷,眯眼瞧了瞧,一脸怒气。天空晴朗,气温和暖。他冷漠地注视着那张意大利地图。果不出所料,他们已经攻占了博洛尼亚。情报室内,科洛尼下士正取出导航工具箱里的博洛尼亚地图。布莱克上尉打了个极响的哈欠,坐了下来,把两脚翘到桌上,于是,挂通了科恩中校的电话。

    “你打电话吵醒我干吗?”科恩中校埋怨道。

    “他们夜里攻下了博洛尼亚,中校。这次轰炸任务是否取消了?”

    “你说什么,布莱克?”科恩中校咆哮道,“干吗要取消轰炸任务?”

    “因为他们攻占了博洛尼亚,中校。难道还不取消轰炸任务?”

    “当然取消啦。你以为我们现在去轰炸自己的部队?”

    “你打电话吵醒我干吗?”卡思卡特上校对科恩中校抱怨道。

    “他们攻占了博洛尼亚,”科恩中校告诉他说,“我想你大概会希望知道这个消息。”

    “谁攻占了博洛尼亚?”

    “是我们。”

    卡思卡特上校狂喜,因为当初是他自告奋勇要求让自己的部下去轰炸博洛尼亚的,从此,他便以英勇闻名,但现在,又解除了这次令他进退维谷的轰炸任务,却丝毫无损他已赢得的名声。攻克博洛尼亚,也着实让德里德尔将军心花怒放,但他对穆达士上校极为恼火,原因是上校为了告诉他这一消息而叫醒了他。司令部同样也很高兴,于是,决定给攻占博洛尼亚城的指挥官授一枚勋章。所以,他们把它给了佩克姆将军,因为佩克姆将军是唯一一位军官主动伸手要这枚勋章的。

    佩克姆将军荣膺勋章后,便即刻请求承当更多的职责。依照他的意见,战区所有作战部队都应归由他亲任指挥官的特种兵团指挥。他时常自言自语——总带着每次与人争执时必定有的那种殉教者的微笑,令人觉着和蔼可亲又通情达理:假如投弹轰炸敌军算不得是特殊工种,那么,他实在不明白,究竟什么工种才是特殊的。

    司令部曾提出,让他在德里德尔将军手下担任作战指挥,可他极和气地婉言拒绝了。

    “我想的可不是替德里德尔将军执行什么作战飞行任务,”佩克姆将军宽容地解释道,笑嘻嘻的,一副和悦的面容。“我更想替代德里德尔将军,或许更想超过德里德尔将军。这样,我也就可以指挥许多其他将军。你知道,我最出色的才能主要在于行政管理。我就有这种高妙的本领,可以让不同的人的意见统一起来。”

    “他倒是有一种高妙的本领,可以让不同的人都觉得他实在是个讨厌透顶的混蛋,”卡吉尔上校曾怀恨地跟前一等兵温特格林吐出了自己的心里话,希望他把这句刺耳的话传扬出去,让第二十六空军司令部上上下下都知道。“假如有谁配接任那个作战指挥的职位,那个人就是我。我甚至还想到过,我们应该伸手向司令部要那枚勋章。”

    “你真想参加作战?”前一等兵温特格林问道。

    “作战?”卡吉尔上校惊呆了。“哦,不——你误解我的意思了。

    当然,真要参加作战,我其实也不在乎,不过,我最出色的才能主要在于行政管理。我同样有这种高妙的本领,可以让不同的人的意见统一起来。”

    “他倒是也有一种高妙的本领,可以让不同的人都觉得他实在是个讨厌透顶的混蛋。”后来,前一等兵温特格林来到皮亚诺萨岛,查实米洛和埃及棉花一事时,曾私下里笑着告诉约塞连。“假如有谁配晋升,那就是我。”其实,他调至第二十六空军司令部担任邮件管理员后不久,便接连升级,升到了下士,可后来,因为妄加品藻自己的上级军官,说了些极不中听的话,给传扬出去,结果,一下子又被降为列兵。成功的喜悦,更让他感觉到必须做有道德的人,同时,又激发出他的勃勃雄心,再创一番更崇高的业绩。“你想买几只齐波牌打火机吗?”他问约塞连,“这些打火机是直接从军需军官那里偷来的。”

    “米洛知道你在卖打火机吗?”

    “这跟他有什么关系?米洛不是现在也不兜售打火机了吗?”

    “他当然还在兜售,”约塞连告诉他说,“不过,他的打火机可不是偷来的。”

    “那是你的看法,”前一等兵温特格林哼了一声,回敬道,“我卖一块钱一只。他卖多少钱?”

    “一块零一分。”

    前一等兵温特格林得意洋洋地窃笑了一下。“我每回都占他的上风。”他颇有些幸灾乐祸。“嗨,他那些脱不了手的埃及棉花怎么样了?他究竟买了多少?”

    “全买了。”

    “全世界的棉花?哦,真他妈见鬼!”前一等兵温特格林十足一副幸灾乐祸的劲儿。”简直是头蠢驴!当时你一块儿跟他在开罗,干吗不阻止他呢?”

    “我?”约塞连耸了耸肩,答道,“他能听我的话?他们那儿所有高档饭店都有电传打字电报机。可米洛以前从未见过自动记录证券行市的收报机,就在他请领班给他作解释的时候,埃及棉花的行情报告正巧传了过来。‘埃及棉花?’米洛用他那种惯有的表情问道,‘埃及棉花的售价多少?’接下来,我就知道,他把那些该死的棉花全都买了下来。现在他可真是吃不了兜着走了。”

    “他真是一点想象力都没有。假如他愿意做买卖,我在黑市上就能抛售许多棉花。”

    “米洛了解黑市行情,根本就不需要棉花。”

    “但需要医药用品。我可以把棉花卷在木牙签上,当做消毒药签卖出去。他愿不愿给个合适的价,卖给我?”

    “不管什么价,他都不会卖给你的,”约塞连答道,“你跟他对着干,他很恼火。其实,他对谁都很恼火,因为上星期大家都拉肚子,把他食堂的名声都给搞臭了。对了,你能帮帮我们大伙儿。”约塞连突然抓住他的胳膊。“你不是可以用你的那台油印机伪造一些官方命令,帮我们逃脱这次去轰炸博洛尼亚的任务吗?”

    前一等兵温特格林很轻蔑地瞧了他一眼,慢慢把手臂抽了回去。“我当然可以,”他自豪他说,“但是我做梦都没想过要做那种事。”

    “为什么?”

    “因为这是你的工作。我们大家都各有各的工作。我的工作就是想办法卖掉这些齐波牌打火机,赚几个钱,还有,再从米洛那里买些棉花来。你的工作就是炸掉博洛尼亚的弹药库。”

    “可我会在博洛尼亚给炸死的,”约塞连恳求道,“我们全都会给炸死的。”

    “那你没办法,只得被炸死了,”前一等兵温特格林回答道,“你干吗不学学我,想开些,这都是命中注定的?假如我注定是卖掉这些打火机,赚几个钱,再从米洛那里买些便宜棉花,那么,这就是我要做的事。假如你注定要在博洛尼亚上空被炸死,那你就会被炸死,所以,你最好还是飞出去,勇敢点去死。我不愿这么说,约塞连,可是,你都快成了牢骚鬼了。”

    克莱文杰很赞同前一等兵温特格林的说法,约塞连要做的事,就是在博洛尼亚上空被炸死。当约塞连供认,是他把那条轰炸路线移到了上面,致使轰炸任务被取消,克莱文杰气得脸色发青,狠狠咒骂了一通。
 


    “干吗不可以?”约塞连咆哮道,越发激烈地替自己争辩,因为他自觉做错了事。“是不是因为上校想当将军,我就该让人把屁股给打烂吗?”

    “意大利大陆上的弟兄们怎么办?”克莱文杰同样很激动地问道,“难道因为你不想去,他们就该让人把屁股给打烂吗?那些弟兄有权得到空中支援!”

    “但不一定非得我去不可。瞧,他们并不在乎由谁去炸掉那些弹药库。我们去那里执行轰炸任务,唯一的理由,就是因为那个狗娘养的卡思卡特自愿要求让我们去。”

    “哦,这些我都知道,”克莱文杰跟他说,那张憔悴的面孔显得极苍白,两只焦虑不安的棕色眼睛却是充满了诚挚。“但事实是,那些弹药库还在那里。我跟你一样,也不赞同卡思卡特上校的做法。

    这一点,你很清楚。”克莱文杰停了停,双唇哆嗦着,再握住拳头,对着自己的睡袋轻击了一下,于是,强调说,“但该炸什么目标,或是由谁去轰炸,或者——,这些都不是我们能决定的。”

    “或是谁在轰炸目标时送了命?为什么?”

    “没错,甚至是送命也没法决定。我们无权质问——”

    “你真是疯啦!”

    “——无权质问——”

    “你真的是说,无论我怎么死,还是为什么死,这都不是我的事,而是卡思卡特上校的事?你真是这个意思?”

    “是的,我是这个意思,”克莱文杰坚持说,但似乎很没什么把握。“那些受命打赢这场战争的人,他们的境遇要比我们好得多。他们将决定该轰炸哪些目标。”

    “我们谈的是两回事,”约塞连极其不耐烦他说,“你谈的是空军和步兵的关系,而我说的是我跟卡思卡特上校的关系。你谈的是打赢这场战争,而我说的是打赢这场战争,同时又能保全性命。”

    “千真万确,”克莱文杰厉声说道,显得颇是沾沾自喜。“那么,你说哪一个更重要?”

    “对谁来说?”约塞连马上接口道,“睁开你的眼好好瞧瞧,克莱文杰。对死人来说,谁打赢这场战争,都无关紧要。”

    克莱文杰坐了一会儿,好像挨了猛的一掌。“祝贺你啦!”他极刻薄地喊道,嘴抿紧了,周围现出极细的苍白得无半丝血色的一圈。“我实在想不出还有别的什么态度,更让敌人感到快慰。”
 


    “敌人,”约塞连斟字酌句地反驳道,“就是让你去送死的人,不管他站的是哪一边,自然也包括卡思卡特上校。这一点你无论如何不能忘记,因为你记住的时间越长,你就可能活得越长。”

    但,克莱文杰终究是忘了这句话,结果,他死了。当初,由于约塞连没敢告诉克莱文杰,也是他约塞连一手造成了中队人人闹肚子,最后致使轰炸任务又一次不必要地给延期,因此,这扰得克莱文杰很是心烦意乱。米洛更是坐卧不安,因为他疑心很可能又有人在中队的食物里下了毒。于是,他便火烧火燎地跑去求助约塞连。

    “请赶快找斯纳克下士查问一下,他是不是又在白薯里放了洗衣皂。”他偷偷摸摸地恳求约塞连。“斯纳克下士信任你,假如你向他保证不告诉别人,他会跟你说实后的。他一告诉你,你就来告诉我。”

    “这还用问,我当然在白薯里放了洗衣皂,”斯纳克下士很坦率地告诉约塞连,“是你让我放的,对不?洗衣皂可真管用。”

    “他对上帝起誓,他跟这件事毫无关系,”后来,约塞连回答米洛说。

    米洛将信将疑地撅起了嘴。“邓巴说根本就不存在上帝。”

    不再有丝毫的希望了。第二个星期刚过一半,中队所有的人看上去就跟亨格利-乔一副模样。亨格利-乔是不需要执行轰炸任务的。他总在睡梦里恐怖地乱叫乱吼,全中队上下能安睡的,惟独他一人,晚上,其余的人仿佛一个个缄口不语的幽灵,叼着烟,彻夜在各自的帐篷外于黑暗中游荡。到了白天,他们就聚在一块,显出一副萎靡不振的模样,徒然地注视着那条轰炸路线;或是一眼不眨地盯着正纹丝不动地坐在紧闭着的医务室帐篷门前的丹尼卡医生,他的头顶上方,是那块可怕的手写的招牌。他们开始自编沉闷无趣的笑话,又捏造灾难性的谣言,说什么粉身碎骨的厄运正在博洛尼亚等着他们呢。

    一天晚上,在军官俱乐部里,约塞连醉醺醺地侧身走近科恩中校,骗他说,德国人把最新发明的那种莱佩奇炮运到了前线。

    “什么莱佩奇炮?”科恩中校很好奇地问。

    “就是最新发明的三百四十四毫米的莱佩奇胶炮,”约塞连回答说,“它可以在半空中把整编队的飞机粘合在一起。”

    科恩中校被约塞连一手紧抓住了胳膊时,很是吓了一跳。他猛地挣脱开,当众羞辱约塞连。“放开我,你这白痴!”他暴怒地叫喊道。这时,内特利突然跑到约寒连的背后,一把将他拖开,科恩中校怒目而视,心里倒是很赞许内特利这么做,因为替他出了这口恶气。“这疯子到底是谁?”
 


    卡思卡特上校高兴得咯咯直笑。“这就是弗拉拉战役结束后,你硬是要我给他一枚勋章的那个家伙。你还让我提升他为上尉,记得吗?你是活该如此!”

    内特利的体重比约塞连的轻,因此,他花了好大的劲,才把约塞连肥硕的身体拖过房间,拉到一张空桌旁。“你是不是疯啦?”内特利早已吓得浑身直打战,不停地发出嘘嘘声。“那是科恩中校,你是不是疯了?”

    约塞连想再喝一杯,并作出保证,只要内特利给他要来一杯,他就悄悄离开俱乐部。于是,他让内特利又要来了两杯。最后,内特利好说歹说总算哄他到了门口,这时,布莱克上尉恰好噔噔地踩着重步从外面走了进来,使劲在木地板上跺着满是泥浆的鞋子,帽檐儿上的雨水,像是从高高的屋顶直往下泻。

    “好家伙,你们这些杂种这下可是没有退路了,”他兴致勃勃地宣布道,边说边离开了脚下那滩污水,他身上的雨水溅得四处都是。“我刚接到科恩中校的电话。你们可知道他们在博洛尼亚准备好了什么迎候你们?哈!哈!他们准备好了最新发明的那种莱佩奇胶炮。它可以在半空中把整编队的飞机粘合在一起。”

    “上帝啊,真有这回事!”约塞连尖声叫道,吓得瘫倒在了内特利的身上。

    “哪里有上帝,”邓巴很镇定他说,一面略有些摇晃地走了过来。

    “嗨,帮我来扶他一把,行吗?我得送他回自己的帐篷去。”

    “谁这么说的?”

    “是我。哎呀,瞧瞧这雨。”

    “我们必须去弄一辆车子来。”

    “去把布莱克上尉的汽车偷来,”约塞连说,“这可是我老做的事。”

    “我们是谁的车也偷不到的。因为以前你每次要车,总是偷偷开走停放最近的车子,现在可没人再把点火开关钥匙留在车上了。”

    “上车吧,”一级准尉怀特-哈尔福特醉醺醺地驾驶着一辆有篷吉普车,开了过来,招呼他们说。等他们全都挤进车子,他便冷不丁地快速开了出去,大伙儿一个个往后仰面倒下去。他们破口大骂,他听了,哈哈大笑。一出停车场,他便笔直往前,疾驶而去,汽车结结实实地撞到了道路另一侧的路堤上。车里的其他人一齐往前倾了过去,一个个叠了起来,无法动弹,对他又是一顿臭骂。“我忘了拐弯,”他解释说。

    “小心点,行吗?”内特利告诫他,“你最好把前灯打开。”

    一级准尉怀特-哈尔福特倒车离开路堤,拐过弯,沿着大路飞驰而去。车轮在沥青路面上飕飕地飞转,发出咝咝的声音。

    “别开这么快,”内特利恳求道。

    “你最好先带我去你们中队,这样,我可以帮你安顿他上床。然后,你再开车送我回我自己的中队。”

    “你到底是谁?”

    “邓巴。”

    “嗨,把前灯打开,”内特利叫道,“注意路面!”

    “前灯都开着。约塞连难道没在这车上吗?所以,我才让你们这几个杂种上车。”一级准尉怀特-哈尔福特一百八十度转身,两眼直盯住后座。

    “注意路面!”

    “约塞连?约塞连在这儿吗?”

    “我在这儿呢,一级准尉。我们回去吧。你怎么那么肯定?你从来就没回答过我提的问题。”

    “你们都瞧见了?我跟你们说过,他在这儿。”

    “什么问题。”

    “我们刚才谈的什么,就是什么问题。”

    “重要吗?”

    “我记不得那问题是否重要。我向上帝发誓,我本来知道是什么问题。”

    “上帝根本就不存在。”

    “这正是我们刚才谈的问题。”约塞连大叫了起来。“你怎么会那么肯定?”

    “喂,你肯定前灯都开了吗?”内特利喊道。

    “开了,开了。他想要我干吗?挡风玻璃上全是雨水,难怪从后座看前面黑咕隆咚的。”

    “这雨实在是美极了。”

    “我真希望这雨一直这样不停地下。雨啊,雨,请走——”

    “——开。改日——”

    “——再——”

    “——来。小约约想要——”

    “——玩耍。在——”

    “——草地上,在——”

    一级准尉怀特-哈尔福特错过了途中的第二个拐弯,一路驶去,直把吉普车开上了一条陡峭路堤的最高处。吉普车往下滑行时,侧翻了,轻轻地陷在了泥地里。车子里,一阵受惊后的寂静。

    “大家没事吧?”一级准尉怀特-哈尔福特压低了声音问道。没人受伤,他便如释重负,长叹了一口气。“你们知道,我就是这个毛病,”他呻吟道,“从来就不听别人的话。刚才有人再三要我把前灯打开,可我就是不愿听。”

    “是我再三要你把前灯打开的。”

    “我知道,我知道。可我就是不愿听,是不是?我真希望有一瓶酒。我是带了瓶酒的。瞧,瓶还没打碎。”

    “雨进来了。”内特利察觉到了。“我身上都湿啦。”

    一级准尉怀特-哈尔福特打开黑麦威士忌酒瓶,喝了一口,于是便把酒瓶递给了别人。大伙叠罗汉似的,横七竖八地躺在车里,全都喝了酒,只有内特利没喝,他一刻不歇地摸索着找车门把手,可就是摸不着。酒瓶噔的一声,落在了他的头上,威士忌直灌他的颈脖。他一个劲地扭动身体。

    “喂,我们得爬出去,”他叫喊道,“我们全都会淹死的。”

    “车里有人吗?”克莱文杰关切地问道,一边打了手电筒从上往下照。

    “是克莱文杰,”他们大叫道。克莱文杰伸过手去,想帮他们一把,可他们却想把他从车窗拖进去。

    “瞧瞧他们!”克莱文杰愤怒地对麦克沃特——正坐在指挥车的方向盘后,咧开了嘴笑——大声说,“就像是一群喝醉了酒的牲畜躺在里边。你也在,内特利?你应该感到害臊!快——趁他们都还没得肺炎死掉,帮我把他们拉出来。”

    “你知道,这主意听起来挺不错,”一级准尉怀特-哈尔福特想了想说,“我想我倒是乐意得肺炎死的。”

    “为什么?”

    “为什么不?”一级准尉怀特-哈尔福特回答道,然后,双臂抱着那瓶黑麦威士忌酒,极其满足地仰躺在泥地里。

    “唉,瞧他在干吗?”克莱文杰恼火地大声叫道,“你们都爬起来上车,我们一起回中队去,行不行?”

    “我们不能都回去。得留下个人在这里,帮一级准尉把车翻过来,因为这车是他签了字从汽车调度场借来的。”

    一级准尉怀特-哈尔福特极舒适地在指挥车里坐了下来,背往后一靠,咯咯地直笑,一副高兴得意劲儿。“那是布莱克上尉的车,”他喜眉笑眼地告诉他们说,“刚才我是用他那串备用钥匙从军官俱乐部把车偷开来的。他还以为这钥匙今天早上丢了呢。”

    “啊,真有你的!咱们该为此喝一杯。”

    “难道你们还没喝够?”麦克沃特刚发动汽车,克莱文杰便开始责骂了起来。“瞧你们这些人。你们是不是不在乎把自己喝死淹死?”

    “只要不在飞行时死就行。”

    “喂,把瓶打开,把瓶打开。”一级准尉怀特-哈尔福特催促麦克沃特。“把前灯关掉。只有这样,才能在车上喝酒。”

    “丹尼卡医生说得一点没错,”克莱文杰接着又说,“有些人的确不知道该如何照顾自己。我实在是很厌恶你们这些人。”
 


    “行了,饶舌鬼,快下车,”一级准尉怀特-哈尔福特命令道,“除约塞连外,其他人全都下车。约塞连在哪儿?”

    “见鬼,别碰我!”约塞连哈哈大笑了起来,一边猛地把他推开。

    “你满身都是泥。”

    克莱文杰把目光集中到内特利身上。“真让我吃惊的是你。你知道自己身上是什么味儿,你不想办法劝阻他惹麻烦,反倒跟他一样喝得烂醉。要是他跟阿普尔比再打一架,你怎么办?”克莱文杰听见约塞连在暗笑,吃惊地瞪大了双眼。“他没有跟阿普尔比再打架,是不是?”

    “这一次没有,”邓巴说。

    “没有,这一次没有。这次我干得更漂亮。”

    “这次他跟科恩中校打了一架。”

    “他没有!”克莱文杰倒抽了一口气。

    “他真干了?”一级准尉怀特-哈尔福特兴奋地大叫了起来。“那该为此喝上一杯。”

    “这事可就糟啦!”克莱文杰很是不安他说,“你们究竟干吗非得去惹科恩中校呢?哎呀,灯怎么啦?怎么那么黑?”

    “我把灯都关了,”麦克沃特回答说,“你知道,一级准尉怀特-哈尔福特说的没错。前灯关了要好得多。”

    “你疯啦?”克莱文杰尖声叫了起来,突然俯身前去,吧咯一声打开了前灯。他几乎歇斯底里般地猛转过身,面对着约塞连。“你瞧你干的好事?你让他们一举一动全跟你一样了!要是雨停了,明天我们就得飞博洛尼亚,那可怎么办?你们得有健康的身体。”

    “雨是再也不会停了。不会,长官,像这样的雨或许真会永远下个不停。”

    “雨已经停了。”有人说,整个车子一片死寂。

    “你们这些可怜的杂种。”几分钟过后,一级准尉怀特-哈尔福特很是同情地低声说了一句。

    “雨真的停了吗?”约塞连怯声怯气地问道。

    麦克沃特关掉挡风玻璃刮水器,想看个清楚。雨早停了。天渐渐晴了。月亮让一片褐色的薄雾给罩住了,轮廊却是清晰可见。

    “唉,行了,”麦克沃特镇静地大声说,“这有啥了不得的。”

    “别担心,弟兄们,”一级准尉怀特-哈尔福特说,“机场跑道这会儿太松软,明天还用不起来。或许还没等机场干透,天就又下起雨来了。”

    “你这讨厌透顶令人恶心的杂种。”当他们快速驶进中队营地时,亨格利-乔在自己帐篷里惊叫了起来。

    “天哪,今天晚上他回来了?我以为他跟那架军邮班机还在罗马呢。”

    “哎哟!哎哎哎哎哟!哎哎哎哎哎哎哎哟!”

    一级准尉怀特-哈尔福特浑身打颤。“这家伙让我心里直发毛,”他低声抱怨道,“嘿,弗卢姆上尉出什么事啦?”

    “这个家伙吓得我心惊胆战。上星期我在树林里看见他在吃野浆果。他再也不在活动房里睡了。他那模样就像是个鬼。”

    “亨格利-乔是害怕代别人参加病号检阅,尽管已经取消了病号检阅。前天晚上,他想宰了哈弗迈耶,没料到自己却一头栽进了约塞连的狭长掩体,你看到了吗?”

    “哎哎哎哎哟!”亨格利-乔惊呼道,“哎哟!哎哎哎哎哟!哎哎哎哎哎哎哎哟!”

    “食堂里不再有弗卢姆在,这实在是桩让人高兴的事。再听不到‘把盐递过来,沃特’这样的话了。”

    “还有‘快把甜菜递给我,彼特’。”

    “还有‘把面包递给我,弗雷德’。”

    “滚开,滚开,”亨格利-乔惊叫道,“我说了,滚开,滚开,你这讨厌透顶令人恶心的杂种。”

    “至少我们知道了他都做些什么梦,”邓巴做了个鬼脸,说道,“他老是梦见那些讨厌透顶令人恶心的杂种。”

    那天深夜,亨格利-乔梦见赫普尔的那只猫睡在自己脸上,差点没把他给闷死。等他醒来,赫普尔的那只猫果真在他脸上睡大觉。当时他的痛苦挣扎也实在令人毛骨悚然。他发出一声尖厉怪异的长嚎,刺破月色皎洁的黑夜,接着,像一阵毁灭性的剧震,回荡了片刻。之后便是让人心惊肉跳的沉寂,紧接着,又是一阵大闹大嚷从亨格利-乔的帐篷里传了出来。

    约塞连是最先到亨格利-乔帐篷的那几个人当中的一个。当他冲进帐篷时,亨格利-乔早就掏出了枪,正使劲挣脱让赫普尔抓住的那只胳膊,朝那猫开枪。那只猫却是不停地发出呼噜呼噜的叫声,极是凶猛地发动佯攻,企图转移亨格利-乔的注意力,不让他开枪打赫普尔。两个人全都穿着军用内衣。头顶上方那只非磨砂灯泡,在那根松了的电线上,正发了疯似地摇来晃去。乱作一团的黑影不停地毫无规律地打转,上下移动,整个帐篷也因此像是在回旋。约塞连本能地伸出双臂,保持身体平衡,然后,猛一个漂亮的鱼跃,往前直扑过去,把三个格斗者撞倒在地,压在了自己的身体下面。他从混战中脱开身来,一手揪住一个家伙的后颈——亨格利-乔的后颈和那猫的颈背。亨格利-乔和那猫恶狠狠地相互瞪了一眼。那猫凶狠地冲着亨格利-乔呼噜呼噜直叫,亨格利-乔抡起拳头,想狠狠地把它揍扁。

    “决斗要公平嘛。”约塞连作出了裁定。这会儿,惊恐万状地跑来看这场混战的那些人全都没有了恐怖感,发出了一阵欣喜若狂的喝彩声。“我们要公平决斗。”约塞连把亨格利-乔和猫带到外面,依旧一手揪住一个后颈,把他们分开。然后,他便正式向他们阐明:

    “拳头,牙齿和爪子都可以用。但不能用枪。”他警告亨格利-乔。“不准呼噜呼噜地叫。”他严厉地警告那只猫。“等我一放开你们,就开始。一旦双方扭在一起,马上分开,接着再打。开始!”

    四周围了一大群专爱看热闹的无聊人,可是,一等约塞连松手,那猫竟害怕了起来,像个懦夫似的,可耻地从亨格利-乔身边逃跑了。亨格利-乔被宣布为胜利者。他高昂起萎缩的头,直挺起皮包骨的胸膛,脸上挂着胜利者自豪的笑容,扬扬得意地大步走了开去。他凯旋而归,重新上床睡觉,可又梦见赫普尔的那只猫睡在他的脸上,把他闷得气都喘不过来
