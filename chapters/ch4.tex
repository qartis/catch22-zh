\chapter{丹尼卡医}
 
    亨格利-乔确实疯了,这一点约塞连比谁都清楚。约塞连尽了一切力帮助他。但亨格利-乔无论如何不听他的。他不愿听信约塞连,是因为在他看来,约塞连也是个疯子。

    “他干吗非听从你不可?”丹尼卡医生连头也不抬地问约塞连。

    “因为他有病。”

    丹尼卡医生轻蔑地哼了一声。“他自己觉得有病吗?那我呢?”

    丹尼卡医生脸沉沉地发出一声讥笑,于是,慢悠悠地接着道,“唉,我倒不是发什么牢骚。我知道,眼下正是战争时期。我也知道,许多人为了打赢这场战争,不得不替我们承受苦难。可是,为什么我也非得跟他们一样受苦呢?他们干吗不征募一些老医生呢?这些人不是时常在公共场合口口声声吹嘘什么医务界随时准备作出重大牺牲吗?我不想作什么牺牲。我想发财。”

    丹尼卡医生是极讲究洁净的人。于他,愠怒便是桩乐事。他皮肤黝黑,脸型极小,却流露出聪慧和阴郁,双目下垂着哀戚的眼袋。

    他始终担忧自己的健康,几乎每天上医务室量体温。轮番替他量体温的,是在那里工作的两个士兵,他俩承担了医务室的一切事务,且把医务室上上下下安置得妥妥当当。于是,丹尼卡医生终日无所事事,整日抽着不通气的鼻子坐在日光下暗自纳闷,其他人为何如此愁眉锁眼。两个士兵,一名叫格斯,另一名叫韦斯,他俩已成功地将医务工作完善为一门精密的科学。门诊伤病员集合时,凡发现体温超过华氏一百零二度者,一概急送医院。除约塞连外,凡在门诊伤病员集合时查出体温低于华氏一百零二度的病号,全部用龙胆紫溶液搽牙龈和脚趾,再就是每人给一颗轻泻片。结果,这药病员们一接到手,便扔进了灌木丛。至于体温不高不低正好是华氏一百零二度的那些人,则一律要求于一小时后回医务室,重新测量体温。约塞连呢,虽然体温只有华氏一百零一度,但是他随时可进医院,只要他自己愿意,原因是,他压根就没把格斯和韦斯这两个人放在眼里。

 


    这一整套制度的推行,于每一位官兵都大有益处,尤其在丹尼卡医生身上,这一点体现得更是充分。他有了足够的时间,尽兴地观看年老的德-科弗利少校在自己的私人蹄铁投掷场掷蹄铁。科弗利少校依旧戴着丹尼卡医生替他制作的透明的赛璐珞眼罩,那一狭条赛璐珞片,是数月前从梅杰少校的中队办公室的窗子上窃来的。当初,德-科弗利少校刚从罗马回来,眼角膜受了伤。在罗马,他租了两套公寓房间,专供军官和士兵休假时享用。丹尼卡医生只有在每天觉着自己患了重病时,才会顺道去一趟医务室,即便去了,也只是让格斯和韦斯替他细细检查一番。然而,他俩无论如何查不出丹尼卡医生有什么不正常。他的体温,始终是华氏九十六点八度,这样的体温于他们实在是极正常的,自然,只要丹尼卡医生自己觉得无关紧要。但,丹尼卡医生确实很在意。他开始对格斯和韦斯失却了信任感,正考虑让人把他俩遣回汽车调度场,再找个人来作替换。当然,这人得有能耐在丹尼卡医生身上查出些毛病来。

    丹尼卡医生自己通晓诸多极不正常的物事。除自己的健康状况外,他还担忧或许某日会被遣往太平洋,以及飞行时间。至于健康,无论是谁,在相当长的时间内,都是把握不了的。而太平洋呢,却是一片汪洋,四周让象皮病及其他种种可怕的疾病严实地围住。

    假如他什么时候让约塞连停飞,由此而得罪了卡思卡特上校,那么,他很有可能突然人不知鬼不觉地给调到太平洋。他所谓的飞行时间,便是为领取飞行津贴,每月坐飞机飞行所必需的时间。丹尼卡医生极讨厌飞行。坐在飞机上,他总有蹲牢房的感觉。人在飞机上,只能从飞机这一端走到另一端,此外,实在是没有别的什么活动余地了。丹尼卡医生曾听人说过,凡是喜钻飞机者,实实在在是想满足一种潜意识的欲望:再次钻进子宫。是约塞连跟他这么说的。幸亏约塞连出面相帮,丹尼卡医生方才免了再次钻进子宫的麻烦,依旧分文不少地领取他的每月飞行津贴。每次执行训练飞行任务,或是飞罗马,约塞连总会说服麦克沃特,让他把丹尼卡医生的名字记入飞行日志。

 


    “你知道这其中的情由,”丹尼卡医生曾花言巧语,哄骗约塞连,同时诡秘地使了个眼色,仿佛与他在一起密谋什么。“非万不得已,我又何必去冒险呢?”

    “那当然,”约塞连表示同意。

    “我在飞机上也好,不在也好,这跟别人有什么相干?”

    “毫不相干。”

    “的确是这样,压根就碍不了别人什么事,”丹尼卡医生说,“这世界要畅运,靠的是润滑。左手帮右手,右手帮左手。你懂我的意思?你替我搔背,我替你搔背。”

    约塞连懂他的意思。

    “我不是这意思,”见约塞连开始替他搔背,丹尼卡医生说道,“我说的是合作、互助;你帮我,我帮你。懂吗?”

    “那就帮我一个忙吧,”约塞连请求道。

    “这绝对不可能,”丹尼卡医生回答说。

    丹尼卡医生时常坐在自己的帐篷外面晒太阳,身穿夏令卡其裤及短袖衬衫——由于每天洗烫,似消了毒一般,差不多褪成了灰色,神情却很沮丧,颇显得怯懦,微不足道。仿佛他一度大受惊吓,魂魄飞散,从此便再也不曾彻底摆脱掉那次惶恐。他蟋缩着身子,坐在那里,半个头埋在单薄的双肩之间,两手给太阳晒得黑黑的,手指却镀成银色,闪光发亮,双臂裸露着交叉胸前,手不时轻柔地抚摩臂背,好像他感觉冷似的。其实,他这人倒是极热心的,颇有些同情心。他始终觉得自己挺倒霉,心中由此而愤愤不平。

    “干吗老是我倒霉?”他常这么悲叹,不过,这话问得实在是好,无法予以即刻的答复。

    约塞连知道丹尼卡医生这话问得好,因为他长于收集这类难以回答的问题,且用这些问题扰乱了克莱文杰和那位戴眼镜的下士一度合办的短训班——地点是布莱克上尉的情报营,每周两个晚上。戴眼镜的下士极可能是一个颠覆分子,这一点大家都很清楚。布莱克上尉确信这家伙就是颠覆分子,因为他架了副眼镜,且又常用“万灵药”和“乌托邦”一类的词。再者,他憎恶阿道夫-希特勒,殊不知,在与德国的非美活动进行的斗争中,希待勒可是立下了汗马功劳。约塞连也参加了短训班,原因是,他极想知道为何竟有那么多人千方百计要害他。此外,还有少数官兵也颇有兴致。克莱文杰和那个被认作是颠覆分子的下士,每次授课毕,总要问大家是否有问题,这一问实在是不该的,其结果,便是引出了一连串极有趣味的问题。

    “谁是西班牙?”

    “为什么是希特勒?”

    “什么时候是正确的?”

    “旋转木马坏掉时,我常叫他爸爸的那个脸色苍白的驼背老头儿在哪里呢?”

    “慕尼黑的王牌怎么样?”

    “嗬——嗬!脚气病。”

    以及:

    “睾丸!”

    大家连珠炮似地发问。于是,便有了约塞连那个没有答案的问题:

    “去年的斯诺登夫妇如今在何方?”

    这问题难住了克莱文杰和下士,因为斯诺登早已丧命于阿维尼翁上空。当时在空中,多布斯发了疯,强夺过赫普尔手中的操纵器,最终导致了斯诺登的一命呜呼。

    下士故意装聋作哑。“你说什么?”他问道。

    “去年的斯诺登夫妇如今在何方?”

    “很遗憾,我没听懂你说的话。”

    约塞连把话说简洁些,想让下士听个明白。

    “看在老天爷面上,”下士说。

    “我也不说法语,”约塞连答道。假如可能,他打算追根究底,千方百计从下士嘴里把问题的答案给“挤”出来,即便竭尽全世界的一切语汇,也不足惜。然而,克莱文杰出面干涉。瘦溜的克莱文杰这会儿脸色苍白,粗重地喘息着,营养不良的双眼里早已噙了一层湿润的晶莹的泪水。

    大队司令部对此却是不胜惊恐,一旦学员们随心所欲地提问题,说不准会有什么秘密让他们给捣出来。卡思卡特上校遂遣科恩中校前去制止这种放肆。最终,科恩中校制订了一条提问规则。在给卡思卡特上校的报告中,科恩中校解释道,他订出的这一规则,实在是天才之举。依照科恩的这一规则,只有从未问过问题的人,方可提问。不久,参加短训班的,便只有那些从未提问过的官兵。终于,短训班彻底解散,原因是,克莱文杰、下士和科恩中校三人取得一致看法,培训那些从不质疑的人,既不可取,亦绝无必要。
 


    和司令部的所有工作人员一样,卡思卡特上校和科恩中校都在大队司令部的办公大楼里生活和工作。唯独随军牧师是个例外。

    司令部办公大楼是一座庞大建筑,由一种易碎的红色石块砌成,且装有极大的管道设备,年久失修,长日当风。大楼后面是一现代化的双向飞碟射击场,由卡思卡特上校下令建筑,专供大队军官娱乐。依德里德尔的命令,现在,凡参战的官兵,每个月至少得在这射击场花上八个小时。

    约塞连射双向飞碟,但从未击中过;阿普尔比却是百发百中的射击能手。约塞连拙于双向飞碟射击,赌博术亦极低劣。赌场上,他向来赢不了钱,即便作弊,也赢不了,因为他的对手的作弊术总是胜他一筹。这便是他平素自认的两桩遗恨:永远成不了双向飞碟射手,永远捞不到钱。

    “想要不捞钱,是要绞尽脑汁的。这年月,傻爪也能捞钱,大多数傻瓜有这能耐。可是,具有才智的人又如何呢?举个例子,说说有哪个诗人会捞钱的。”卡吉尔上校在一份说教备忘录——由卡吉尔上校定期撰写、佩克姆将军签发、大队官兵传阅——里写下了以上这段话。

    “T.S.艾略特,”前一等兵温特格林答道。当时,他正在第二十七空军司令部的邮件分类室里,说罢,连自己的姓名也没留与对方,便砰地挂上电话。

    卡吉尔上校,人在罗马,听了电话,大惑不解。

    “是谁?”佩克姆将军问。

    “不知道,”卡吉尔上校答道。

    “他想干什么?”

    “不知道。”

    “那他说了些啥?”

    “T.S.艾略特,”卡吉尔上校告诉他。

    “什么?”

    “T.S.艾略特,”卡吉尔上校又说了一遍。

    “只说了‘T.S——’”“是的,将军。他啥也没说,只说了‘T.S.艾略特’。”

    “真不明白他说这是啥意思,”佩克姆将军思忖道。

    卡吉尔上校也很纳闷。

    “T.S.艾略特。”佩克姆将军若有所思。

    “T.S.艾略特。”卡吉尔上校复述了一遍,语调是同样的阴郁、困惑。

    待过片刻,佩克姆将军重新振作起来,露出令人宽慰的慈祥的笑容,表情精明狡黠,两眼透出恶狠狠的光芒。“让人替我接通德里德尔将军,”他对卡吉尔上校说,“别让他知道是谁打的电话。”

    卡吉尔上校把话筒递给他。

    “T.S.艾略特。”佩克姆将军说罢,便挂断了电话。

    “谁?”穆达士上校问道。

    在科西嘉的德里德尔将军没有答复。穆达士是德里德尔将军的女婿。将军经不住妻子的软磨,终于违心地把女婿弄进了军队。

    德里德尔将军狠狠地逼视穆达士上校。一见到女婿,他便心起厌恶,但女婿是他的副官,所以时常得随从他。当初,他就不赞成女儿嫁给穆达士上校,原因是,他讨厌参加婚礼。德里德尔将军紧锁眉头,心事重重,一脸凶气。他移步走到办公室的大穿衣镜前,注视着自己矮墩墩的镜中影像。他,头发花白,脑门宽阔,几缕铁灰色头发垂下遮住双眼,下巴方正,好斗。将军苦苦思索着适才接到的那个神秘电话。他计上心头,愁容亦随之缓缓地舒展了开来,于是,现出恶作剧般的兴奋,撅起了嘴唇。

    “接佩克姆,”他对穆达士说,“别让那狗杂种知道是谁打的电话。”

    “是谁?”在罗马那边的卡吉尔上校问。

    “还是那个人,”佩克姆将军答道,满脸的惊讶。“这下他缠住我了。”

    “他想干什么?”

    “我不知道。”

    “他说啥?”

    “还是那句话。”

    “‘T.S.艾略特’?”

    “没错,‘T.S.艾略特’。此外什么也没说。”佩克姆将军有了一个挺妙的主意。“说不定是个新密码,或是别的什么,比方说,当日的旗号。为何不叫人跟通讯司令部核实一下,查查清楚究竟是不是新密码或类似的什么,还是当日的旗号?”

    通讯司令部回复道,T.S.艾略特既非新密码,亦非当日旗号。

    卡吉尔上校亦有了个主意。“也许我该给第二十七空军司令部打个电话,问问他们是否知道这事。他们那儿有一个叫温特格林的办事员,跟我挺熟的。他私下告诉我说,我们送上去的报告,写得太罗嗦。”

    前一等兵温特格林告诉卡吉尔上校说,第二十七空军司令部的档案不见有一个名叫T.S.艾略特的人的记录。

    “我们的报告最近怎么样?”趁前一等兵温特格林还没放下话筒,卡吉尔上校便决定探问一下。“比先前写得好多了,是不是?”

    “还是太罗嗦,”前一等兵温特格林答道。

    “假如是德里德尔将军幕后策划了这一切,那我就丝毫不感到奇怪了,”佩克姆将军最终坦言道,“你记不记得上回他是怎么处置双向飞碟射击场一事的?”

    当初,卡思卡特私建了一片双向飞碟射击场。结果,德里德尔将军开放了射击场,供大队的所有参战官兵享用。他要求自己的部下,只要射击场设备和飞行时刻表许可,尽可能在那儿多泡上些时辰。每月作八小时的双向飞碟射击,于他们实在是极好的训练。训练他们射击飞靶。
 


    邓巴极喜射击双向飞碟,是因为他极其讨厌这一运动,所以,时间过起来就显得很慢。他曾计算过,只要在双向飞碟射击场同哈弗迈耶和阿普尔比这样的人呆上一个小时,就好像是熬过了一百八十六年。

    “我想你准是疯了。”对邓巴的发现,克莱文杰曾作如是说。

    “谁在乎这个?”邓巴答道。

    “我想你是疯了,”克莱文杰坚持自己的看法。

    “管它呢!”邓巴回答说。

    “我真是这么想的。我甚至想承认,生命似乎漫长了些,假——”

    “——是漫长了些,假——”

    “——是漫长了些——是漫长了些吗?没错,确实是漫长了些,假如生活枯燥乏味,满是痛苦烦恼,因——”

    “你猜猜看有多快?”邓巴冷不防问了一句。

    “你说啥?”

    “它们过得很快,”邓巴解释道。

    “谁?”

    “年月呗。”

    “年月?”

    “年月,”邓巴说,“年月,年月,年月。”

    “克莱文杰,你干吗老是纠缠邓巴?”约塞连插话道,“难道你不清楚像你这样喋喋不休是要折寿的?”

    “没关系,”邓巴宽宏他说,“我还有好几十年可活呢。你可知道,一年的时间流逝有多长?”

    “你也给我闭嘴吧,”约塞连对奥尔说。奥尔正在一旁窃笑。

    “我刚才想起了那个姑娘,”奥尔说,“西西里的那个姑娘。那个秃头的西西里姑娘。”

    “你最好也闭上嘴巴,”约塞连警告他说。

    “这可是你的不是了,”邓巴对约塞连说,“他想笑,你又何必阻止他呢?与其让他开口说话,还不如听他笑。”

    “好吧。想笑,你就继续笑吧。”

    “你可知道,一年的时间流逝有多长?”邓巴又问了克莱文杰一遍。“这么长。”他打了个榧子。“一秒钟以前,你还是个年轻人,朝气蓬勃地跨进了高等学府的大门。如今,你却已是老态龙钟了。”

    “老态龙钟?”克莱文杰吃惊地问,“你说什么?”

    “老态龙钟。”

    “我还没老呢。”

    “你每次执行飞行任务,死神与你便是近在咫尺。到了你这般年纪,你还能长多少岁?半分钟以前,你还在上中学,一只解了扣子的奶罩便是你心中的伊甸园。仅五分之一秒钟以前,你还是个小孩,过一个十星期的暑假,尽管似十万年一般长,却仍旧去得匆匆。

    嗖!飞逝而过。你究竟有什么其他高招让时间减速?”说罢,邓巴差些动起了肝火。

    “嗯,或许是这个理儿,”克莱文杰低声附和道,心里却是极不服气的。“也许人的一生越漫长,就必定会时时遇上许多的不愉快。

    但既然如此,谁又希望长命百岁呢?”

    “我希望,”邓巴跟他说。

    “为什么?”克莱文杰问。

    “除此,还能有别的什么呢?”
