\chapter{约-约的同帐篷伙伴}
 
    天气变冷了,约塞连却感到很暖和。几乎连绵不绝的鲸鱼状云彩低低飘浮在阴沉灰暗的天空中。约塞连觉得它们看上去很像两个月前进攻法国南部那一天天上黑压压的Bl7型和B24型轰炸机群。这些飞机从意大利各远程空军基地起飞,轰轰隆隆、密密麻麻地飞过天空。中队里人人都知道基德-桑普森的两条细腿被潮水卷到潮湿的沙滩上,而且已经腐烂了,看上去就像一截弯曲的紫色的鸟的胸叉骨。不论是格斯、韦斯还是太平间的收尸员,谁都不愿意去收拾它们。大家全都装作不知道基德-桑普森的腿还在那里,好像它们早已像克莱文杰和奥尔的尸体那样,随着潮水永远地向南漂去了。现在,天气又不好,几乎没有人会再独自溜出来,像个有怪癖的人一样钻到灌木丛中窥探那堆腐烂的残肢了。

    再也没有晴朗的天气了,再也没有轻松的飞行任务了。只有令人恼火的淫雨和阴沉冰冷的浓雾。天只要一放晴,飞行员们就得连着飞上一个星期。到了夜里,寒风呼啸,扭曲多节的矮树丛吱吱嘎嘎地呻吟着,就像滴答作响的时钟一样每天凌晨准时把约塞连从似睡非睡的状态中唤醒,使他想起基德-桑普森的两条泡胀了的腐烂的细腿,想起在十月这种寒风呼啸、冷气袭人的黑夜里,那两条腿正躺在湿漉漉的沙滩上,任凭冷雨浇洒。从基德-桑普森的腿,约塞连又会联想起可怜的、呜咽不止的斯诺登在飞机尾舱里冻得要死的情景。约塞连始终没有发现遮盖在斯诺登鸭绒防弹衣里面的那个伤口,错误地以为他只是腿上负了伤。等到他把这个伤口消毒包扎好,斯诺登的内脏突然喷涌而出,弄得满地都是。晚上,当约塞连努力入睡时,他会把他所认识的、但现在已经死掉的男女老少的名字统统在脑子里过一遍。他回忆起所有的战友,在脑海里唤起他从童年时代起就认识的长辈们的形象——他自己的和所有别人的大伯、大娘、邻居、父母和祖父母,以及那些可怜的、总是受骗上当的店小二——天一亮就起身打开铺门,在那狭窄肮脏的铺子里傻乎乎地一直干到深夜。这些人现在也都死了,死人的数字看来正在不断地增加,德国人仍然在抵抗。他暗自猜想,死是不可逆转的趋势,他开始认为自己也快要死了。

    由于奥尔精心制作的那个火炉,天气转冷时,约塞连却仍然感到很暖和。要不是因为怀念奥尔,要不是因为有一天一帮精力旺盛的伙伴强行闯入他的帐篷的话,他本来会在他这顶温暖的帐篷里过得非常舒适的。这些人是卡思卡特上校为了填补基德-桑普森和麦克沃特留下的空缺,在四十八小时内从两个满员的战斗机组调过来的。约塞连执行完飞行任务,拖着沉重的脚步走回帐篷时,发现他们已经搬进来了,他只好发出一声嘶哑的长叹,以表示抗议。

    这帮人一共四个,他们有说有笑地互相帮着搭起行军床,吵吵闹闹的,快活极了,约塞连一看见他们,就知道自己受不了他们那一套。这帮人活泼好动,热情洋溢,精力充沛,在国内时就已经结为朋友。他们简直令人不可思议,他们都是些刚满二十一岁的小伙子,喜欢咋咋唬唬,过分自信,头脑简单。他们都上过大学,跟漂亮、单纯的姑娘订了婚,未婚妻的照片已经摆在奥尔装修过的粗糙的水泥壁炉架上了。他们开过快艇,打过网球,骑过马。他们中的一个还跟一个比他年龄大的女人睡过觉。他们在国内不同的地方有着共同的朋友,他们曾经和彼此的表兄弟一块上过学。他们都喜欢听世界棒球锦标赛的实况转播,都很关心哪一支橄揽球队赢了球。

 


    他们的感觉虽然迟钝,斗志却很旺盛。他们对战争的延续感到十分高兴,因为这样他们就可以亲眼看看打仗究竟是怎么一回事。他们的行李刚打开一半,约塞连就把他们全轰了出去。

    约塞连态度强硬地向陶塞军士表示,让他们住进来是根本不可能的。陶塞军士那张灰黄瘦长的马脸露出一副沮丧相,他告诉约塞连必须让这些新来的军官住进来。只要约塞连一个人独自住着一顶帐篷,他就不能向大队另外申请一顶六人住的帐篷。

    “我不是一个人独自住在这里的,”约塞连气呼呼地说,“我这儿有个死人跟我一块住呢。他叫马德。”

    “行行好吧,长官,”陶塞军士恳求道,他疲倦地叹了口气,斜眼瞟了瞟那四个就站在帐篷门外的新来的军官。他们正困惑不解地默默听着他们俩的谈话。“马德在奥尔维那托执行飞行任务时战死了,这你是知道的。他是紧挨着你飞行的。”

    “那你为什么不把他的东西搬走?”

    “因为他从来没到这帐篷来过。上尉,请你不要再提这件事了。

    要是你愿意,你可以搬过去跟内特利上尉一块住,我还可以从中队传达室叫几个士兵过来帮你搬东西。”

    但是,抛弃奥尔的帐篷就等于抛弃奥尔,那样一来,奥尔会遭到这四个急等着往里搬的笨蛋军官的排挤和侮辱。这些咋咋唬唬、嘴上没毛的年轻人偏偏等到一切都安排就绪才露面,而且居然获准进驻这岛上最舒适的帐篷,这实在太没道理了。但陶塞军士却解释说,这是军规,因此约塞连只能是在给他们腾地方时用狠毒而又抱歉的目光瞪着他们。待到他们搬进他独居的帐篷并成为主人时,他又主动凑上前指指点点地帮忙,以表示他的歉意。

    在约塞连接触过的人当中,这几个家伙是最叫人泄气的一伙了。他们总是兴高采烈的,见了什么东西都觉得可笑。他们开玩笑地把他叫做“约-约”。他们总是要到半夜三更才回来。他们踮起脚尖,竭力不弄出声响,可还是笨手笨脚地不是踢到这个就是撞上那个,或者干脆格格地笑起来,最后总要把他吵醒。当他坐起身来骂骂咧咧地抱怨时,他们发出驴叫般的欢笑声,像老朋友似的跟他打哈哈。他们每回这么胡闹时他就想全杀了他们。他们使他想起唐老鸭的侄儿们。他们都很怕约塞连,天天没完没了唠唠叨叨地竭力讨他欢心,并且争着为他做这做那。这更使他恼火,觉得自己真是活受罪。他们鲁莽幼稚,臭味相投;他们既天真又放肆,既恭顺又任性;他们愚笨无知,从不叫苦抱屈。他们钦佩卡思卡特上校,他们认为科恩中校聪明机智。他们害怕约塞连,可是一点也不害怕卡思卡特上校规定的七十次战斗飞行任务。他们是四个潇洒英俊、诙谐幽默的小伙子,他们快要把约塞连逼疯了。他无法使他们理解,他是一个二十八岁的古怪的守旧分子,属于另一代人,另一个时代,另一个世界。他更无法使他们理解,他不喜欢把时间花在玩乐享受上,他觉得这不值得,至于他们四个更是叫他心烦,他没有办法叫他们闭上嘴不讲话。他们比女人还糟糕,他们没有头脑,不知道内省和自我抑制。

 


    他们在其它中队的朋友开始恬不知耻地过来串门聊天。他们把他的帐篷当做聚会地点,弄得他常常没有地方呆。最糟糕的是,他再也不能把达克特护士带到帐篷里睡觉了,眼下天气这么坏,他实在也没有别处可去了!这真是一场他始料不及的灾难。伦恨不得用拳头砸碎他帐篷里这些家伙的脑袋,或者挨个抓住他们的裤子后腰和后脖领,把他们揪起来扔出去,扔到那些潮湿绵软的多年生野草丛中去,永远不许他们回来。那野草丛的一侧搁着他那个锈迹斑斑、底部有几个小沉的尿壶,这尿壶原本是个汤盆;另一侧是中队用多节松木板搭成的厕所,那厕所看上去跟近处海滩上的更衣室相差无几。

    然而,他并没有砸碎这些家伙的脑袋,而是穿上高统胶靴和黑雨衣,冒着蒙蒙细雨,黑灯瞎火地跑去邀请一级准尉怀特-哈尔福特搬来跟他一起住,打算借助他的恐吓诅咒和下流习惯把这帮衣食讲究、生活严谨的狗杂种赶出去。但是,一级准尉怀特-哈尔福特冻得生了病,正打算搬去住院,万一转成肺炎,还是死在医院里好。直觉告诉一级准尉怀特-哈尔福特,他的死期就要到了。他胸部疼痛,咳嗽个不停。威士忌已经不能使他暖和起来了。最要命的是,弗卢姆上尉已经搬回到他的活动房子里去了。这是一个含义明确无误的预兆。

    “他会搬回来的,”约塞连争辩道。他竭力想使这个忧郁的宽胸脯印第安人振作起来,可是做不到。他那张结实的红褐色脸蒙上了一层死灰色,显得衰老憔悴。“在这种天气里,他要是还住在树林里,准会冻死的。”

    “不,那也不会把这个胆小鬼赶回来的,”一级准尉怀特-哈尔福特固执地反驳道。他摆出一副神秘莫测的样子,敲了敲前额。

 


    “不,先生,他心里很清楚。他知道现在是我染上肺炎死去的时候了,这就是他知道的事情,这也就是我怎么会知道我的死期到了的。”

    “丹尼卡医生怎么说?”

    “他们什么话都不让我说,”丹尼卡医生坐在他那张放在阴暗角落里的凳子上,伤心他说。在摇曳不定的烛光里,他那张光滑、细长的小脸呈现出一种龟绿色。帐篷里到处散发着霉味。电灯泡几天前就烧坏了,可两个人谁也不愿意动手换一个。“他们再也不让我开药方了。”丹尼卡医生又加上一句。

    “他已经死了,”一级准尉怀特-哈尔福特幸灾乐祸地说。他从被痰堵住的嗓子里发出一声嘶哑的大笑。“这真是可笑极了。”

    “我甚至连军饷也领不到了。”

    “这真是可笑极了。”一级准尉怀特-哈尔福特又说了一遍。

    “这些日子里,他一直在糟踏我的肝,看看他自己出的事吧,他已经死了,他是因为太贪心才死去的。”

    “我不是因为这个才死的,”丹尼卡医生语调平淡地说。贪心并没有什么错。这全是斯塔布斯医生那个讨厌鬼惹的事。他激起了卡思卡特上校和科恩中校对全体航空军医的怒火。他倒是坚持住原则了,可医务界的名声全让他给败坏了。他要是再不小心点,他那个州的医学协会就会开除他的会籍,他就再也别想在医院里干下去了。

    约塞连看着一级准尉怀特-哈尔福特小心地把威士忌倒入三个空的洗发香波的瓶子里,又把瓶子放到他正在收拾的军用背包里。

    “你去医院的路上能不能顺路到我的帐篷走一趟,替我往他们中不管哪一个的鼻梁上揍上一拳?”他沉思着大声说,“我那儿一共住进去四个家伙,他们要把我从我的帐篷里挤出去了。”

    “你知道,我那个部落从前发生过一件类似的事情,”一级准尉怀特-哈尔福特快活地开玩笑说。他一屁股坐到他的行军床上,抿着嘴笑起来。“你为什么不去叫布莱克上尉把他们踢出去呢?布莱克上尉就喜欢干这种事。”

    听到布莱克上尉的名字,约塞连愁眉不展地做了个鬼脸。每回新来的飞行员到布莱克上尉的情报室帐篷去取地图或资料时,他都要欺侮他们一番。一想到布莱克上尉,约塞连对他的这些同帐篷伙伴的态度变得宽容起来,竟转而护着他们了。当他在黑暗中晃动着手电筒的光束往回走时,他提醒自己说,他们年轻、生气勃勃,这不是他们的过错。他真希望自己也年轻、生气勃勃。他们勇敢、自信、无忧无虑,这也不是他们的过错。他应当对他们有耐心,等到他们中有一两个阵亡,其余人受伤时,他们就会成熟起来。他发誓要更加忍让,更加仁慈。但是,当他态度比以往更加友好地钻进自己的帐篷时,却被壁炉里熊熊燃烧的火舌惊得瞠目结舌。奥尔那些美丽的银杉回木正在化为灰烬!他的同帐篷伙伴已经把它们烧掉了!

    他目瞪口呆地盯着这四张麻木迟钝、兴高采烈的面孔,恨不得狠狠骂他们一顿,恨不得揪住他们的脑袋往一块猛撞,可他们却开心地大叫着迎接他,殷勤地搬过一把椅子请他坐下来吃栗子和烤土豆。

    他能把他们怎么样呢?

    就在第二天早晨,他们把帐篷里的死人也给弄出去了!他们就那样把他往外一扔!他们把他的行军床和他所有的行李物品全都搬到外面,往灌木丛那儿随便一扔,轻松地拍了拍手,转身就往回走,心里还觉得这件事办得挺圆满。他们精力过人,热情充沛,办起事来既讲究实际,又干脆利落,效率高极了。约塞连差点给吓晕过去。仅仅一转眼的工夫,他们就把约塞连和陶塞军士几个月来费尽心机都没能解决的问题一下子全解决了。约塞连惊慌起来,他真怕他们也许会同样干脆利落地把他给扔出去。于是,他跑到亨格利-乔那里,和他一起逃到罗马去了。第二天,内特利的妓女终于睡了一夜好觉,并从柔情蜜意中醒来
