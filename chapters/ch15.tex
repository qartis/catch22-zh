\chapter{皮尔查德和雷恩}
 
    皮尔查德上尉和雷恩上尉是两个不讨人厌的负责中队协同作战的军官。他俩性格温和,说起话来轻声慢语,个子中等偏矮,并且都喜欢战斗飞行。他俩唯一希望的就是能得到机会,继续执行战斗飞行任务。除此之外,无论是对生活还是对卡思卡特上校,他俩都别无他求。他们已经完成了几百次作战飞行任务,却还想能再飞上几百次。他们每一次都将飞行任务分配到自己头上。以前他俩从未经历过像战争这样奇妙的事情,生怕以后再也经历不到了。每次他们执行任务时,那态度很是谦卑,总是不声不响的,尽量避免张扬,而且尽力不惹恼任何人。无论从谁身旁走过,他俩总是很快地露出微笑。他们说话时,也总是咕咕哦哦的,从不粗声大气。他俩同属那类惯于随机应变、不管做什么事都心甘情愿、乐于屈从他人的人。

    只有他们两人单独相处时,他们才感到自在。他们从不正视其他人的目光,即使那天在“露天会议”上他们公开谴责约塞连,说他不该唆使基德-桑普森在执行轰炸博洛尼亚的任务时中途返航的时候,他们也不同约塞连的目光接触。

    “弟兄们,”头上的黑发已变得稀落的皮尔查德上尉开口说道,并局促不安地笑了一下。“当你们想在执行任务的中途返航时,尽量搞搞清楚,是不是有什么重大的理由,行吗?不要为了一点无关紧要的小事……比方说对讲机出了点故障……或诸如此类的小事,就返航了,你们说好不好?关于这事,雷恩上尉还要补充说几句。”

    “弟兄们,皮尔查德上尉说得对,”雷恩上尉说,“关于这事,我要对你们说的也就是这些。好啦,我们今天总算去过了博洛尼亚,大家也知道了这次飞行任务只不过是一次常规轰炸。我想咱们大伙是有点紧张了,所以没有对那儿造成多大的破坏。现在,听着,卡斯卡特上校已经得到了上级的许可,让咱们重新干一次。明天咱们可真的要去将那些弹药库好好收拾掉。好了,对这事你们有什么想法?”

    为了向约塞连证明他俩对他并无敌意,第二天重返博洛尼亚执行轰炸时,他俩甚至派他同麦克沃特一起飞,让他们的飞机在第一飞行编队里担任领队轰炸机。当约塞连飞至目标上空时,他表现得像哈弗迈耶那样自信,根本就不做规避动作,可突然间炮火从四面八方向他袭来,吓得他屁滚尿流。

 


    到处都是密集的高射炮火!约塞连原来受了骗,中了计,上了大当。此时他毫无办法,只能像个白痴似地坐在那里,眼睁睁地看着那丑陋的团团黑烟向上升腾,朝着他猛扑过来杀死他。然而在炸弹扔完之前,他什么也不能干,只好将视线转回到轰炸瞄准器上;

    瞄准器透镜上那细细的十字线像是有磁铁吸住似的,同他先前调整好的样子丝毫不差,牢牢地对准着目标;那两条线的相交处不偏不倚地正对着他负责轰炸的那个场院的中央,那是一个经过伪装的仓库,就建在第一排房屋的前面。当他的飞机悄悄地朝前飞着的时候,约塞连一个劲地发起抖来了。他先是听到了那些在他的飞机四周爆炸的高射炮弹发出的四声沉重的嘣——嘣——蹦——蹦的声音,后又听见了夹杂在这些声音中的一声刺耳而又尖厉的爆炸声,原来又有一颗炮弹猛然间就在距他咫尺的地方炸开了。在他祈求炸弹赶快落下去的时候,他的心里涌出上千种互不相干的冲动,脑袋几乎都要裂开。他真想哭。发动机继续发出单调的嗡嗡声,就像一只又肥又懒的苍蝇在哼哼。最后,瞄准器上的指针交叉到了一起,八颗五百磅的炸弹接连投了下去。由于卸掉了重负,飞机轻快地忽闪着向上飞去。约塞连将低着的脑袋从瞄准器上移开,偏过头去看左边的指示器。当指针指到零的时候,他关上了弹舱门,然后朝着对讲机,将嗓门提高到最大,尖叫道:

    “向右急转!”

    麦克沃特立即响应。随着引擎发出一阵难听的吼叫,他将飞机的一侧机翼朝下,使整个机身侧转过来,然后毫不留情地让飞机呼啸着就地来了个三百六十度的大转弯,避开了约塞连刚才发现的两道对准他们飞过来的高射炮火。然后约塞连又叫麦克沃特让飞机爬高,并不断地催他爬高、再爬高些,直至他们终于挣脱了炮火,飞进了一片宁静的、犹如蓝宝石一般湛蓝的天空。那里阳光灿烂,只有远处飘浮着些许长长的白纱一样纤薄的浮云。风吹打在飞机那圆柱形的舷窗上,那声音就像杂乱的琴声,不过让人听了感到宽心。飞机又重新加快了速度,直到这时约塞连才轻松下来,并感到一阵欣喜。后来他又吩咐麦克沃特让飞机向左拐,然后再快速向下俯冲。这时他瞥见有高射炮弹穿过他的头顶和右后上方,呈蘑菇形爆炸开来。要不是刚才向左转弯,紧接着又向下俯冲,他们准会被这阵炮火击中。为此,约塞连不禁感到一阵极短暂的狂喜。紧接着他又用刺耳的喊叫声让麦克沃特将飞机拉平,然后又催他赶快往上飞,在空中绕了一大圈,重新回到一片没有硝烟、四周参差不齐的蓝天里。与此同时,他刚才投下的那些炸弹也开始炸响了。第一颗正好落在约塞连先前瞄准的那个场院里,紧接着,其余几颗从他的和他的小队的其他飞机里投下的炸弹也都在地面上炸开。只见橘红色的火焰迅速掠过建筑物的顶部,顷刻之间变成一团团巨大无比、翻腾不已的粉红色、灰色和黑色的烟云,并四下蔓延开来,同时发出隆隆巨响,就好像是一阵阵伴随着红色、白色和金黄色的闪电而来的巨雷声。

    “哈,你看那儿,”阿费挨着约塞连大声惊叹道,他那胖胖的圆脸上闪出兴奋而又着迷的神情。“那儿原先准是个弹药库。”

    约塞连刚才早已把阿费给忘了。“滚走!”他大声朝阿费喝道,“快滚出机头!”

    阿费彬彬有礼地微笑着,指着下面的目标,十分大度地敦请约塞连朝下看。约塞连接连不断地用手拍打着阿费,并一个劲地对着那条爬行通道做着手势。

    “快回机舱去!”他狂乱地大声喊道,“回机舱去!”

    阿费和气地耸了耸肩。“我听不见你在说什么,”他解释说。

    约塞连抓住阿费身上的降落伞具的皮带,将他推回到爬行通通。也就在这时,飞机猛然间剧烈地抖动了一下,被击中了。这一抖动使得约塞连感到全身的骨头全散架了,连心脏也停止了跳动,他立即意识到这下子他们全完了。“快爬高!”他看到麦克沃特还活着,便冲着对讲机朝他尖声大叫起来。“快爬高,你这个杂种!爬高,快爬高,爬呀,快爬!”

    飞机立即陡直地向上飞去,爬得迅速而又吃力。后来约塞连又用刺耳的声音对麦克沃特大喊了一阵,要他把飞机拉平,然后又一次扭转机身,毫不怜惜地让飞机在一阵轰响中做了一个四十五度的急转弯。这个急转弯就像是一次强有力的吸气,差点没把约塞连的五脏六肺给吸出来,让他感到浑身瘫软,像一件失去了物质形体的东西那样在半空中不住地飘浮着,直到后来他叫麦克沃特再次把飞机拉平。飞机平飞后刚来得及转回右后方,就又带着一阵尖叫声向下俯冲过去。飞机急速地穿过那数不尽的一团团幽灵似的黑色烟雾向下冲着。那些飘浮在空中的黑色烟尘飘落在机头光滑的有机玻璃舱罩上,那情景就像是一片片邪恶、阴湿、肮脏的雾尘拂拭着约塞连的脸颊。此时地面上的高射炮又重新开火,一束束的炮火盲目并且杀气腾腾地朝着天空飞来,随后又无力地落下去,飞机就在这片炮火中忽上忽下地急飞着。在这种钻心揪肺的恐惧中,约塞连的心像是一把锤子似的,咚咚地敲个不停。汗水从他的脖子上大把大把地涌出,直朝着他的胸口和腰间奔流,又热又粘。有那么一会,他模模糊糊地意识到他这一编队里的其他飞机都已不在了,随后他能意识到的就只有他自己了。他感到自己的嗓子眼发堵,透不过气来,并刀割似地疼痛。他带着这种钻心的疼痛对麦克沃特尖叫着,向他发出一个又一个指令。麦克沃特每改变一下航向,发动机便发出震耳欲聋、痛苦不堪的尖声长啸。前方远处,另一群高射炮还在朝着天空接连不断地密集射击着,同时炮口还在不断地移动,以便调整到最精确的高度,恶狠狠地等待着约塞连飞入他们的射程。

 


    突然随着另一声震天动地的爆炸巨响,飞机又震动了一下,几乎翻了个身,机头里立刻充满了带有一股甜味的蓝烟。什么东西着火了!约塞连调脸想逃,却撞到了阿费身上。原来刚才是阿费划了根火柴,这会儿正若无其事地点着了他的烟斗呢。约塞连睁大眼睛看着这个生就一张笑嘻嘻的圆脸的领航员,心里既惊恐又疑惑。他心想,他们两人当中准有一个疯了。

    “天哪!”他痛苦而又吃惊地朝阿费大叫。“你给我从机头滚出去!你疯了吗?滚走!”

    “什么?”阿费问。

    “滚走!”约塞连歇斯底里地大叫,一面捏起双拳,用手背狠狠地揍着阿费,想把他赶走。“滚!”

    “我还是听不见你说什么,”阿费说。他说话时态度温和,口气里既带着困惑不解,又含有几分责难,一副清白无辜的样子。“你得说大声一点才行。”

    “从机头滚出去!”约塞连拿他没办法,只得再次尖声高叫。“他们想打死咱们!你明不明白?他们想打死咱们!”

    “该死的,我该往哪飞?”麦克沃特用一种痛苦的声音尖着嗓子朝着对讲机怒喊道,“我该往哪飞?”

    “向左拐!向左,你这该死的狗娘养的!赶快向左拐!”

    阿费爬到约塞连的身后,用烟斗柄朝他的肋部猛戳了一下。随着一声嘶哑的叫喊,约塞连一下子跳了起来,脑袋撞着了机舱顶,接着又双膝跪地,在地上蹦了一大圈,脸色像纸一样苍白,整个人气得浑身发抖。阿费则带着一种鼓励的神情朝他眨了眨眼,然后竖起大拇指朝麦克沃特做了个诙谐幽默的怪相。

    “难道有什么东西在吃他?”他出声地笑着问。

    突然一种不可名状的感觉攫住了约塞连,使得他一反常态。

    “请你离开这儿好吗?”他哀求似地大声喊道,并使出全身的力气将阿费推转身去。“你是聋了还是怎么了?回到机舱里去!”然后他又冲着麦克沃特尖叫,“俯冲!俯冲!”

    他们再度陷入了由不断爆炸着的高射炮弹交织成的砰砰作响的巨大火网之中。这时阿费又一次爬到了约塞连的身后,再次用烟斗使劲捅了一下他的肋部。约塞连又嘶哑着嗓子叫了一声,并惊跳起来。

    “我还是没听清你刚才说的话,”阿费说。

    “我说离开这里!”约塞连大叫道,禁不住哭了起来。他使出全部的力气,用双手狠劲地捶打着阿费的身体。“从我这里滚开!滚开!”

    拳头捶打在阿费身上就像是打在一只软软的充了气的橡皮口袋上。这一大堆柔软的、毫无知觉的物体既无丝毫反抗,也没任何反应。过了一会,约塞连的冲动平息了,他的双臂也因疲惫而无力地垂了下来。此时他感到十分丢脸,因为他竟拿阿费毫无办法,他为自己感到可怜,并几乎为此而哭了出来。

    “你刚才说什么?”阿费问。

    “从我这儿走开,”约塞连回答说,现在他用的是恳求的口吻。

    “回飞机后舱去吧。”

    “我还是听不见你说什么。”

    “没关系,”约塞连呜咽着说,“没关系。你别再招我就行了。”

    “什么没关系?”

    约塞连开始拍打自己的脑门。他抓住阿费衬衫的前襟,挣扎着站起身来,用力把他拖到机头的后部,像扔一只臃肿笨重的大口袋似地把他推倒在爬行通道的入口处。当他朝着机头爬回来的时候,一枚炮弹带着一声巨响就在他的耳边爆炸了。靠着没被完全摧毁的、残留在大脑深处的那一点理智,约塞连感到纳闷,这枚炮弹怎么没一下子把他们全都炸死。他们的飞机仍旧在爬升。发动机又开始发出了难听的嚎叫声,好像正处于极大的痛苦之中。机舱内的空气中充满了机器发出的呛鼻气味和汽油散发出的恶臭。他意识到的下一桩事就是,下雪了。

    成千上万的细小的白纸片像雪花一样在飞机里飘落下来,密密麻麻地绕着约塞连的头乱转、每当他惊慌地眨一下眼,这些纸片便立即粘到他的眼睫毛上;他每呼吸一下,它们就贴着他的鼻孔和嘴唇翻飞。他感到晕头转向,不知所措,可阿费却得意洋洋地咧嘴大笑,那样子简直就不像个人,手里还高举着一份破破烂烂的地图叫约塞连快看。一大团高射炮火刚才击穿了机舱底,穿过阿费那一大堆乱七八糟的地图,然后又在距他们的脑袋只几英寸的地方穿透舱顶飞了出去。阿费的那股高兴劲简直不可名状。
 


    “你要瞧瞧这个吗?”他嘁嘁喳喳他说着,两根又粗又短的手指头透过一张地图的破洞,朝着约塞连开玩笑地乱晃着。“你要瞧瞧这个吗?”

    阿费那副欢天喜地、心满意足的样子让约塞连看了直发呆。阿费就像梦中的可怕的吃人妖魔,你既伤不了他,也躲不开他。约塞连害怕他的原因很复杂,这会儿他被吓得魂飞魄散,也就无法去弄清楚其中的原因了。风从舱底被炮弹打穿的齿形裂口呼啸而入,使无数纸片像石膏碎粒一样在空中回旋不已,给人一种飞机里新上了一层漆,并且灌满了水的假相。一切看上去都很怪异,都是那么花哨,那么荒唐。这时传来了一声尖厉的叫嚷声,约塞连的头不禁猛然抽动了一下。这声音无情地钻透他的脑袋,直达他的双耳。原来这是麦克沃特在叫喊,他这是在求约塞连快下指令,因为刚才的这一片慌乱使一切都乱了套。约塞连仍旧痛苦而又惶惑地盯着阿费那张圆鼓鼓的面孔,这面孔透过那些在空中飞舞的无数白纸片,正从容而又茫然地冲着他笑呢。由此约塞连得出了一个结论:阿费是个只知道胡言乱语的白痴。就在这时,八枚高射炮弹在他们齐眉高的机外右方爆炸开来,紧接着又来了八枚,跟着又是八枚。这最后八枚炮弹是朝飞机的左方打来的,所以他们差点就撞上了这些炮弹。

    “向左急转!”约塞连冲着麦克沃待叫喊道,而阿费则仍然在对着他龇牙咧嘴地笑个不停。麦克沃特的确向左急转了,然而那些炮弹也跟着往左急转,紧紧地尾随着他们。约塞连急得大叫:“我是说要急转,急转,急转,急转,你这狗娘养的,要急转!”

    麦克沃特让飞机更加迅速地转了一个弯。忽然间,像出现奇迹似的,他们飞出了炮火的射程。火网没有了。那些高射炮也停止了对他们的轰击。而他们仍旧活着。

    在他的后面,人们正在死去。其他几个小队的飞机在高射炮的轰击下,排成了一个长条,有好几英里长,弯弯曲曲的,并不断蠕动着,仍然在目标上空做着与他们刚才一样危险的飞行。它们快速穿过天空中新老高射炮火留下的巨大烟云,就像一群老鼠穿过它们自己的一堆堆粪便在疾走狂奔,有一架飞机着火了,晃动着机翼摇摇摆摆地飞离了队伍,并不断大幅度地翻滚着,就像一颗巨大的血红色的流星。在约塞连的注视下,这架燃烧着的飞机先是侧着机身在空中飘动,然后开始呈螺旋状慢慢地向下兜起大大的圈子,并且圈子渐渐地变得越来越窄。那着了火的庞大机身吐着桔红色的火舌,而飞机的后部则火光闪闪,就像拖着一条长长的、波动不已的、由火和烟形成的斗篷。天空中开始出现了降落伞,一、二、三——四顶降落伞,接着这架飞机由转圈变成了高速的旋转,然后就一路向下栽去,直落地面,像一大片彩色皱纹纸似的在那堆熊熊烈火中无声无息地抖动着。另一中队里的整整一个小队的飞机已经给打得散了队形。
 


    约塞连兴致索然地叹了口气,他这一天的活算是干完了。这会儿他无精打采,心里极不愉快。此刻他们飞机的发动机正甜美地低声吟唱着,麦克沃特放慢了速度,慢悠悠地飞着,好让他们小队里的其他飞机跟上来。这突如其来的宁静显得是如此地陌生,如此地不自然,好像有那么一点隐含杀机的味道。约塞连劈劈啪啪地解开了防弹衣的纽扣,又摘下头上的钢盔。他又叹了口气,依旧感到心神不安,于是便合上双眼,试图让自己放松一下。

    “奥尔上哪儿去了?”突然有人通过对讲机问了他一句。

    约塞连一下子弹跳了起来,嘴里大声地吐出了一个音节:奥尔!这一喊声里透着焦虑,这一声喊也是对他们在博洛尼亚上空所遭遇到的不可思议的高射炮火袭击所作出的唯一合乎情理的解释。他猛地俯身向前,扑到他的轰炸瞄准器上,透过上面的有机玻璃朝下看,企图找到奥尔的确切踪影。奥尔像磁铁一样会吸引高射炮火,而且毫无疑问,当他一天前人还在罗马的时候,就在一夜间将赫尔曼-戈林所率的整整一个师从天知道的什么鬼驻扎地给吸引到博洛尼亚来了,并且还将他们所射出的全部劈啪作响的炮弹都引来了。这时阿费的身体也朝前俯了过来,他头盔的锋利帽边恰好砸到了约塞连的鼻梁。顿时,约塞连的双眼泪水横流,于是他便狠狠地咒骂起阿费来。

    “他在那儿,”阿费装腔作势地用悲哀的语气说,一面戏剧性地指着下面一幢灰色石头农舍的牲口棚前停着的一辆装干草的大车和两匹马。“已经粉身碎骨。我想那些碎片也已荡然无存了。”

    约塞连又咒骂起阿费来,同时继续专心地寻找着。他心里很同情他那位平日里总是欢蹦乱跳、行为古怪、生着一对龅牙的同帐篷伙伴,因而为他感到恐惧,感到担忧。他的那位伙伴曾经用乒乓球拍子将阿普尔比的脑袋砸开了花,而这会儿他又一次让约塞连吓得灵魂出窍。最后,约塞连发现了一架双引擎、双舵的飞机,这架飞机从一片苍翠的森林里飞了出来,来到一块黄澄澄的田野的上空。

    飞机的两个螺旋浆有一个变了形,已经完全不转了,然而飞机却还能维持适当的高度,保持着正确的航向。约塞连不知不觉地低声祈祷起来,感谢上帝。可随后又对奥尔感到无比的恼火,不觉又破口大骂起来,不过这种咒骂中既夹杂着怨恨,也夹杂着宽慰。

    “这个杂种!”他骂道,“这个该死的长不高的红脸蛋、大脸盘、卷头发、一嘴龅牙的狗杂种!”

    “你在说什么?”阿费问。

    “这个肮脏而又该死的傻瓜侏儒,这个鼓腮帮、金鱼眼、矮冬瓜、大龅牙、整天就会嬉皮笑脸、疯子一样的狗娘养的杂种!”约塞连唾沫四溅地骂着。

    “什么呀?”

    “没什么!”

    “我还是听不清你说什么,”阿费回答说。

    约塞连缓慢而又艰难地转过身来,面朝着阿费,开口道:“你竖耳听着。”

    “我?”

    “你这个自以为了不得的家伙,胖得像水桶,专会讨好,愚蠢透顶,还自鸣得意……”

    阿费泰然自若。他镇静地划了根火柴,然后吧咯吧喀地吸着他的烟斗,脸上明显地挂着一副能够包容一切、原谅一切的宽厚表情。他亲切地微笑着,张开嘴准备说话。可约塞连伸手捂住了他的嘴,厌烦地将他推开了。在回机场的途中,约塞连一直闭着两眼假装睡觉,这样他就可以不用听阿费说话,或看到阿费了。

    在简令下达室,约塞连向布莱克上尉汇报了作战情况,然后便和其他人等在那里;大家一直在心神不安地窃窃私语着,直到奥尔最终架着飞机嘎嚓嘎嚓地出现在上空,进入了他们的视野,方才住口。那架飞机虽然只有一个发动机是好的,但仍能让奥尔神气活现地在天上飞着。大家屏住呼吸。奥尔的起落架放不下来。约塞连一直守在那里,直到奥尔将机身贴着地面安全着陆为止。然后他顺手偷了一辆他能见到的发动机钥匙尚未拔走的吉普车,一溜烟地赶回他的帐篷,急切地开始打点行装。每逢紧急战斗过后他们都会有一次例行休假,约塞连决定这次休假去罗马。就在当天晚上,约塞连在罗马找到了露西安姻,并发现了她身上的那块一般人见不到的疤痕
