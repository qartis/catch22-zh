\chapter{斯诺登}
 
    “切开,”一个医生说。

    “你切开吧,”另一个说。

    “别切开,”约塞连舌头僵硬、口齿不清地说。

    “这是谁在乱插嘴,”一个医生抱怨道,“这儿没你说话的地方。

    我们是动手术还是不动手术?”

    “他不需要动手术,”另一个医生抱怨他说,“这不过是个小伤口,我们只要止住血,清洗一下伤口,再缝几针就行了。”

    “可我还从来没有过动手术的机会呢。哪一把是手术刀?这一把是手术刀吗?”

    “不,那一把才是手术刀。好吧,要是你想动手术,就下手吧。切开吧。”

    “就这样切开吗?”

    “不是切开那儿,你这个笨蛋!”

    “不要切开。”约塞连昏昏沉沉地感觉到有两个陌生人要把自己切开,急忙喊叫起来。

    “这儿没你说话的地方,”头一个医生挖苦地抱怨道,“我们给他动手术时,他要一直这么不停地唠叨下去吗?”

    “你们得等我收他住院后才器给他动手术,”一个职员说。

    “你得等我把他审查清楚了才能收他住院,”一个口气生硬的胖上校说。他留着小胡了,长着一张红润的硕大脸盘。这张脸几乎快要贴到约塞连的脸上了,就像一只大煎锅的平锅底似的,散发着烤人的热气。“你出生在什么地方?”

    见到这个口气生硬的胖上校,约塞连联想起那个审问牧师并裁决他有罪的口气生硬的胖上校。他瞪大眼睛,透过眼前的一层簿雾,盯着胖上校。空气中弥漫着甲醛和乙醇的清香气味。

    “我出生在战场上,”他回答说。

    “不,不,你出生在哪一个州?”

    “我出生在清白无辜的情况下。”

    “不,不,你没听明白。”

    “让我来对付他吧,”另一个人急不可耐他说。这个人瘦长脸,深眼窝,薄嘴唇,显得刻薄歹毒。“你大概是个机灵鬼吧?”他问约塞连。

    “他已经精神错乱了,”其中一个医生说,“你们为什么不让我们把他带回到里面去治疗呢?”

    “如果他精神错乱,就让他这么呆在这儿吧。他或许会说出什么能证明他有罪的话来呢。”

    “可他仍在流血不止,你难道看不见吗?他甚至会死掉的。”

    “那对他才好呢!”

    “那是这个下流杂种应得的报应,”口气生硬的胖上校说,“好吧,约翰,全都交待出来吧。我们要知道事实。”

    “大家都叫我约-约。”

    “我们要求你和我们合作,约-约。我们是你的朋友,你要信任我们。我们是到这儿来帮助你的。我们不会伤害你。”

    “我们把大拇指伸到他的伤口里戳几下,挖出点肉来,”那个瘦长脸的家伙提议道。

    约塞连闭上眼睛,好让他们以为他失去知觉了。

    “他昏过去了,”他听见一个医生说,“能不能让我们先给他治疗,要不然就太晚了。他也许会死的。”

    “好吧,带他进去吧。我真希望这杂种死掉。”

    “你得等我收他住院后才能给他治疗,”那职员说。

    当那个职员翻弄着一张张表格给他办住院手续时,约塞连闭上眼睛假装昏死了过去。随后,他被慢慢推到一间又闷又黑的房间里。房间的上空悬挂着许多灼热的聚光灯,在这里,清香的甲醛和乙醇味更加浓重了,沁人心脾的香气熏得人昏昏沉沉的。他还闻到了乙醚的气味,听到玻璃器皿的了当响声。他听见两个医生的沙哑呼吸声,心中一阵窃喜。叫他高兴的是,他们以为他失去了知觉,根本不知道他在偷听。在他听来,他们的那些对话全都无聊透顶,直到后来一个医生说:

    “喂,你认为我们应该救活他吗?我们要是救了他,他们也许会对我们怀恨在心的。”

    “我们动手术吧,”另一个医生说,“我们把他切开,看看里面究竟是怎么回事。他一直抱怨说,他的肝有毛病,可在这张调光照片上,他的肝看上去挺好的。”

    “那是他的胰腺,你这笨蛋,这儿才是他的肝呢。”

    “不,这不是,这是他的心脏。我敢拿一个五分硬币跟你打赌,这才是他的肝。我要开刀把它找出来,我应该先洗手吗?”

    “别动手术,”约塞连说、他睁开眼睛,挣扎着要坐起来。

    “这儿没你说话的地方,”其中一个医生愤愤地训斥道,“难道我们就不能叫他住嘴吗?”

    “我们可以给他来个全身麻醉。乙醚就在这里。”

    “不要全身麻醉。”约塞连说。

    “我们给他来个全身麻醉,叫他昏睡过去,那样我们想把他怎么样就怎么样。”

    “他们给约塞连做了全身麻醉,使他昏睡过去。他醒来时发现自己躺在一个弥漫着乙醚气味的僻静房间里、直觉得口干舌燥;科恩中校坐在他床边的一张椅子上,正安安静静地等着约塞连醒来呢。

    他穿着宽松肥大的橄榄绿衬衣和裤子,胡须密匝匝的棕色脸庞上挂着人丝和蔼而淡漠的微笑:他正用双手轻轻抚摸着自己的秃脑门呢。约塞连一醒过来,他便俯下身格格笑着,语气极为友好地向约塞连保证说,只要约塞连不死,他们之间的那笔交易就仍然有效。约塞连哇的一声呕吐起来。科恩中校一听到声音马上跳起身,厌恶地逃了出去。约塞连心想,乌云之中总还是有一线光明的。随后,他觉得透不过气来,便又昏昏沉沉地睡过去了,一只长着尖指甲的手粗暴地把他摇醒了。他翻过身,睁开眼睛,看见一个面容猥琐的陌生人轻蔑地撇着嘴,不怀好意地瞪着他。那人得意地说:

    “我们抓到你的伙伴了,老弟。我们抓到你的伙伴了。”

    约塞连顿时浑身冰凉,一阵晕眩。他出了一身冷汗。

    “谁是我的伙伴?”当他看到牧师坐在刚才科恩中校坐的地方时,他问道。

    “也许我是你的伙伴,”牧师回答道。

    但是,约塞连没能听见他的话。他又闭上了眼睛。有人拿过水来喂他喝了几口,又踮着脚尖走开了。他睡了一阵,醒来时觉得情绪很好,便转过头去想对牧师笑笑,却发现换了阿费坐在那里。约塞连不由自主地叹了口气。阿费哈哈大笑,问他眼下感觉如何。约塞连异常烦恼地沉下脸,反问阿费为什么不在监狱里呆着,一下子把阿费给问糊涂了,约塞连闭上眼睛,想赶阿费走,等到他再睁开眼睛时,阿费已经走了,牧师又坐在那里了。他一眼瞥见牧师兴高采烈的笑模样,不由哈哈大笑起来,一边笑一边问牧师到底为了什么这么高兴。

    “我为你高兴呀,”牧师激动、快活而又坦率地回答道。“我在大队部里听说你受了重伤,如果你活下来的话,就送你回国。”科恩中校说,你的情况很危险。可我刚刚从一位医生那儿得知、你受的伤非常非常轻,过一两天你大概就可以出院了。你一点危险都没有,情况好得很。”

    听了牧师带来的这个消息,约塞连大大地松了一口气。“这好极了。”

    “是啊,”牧师说。两片绊红悄悄爬上他的面颊,使他看上去显得既顽皮又快乐。“是啊;这好极了。”

    约塞连回想起自己第一次与牧师谈话的情景,不由哈哈大笑起来。“瞧,我第一次遇见你是在医院里,现在我又在医院里了。最近一次我见到你也是在医院里。你这一向呆在哪儿?”

    牧师耸了耸肩。“我一直在祷告,”他坦白道,“我尽可能呆在自己的帐篷里。每一回惠特科姆中士离开这个地区时我都要祷告,这样他就不会抓住我了。”

    “这样做有用处吗?”

    “这样做可以减轻我的烦恼,”牧师又耸了耸肩回答道,“再说,这样的话,我也有事可干了。”

    “噢,这很不错,不是吗?”

    “是呀,”牧师热烈地赞同道,好像他原先从来没有想到过这一点,“是呀,依我看,这确实不错。”他兴奋地俯下身来,显得既尴尬又焦虑。“约塞连,在你住院期间,有没有什么我可以帮忙的地方,需要我为你带些什么东西来吗?”

    约塞连快活地取笑他说:“像玩具、糖果或者口香糖之类吗?”

    牧师的脸又红了。他不自然地咧嘴笑笑,然后又变得恭恭敬敬的。“像书籍啦,也许别的什么东西。我希望我能做点什么让你高兴的事。你知道,约塞连,我们大伙都很为你感到骄傲。”

    “骄傲?”

    “是啊,当然啦。是你冒着生命危险拦住了那个纳粹刺客。这是非常崇高的行为。”

    “什么纳粹刺客?”

    “就是那个来这儿暗杀卡思卡特上校和科恩中校的家伙呀。是你救了他们的命。你在楼厅上跟他扭打成一团时,他差一点把你刺死。你能活下来真是命大。”

    约塞连弄明白是怎么回事后,不由得冷笑起来。”那人根本不是什么纳粹刺客。”

    “没错,是的。科恩中校说他是的。”

    “那人是内特利的女朋友。她是来找我的,不是来找卡思卡特上校和科恩中校的。自从我把内特利的死告诉她以后,她就一直想杀我。”

    “可这怎么可能呢?”牧师脸色发青地愤然反驳道。他给弄得有点糊涂了。“他逃走时,卡思卡特上校和科恩中校两个人全都看见了。官方的报告说,你拦住了一个前来暗杀他们的纳粹刺客。”

    “别相信官方的报告。”约塞连冷冰冰地提醒他。“那是这笔交易的一部分。”

    “什么交易?”

    “是我跟卡思卡特上校和科恩中校做的一笔交易。如果我见人就讲他们的好话,并且永远不在任何人面前批评他们叫其余的官兵执行更多的飞行任务的话,他们就把我当做一个大英雄送回国去。”

    牧师大吃一惊,差点从椅子上跳起来。他既恼怒又沮丧,摆出一副好斗的架势嚷嚷起来。“但这太可怕了!这是一笔见不得人的丑恶交易,难道不是吗?”

    “令人作呕,”约塞连回答说。他把后脑勺靠在枕头上,毫无表情地盯着天花板。“我想,我们都同意用‘令人作呕’这个字眼来形容它。”

    “那你干吗要同意这笔交易呢?”

    “要么接受这笔交易,要么接受军法审判。”

    “噢,”牧师露出一副懊悔不已的神情,用手捂着嘴叫道。他局促不安地欠身坐回到椅子上。“我真不应该说刚才那番活的。”

    “他们会把我关到监狱里,让我和一帮罪犯呆在一起。”

    “当然啦。凡是你认为正确的,你就应当做。”牧师点点头,好像就此了结了这场争论,随即便陷入了窘迫的沉默之中。

    “别担心,”过了几分钟,约塞连凄惨地笑了笑说,“我并不打算这么做。”

    “但你必须这么做,”牧师关心地俯下身来坚持道,“真的,你必须这么做。我没有权利影响你。我真的没有权利说三道四。”

    “你没有影响我。”约塞连吃力地翻过去侧身躺着,既庄重又嘲讽地摇了摇头。“耶稣啊,牧师!你难道不认为那是一种罪孽吗?救了卡思卡特上校的命!我决不允许在自己的档案里出现这桩罪行。”

    牧师小心翼翼地回到原先的话题上;“那你将怎么办呢?你不能让他们把你关进监狱。”

    “我要执行更多的飞行任务。要么,我也许真的会临阵脱逃,让他们抓住我。他们大概会的。”

    “那样他们就会把你关进监狱。你不想进监狱吧。”

    “那么,我想我只好继续执行飞行任务,直到战争结束。我们中总有些人会活下来的。”

    “可你也许会送命的。”

    “那么,我想我还是不执行飞行任务吧。”

    “那你将怎么办呢?”

    “我不知道。”

    “你会让他们送你回国吗?”

    “我不知道。外面热吗?这儿倒是很暖和的。”

    “外面很冷,”牧师说。

    “你知道,”约塞连回忆说,发生了一件希奇古怪的事——也许是我做梦吧。我觉得刚才来过一个陌生人,他告诉我他抓住了我的伙伴。不知道这是不是我想象出来的。”

    “依我看,这不是你想象出来的,”牧师对他说,“我上一次来的时候,你就把这件事讲给我听了。”

    “那么,那个人真的说过这话了。‘我们抓到你的伙伴了,老弟,’他说,‘我们抓到你的伙伴了。’我以前还从来没有见到过像他那么凶恶的样子。我很想知道谁是我的伙伴。”

    “我倒认为我是你的伙伴,约塞连,”牧师既谦卑又诚恳地说,“他们肯定是抓住我了。他们记下了我的号码,一直在监视着我。他们要叫我到哪儿去,我立刻就得到哪儿去。他们审问我的时候就是这么说的。”

    “不,我不认为他们指的是你,”约塞连肯定地说,“我认为他们准是指内特利或者邓巴这一类的人。你知道,是指某一个在这场战争中送命的人,像克莱文杰、奥尔、多布斯、基德-桑普森或者麦克沃特。”约塞连突然吃惊地长叹一声,摇了摇脑袋。“我这才明白,”他叫道,“他们抓走了我所有的伙伴,不是吗?剩下的只有我和亨格利-乔了。”当他看见牧师的脸色变得煞白时,他不由得惊慌起来。

    “牧师,出了什么事?”

    “亨格利-乔死了。”

    “上帝啊,不!是执行任务时死的吗?”

    “他是睡觉时做梦死去的,他们看见一只猫趴在他的脸上。

    “可怜的家伙。”约塞连说着便哭了起来,他把脸伏在臂膀里,不想让人看见他的眼泪。牧师没说再见就走了。约塞连吃了点东西后睡着了。半夜里,一只手把他摇醒过来、他睁开眼睛,看见一个面容猥琐的瘦子。那人穿着病员的浴衣和睡衣裤,一边狞笑着,一边嘲弄地对他说。

    “我们抓到你的伙伴了,老弟。我们抓到你的伙伴了。”

    约塞连心烦意乱起来、“你他妈的到底在说些什么?”他略显恐慌地追问道。

    “你会发现的,老弟,你会发现的。”

    约塞连伸出一只手去掐那个折磨自己的人的脖子,可那人毫不费劲地避开了他的手,怪笑一声逃到走廊里不见了。约塞连躺在床上一个劲地哆嗦,脉搏直跳个不停,他出了一身的冷汗。他很想知道谁是他的伙伴。医院笼罩在黑暗之中,一片寂静。他没戴手表,不知道几点了。他已经完全清醒了。他知道,自己是个整夜卧床不起却又无法入睡的囚徒,在永无尽头的黑夜里企盼着黎明的到来。

    阵阵寒气从他的双腿直往上逼来,他想起了斯诺登。斯诺登从来都不是他的伙伴,只不过是个他稍微有点熟悉的小伙子罢了。那一回,多布斯在内部对讲机里向约塞连呼叫,救救轰炸手、救救轰炸手。约塞连从炸弹舱的舱顶上爬过去,爬到机尾舱里,看见斯诺登受了重伤,眼看就要冻死了。一圈刺眼的金色阳光透过侧炮眼照射到他躺着的地方,在他的脸上跳动着。约塞连第一眼看见那种令人毛骨悚然的情景时,胃里就立刻翻腾起来了,他觉得恶心。他心惊胆战地愣了几分钟才往下爬,匍匐着穿过炸弹舱顶上的狭窄通道,从装着急救药箱的密封皱纹纸板箱旁边爬过去。斯诺登双腿叉开仰面躺在舱板上,身上仍然裹着笨重的防弹衣、防弹钢盔、降落伞背带和飞行救生衣。离他不远处躺着那个不省人事的小个子尾舱机枪手。约塞连看见斯诺登的大腿外侧有一个伤口,看上去足有一只橄榄球那么大,那么深。鲜血浸透了他的工作服,根本分不清楚哪些是碎布条,哪些是烂糊糊的血肉。

    急救药箱里没有吗啡,也没有别的可以帮斯诺登止痛的药品。

    约塞连只好眼睁睁地、心惊胆战地盯着那个裂开了的伤口。药箱里的十二支吗啡针全被人偷走了。在原来放针的地方有一张字迹工整的纸条,上面写着:“有益于M&M辛迪加联合体就是有益于国家。米洛-明德宾德”。约塞连一边诅咒米洛,一边拿起两片阿司匹林硬往斯诺登那两片毫无反应的苍白嘴唇里塞。不过,他先是匆匆忙忙地抓起一条止血带绑住斯诺登的大腿,因为在最初几分钟的慌乱之中,他的脑子里一片混乱,只知道自己必须采取适当的措施,却一时想不出具体应该做些什么。他真怕自己会完全垮掉。斯诺登一声不吭,静静地看着他。并没有动脉出血的迹象,可约塞连却装出一副全神贯注绑扎止血带的模样,因为他根本不懂得如何使用止血带。他假装成熟练和内行的样子摆弄着止血带,他能够感觉出斯诺登那暗淡无光的眼睛正盯着自己。止血带还没绑扎好,他就恢复了镇定。他立即把止血带松开,以防产生坏疽。此时,他的头脑已经清楚,他知道该怎么办了。他在急救药箱里翻来翻去,想找一把剪刀。

    “我冷,”斯诺登轻声说,“我冷。”

    “你很快就没事了,小伙子,”约塞连笑着安慰他说,“你很快就没事了。”

    “我冷,”斯诺登又虚弱无力他说,他的嗓音听起来像个天真的孩子。“我冷。”

    “好啦,好啦。”约塞连不知道再说什么好,只得这样答应着。

    “好啦,好啦。”

    “我冷。”斯诺登鸣咽着。“我冷。”

    “好啦,好啦。好啦,好啦。”

    约塞连害怕起来,动作也加快了。终于,他找到了一把剪刀。他小心翼翼地把斯诺登的工作服从伤口处往上剪开,一直剪到他的大腿根。随后,他又绕着他的大腿笔直地剪了一圈,把那件厚厚的华达呢工作服一截为二。他正剪着,那个小个子尾舱机枪手醒了过来,看了看他,便又昏过去了。斯诺登把头扭到另一边,以便更加直接地盯着约塞连。他那疲倦、无精打采的眼睛里闪动着一丝暗淡的光。约塞连心里有点发虚,竭力不去看他。他又顺着工作服的内侧接缝往下剪。从那个裂开的伤口里——那些疹人的肌肉组织仍在抽搐着、跳动着——殷红的鲜血不停地往外涌。透过这些,他看到的是不是一根粘糊糊的骨管呢,——鲜血就像房檐上融化的雪水那样分成许多细流往外流淌,不过,他的血又粘又红,一流出来就凝固住了。约塞连沿着工作服的裤管一直剪到底,又动手把剪断下来的裤管从斯诺登的腿上褪下来。裤管扑的一声落在地上,里面的卡其布短衬裤的底边露了出来,其中一侧浸透了血污,好像要用鲜血解渴似的。约塞连吃惊地看见,斯诺登赤裸的大腿是那样光滑而苍白,而他那白得出奇的小腿则毛茸茸地长满细细的、卷曲的淡黄汗毛,看上去令人厌恶又毫无生气,显得很特别。这时他看清了,这个伤口并没有橄榄球那么大,但却有他的手掌那么长、那么宽,而且非常深,里面血肉模糊,只能看见血淋淋的肌肉不停地抽搐着,就像新鲜的碎牛肉。约塞连看出斯诺登没有生命危险,长长地舒了一口气,放下心来。伤口内的鲜血已经开始凝固了。在飞机降落之前,只要给他包扎一下,使他保持镇静就可以了。约塞连从急救药箱里拿出几包磺砖药粉来。当他轻轻地推着斯诺登,想叫他稍微侧一侧身子时,斯诺登哆嗦起来。

    “我弄痛你了吗?”

    “我冷。”斯诺登呜咽着。“我冷。”

    “好啦,好啦,”约塞连说,“好啦,好啦。”

    “我冷,我冷。”

    “好啦,好啦。好啦,好啦。”

    “伤口开始痛了,”斯诺登猛地缩了一下,突然哀怨地叫了起来。

    约塞连又发疯似地在急救药箱里乱翻一通,想找支吗啡针:可是只找到了米洛的纸条和一瓶阿司匹林。他一边诅咒着米洛,一边把两片阿司匹林递到斯诺登的嘴边。他没有水给他服药。斯诺登几乎令人察觉不出地轻轻摇了摇头,表示他不愿意吃阿可匹林:他的脸苍白苍白的。约塞连摘下斯诺登的防弹钢盔,让他的头枕在舱板上。

    “我冷。”斯诺登半闭着眼睛呻吟道,“我冷。”

    他的嘴唇开始发青。约塞连有点惊慌失措了,他不知道该不该扯开斯诺登的开伞索、把尼龙降落伞布盖在他的身上。机舱里非常暖和、出乎他的意料,斯诺登突然抬了抬眼睛,疲倦而友好地冲他微微一笑,随后挪了挪屁股,好让约塞连给他的伤口敷上磺安药粉。约塞连干着干着便恢复了信心,重新变得乐观起来,飞机闯进一股垂直气流之中、剧烈地颠簸起来:约塞连突然吃惊地想起来,他把自己的降落伞忘在机头那边了。但是,这会儿已经没有什么办法好想了。他一包接一包地把结晶状的白色药粉倒入那个血肉模糊的椭圆形伤口里,直到把殷红色全部盖住。接着,他忧心忡忡地深吸一口气:咬紧牙关,壮起胆子伸出一只赤裸的手抓起那些垂在外面的、渐渐变干巴了的缕缕肌肉,把它们塞回到伤口中去。他急急忙忙地用一大块药棉纱布盖住伤口,随即把手缩了回去。这场短暂的严峻考验总算过去了,他神经质地笑了笑。直接接触无生命的肉体并不像他所预料的那么令人恶心,于是,他一再找借口一次次用手指头去抚摸那个伤口,以确认自己是勇敢的。

    然后,他动手用一卷绷带绑住那块纱布。当他第二次把绷带绕过斯诺登的大腿时,他看见在他的大腿内侧还有个小洞。这是个圆圆的、有两角五分硬币那么大的伤口,青紫的边缘卷缩着,中间黑洞洞的,血已经凝固了。弹片就是从这儿穿进去的。约塞连在这个伤口上也敷上一层磺安药粉,又继续往斯诺登的大腿上缠绷带,直到把那块纱布包扎紧为止。接着,他用剪刀剪断绷带,把绷带头塞到里面,打了个十分整齐的方结,紧紧系住绷带。他觉得自己包扎得很好,得意地跪坐在自己的后脚跟上,一边擦着额头上的汗珠,一边真诚而友好地对斯诺登咧嘴笑着。

    “我冷。”斯诺登呻吟着。“我冷。”

    “你很快就没事了,小伙子,”约塞连安慰地抬了抬他的胳膊,向他保证道,“一切全都控制住了。”

    斯诺登无力地摇了摇头。“我冷。”他又说。他的眼睛呆滞、暗淡,就像两块石头,“我冷。”

    “好啦,好啦,”约塞连说。他越来越感到疑虑和惊慌。“好啦,好啦。不一会儿我们就着陆了,丹尼卡医生会来照料你的。”

    可是,斯诺登还是不停地摇头。最后,他稍微扬了扬下巴,朝自己的腋窝示意了一下。约塞连弯下腰盯住那儿,看见就在防弹衣的袖筒上方,一片颜色奇怪的污迹从工作服里渗透出来、他觉得自己的心一下子停住不跳了,接着又激烈地咚咚跳个不停、跳得他透不过气来。斯诺登的防弹衣里面还有伤口。约塞连一把扯开斯诺登防弹衣的扣子,不由得尖声叫了起来。斯诺登的内脏涌了出来,湿漉漉地堆在地板上,而且伤口里面的血仍然滴滴答答地往外流淌着。一块三英寸多长的弹片正巧从他另一侧的腋窝处射了进去。

    这块弹片穿过他的腹腔,又在这边的肋骨处打通一个大洞,把他肚子里杂六杂八的东西全都带了出来。约塞连又尖叫了一声,伸出双手使劲捂住眼睛。他吓得浑身战栗,牙齿格格打战。他强迫自己再次抬眼看过去。他一边看一边痛苦地想,上帝造出的一切都在这儿了——肝、肺、肾、肋骨、胃,还有斯诺登那天午饭吃的煨番茄。约塞连最讨厌煨番茄。他头晕目眩地转过身去,一手按住热乎乎的喉咙,大口大口呕吐起来。他正吐着,那个尾舱机枪手醒了过来,看了他一眼,就又昏过去了。约塞连吐完之后,感到浑身疲乏无力,内心既痛苦又绝望。他虚弱地转回身对着斯诺登。斯诺登的呼吸变得越来越微弱、急促,他的脸也变得越来越苍白。约塞连不知道到底该怎么做才能够救活他。

    “我冷,”斯诺登呜咽着说,“我冷。”

    “好啦,好啦,”约塞连机械地嘟哝着。他的声音小得根本听不见。

    约塞连也感到冷,他不由自主地哆嗦起来。斯诺登那可怕的五脏六腑脏兮兮地淌了一地。他死死盯住它们,浑身起了一层鸡皮疙瘩。它们所包含的寓意是很容易领会的。人是物质,这就是斯诺登的秘密。把他从窗口扔出去,他就会摔下去;把他点燃了,他就会烧起来;把他埋入地下,他就会和别的各种垃圾一样腐烂。灵魂离去之后,人就变成了垃圾。这就是斯诺登的秘密。成熟的时机决定一切。

    “我冷,”斯诺登说,“我冷。”

    “好啦,好啦,”约塞连说,“好啦,好啦。”他扯开斯诺登的开伞索,把白色的尼龙降落伞布盖在他的身上。

    “我冷。”

    “好啦,好啦。”
