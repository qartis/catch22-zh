\chapter{不朽之城}
 
    约塞连未经上司许可就擅自离队,搭乘米洛的飞机跟他一块飞往罗马。在飞机上,米洛责备地晃着脑袋,虔诚地咂起嘴唇,以教士的口吻对他说,他为他感到羞愧。约塞连点点头,米洛接着说,约塞连把枪挎在屁股后面倒退着走路,并拒绝执行更多的飞行任务,这是自己给自己出丑。约塞连点点头。米洛又说,这种做法是对他自己中队的背叛,既让他的上司感到为难,又使米洛处于一种极为难堪的境地。约塞连又点点头。米洛又说,官兵们已经开始抱怨了。约塞连仅仅考虑他自身的安全,而像米洛、卡思卡特上校、科恩中校和前一等兵温特格林这样的人却都在全力以赴打赢这场战争,这未免太不公平了。已经执行了七十次飞行任务的人也开始抱怨了,因为他们不得不飞满八十次。危险的是,他们中的某些人可能也会挎上枪,开始倒退着走路。士气正变得越来越低落,这全都是约塞连一手造成的。国家正处在生死存亡的关头,他却胆敢滥用自由、独立等等传统权利,从而危及到这些权利本身。

    米洛没完没了地唠叨着,约塞连坐在副驾驶员的座位上,一边不住地点着头,一边却竭力不去听他的唠叨。约塞连满脑子想的全是内特利的妓女,还有克拉夫特、奥尔、内特利、邓巴、基德-桑普森、麦克沃特,以及他在意大利、埃及和北非见到过的那些贫穷、愚笨、疾病缠身的人。他知道,在世界上别的地区也有这样的人。斯诺登和内特利的妓女的小妹妹也使他感到良心不安。约塞连觉得,他现在明白了内特利的妓女为什么认为他对内特利的死负有责任,为什么要杀死他。她为什么不应该这样做呢?这是一个男人的世界,各种非自然的灾祸全都降临到她和其他所有年纪较轻的人的头上,为此,她们每个人都有充分的权利谴责他和其他所有年纪较大的人,正如她自己,即使她正处于悲伤之中,也应当为降临到她的小妹妹和其他所有孩子头上的种种人为的苦难而受谴责一样。某人某时总得做某件事。每个受害者都是犯罪者,每个犯罪者又都是受害者。总得有某个人在某个时候站出来打碎那条危及所有人的传统习俗的可恶锁链。在非洲的某些地方,幼小的男孩子仍然被成年的奴隶贩子偷去卖掉赚钱。那些买主把他们开膛破肚,然后吃掉他们。约塞连感到不可思议,这些孩子怎么能够身受如此野蛮的残害却未曾流露出丝毫的惧怕和痛苦呢?他认定这是他们的忍受力特别强的缘故。他想,要不然的话,这种习俗肯定早已消亡,因为,他觉得,无论人们对财富或长生不老的渴望多么强烈,都不至于使他们拿孩子们的痛苦去换取这些。

    米洛说,约塞连是在捣乱。约塞连又一次点点头。米洛说,约塞连不是队里的一个好成员。约塞连点点头,听着米洛告诉他,如果他不喜欢卡思卡特上校和科恩中校管理大队的方式,那么他应该做的是离队去俄国,而不是留在这儿兴风作浪。约塞连本来想说,如果卡思卡特上校、科恩中校和米洛不喜欢他在这儿兴风作浪的话,他们可以统统去俄国,但他还是忍住了没说出口。米洛说,卡思卡特上校和科恩中校两个人一直对约塞连很好,上一次执行轰炸弗拉拉的任务之后,他们不是还发给他一枚勋章并提拔他为上尉吗?约塞连点点头。难道不是他们供给他吃的并按月发给他军饷的吗?约塞连又点点头。米洛确信,如果他前去向他们赔罪认错,答应执行八十次飞行任务,他们肯定会宽大为怀的。约塞连说,这件事他会考虑的。当米洛放下飞机轮子,朝着跑道滑降下去时,约塞连屏住呼吸,祈求上帝保佑平安降落。真是可笑,他怎么竟会变得这么厌恶飞行呢?

 


    飞机降落后,他看到罗马已是一片废墟。飞机场八个月前曾遭到轰炸。在机场入口的两侧可以看见一个个推土机推成的平顶白色碎石瓦砾堆,机场周围的铁丝网也全给推土机推倒了。圆形剧场只剩下残垣断壁,君士但丁拱门也已经倒塌了。内待利的妓女的公寓墙倒屋塌,窗玻璃全都砸破了。妓女们都不在了,只剩下那个老太婆守在那儿。她身上左一层右一层地裹着毛线衣和裙子,头上蒙着一条深色的围巾。她双臂抱拢在胸前,坐在电炉旁边的一张木头椅子上,正用一只破铝锅烧开水呢。约塞连进门时,她正在大声地自言自语。一看见他,她就呜咽开了。

    “走了,”他还没开口问话,她就呜咽着说。她抱住自己的胳膊时,在那张吱嘎作响的椅子上悲伤地前后摇晃着。“走了。”

    “谁走了?”

    “全都走了。所有可怜的年轻姑娘都走了。”

    “去哪儿了?”

    “外面。全都被赶到外面大街上去了。她们全都走了,所有可怜的年轻姑娘都走了。”

    “被谁赶走了?是谁干的?”

    “是那些下流的高个子士兵,他们戴着硬邦邦的白帽子,手里拿着棍子。还有我们的宪兵。他们拿着棍子把她们往外赶,连外衣也不让她们穿。可怜的姑娘们。他们就这么把她们全都赶到外面去挨冻。”

    “他们逮捕她们了吗?”

    “他们把她们赶走了,他们就这么把她们赶走了。”

    “如果他们没有逮捕她们,那为什么要把她们赶走呢?”

    “我不知道,”老太婆抽泣着说道,“我不知道。谁来照顾我呢?

    现在所有那些可怜的年轻姑娘都走了,还有谁来照顾我呢?谁来照顾我呢?”

    “这总得有个理由,”约塞连固执地说。他用一只拳头使劲捶着另一只手掌。“他们总不能就这么闯进来把所有的人都赶出去吧。”

    “没有理由,”老太婆呜咽道,“没有理由。”

    “那他们有什么权利这么做?”

    “第二十二条军规。”

    “什么?”约塞连惊恐万状,一下子愣住了。他感到自己浑身上下针扎般地疼痛。“你刚才说什么?”

    “第二十二条军规。”老太婆晃着脑袋又说了一遍。“第二十二条军规。第二十二条军规说,他们有权利做任何事情,我们不能阻止他们,”“你到底在讲些什么?”约塞连困惑不解,怒气冲冲地朝她喊叫道,“你怎么知道是第二十二条军规?到底是谁告诉你是第二十二条军规的?”

    “是那些戴着硬邦邦的白帽子、拿着棍子的大兵。姑娘们在哭泣。‘我们做错了什么事?’她们问。那些兵一边说没做错什么,一边用棍子尖把她们往门外推。‘那你们为什么把我们赶出去呢?’姑娘们问。‘第二十二条军规,’那些兵说。他们只是一遍又一遍地说‘第二十二条军规,第二十二条军规’。这是什么意思,第二十二条军规?什么是第二十二条军规?”

 


    “他们没有给你看看第二十二条军规吗?”约塞连问。他恼火地跺着脚走来走去。“你们就没有叫他们念一念吗?”

    “他们没有必要给我们看第二+条军规,”老太婆回答道。

    “法律说,他们没有必要这么做。”

    “什么法律说他们没有必要这么做?”

    “第二十二条军规。”

    “唉,真该死!”约塞连恶狠狠地嚷道,“我敢打赌,它根本就不存在。”他停住步,闷闷不乐地环顾了一下房间。“那个老头在哪?”

    “不在了,”老太婆悲伤地说。

    “不在了?”

    “死了,”老太婆对他说。她极为悲哀地点点头,又把手掌朝着自己的脑袋挥了挥。“这里面有什么东西破裂了。一分钟前他还活着,一分钟后他就死了。”

    “但他不可能死!”约塞连叫道。他很想坚持自己的观点,可他当然知道那是真的,知道那是合乎逻辑的,是符合事实的:这个老头和大多数人走的是一条路。

 


    约塞连转身出去,步履沉重地在公寓里转了一圈,他阴沉着脸,既悲观又好奇地把所有的房间窥视了一遍。玻璃制品全都被那些兵用棍子砸碎了。撕成一条条的窗帘和被单乱七八糟扔了一地。

    椅子、桌子和梳妆台全都给打翻了。所有能砸碎的东西全部给砸碎了。这场破坏真是干净彻底,野蛮的汪达尔人也只能干到如此地步。所有的窗子都打破了,乌云般的黑暗穿过破碎的窗格玻璃涌入每个房间。约塞连能够想象得出那些戴着硬邦邦的白色钢盔的高个子宪兵砰砰的沉重脚步声,能够想象得出他们乱砸乱摔时那副狠毒而又兴致勃勃的样子,以及他们那种伪善的、冷酷的所谓正义感和献身精神。所有可怜的年轻姑娘都走了。所有人都走了,只剩下这个穿着一层层肥大的褐色和灰色的毛线衣、戴着黑色围巾的老太婆。她很快也会走的。

    “走了,”约塞连走了回来,还没来得及开口讲话,她就悲伤他说道,“现在谁来照顾我呢?”

    约塞连没有理会她的问话。“内特利的女朋友——有人听到过她的消息吗?”他问。

    “走了,”“我知道她走了。可有人听到过她的消息吗?有人知道她在哪儿吗?”

    “走了。”

    “还有她那个小妹妹,她怎么样了呢?”

    “走了。”老太婆的声调没有任何变化。

    “你知道我在说什么吗?”约塞连严厉地问道。他逼视着她的眼睛,想弄清楚她对他讲话时头脑是否清醒。他提高了嗓门。“那个小妹妹怎么样了,那个小姑娘?”

    “走了,走了,”老大婆被他的追问惹火了,生气地耸了耸肩回答道。她低低的呜咽声变得越来越高。“和其他人一块被赶出去了,赶到大街上去了。他们甚至不让她带上自己的外衣。”

    “她到哪儿去了?”

    “我不知道,我不知道。”

    “谁来照顾她呢?”

    “谁来照顾我呢?”

    “她不认识别的什么人,是吗?”

    “谁来照顾我呢?”

    约塞连往老太婆膝盖上扔了些钱——说来可笑,留下钱又能补救多少过失呢——便大踏步地走出了公寓。他一边走下楼梯,一边在心里狠狠地诅咒第二十二条军规,尽管他心里明白,根本不存在这么条军规。第二十二条军规不存在,对此他确信无疑,可那又有什么用呢?问题在于每个人都认为它存在,而更糟糕的是,它没有什么实实在在的内容或条文可以让人们嘲笑、驳斥、指责、批评、攻击、修正、憎恨、谩骂、啐唾沫、撕成碎片、踩在脚下或者烧成灰烬。

    外面又冷又黑,空气中弥漫着死气沉沉的薄雾,四处渗透,把一排排用粗糙大石块建成的房子和一座座纪念碑的底座笼罩得严严实实。约塞连急急忙忙赶回米洛那儿认错。他明知故犯地撒谎说,他很抱歉,并答应米洛,只要米洛愿意利用他在罗马的全部影响,帮助找出内特利的妓女的小妹妹在哪里,那么,卡思卡特上校叫他再执行多少次飞行任务他就执行多少次。

    “她还只是个十二岁的小处女,米洛,”他焦虑地解释道,“我想立刻找到她,不然就太晚了。”

    听了他的请求,米洛宽厚地笑了笑。“我这儿正好有个你正在寻找的十二岁的小处女,”他眉开眼笑地说,“这个十二岁的小处女其实刚刚三十四岁,但她是靠吃低蛋白饮食长大的,她的父母又非常严厉,她一直没有跟男人睡过觉,直到——”

    “米洛,我说的是一个小姑娘!”约塞连极不耐烦地打断他的话。“你难道不明白吗?我不是想跟她睡觉。我是想帮助她。你也有女儿吧。她还是个小孩子,她在这座城市里举目无亲,没有任何人照顾她。我是要保护她不受伤害。你难道不明白我在说什么吗?”
 


    米洛终于明白了,而且深受感动。“约塞连,我为你而骄做,”他大为激动地叫道,“我真的为你而骄做。当我看到你并不总是一门心思考虑性生活时,你不知道我是多么地高兴。你是个讲义气的人。我当然有女儿,我完全明白你在说些什么。我们一定要找到那个女孩。你别着急。你跟我来,哪怕把这座城市翻个底朝天,我们也要找到那个女孩。来吧!”

    约塞连坐着米洛-明德宾德开得飞快的M&M指挥车来到警察总部,会见一个警察专员。那人皮肤黝黑,长着两撇细细的小胡子,上衣敞开着,显得邋里邋遢。他们走进他的办公室时,他正跟一个长着肉赘和双下巴的矮胖女人调情呢。看到米洛,他喜出望外,奴颜婢膝地朝着米洛又是鞠躬又是作揖,好像米洛是什么高官显贵似的。

    “啊,米洛侯爵,”他热情洋溢地叫道,看也不看一眼就把那个满脸不高兴的矮胖女人推出了门。“你为什么不早告诉我你要来呢?如果我事先知道,我会为你举行一个盛大宴会的。请进,请进,侯爵,你怎么这么长时间都不到我们这里来了呢?”

    米洛知道眼下一分钟都不能浪费。“喂,卢吉,”他边说边急匆匆地点点头,几乎显得有些粗暴无礼。“卢吉,我需要你的帮助。我这个朋友要找个女孩。”

    “找个女孩,侯爵?”卢吉问。他用手抓了抓脸,沉思了一下。

    “罗马有这么多的女孩。对一个美国军官来说,找一个女孩不会是很困难的。”

    “不,卢吉,你没明白。是个十二岁的小处女,他必须马上找到她。”

    “噢,是这样,我明白了,”卢吉领悟地说,“找个处女也许要花点时间。不过,在公共汽车终点站那儿有许多进城来找工作的年轻农村姑娘,如果他在那儿等的话,我——”

    “卢吉,你还是没明白。”米洛烦躁而粗暴地打断了警察专员的活,后者不禁面红耳赤,急忙跳起来立正站好,胡乱地系上制服的扣子。“这小姑娘是一个朋友,是家人的一个老朋友。我们要帮助她。她还是个孩子。她眼下在这座城市里的某一个地方,无依无靠的。我们得在她受到伤害之前找到她。现在你明白了吗?卢吉,这件事对我极为重要。我有个女儿跟这个小姑娘一样大。眼下对我来说,世界上再也没有比及早救出这个可怜的孩子更为重要的事情了,你愿意帮忙吗?”
 


    “是的,侯爵,现在我明白了,”卢吉说,“我将尽我所能去寻找她。不过,今晚我这儿没有什么人了。今晚所有的人都忙着去打击非法烟草买卖了。”

    “非法烟草买卖?”米洛问。

    “米洛。”约塞连声音微弱地叫了一声。他的心沉下去了,他当时就明白一切全完了。

    “是的,侯爵,”卢吉说,“非法烟草买卖的利润非常高,所以走私活动几乎无法控制。”

    “非法烟草买卖的利润真的这么高吗?”米洛极感兴趣地问。他贪婪地高高挑起铁锈色的眉毛,直往鼻孔里吸气。

    “米洛,”约塞连冲他叫道,“听我说,好吗?”

    “是的,侯爵,”卢吉回答道,“非法烟草买卖的利润非常高。走私引起了全民的公愤,侯爵,这真是国人的耻辱。”

    “这是事实吗?”米洛出神地笑着说,着魔似地迈步朝门口走去。

    “米洛!”约塞连大叫道,冲动地奔上去拦住他。“米洛,你必须帮助我。”

    “非法烟草买卖,”米洛露出癫痫患者般的贪婪神色对他解释道,倔强地甩开他往外走。“让我走,我必须去非法走私烟草。”

    “留在这儿帮我找到她吧,”约塞连恳求道,“你可以明天再去非法走私烟草。”

    但是,米洛根本没听见他的恳求。他大步流星地往外冲去,虽然算不上来势凶猛,可也无法阻拦。他满头大汗,双眼闪闪发光,嘴唇抽搐,口水直淌,仿佛他已经深深陷入某种盲目的情结之中了。

    他平静地呻吟着,好像处在某种出自本能的、模糊不清的痛苦感觉之中。他一遍又一遍地重复道:“非法烟草,非法烟草。”约塞连最后终于看出来了,和他根本讲不通道理,只好无可奈何地给他让开条路。米洛像出膛的子弹猛冲了出去。警察专员又解开了制服的扣子,轻蔑地看了看约塞连。

    “你还在这儿干什么?”他冷冷地问,“你是要等我逮捕你吗?”

    约塞连走出办公室,走下楼梯,来到昏暗的、墓地般的街道上。

    经过门厅时,他遇上那个长着肉赘和双下巴的矮胖女人进门往里走。外面根本没有米洛的影子。所有的窗子里面都没有灯光。空无一人的人行道形成一个陡峭的斜坡,向前延伸了好几个街区。他能够看见,在长长的鹅卵石斜坡的顶端,有一条灯火通明的宽阔大道。警察总部差不多位于这斜坡的最低处,人口处的黄色灯泡像湿火把似的在潮湿的夜晚里噬噬作响。空中飘洒着寒冷的细雨。他慢慢地顺着斜坡往上走,不一会便来到一家安静、舒适、诱人的餐厅前面。餐厅的窗户上挂着大红天鹅绒窗帘,门旁有块天蓝霓虹灯招牌,上面写着:“托尼餐厅,佳肴美酒,请勿入内。”有那么一瞬间,天蓝霓虹灯招牌上的这几个字使他稍稍有点惊讶。在他身处的这个不可思议的畸形世界里,无论什么反常的东西都不再显得稀奇古怪了。那些矗立在街道两侧的建筑物的顶部全都以一种奇特的、超现实主义的比例修建成斜面,结果使得街道本身看上去也是倾斜的。他翻起暖和的羊毛外套的衣领,让它紧紧地裹住自己。这个夜晚阴湿寒冷。一个穿着薄薄的衬衫和薄薄的破裤子的男孩赤着脚从黑暗中走了出来。他长着黑黑的头发,他需要理发了,他还需要鞋子和袜子。他面带病容,脸色苍白,一副凄惨的模样。他走在湿漉漉的人行道上。他的脚踩在雨水坑里,发出吮吸般的轻微声响,听起来十分可怖。这男骇的穷困深深地打动了约塞连,他从心底里同情他,他真想一拳把男孩那张苍白、凄惨、面带病容的脸打个满脸开花,真想一拳把他打出人世间,因为,看见这男孩使他想起所有生活在意大利、生活在这同一个夜晚的苍白、凄惨、面带病容的孩子,想起他们全部需要理发,需要鞋子和袜子。这男孩还使约塞连想起那些残废人,想起那些饥寒交迫的男男女女,想起那些寡言少语、逆来顺受的虔诚母亲,她们在这同一个夜晚目光紧张地坐在户外,毫不在乎地在阴冷的雨中袒露前胸,用冻得冰凉的动物般的Rx房给婴儿喂奶。奶牛。恰恰在这个时候,一个正在喂奶的母亲抱着用黑色破布裹着的婴儿缓步走过。约塞连真想也把她打得满脸开花,因为她使他想起了刚才那个穿着薄薄的衬衣和薄薄的裤子的男孩,以及这个世界上所有令人不寒而栗、目瞪口呆的悲惨事件。在这个世界上,除了那些擅长权术、卑鄙无耻的一小撮人之外,其他所有的人全都得不到温饱和公正的待遇。这是一个多么令人憎恶的世界啊!他想知道,即使在他自己那个繁荣的国度里,在这同一个夜晚,有多少人缺吃少穿,有多少住房四壁透风,有多少丈夫喝得烂醉,有多少妻子遭受毒打,有多少孩子被欺侮、被辱骂、被遗弃。有多少家庭忍饥挨饿买不起食物?有多少人伤心欲绝?在这同一个夜晚,发生了多少起自杀事件,又有多少人精神失常?有多少奸商和店老板欣喜若狂?有多少赢家变为输家,多少成功者变为失败者,多少富人变为穷人?有多少聪明人其实愚蠢透顶?有多少美满的结局其实充满了不幸?有多少老实人其实是骗子,多少勇敢的人其实是胆小鬼,多少忠心耿耿的人其实是叛徒,多少圣徒其实道德败坏,多少身居要职的人为了几个小钱向恶魔出卖灵魂?又有多少人根本没有灵魂?有多少笔直的窄道其实弯弯曲曲?有多少最美好的家庭其实是最糟糕的家庭,多少好人其实是坏人?你要是把这些人全都加起来,然后再把他们从总人数中减掉,剩下的也许就只有孩子们了,或者还有个艾尔伯特-爱因斯但,再加上什么地方的一个老提琴手或雕刻家。约塞连孤零零地走着,内心非常痛苦。他觉得自己似乎与世隔绝了。他心里老是想着那个面带病容的赤脚男孩。直到他拐了个弯走到大道上时,他才终于把男孩那令人惨不忍睹的形象从脑海里摆脱掉。在大道上,他碰到一个盟军士兵躺在地上抽搐。这是个年轻的中尉,长着一张小小的、苍白的、孩子气的脸。六个来自不同国家的士兵使劲按住他身体的不同部位,努力想帮他平静下来。他咬紧牙关,语无伦次地喊叫着、呻吟着,一个劲地翻白眼。“别让他把舌头咬掉了,”约塞连身旁一个矮个中士机灵地提醒道。又一个士兵立即扑上去加入了这场混战,他使劲按住了中尉那张痉孪的脸。突然间,这帮人的目的达到了,被他们牢牢压在身下的中尉一下子僵直不动了。可他们反而没了主意,你望望我,我望望你,谁也不知道该拿他怎么办才好。他们粗野的面孔全都绷得紧紧的,不约而同地流露出痴呆呆的恐慌神色。“你们为什么不把他抬起来放到那辆汽车的引擎盖上去呢?”一个站在约塞连背后的下士拖着腔说。这话似乎有道理,于是那七个士兵抬起年轻的中尉,一边仍然按住他身上抽搐的各个部位,一边小心翼翼地把他平放在旁边一辆停着的汽车的引擎盖上。可把他放在引擎盖上以后,他们又开始紧张不安地互相望着,不知道接下来该拿他怎么办才好。“你们为什么不把他从那汽车的引擎盖上抬下来放到地上呢?”约塞连背后的那个下士又拖着腔说。这似乎也是个好主意,于是他们又动手把他抬回到人行道上。他们还没有把他放好,就飞快地开过来一辆闪着红色聚光灯的吉普车。吉普车前座上坐着两个宪兵。

    “出了什么事?”司机叫道。

    “他正抽风呢,”一个正握住年轻中尉一条腿的士兵回答道,“我们在帮他平静下来。”

    “很好。他被逮捕了。”

    “我们应该拿他怎么办?”

    “逮捕他!”宪兵大叫道。他为自己开的这个玩笑而声音粗哑地大笑起来,直笑得弯下了腰,然后开着吉普车一溜烟走了。

    约塞连这才想起来自己没有准假条,便谨慎地从这帮陌生人身边走过,朝着前面远处漆黑的夜色中传来低沉人声的地方走去。

    在被雨水淋透了的宽阔的林荫大道上,每隔半个街区就有一盏低低弯垂的路灯,灯光透过褐色的烟雾,闪烁着怪异的光芒。他听到在他头顶的窗户里,有一个不幸的女人在恳求道:“请不要,请不要。”一个垂头丧气的年轻妇女穿着黑色雨衣,脸上垂着一缕缕黑发,耷拉着眼皮走了过去。在位于下一个街区的公共事务部的门外,一个醉醺醺的年轻士兵把一个醉醺醺的女郎一步步逼退到一根科林斯式凹槽圆柱上,他的三个醉醺醺的伙伴则两腿夹着酒瓶,坐在附近的台阶上看着他们俩。“请不要,”醉醺醺的女郎哀求道,“我现在要回家去,请不要。”约塞连转过身朝他们望去,其中一个坐着的士兵挑衅地骂了一声,抓起一个酒瓶子朝着约塞连扔了过去。酒瓶没有伤着他,而是落到远处,发出一声闷响,碎了。约塞连双手插在衣袋里,无精打采,不慌不忙地走开了。“来吧,宝贝,”他听见那个醉醺醺的士兵口气坚决地催促道,“现在轮到我了。”“请不要,”那个醉醺醺的女郎哀求道,“请不要。”就在下一个拐弯处,从一条弯弯曲曲的窄街深处,从漆黑漆黑的阴影里,传来神秘的、清晰的铲雪的声音。他走下人行道从这条凶险的胡同口穿过时,那种铁铲刮擦水泥地面发出的有节奏的、令人心里发毛的缓慢声响吓得他起了一身的鸡皮疙瘩。他急忙快步往前走去,直到那折磨人的刺耳声音被远远地抛在后面。现在他知道自己走到哪儿了,如果他一直往前走,很快就会到达林荫大道中央那口干涸的喷泉处,然后再往前走七个街区,就是军官公寓了。突然,他听到从前面阴森可怖的黑暗当中传来动物的嗥叫声。拐弯处的路灯已经熄灭了,整整半条街笼罩在黑暗之中,一切东西看上去全都模模糊糊、歪歪扭扭的。在十字路口的另一边,一个男人正用一根棍子打一条狗,就像拉斯科尔尼科夫梦中的那个人拿一条鞭于抽那匹马一样。约塞连努力想做到既不行也不听,可是办不到。那条狗被一条破旧的白棕绳拴着,声嘶力竭、惊恐万状地时而哀号,时而尖叫,毫无反抗地匍匐在地上扭来扭去,可那人仍然不停地用那根粗粗的扁棍一个劲地打它。一小群人在围观。有一个矮胖的女人走上前去,请求他往手。“少管闲事,”那人生硬地叫道,举起棍子,好像要连她一块打似的。那女人满面羞愧,胆怯而猥琐地退了回去。约塞连加快脚步,几乎跑着离开了。这个夜晚充满了种种恐怖景象。他在心里想,如果耶稣降临久这个世界上走一遭的话,他的感觉准跟精神病医生穿过到处是疯子的精神病房,或跟被盗者穿过到处是盗贼的牢房时的感觉一模一样。即使此时出现一个麻风病人,也没有人会觉得他丑陋难看的!在下一个拐弯处,一个男人正在野蛮地殴打一个小男孩,一群成年人无动于衷地围观着,没有一个人出来干预。一种似曾相识的感觉使约塞连感到恶心,他急忙向后退去。他肯定自己从前什么时候曾经目睹过与此相同的可怕情景。是记忆错觉吗?这种不祥的巧合使他震惊,使他内心充满了疑虑与恐慌。这情景与他在前一个街区看到的情景非常相似,尽管其中的具体人物似乎完全不同。这世界上究竟正在发生什么事情?会有一个矮胖的女人站出来请求那男人住手吗?那男人会扬起手打她,把她吓退吗?谁也没有动一动。那男孩不停地哭叫着,好像沉浸在痛苦之中。那男人一次次扬起巴掌,响亮地、狠狠地朝着他的脑袋打下去,把他打倒在地,然后又猛地把他揪起来,再一次把他打倒。那帮绷着脸、缩着脑袋的围观者当中似乎没有人关心这个被打得晕头转向的男孩,没人愿意站出来加以制止。这男孩最多只有九岁。一个面色灰黄的妇女正捧着一块肮脏的洗碗布在哭泣。这男孩皮包骨头,他需要理发了,鲜血从他的两只耳朵里涌出来。约塞连快步穿越宽阔的大道,来到另一侧,远远躲避开这幕令人作呕的情景,不料却又发现脚下踩上了一些人的牙齿。在被雨水冲刷得闪闪发亮的人行道上,这些牙齿散落在一滩滩被劈啪降落的雨点淋得醚糊糊的、血迹周围,就像尖尖的手指甲那样你戳着我,我指着你。地上到处是臼齿和门牙的碎片。他踮起脚尖绕过这片怪异的废墟,来到一个门前。门洞里面一个士兵正用一块湿透了的手帕捂着嘴哭泣。他摇摇晃晃地站着,身旁还有两个士兵搀扶着他。他们严肃而焦虑地等待着军用救护车。可当它终于闪烁着琥珀色的雾灯当当地驶过来时,却没在他们面前停下来,而是一直开到了前面一个街区。那儿有个拿着几本书的意大利平民和一群拿着手铐和警棍的便衣警察发生了冲突。那个尖叫着、挣扎着的平民本来是个皮肤黝黑的人,眼下却吓得面如白纸。当许多身材高大的警察抓住他的四肢,把他举起来时,他的眼睛像蝙蝠拍打翅膀似的,紧张而绝望地扑闪个不停。他的书撤了一地。“救命啊!”当警察把他抬到救护车后面敞开的门前往车里扔去时,他尖声大叫着。他的嗓子因为激动而哽噎住了。“警察!救命!警察!”车门被关上拴住了,救护车飞驰而去,当警察把他团团围住时,他竟然荒唐地向警察喊叫救命,这真是一个毫无幽默的讽刺。想到这种呼救的徒劳和荒谬,约塞连不禁苦笑了一下。随后,他猛然悟出,这呼救声有着不止一层的含义。他惊恐地意识到,这也许不是向警察发出的呼救,而是一个命在旦夕的朋友勇敢地从坟墓里发出的警告。他是在呼喊那些除了佩带警棍和手枪的警察以外的人,以及另外一些佩带警棍和手枪的警察前来支持他。“救命!警察!”那人这样喊叫着,他可能是在大声提醒别人有危险。想到这儿,约塞连赶快蹑手蹑脚地从警察身旁溜走,却又差点被一个四十岁的粗壮女人的脚绊倒。这女人正一边心慌意乱地穿过十字路口,一边鬼鬼祟祟地、存心不良地回头扫视跟在她身后的一个八十岁的老妇人。这老妇人脚踝上缠着厚厚的绷带,步履瞒珊地追赶着她,可怎么也迫不上,老妇人摇摇晃晃地往前走,大口大口地喘着气,心烦意乱、焦虑不安地自语着。这幕情景的性质是明确无误的:这是一场追逐。前面的女人已经成功地穿越了一半宽阔的大道,而后面的老妇人却还没有走下人行道。那女人扭头看后面步履艰难的老妇人时,流露出一种恶意的、卑劣的、幸灾乐祸的微笑,显得很恶毒,却又疑惧重重。约塞连知道,只要那个身陷困境的老妇人叫喊一声,他就会上前帮她的忙。他知道,只要她发出一声痛苦的尖叫向他求援,他就会扑上前去抓住前面那个粗壮的女人,把她交给附近那帮警察。但是,那老妇人悲伤而苦恼地嘟囔着,甚至看也没看他就走了过去。不一会,前面的那个女人消失在越来越深的黑暗之中,撇下那老妇人一个人孤零零地、茫然不知所措地站在大路中间,拿不准该走哪条路。约塞连因为自己没能给她任何帮助,羞愧得不敢多看她一眼,急匆匆转身离开了。他一边垂头丧气地逃走,一边鬼鬼祟祟、心慌意乱地回头看,唯恐那老妇人现在会跟着他走。他暗自感谢飘洒着毛毛细雨、没有光亮、几乎伸手不见五指的漆黑夜幕,因为它正好把他给遮掩了起来。一帮帮……一帮帮警察——除了英国,别处全都在一帮帮、一帮帮、一帮帮的暴徒掌握之中。到处都在一帮帮手持警棍的暴徒控制之下。

    约塞连外套的领子和肩膀全都淋透了。他的袜子潮湿冰冷。前面的一盏路灯也灭了,玻璃灯泡给打碎了。建筑物和面容模糊的人影无声无息地从他身旁一一闪过,好像是浮在某种恶臭扑鼻、永无尽头的浪潮之上一去不复返地漂走了。一个高个子僧侣走了过去,他的脸被一块粗糙的灰色蒙头斗篷包得严严实实,甚至连眼睛都藏在里面。前面传来脚踩在泥水里走路发出的扑哧扑哧的声响,他真怕这又是一个赤脚的男孩。他与一个瘦削枯槁、表情忧郁的男人擦肩而过。那人穿着件雨衣,面颊上有一个星状的伤疤,一侧的太阳穴上有一块凹陷的、表面光滑的残缺处,足有鸡蛋般大小。一个年轻女人穿着咯吱作响的草鞋突然出现了。她的整张脸丑陋不堪,一大片烧伤留下的粉红花斑伤痕刚刚脱痴,皱皱巴巴地从脖颈向上伸展,经过双颊,一直延伸到眼睛上面,真是可怕极了!约塞连吓得浑身哆嗦,不敢抬头多看一眼。不会有人爱上这个女人的。他感到懊丧。他渴望跟某个他会爱上的姑娘睡觉,那姑娘会抚慰他,使他兴奋,然后把他哄睡着。一帮手持警棍的家伙正在皮亚诺萨岛上等着他。所有的姑娘都走了。伯爵夫人和她的儿媳已经失去了魅力;他已经老了,没有兴趣玩乐了,也没有时间玩乐了。露西安娜走了,也许死了;即使没死,大概也快了。阿费的那个丰满的浪荡女人连同她那枚下流的浮雕宝石戒指一起消失了。达克特护士嫌他丢人,因为他拒绝执行更多的战斗飞行任务,会引起公愤。这附近他认识的姑娘就只剩下军官公寓里的那个相貌平平的女佣,没有一个男人曾经跟她睡过觉。她的名字叫米恰拉,但男人们给她起了不少下流的绰号。当他们用悦耳的讨好声调叫她的这些绰号时,她高兴得格格傻笑,因为她不懂英语,还以为他们是在奉承她,是在善意地和她开玩笑呢。每当她看到他们胡作非为时,她的内心便充满了喜悦。她是个快活、纯朴、手脚勤快的姑娘。她不识字,只能勉强写下自己的名字。她的头发直直的,看上去就像因受潮而腐烂的麦秆。她的皮肤灰黄,眼睛近视,从来没有男人跟她睡过觉,因为他们谁也不想跟她睡觉,只有阿费例外。就在这同一个晚上,阿费强xx了她,然后用手捂住她的嘴,把她按在衣橱里关了将近两个小时,直到响起宵禁的汽笛才住手。此时她若是到外面去便是违法的了。

    然后,他把她从窗户里扔了出去。约塞连赶到时,她的尸体仍然躺在人行道上,四周围了一圈板着面孔、手举暗淡提灯的邻居。

    约塞连彬彬有礼地往圈里挤,邻居们一面给他让出一条路,一面目光狠毒地盯着他。他们怨愤地指着二楼的窗户,严厉地轻声指责着。看到那具摔得血肉模糊的尸体,那种可怜的、血淋淋的惨景,约塞连吓得浑身战栗,心扑通扑通直跳。他闪身钻进门厅,冲上楼梯、进了公寓房间,看到阿费正心绪不宁地来回踱着步,脸上带着一种外强中干、略显不自在的笑容。阿费心不在焉地玩弄着自己的烟斗,看上去有点心烦意乱。不过,他向约塞连保证说,一切全都正常,没有什么可担心的。

    “我只强xx了她一次,”他辩解道。

    约塞连吓了一跳。“可你杀了她,阿费!你杀了她!”

    “唉,强xx了她之后,我不得不这么干,”阿费态度极为傲慢地回答道,“我不能让她到处去讲我们的坏活,对吧?”

    “可你干吗要去碰她呢,你这个愚蠢的杂种?”约塞连叫道,“你要是需要姑娘,难道不能到大街上去找一个来吗?这座城市里到处是妓女。”

    “哦,不,我不能,”阿费吹嘘道,“我一辈子没有花钱干过这种事。”

    “阿费,你疯了吗?”约塞连几乎说不出话来了。“你杀了一个女人。他们会把你关进监狱的!”

    “噢,不,”阿费强挤出一个笑容回答道,“不会把我关起来的。

    他们不会把好心的老阿费关进监狱的。不会因为杀了她就把我关起来的。”

    “可你把她从窗户扔了出去。她的尸体还在街上躺着呢。”

    “她没有权利躺在那儿,”阿费回答道,“已经过了宵禁时间了。”

    “笨蛋!你难道不知道你干了什么事吗?”约塞连真想抓住阿费那毛毛虫般柔软的肥实肩膀使劲摇晃几下,好叫他清醒清醒。“你谋杀了一个人。他们就要把你关进监狱了。他们甚至可能会绞死你的!”

    “噢,我可不认为他们会这么做,”阿费回答道。他开心地抿嘴笑了笑,不过看得出来,他越来越紧张了。他用粗短的手指笨拙地摆弄着烟斗,无意识地把烟丝全部抖落出来了。“不,长官。他们不会绞死好心的老阿费的。”他又格格地笑了起来。“她不过是个女佣人。我可不认为他们会因为一个下贱的意大利女佣人的死而大惊小怪的。现在每天都要死掉成千上万的人呢。你说呢?”

    “你听!”约塞连几乎是高兴地叫了起来。他竖起耳朵听远处哀鸣般的警笛声。是警车的警笛声。然后,几乎在刹那之间,警笛声越来越响,变成一种嘈杂刺耳、气势汹汹的曝叫。这曝叫盖过其它一切声音,似乎从四面八方撞入室内,把他们团团围住。约塞连看到,阿费的脸上没有一点血色。“阿费,他们是来抓你的。”为了能让阿费在一片警笛声中听见,他可着嗓子叫喊。他的心底涌起一阵同情。“他们是来逮捕你的,阿费,你难道不懂吗?你不能害死另一个人而逍遥法外,即便她是个下贱的女佣人也不行。你难道不明白吗?你不懂吗?”

    “噢,不,”阿费说。他勉强挤出一丝笑容,干巴巴地哈哈一笑。

    “他们不是来逮捕我的。不会逮捕好心的老阿费的。”

    突然间,他面呈病容,瘫坐在椅子上。他表情呆滞,浑身哆嗦,两只又粗又短、肌肉松弛的手在膝盖上抖个不停。汽车在门外刹住了,聚光灯随即射向窗口,车门砰地关上,警笛尖叫起来。有人刺耳地大声喊叫着。阿费吓得脸色发青。他机械地摇着脑袋,脸上浮现出一种古怪而生硬的微笑,声音微弱而空洞地一遍遍重复着,他们不是来抓他的,不是来抓好心的老阿费的,不,长官。甚至当有人脚步沉重地冲上楼梯,跑过楼梯平台时,甚至当有人使足劲在门上用拳头猛捶了四下,差点把他们的耳朵震聋时,他仍然在努力使自己相信,这些人不是来抓他的。随后,公寓房间的门被猛地推开,两个粗野强壮的大块头宪兵冲进房间。他们的目光冷冰冰的,肌肉发达的下巴绷得紧紧的,显得十分严厉。他们大踏步穿过房间,逮捕了约塞连。

    他们是因为约塞连未持有通行证便呆在罗马而逮捕他的。

    他们因擅自闯入而向阿费道歉,随后便一边一个夹住约塞连,把铁铐般的手指伸到他的腋下牢牢掐住,将他带了出去。下楼梯时,他们一句话也没有说。外面车门紧闭的汽车旁边,还有两个身材高大、戴着硬邦邦的白色钢盔的宪兵正在等着他们。他们把约塞连推到汽车后座上,汽车立刻轰呜着穿过雨雾朝警察所开去。宪兵们把他锁在一间四面都是石头墙壁的牢房里关了一夜。到了黎明时分,他们递给他一只桶解小便,接着便开车把他押送到飞机场。

    在那儿的一架运输机旁边,另外有两个手持警棍、头戴白色钢盔的膀大腰圆的宪兵正在等着他们。他们到达时,飞机的引擎已经发动起来了,绿色的圆柱形整流罩表面上,渗出的水汽凝聚而成的小水珠微微颤动着。那些宪兵互相之间也不说一句话,甚至连头也不点一下。约塞连从来没有看见过这么冷冰冰的面孔。飞机直接飞往皮亚诺萨岛。在简易跑道上,还有两个沉默不语的宪兵正在等着他们。现在,一共有八个宪兵了。他们准确地遵行着无声的命令,列队分别进入两辆汽车。汽车轰呜着奔驰而去。他们穿过四个中队的驻地,来到大队司令部的大楼前面。在那儿的停车场上,另外有两个宪兵正在等着他们。这样,当他们转弯走向大楼人口时,一共有十个高大强壮、意志坚强、沉默不语的宪兵严严实实地簇拥着他。他们在煤渣路上迈着整齐的步伐,脚下发出嘎吱嘎吱的声响。

    约塞连觉得,他们似乎走得越来越炔。他惊恐不安起来。这十个宪兵中的任何一个看上去都力大无比,一拳就可以把他打死。他们只需把他们宽阔的、强健的、巨石般的肩膀朝他身上猛劲挤压过去,即刻就能叫他断气。他没有任何救自己性命的办法。当他们紧紧排成两行,把他夹在中间快步往前走时,他甚至弄不清楚是哪两个宪兵把手伸到他的腋下牢牢掐住的。他们的脚步越来越快。当他们果断而有节奏地疾步走上宽阔的大理石楼梯,来到上面的楼梯平台时,约塞连觉得自己好像是脚离了地在飞似的。在楼梯平台处,另外有两个表情冷酷、令人难以捉摸的宪兵正在等着他们。这两个宪兵领着他们以更快的速度沿着长长的、悬在宽阔门厅上方的楼厅往前走。在暗色的瓷砖地面上,他们的脚步轰然作响,犹如一阵令人肃然起敬的、节奏越来越快的鼓声回荡在空荡荡的大楼中央。当他们走向卡思卡特上校的办公室时,他们前进的速度更快,步伐更整齐了。他们把他推进办公室时,约塞连以为自己这回死定了,吓得两只耳朵里嗡嗡直响。在卡思卡特上校办公桌的一角,科恩中校正舒舒服服地仰坐着。他和蔼可亲地笑着朝约塞连打了个招呼,然后说道:

    “我们要送你回国啦。”
