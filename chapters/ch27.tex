\chapter{达克特护士}
 
    苏-安-达克特护士是个成年女性,又瘦又高,腰板笔直,长着一个圆滚滚的翘屁股和一对小巧的Rx房。她的脸庞棱角分明,皮肤白里透红,眼睛小小的,鼻子和下巴尖细瘦削,一副新英格兰禁欲主义者的模样,看上去既非常可爱又非常平庸。达克特护士成熟老练,精明能干,办事果断严格。她喜欢独当一面,一向遇事不慌,无论大事小事都是自己拿主意,从来不需要别人帮忙。约塞连觉得她可怜,打算帮她一把。

    第二天一早,当她站在约塞连的床脚边整理床单时,他悄悄把手伸到她双膝间的窄缝里,随即飞快地在她的裙子里面尽力向上摸去。达克特护士尖叫一声,猛地往上跳去,可是跳得不够高。她扭动着身体,弓着腰,以自己那神圣的部位为支点,前旋后转,左扭右摆,整整折腾了十五秒钟,才终于挣脱出来。她惊惶失措地后退到走道中间,面如纸灰,双颊抽搐个不停。她后退得太远了。一直在走道另一侧看热闹的邓巴一声不吭地从床上跃起直扑她的身后,伸出双臂一下子揽住她的胸脯。达克特护士又尖叫了一声。她甩开邓巴,远远地躲到走道的这一侧。不料约塞连又趁机扑上去一把抓住了她。她只好又一次蹦过走道,活像一只长着脚的乒乓球。

    正严阵以待的邓巴立刻朝她猛扑过来,幸好她反应及时,闪身跳到一旁。邓巴扑了个空,从她身边蹿过病床,一头撞到地上。只听扑通一声,他便昏了过去。

    他在地上醒来时,鼻子正在流血,这倒正和他一直假装的那种折磨人的脑病的症状一模一样。病房里闹哄哄乱成一团。达克特护士在哭泣,约塞连挨着她坐在床边,一个劲地向她赔不是。主管上校怒气冲冲地朝约塞连大喊大叫,说他绝对不能允许病人肆意调戏护士。

    “你要他怎么样?”躺在地上的邓巴可怜巴巴地问。他一开口说话太阳穴便感到一阵阵的疼痛,疼得他身体缩成一团。“他又没干什么。”

    “我是在说你呢!”这位很有派头的瘦上校放开嗓门吼叫道,“你要为你的所作所为受处分的。”

    “你要他怎么样?”约塞连叫喊起来。“他不就是头朝下摔到地上去了嘛。”

    “我也正在说你呢!”上校一转身冲着约塞连发起火来。“你抱住了达克特护士的胸脯,等着吧,你会为此而后悔的。”

    “我没有抱住达克特护士的胸脯,”约塞连说。

    “是我抱住达克特护士的胸脯的,”邓巴说。

    “你们两个都疯了吗?”医生面色苍白,一边尖叫着,一边慌慌张张地向后退去。

    “是的,医生,他的确疯了,”邓巴肯定他说,“他每天夜里都梦见自己手里拿着一条活鱼。”

    正在后退的医生停了下来,露出既惊奇又厌恶但又不失优雅的表情,病房里静了下来,“他梦见了什么?”医生质问道。

    “他梦见自己手里拿着一条活鱼。”

    “是什么样的鱼?”医生转向约塞连,厉声发问道。

    “我不知道,”约塞连答道,“我不会分辨鱼的种类。”

    “你哪一只手拿的鱼?”

    “不一定。”

    “那是随着鱼而变化的,”邓巴帮腔道。

    上校转过身,眯起眼睛怀疑地盯着邓巴。“是吗?你是怎么知道这么多的?”

    “因为我在梦里呀,”邓巴一本正经地答道。

    上校窘得面红耳赤。他恶狠狠地瞪着他们俩,一副决不手软的样子。“爬起来,回到你的床上去。”他咧开两片薄嘴唇命令邓巴。

    “关于这个梦,我再也不想听你们俩讲一个字了。我手下有人专门负责听你们这类令人讨厌的疯话。”

    上校把约塞连打发到精神病专家桑德森少校那儿。这位少校长得敦敦实实,总是笑眯眯的,显得十分和蔼可亲。他小心翼翼地问约塞连:“你究竟为什么认为费瑞杰上校讨厌你的梦呢?”

    约塞连恭恭敬敬地回答道:“我认为,这或者是由于这个梦的某种特性,或者是由于费瑞杰上校的某种特性。”

    “你讲得很好,”桑德森少校拍手称赞道。他穿着一双咯吱作响的步兵军鞋,一头木炭般乌黑的头发几乎朝天直竖着。“由于某种原因,”他推心置腹地说,“费瑞杰上校总是使我想起海鸥。你知道,他不大相信精神病学。”

    “你不大喜欢海鸥吧?”约塞连问。

    “是的,不怎么喜欢,”桑德森少校承认道。他发出一种神经质的尖笑,伸出手爱抚地摸摸他那胖得垂挂下来的双下巴,仿佛那是一把长长的山羊胡子。“我认为你的这个梦很迷人。我希望这个梦经常出现,这样我们就可以继续不断地讨论它。你想抽支烟吗?”当约塞连拒绝时,他笑了笑。“你认为究竟是什么使你产生这么大的反感,”他故意问,“连我的一支烟都不肯接受?”

    “我刚刚熄掉一支,它还在你的烟灰缸里冒烟呢。”

    桑德森少校抿嘴笑笑。“这个解释很巧妙。但我想我们很快就会找出真正的原因的。”他把松开的鞋带系成一个松松垮垮的蝴蝶结,然后从桌上拿过一本黄色横道拍纸簿放到膝上。“让我们谈谈你梦见的那条鱼吧。总是同一条鱼,是吗?”

    “我不知道,约塞连回答道,“我不大会辨认鱼。”

    “这鱼使你想到了什么?”

    “其它的鱼。”

    “其它的鱼又使你想到了什么?”

    “其它的鱼。”

    桑德森少校失望地往后一靠。“你喜欢鱼吗?”

    “不是特别喜欢,”“那么你认为究竟是什么使你对鱼产生这样一种病态的反感呢?”桑德森少校得意洋洋地问。

    “它们太乏味了,”约塞连回答说,“刺又太多。”

    桑德森理解地点点头,露出讨人喜欢的、虚假的微笑。“这个解释很有意思。但我想我们很快就会找出真正的原因的。你喜欢那条鱼吗?那条你拿在手里的鱼?”

    “我对它没有一点感情。”

    “你不喜欢那条鱼吗?你对它怀有什么故意的或者对抗的情绪吗?”

    “不,完全没有。事实上,我还是喜欢那条鱼的。”

    “那么,你确实喜欢那条鱼咯?”

    “哦,不,我对它没有一点感情。”

    “但你刚才还说你喜欢它呢。现在你又说你对它没有一点感情。我把你的自相矛盾之处抓住了,你明白吗?”

    “是的,长官,我想您是把我的自相矛盾之处抓住了。”

    桑德森少校拿起他那枝粗粗的黑铅笔,得意洋洋地在拍纸簿上一笔一划地写下“自相矛盾”几个字。写完之后,他抬起头来继续问道:“你这两句话表达了你对那条鱼的自相矛盾的情绪反应,究竟是什么使你说出这两句话来的呢?”

    “我想我对它持有一种既爱又恨的矛盾态度。”

    听到“既爱又恨的矛盾态度”这几个字,桑德森少校高兴得跳了起来。“你的确理解了!”他喊道,欣喜若狂地把两只手放在一起拧来拧去。“唉,你想象不出我是多么孤独,天天跟那些毫无精神病常识的人谈话,想方设法给那些对我或者我的工作丝毫不感兴趣的人治病!这使我产生了一种无能为力的可怕感觉。”一丝焦虑的阴影在他的脸上一闪而过。“我似乎无法摆脱这种感觉。”

    “真的吗?”约塞连问,他不知道还有什么话好说。“你为什么要为别人缺乏教育而责怪你自己呢?”

    “我知道这很愚蠢,”桑德森少校心神不安地回答道,脸上带着不很雅观的、无意识的笑容。“可我一向十分看重别人的好主意。你瞧,比起我的同龄人来,我的青春期来得晚一些,这就给我带来某种——嗯,各种问题。我清清楚楚地知道,和你讨论我的这些问题将会给我带来乐趣,我真希望马上开始这种讨论,所以我不大愿意现在就把话题扯到你的问题上去。可恐怕我必须这样做。要是费瑞杰上校知道我们把全部时间都花在我的问题上的话,他准会发火的。我现在想给你看一些墨水迹,看看某些形状和颜色会使你联想起什么来。”

    “你就别操这份心了吧,医生,不管什么东西都会使我联想起性来的。”

    “是吗?”桑德森少校高兴得叫了起来,好像不敢相信自己的耳朵似的。“现在我们的确有了进展!你做没做过有关性生活的美梦呢?”

    “我那条鱼的梦就是性生活的梦。”

    “不,我的意思是真正的性生活的梦——在这种梦里,你抱住一个光屁股女人的脖子,拧她,使劲打她的脸,直打得她浑身是血,后来你就扑上去强xx她,再后来你突然哭了起来,因为你爱她爱得这么深,恨她也恨得这么深,真不知该怎么办才好。这就是我想跟你讨论的性生活的梦,你没有做过这类性生活的梦吗?”

    约塞连摆出一副精明的神情,想了一想,下结论说:“这是鱼的梦。”

    桑德森少校往后缩了一下,好像被人打了一巴掌似的。“对,对,当然罗,”他冷淡地随声应道,他的态度变得急躁起来,带有一种自我防护性质的对立情绪。“但不管怎么说,我希望你能做这一类的梦,也好让我看看你如何反应。今天就谈到这里吧。还有,我问你的那些问题,我希望你能梦见它们的答案。你知道,这些谈话对我和对你一样不愉快。”

    “我会把这个说给邓巴听的,”约塞连说。

    “邓巴?”

    “这一切都是他开的头。是他做的梦。”

    “噢,是邓巴,”桑德森少校冷笑道。他的自信心又恢复了。“我敢肯定,邓巴就是那个干了那么多下流事却总是让你替他受过的坏家伙,是不是?”

    “他没有那么坏。”

    “你到死也护着他,是不是?”

    “倒是没达到那种程度。”

    桑德森少校嘲讽地笑着,把“邓巴”两字写在他的拍纸簿上。

    “你怎么一瘸一拐的?”约塞连朝门口走时他厉声问道,“你腿上究竟为什么要缠着绷带?你是疯了还是怎么的?”

    “我的腿受了伤,就是为了这个我才住院的。”

    “噢,不,你没受伤。”桑德森少校幸灾乐祸地盯着他,目光中充满了恶意。“你是因为唾液腺结石才住院的。说到底,你还是不够聪明,对吧?你甚至不知道自己是为什么住院的。”

    “我是因为腿伤才住院的,”约塞连坚持道。

    桑德森少校发出一声嘲笑,不再理会他的辩解。“好吧,请代我问候你的朋友邓巴,并请告诉他为我做一个那样的梦,行吗?”

    但是,邓巴由于经常性的头痛而感到恶心和晕眩,无心跟桑德森少校合作。亨格利-乔倒是常做噩梦,因为他已经完成了六十次飞行任务,又在等着回家呢。可是,当他到医院里来时,他坚决不肯跟任何人谈论他的梦。
 


    “难道就没有人为桑德森少校做过什么梦吗?”约塞连问,“我真的不想让他失望,他本来就已经感到被人抛弃了。”

    “自从听说你受伤后,我一直在做一个非常奇特的梦,”牧师坦白说,“我从前每天夜里不是梦见我老婆要咽气,或者被人害死,就是梦见我孩子被一小口营养食品给噎死了。最近我梦见我在没顶的深水里游泳,一条鲨鱼正在咬我的腿,咬的部位和你缠绷带的地方正相同。”

    “这是个美妙的梦,”邓巴大声宣布,“我敢打赌,桑德森少校肯定会爱上这个梦的。”

    “这是个可怕的梦!”桑德森少校叫道,“里面全是些痛苦、伤残和死亡。我敢肯定,你做这个梦就是为了惹我生气。你竟然做出这种可恶的梦来,我真的说不准你该不该留在美国军队里。”

    约塞连认为自己看到了一线希望。“也许你是对的,长官,”他狡猾地暗示道,“也许我应该停飞,回到美国去。”

    “难道你从来都没有想到过,你不加选择地乱追女人,不过是为了缓解你下意识里对性无能的恐惧吗?”

    “是的,长官,想到过。”

    “那你为什么还要这样做呢?”

    “为了缓解我对性无能的恐惧。”

    “你为什么不能给自己另找一项有益的业余爱好呢?”桑德森少校友好而关切地问道,“比方说,钓鱼。你真的觉得达克特护士有那么大的吸引力?我倒认为她太瘦了,相当乏味,相当瘦,你明白吗?像条鱼。”

    “我几乎不了解达克特护士。”

    “那你为什么抱住她的胸脯呢?仅仅因为她有个胸脯吗?”

    “那是邓巴干的。”

    “喂,别又来这一套,”桑德森少校嘲弄地叫道,话音十分尖刻。

    他厌恶地把笔猛地往下一摔。“你真的认为假装成另一个人就能开脱掉自己的罪责吗?我不喜欢你,福尔蒂奥里。你知道这一点吗?

    我一点也不喜欢你。”

    约塞连感到一阵冰冷潮湿的恐慌风一般穿胸而过。“我不是福尔蒂奥里,长官,”他战战兢兢地说,“我是约塞连。”

    “你是谁?”

    “我的姓是约塞连,长官,我是因为一条腿受了伤而住院的。”

    “你的姓是福尔蒂奥里,”桑德森少校挑衅地反驳道,“你是因为唾液腺结石而住院的。”

    “喂,得啦,少校!”约塞连火了。“我应该知道我是谁。”

    “我这儿有一份军方的正式记录可以证明这一点,”桑德森少校反唇相讥道,“你最好趁着还来得及赶快抓住你自己。起先你是邓巴,现在你是约塞连,下回你也许会声称你是华盛顿-欧文了。

    你知道你得了什么病吗?你得的是精神分裂症,这就是你的病。”

    “也许你是对的,长官,”约塞连圆滑地赞同道。

    “我知道我是对的。你有一种严重的迫害情结,你以为大家都想害你。”

    “大家是都想害我。”

    “你瞧见了吧?你既不尊重极度的权威,又不尊重旧式的传统。

    你是危险的,是堕落的,应当把你拉到外面去枪毙!”

    “你这话当真吗?”

    “你是人民的敌人!”

    “你是疯子吗?”约塞连叫喊起来。

    “不,我不是疯子。”多布斯在病房里怒吼着答话,他还以为自己不过是在偷偷摸摸地耳语呢。“我告诉你吧,亨格利-乔看见他们了。他是昨天飞往那不勒斯去给卡思卡特上校的农场装运黑市空调器的时候看见他们的。他们那儿有一个很大的人员补充中心,里面住满了正预备回国的几百个飞行员、轰炸手和机枪手。他们完成了四十五次飞行任务,只有四十五次。有几个戴紫心勋章的人完成的次数还要少。从国内来的补充机组人员一批接一批地到达,全都补充到别的轰炸机大队去了。他们要求每个人至少在海外服役一次,行政人员也是这样。你难道没读报纸吗?我们应该马上杀了他!”

    “你只要再飞两次就完成任务了。”约塞连低声劝解他。“为什么要冒这个险呢?”

    “只飞两次也有可能被打死,”多布斯摆出一副寻衅闹事的架势回答道。他的嗓音嘶哑颤抖,显得很紧张。“明天早上我们干的第一件事就是趁他从农场开车回来时杀掉他。我这儿有枝手枪。”

    约塞连吃了一惊,瞪大眼睛看着多布斯从衣袋里抽出手枪来,高高地举在空中摇晃着。“你疯了吗?”约塞连惊惶失措地低声制止他。“快收起来,把你那白痴嗓门放低点。”

    “你担什么心?”多布斯傻乎乎地问,他有点不高兴了。“没有人会听见我们。”

    “喂,你们那边说话小点声。”一个声音远远地从病房那一头传过来。“你们难道没看见我们正想睡午觉吗?”

    “你他妈算什么人,你这个自高自大的家伙!”多布斯高声回敬道。他猛地转过身去,握紧拳头,摆出一副打架的姿势。接着他又扭转身对着约塞连,还没来得及说话,就一连打了六个响雷般的喷嚏。每打完一个喷嚏,他都要左右晃动着他那橡胶般柔韧的双腿,徒劳地抬起胳膊肘想把下一个喷嚏挡回去。他的眼睛水汪汪的,眼睑又红又肿。“他以为他是谁,”他质问道。他一边抽抽搭搭地用鼻子吸气,一边用粗壮的手腕背揩着鼻子。“他是警察还是什么人?”

    “他是刑事调查部的人,”约塞连平静地告诉他,“我们这儿眼下有三个这样的人,还有更多的人正要来呢。嗨,别给吓住了。他们是来找一个名叫华盛顿-欧文的伪造犯的。他们对谋杀犯不感兴趣。”

    “谋杀犯?”多布斯觉得受到了侮辱。“你为什么把我们叫做谋杀犯?就是因为我们打算杀掉卡思卡特上校吗?”

    “闭嘴,你这该死的!”约塞连喝道,“你就不能小点声说话吗?”

    “我是在小声说话呢。我——”

    “你仍然在大声嚷嚷呢。”

    “不,我没有。我——”

    “嗨,闭上你的嘴,行不行?”病房里所有的病人都朝着多布斯叫喊起来。

    “我跟你们这帮家伙拼了!”多布斯冲着他们尖叫道。他站到一把摇摇晃晃的木椅子上,疯狂地挥舞着他的手枪。约塞连抓住他的胳膊,使劲把他揪下来。多布斯又开始打喷嚏。“我有过敏症,”打完喷嚏后他抱歉地说。他的鼻涕直流,泪水盈眶。

    “这太糟了,要是没有这毛病,你满可以成为一个伟大的领袖人物。”

    “卡思卡特上校才是谋杀犯呢。”多布斯嗓音嘶哑地发着牢骚,把一条又脏又皱的土黄色手帕塞到口袋里。“就是他想要害死我们大家,我们必须想办法制止他。”

    “也许他不会再增加飞行任务的次数了,也许他最多就增加到六十次。”

    “他一直在增加飞行任务的次数,这你比我知道得更清楚。”多布斯咽了口唾沫,俯下身去,几乎把脸贴到了约塞连的脸上。他的脸绷得紧紧的,石头块般的古铜色腮帮子上鼓起一个个微微颤抖的肉疙瘩。“你只要说声行,明天早上我就把这件事全办好了。我跟你说的话你明白吗?我现在可是在小声说话,对不对?”

    多布斯紧紧盯住约塞连,目光中饱含着热切的恳求。约塞连好不容易才把自己的目光移开。“你他妈的干吗不出去干了这件事?”

    他顶撞道,“你为什么非得对我说不行,你自己一个人干不就得了?”

    “我一个人不敢干。不论什么事,我都不敢一个人干。”

    “那么,别把我扯进去。我现在要是搀和到这种事情当中去,那可是傻透了。我腿上的这个伤口值一百万美元呢。他们就要把我送回国去了。”

    “你疯了吗?”多布斯不相信地叫起来。“你那腿上不过擦破点皮。你只要一出院,他马上就会安排你参加战斗飞行,哪怕你得了紫心勋章什么的也得参加。”

    “到那时候我会真的杀了他的,”约塞连咬牙切齿地说,“我会去找你一块干的。”

    “趁着现在有个机会咱们明天就干了吧,”多布斯恳求道,“牧师说卡思卡特上校又去主动请战了,要求派咱们轰炸大队去轰击阿维尼翁。也许你还没出院我就被打死了。瞧瞧,我这双手直打颤,我不能开飞机了,我不行了。”

    约塞连不敢答应他。“我想再等一等,先看看会发生什么事情。”

    “你的毛病就是你什么都不愿意干。”多布斯给惹火了,粗声粗气地发作起来。

    “我正在尽我的最大努力呢,”多布斯离开后,牧师向约塞连解释道,“我甚至到医务室找丹尼卡医生谈过,叫他想法帮帮你。”

    “是的,我明白。”约塞连强忍住笑。“结果怎么样?”

    “他们往我的牙龈上涂了紫药水。”牧师不好意思地说。

    “他们还往他的脚趾头上涂了紫药水。”内特利愤愤地加上一句。“然后他们又给他开了轻泻剂。”

    “可我今天早上又去见了他一次。”

    “他们又往他的牙龈上涂了紫药水。”

    “可我到底还是对他讲了,”牧师用自我辩解的悲哀语调争辩道,“丹尼卡医生是个忧郁的人,他怀疑有人正在策划着把他调到太平洋战区去。这些日子,他一直想来求我帮忙。当我告诉他,我需要他帮忙时,他感到很奇怪,怎么就没有一个可以让我去见见的牧师呢?”约塞连和邓巴放声大笑,牧师则垂头丧气而又耐心地等着他们笑个够。“我原来一直以为忧郁是不道德的,”他继续说下去,好像是一个人在独自大声哭泣似的。“现在我也不知道该怎样看待这个问题了。我想把不道德作为我这个礼拜天的布道主题。可是我拿不准我该不该带着涂了一层紫药水的牙龈去布道。科恩中校非常讨厌涂着紫药水的牙龈。”

    “牧师,你为什么不到医院来跟我们一块住上一阵散散心呢?”

    约塞连怂恿地说,“你在这儿会非常舒服的。”

    有那么一会儿,这个轻率的馊点子曾引起了牧师的兴趣。“不,我想这不行。”他犹豫地作出了决定。“我打算到大陆去一趟,去找一个叫温特格林的邮件收发兵。丹尼卡医生告诉我,他能帮忙。”

    “温特格林大概是整个战区最有影响的人物了。他不仅仅是个邮件收发兵,他还有机会使用一台油印机。但是他不愿意帮任何人的忙,这正是他成功的原因之一。”

    “无论如何,我还是想跟他谈谈。总会有一个愿意帮你忙的人。”

    “找个人帮帮邓巴吧,牧师,”约塞连态度傲慢地纠正他说,“我腿上这个值百万美元的伤口会帮我离开战场的。再不然的话,还有位精神病专家认为我不适合留在军队里呢。”

    “我才是那个不适合留在军队里的人呢,”邓巴嫉妒地嘟囔着,“那是我的梦。”

    “不是因为梦,邓巴,”约塞连解释说,“他挺喜欢你的梦。是因为我的精神。他认为我的精神分裂了。”

    “你的精神正好从中间一分两半,”桑德森少校说。为了这次谈话,他把他那双笨重的步兵军鞋的鞋带系得整整齐齐,又用粘糊糊的芳香发油把他那木炭般乌黑的头发抹得光溜溜的。他假惺惺地笑着,装出一副通情达理有教养的样子。“我这么说并不是为了折磨你,侮辱你,”他带着折磨人、侮辱人的得意神情继续说,“我这么说也不是因为我恨你,想报复你,我这么说更不是因为你拒绝了我的建议,深深地伤害了我的感情。不,我是个医务工作者,我是冷静客观的。我有一个非常坏的消息要告诉你。你有足够的勇气听我说吗?”

    “上帝啊,千万别说!”约塞连叫道,“我马上就会崩溃的。”

    桑德森少校顿时大怒。“你就不能认认真真地做一件事吗?”他恳求道。他气得涨红了脸,两只拳头一起朝桌面捶去。“你的毛病在于你自以为了不起,什么社会习俗都不遵守。你大概也瞧不起我吧,我不就是青春期来得迟一点嘛。好吧,你知道你是什么东西吗。

    你是个屡遭挫折、倒霉透顶、灰心丧气、目无法纪、适应不良的毛孩子!”桑德森少校放连珠炮似他说出这一长串贬意词之后,火气似乎逐渐平息下来了。

    “是的,长官,”约塞连小心翼翼地附和道,“我想您是对的。”

    “我当然是对的。你还不成熟,还不能适应战争的观念。”

    “是的,长官。”

    “你对死有一种病态的反感,对打仗随时可能掉脑袋这一实际情况,你大概也心怀怨恨吧。”

    “岂止是怨恨,长官,我满腔怒火。”

    “你的生存欲望根深蒂固。你不喜欢固执已见的人,也不喜欢恶棍、势利小人和伪君子。你下意识地恨许多人。”

    “是有意识地,长官,”约塞连帮着纠正道,“我是有意识地恨他们的。”

    “一想到被剥夺、被剥削、被贬低、受侮辱和受欺骗这种种现象,你就愤愤不平。痛苦使你感到压抑,无知使你感到压抑,迫害使你感到压抑,罪恶使你感到压抑,腐化使你感到压抑。你知道吗,你要不是个抑郁症患者,那我才会感到吃惊呢!”

    “是的,长官,也许我是的。”

    “你别想抵赖。”

    “我没抵赖,长官,”约塞连说。他很高兴,他们俩之间终于达到了这种奇迹般的和睦关系。“我同意你所说的一切。”

    “那么,你承认你疯了,是吗?”

    “我疯了?”约塞连大为震惊。“你在说什么呀?我为什么要疯呢,你才疯了呢?”

    桑德森少校又一次气得涨红了脸,两只拳头一起朝大腿上捶去。“你竟敢骂我疯了,”他气急败坏地大声嚷道,“你这是典型的施虐狂、报复狂、偏执狂的反应!你真的疯了!”

    “那你为什么不把我打发回国去呢?”

    “我是要打发你回国去的!”

    “他们要打发我回国去啦!”约塞连一瘸一拐地走回病房,兴高采烈地宣布了这个消息。

    “我也要回国了!”安-福尔蒂奥里高兴地说,“他们刚才到病房里来告诉我的。”

    “那我怎么办?”邓巴气愤地质问医生们。

    “你吗?”他们粗暴地回答道,“你和约塞连一块走,马上回到战斗岗位上去!”

    于是,他们俩都回到战斗岗位上去了。一辆救护车把约塞连送回到中队。他怒气冲冲,一瘸一拐地去找丹尼卡医生评理。丹尼卡一脸愁容,痛苦而轻蔑地盯着他。

    “你!”丹尼卡医生悲哀地大声训斥他。他一脸厌恶的表情,连两只眼睛下面的蛋形眼袋都显得严厉而苛刻。“你只想着你自己。

    你要是想知道自从你住院之后发生了什么事情,就到那条轰炸线那儿去看看吧。”

    约塞连吃惊地问:“我们输了吗?”

    “输了?”丹尼卡医生叫道,“自从我们攻占巴黎以后,整个军事形势变得糟糕透顶。”他停顿了一会,一腔怒火渐渐变成了忧愁烦恼。他烦躁地皱起眉头,好像这一切全是约塞连的错误似的。“美国军队正在德国人的土地上向前推进,俄国人已经夺回了整个罗马尼亚。就在昨天,第八军团的希腊部队攻占了里米尼。德国人正在四面挨打!”丹尼卡医生又停顿了一下,深深地吸了一口气,憋足劲,突然发出一声痛苦的尖叫。“德国空军完蛋了!”他呜咽道,泪水似乎马上就要夺眶而出。“哥特人的整条战线一触即溃!”

    “怎么啦?”约塞连问,“这有什么不好吗?”

    “这有什么不好吗?”丹尼卡医生叫了起来。“如果不会很快出现什么新情况的话,德国人就可能投降。我们这些人全都会被派到太平洋去!”

    约塞连吓了一跳。他怪模怪样地傻盯着丹尼卡医生问:“你疯了吗?你知道你在说什么吗?”

    “嘿,你就可以放心大笑了,”丹尼卡医生讥讽道。

    “谁他妈的笑了?”

    “至少你还有活的机会。你是在参加战斗,有可能被打死。可我怎么办?我一点指望都没有了。”

    “你这该死的家伙真的神经失常了!”约塞连一把揪住他的衬衫领子,使劲冲他嚷道,“你知道什么?现在,闭上你的笨嘴,听我说。”

    丹尼卡医生猛地挣脱开来。“你怎么敢这样对我说话。我是个有开业执照的医生。”

    “那么,闭上你这个有开业执照的医生的笨嘴,听听他们在医院里对我说些什么吧。我疯了,你知道吗?”

    “那又怎么样?”

    “我真的疯了。”

    “那又怎么样?”

    “我是个神经病,是个疯子,你懂不懂?我神经失常了。他们错把另一个人当成我,把那个人打发回国了。他们医院里有一个有开业执照的精神病专家,他给我做了检查,这就是他的诊断结果。我真的疯了。”

    “那又怎么样?”

    “那又怎么样?”约塞连不明白为什么丹尼卡医生理解不了这一点。“你难道不明白这意味着什么吗?现在,你可以把我从战斗岗位上撤下来,打发我回国。他们不会派一个疯子飞出去送死,对不对?”

    “那么还有谁愿意飞出去呢?”
