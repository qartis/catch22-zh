\chapter{沙伊斯科普夫少尉}
 
    七分钱一只买进的鸡蛋,又以每只五分钱的价格售出,最终还赚了钱,米洛何以能做到这一点,就连万事通克莱文杰也犯了难。

    有关战争的一切,克莱文杰了如指掌,惟独一事他不甚明白:为何一旦斯纳克下士可以活下去,约塞连就非死不可,抑或,为何一旦约塞连可以活下去,斯纳克下士便只有死路一条。这是一场卑鄙肮脏的战争。假定没有这场战争,约塞连是本可以活下去的——或许能长寿。他的同胞中,只有极少数人甘愿为赢得这场战争的胜利而捐躯,至于约塞连自己,他实在是没有这个奢望成为其中的一分子。是死还是生,这是需要深思的问题,而克莱文杰倒是越发懒得回答这个问题了。历史并没有要求约塞连英年早逝;没有他的早逝,正义同样会得到伸张;无论是人类的进步,抑或是战争的胜败,都不取决于这一点。凡人皆难免一死,这是必然的事;但,哪些人该死,却全在天命。无论怎么个死法,约塞连都心甘情愿,但他就是不甘做天命的牺牲品。然而,这是战争。依他看,付出了巨大的血的代价,同时又把孩子们从父母有害的影响中解救出来,这便是这场战争唯一的可取之处。

    克莱文杰之所以通晓那么多事,是因为他是个天才。他心跳剧烈,脸色苍白。尽管长得瘦长难看,可他浑身是劲,两眼射出渴求的光芒,是个聪明绝顶的人。当年在哈佛上学时,他差不多所有科目都得过学术奖,至于另外几门功课没得奖,唯一的原因是,他实在太忙了:既要在请愿书上签名,又要分发请愿书,还得就请愿书内容提出质疑;一会儿参加小组讨论,一会儿又退了出来;不是参加青年代表大会,就是替别的青年代表大会担任纠察,或是组织学生委员会,保护被开除的教员。克莱文杰日后必定在学术界大有作为,这是大家一致公认的。说到底,克莱文杰属于那种聪颖绝顶却全无智谋的人。这一点谁都知道,而那些过不多久才会发现这一点的人,是不会明白的。

    总而言之,克莱文杰是个傻子。在约塞连眼里,他往往就跟那些整日在现代博物馆门前东荡西逛的人一样,两只眼睛都长在一张脸的同一侧。这自然是一种错觉,而这种错觉则完全是因克莱文杰本人而起,因为他偏好死盯着问题的一面,一向忽视其另一面。

    政治上,他是一个人道主义者,很能识别左翼和右翼,却又极不自在地夹在两者之间。他时常当着右翼敌人的面,替左翼朋友辩护;

 


    又当着左翼敌人的面,替右翼朋友辩护。可是,无论是左翼还是右翼,都对他深恶痛绝,从来就不愿在任何人面前替他辩护,因为,在他们看来,他实在是个傻子。

    不过,他是个极严肃认真且专心一意的傻子。假如同他去看一场电影,散场后他非缠住你不可,同你讨论什么移情啦,什么亚里士多德啦,什么全称命题啦,什么寓意啦,还有作为艺术形式的电影在物质第一的社会中应尽的责任,等等。他每次带女孩子上剧院看戏,总得让人家等到第一次幕间休息,才肯说出看的戏是好是坏,而且用不着她们多费口舌,他就一下子和盘托出。此外,他还是一个战斗性颇强的理想主义者,投身于消灭种族歧视的斗争,其斗争方式是,凡遇到这种事例,他便当即昏厥。他于文学颇是精通,却不懂得怎么欣赏。

    约塞连曾设法开导他。“别做傻子啦。”他这样劝过克莱文杰。

    当时,他俩还在加利福尼亚州圣安娜的一所军校学习。

    “我去跟他说。”克莱文杰一再坚持。当时,他和约塞连正高高地坐在检阅台上,俯视辅助阅兵场上的沙伊斯科普夫少尉——活像没长胡须的李尔,正怒气冲冲地来回走动。

    “干吗是我?”沙伊斯科普夫少尉悲叹道。

    “别作声,傻瓜。”约塞连长辈似地劝说克菜文杰。

    “你不知道自己在说什么。”克莱文杰很是反感。

    “我当然知道,所以才不作声的,傻瓜。”

    沙伊斯科普夫少尉咬牙切齿地撕扯着自己的头发;橡胶似的两颊因阵阵极度的痛苦而不时地颤动。令他如此苦恼的是,一中队航空学校学员士气消沉,在每周日下午举标的阅兵比赛中;表现极其恶劣。他们之所以士气消沉,一是因为他们讨厌每周日下午列队接受检阅,二是因为沙伊斯科普夫少尉不允许他们选自己的学员军官,而是由他从他们中间任命。

    “我希望有人当面跟我说。”沙伊斯科普夫少尉极诚恳地请求全体学员。“假如我有什么过错,我希望你们直接跟我说。”

    “他希望有人当面跟他说,”克莱文杰说。

    “他是希望谁都不要吭气,傻爪,”约塞连回答说。

    “难道你没听见他说?”克莱文杰反驳道。

    “当然听见,”约塞连答道,“我听见他说得很响,很清楚,假如我们知道什么对我们有利,他希望我们每个人都把嘴闭起来。”

    “我决不惩罚你们,”沙伊斯科普夫少尉向全体学员保证道。

    “他说他不会惩罚我的。”克莱文杰说。

    “他会阉割了你。”约塞连说。

    “我保证决不惩罚你们,”沙伊斯科普夫少尉说,“谁要是跟我说了实话,我一定会很感激的。”

    “他会恨你的,”约塞连说,“到死都会恨你。”

    沙伊斯科普夫少尉是后备军官训练队的毕业生。战争的爆发,于他颇是桩喜事,因为这一来,他便有机会天天穿上军官制服、冲着一群群小伙子——上战场送命之前,每八周便有一批落入他的手掌,以军人特有的清脆快速的嗓音,喊道:“弟兄们!”沙伊斯科普夫少尉极有野心,一向不苟言笑,从来都是极谨慎持重地面对自己的职责。只有当圣安娜陆军航空基地某个与他对立的军官,染上了什么缠绵的疾病,他才会露一丝笑容。他视力极差,又患有慢性瘘管病,然而,这反倒让他觉得战争格外刺激,因为他不可能去海外作战,也就没有了丝毫的危险。沙伊斯科普夫少尉唯一令人满意之处是他的太太,而他太太最让人称心的,是有一个名叫多丽-达兹的女友。多丽-达兹只要有机会,便要与人风流快活。她有一套陆军妇女队的制服,沙伊斯科普夫少尉的太太一到周未,便穿上这套制服;假如一到周未,她丈夫中队里的学员,无论是谁,想跟她上床,她便会为他脱了这套制服。

    多丽-达兹是个活泼的浪荡少女,紫铜色的皮肤,金黄色的头发。工具房、公用电话亭、更衣室和公共汽车候车亭,都是她最喜欢的做爱场所。几乎没什么事她不曾尝试过,而她不愿尝试的事则更是少有。她年方十九,体形苗条,却淫荡不羁,不知羞耻。不少男人让她给弄得全无了自尊心,到了早晨便憎恶自己,因为她揭破了他们的真面目,利用了他们,却又把他们弃置一旁。约塞连倒是挺爱她。作为性交对象,她实在是个绝妙的女人,不过,依她看,约塞连也就如此而已。多丽-达兹只让约塞连碰过她一次,她浑身上下的肌肤极富弹性,那种感觉着实令约塞连爱不释手。约塞连很爱多丽-达兹,因此,他总是控制不住自己,每个星期必定会感情热烈地扑到沙伊斯科普夫少尉的太太身上,以此报复沙伊斯科普夫少尉,就像沙伊斯科普夫少尉报复克莱文杰一样。

 


    沙伊斯科普夫少尉曾造下一桩难忘的孽,他太太倒是记不得了,不过,她还是为此在报复自己的丈夫。她丰满、肌肤白皙、不好动,喜读好书,又不时地力劝约塞连,不要太庸俗,连书都不读。她自己手边从来是少不了一本好书的,即便赤条条躺在床上,身上只有约塞连及多丽-达兹的身份识别牌时,也不例外。她让约塞连感到厌倦,可他也照样爱上了她。她毕业于沃顿商业学校,主修的是数学,可笨得出奇,每个月竟连二十八都数不清。

    “亲爱的,我们再生个孩子吧,”她月月都这么跟约塞连说。

    “你在说胡话吧,”他总这么回答。

    “我可是当真的,宝贝,”她坚持说。

    “我也一样。”

    “亲爱的,我们再生个孩子吧,”她常跟自己的丈夫说。

    “我没时间,”沙伊斯科普夫少尉老是没好气地咕哝道,“难道你不知道在进行阅兵吗?”

    沙伊斯科普夫少尉最为关心的,是如何在阅兵比赛中获胜,如何把克莱文杰送至裁定委员会,指控他密谋打倒由他任命的学员军官。克莱文杰专爱闹事,又自命不凡。沙伊斯科普夫少尉知道,假如对他不小心防范,这家伙很有可能闹出更大的乱子来。昨天是想阴谋打倒学员军官,明天或许企图颠覆整个世界。克莱文杰颇有头脑,而沙伊斯科普夫少尉发现,凡是有头脑的人往往相当精明。这种人很危险,就连那些由克莱文杰扶掖的新上任的学员军官,也急不可耐地想出来作证,指控克莱文杰,欲置他于死地。指控克莱文杰一案,显然是成立的。唯一缺少的,就是以什么罪控告他。

    但无论如何不能牵涉阅兵比赛,因为克莱文杰几乎同沙伊斯科普夫少尉本人一样,极为重视那些阅兵比赛。每周日下午,学员们早早便出来参加阅兵比赛,摸索着在营房外排成十二人一列的队伍。于是,他们宿酒未醒地哼唧着,深一脚浅一脚地走向大阅兵场各就各位。然后,他们就和其他六七十支中队的学员纹丝不动地站在烈日下,一站便是一两个小时,直到不少学员支持不住晕倒在地,队伍才被解散。阅兵场边上,停放了一排救护车,还站着一队队担架兵,他们手持步话机,个个训练有素。救护车车顶上,是手持望远镜的观察员。一名记分员负责记录比分。这一阶段比赛的全过程,由一名精通会计的军医负责监督。每分钟脉搏跳多少次可视作晕厥,必须得到军医的认可,记分员记录的比分,也必须经他核实。
 


    一旦救护车载满了昏迷的学员,军医便示意乐队指挥开始奏乐,结束比赛。于是,所有中队一个紧跟着一个,向前走去,绕检阅台拐个大弯,退出阅兵场,返回各自的营房。

    所有参加检阅的中队齐步走过检阅台时,都被打了分。检阅台上,坐着一名上校——留着两撇又浓又粗的八字须,摆出一副狂妄自大的尊容——和其他几位军官。各联队的最佳中队得一面插上旗杆的黄色锦旗——实在是毫无用处。基地的最佳中队则获一面红色锦旗,旗杆略长一些——更是没什么价值,因为旗杆的分量重了,下周日由其他中队夺走之前,足足一个星期他们必须得扛东扛西,实在很是令人头疼。在约塞连看来,以锦旗代奖品是颇有些滑稽可笑的。锦旗不代表金钱,也不代表等级特权。它们就跟奥林匹克运动会奖章和网球赛奖杯一样,仅仅表明,获奖者做了一桩于谁都无甚益处的事情,只不过比任何别的人做得出色罢了。

    阅兵比赛这件事本身看来也同样滑稽可笑。约塞连讨厌受人检阅。阅兵大过军事化。他讨厌听到有关阅兵的消息;讨厌看到阅兵的场面,讨厌让接受检阅的队伍给困在半途,动身不得;也讨厌被迫参加阅兵活动。当一名航空学校学员已经是触尽了楣头,每星期天下午还得跟士兵一样,在炎炎的赤日下接受检阅。当一名航空学校学员确实是桩相当倒霉的事,因为现在看来,军训结束之前,战争显然是打不完的。而约塞连之所以自愿报名进航空学校接受训练,唯一的原因就是他以前一直以为,战争必定先他的军校训练而结束。约塞连作为一名大兵,早具备了条件进航空学校接受训练,但得等上若干星期,才会被选派到某个班:再等上若干星期,便做一名轰炸领航员;之后,又得接受若干星期的作战训练,为执行海外任务做准备。当时,似乎根本就想不到,战争竟会打那么长时间。有人曾跟他说,上帝和他站在一边;有人还跟他说,上帝无事不成。可是,战争根本就没个结局,而他的训练倒是差不多近了尾声。

    沙伊斯科普夫少尉一心想在阅兵比赛中获胜,于是,熬了大半个晚上、琢磨来琢磨去。他妻子躺在床上,含情脉脉地企盼着他,一边迅速翻阅克拉夫特-埃宾的书,找自己最爱读的章节。沙伊斯科普夫看的则是有关行进方面的书。他拿了一盒盒小兵巧克力糖摆弄来摆弄去,直到所有的巧克力糖都化在了他的手里,于是,又取出一套塑料牧童,极熟练地把它们排成若干十二人一列的队伍。

    这套塑料玩具是他以化名从一家邮购商店买来的,为了不让人看见,白天他总是把它锁藏起来。列奥纳多的解剖练习原来也是不可或缺的。一天晚上,他觉得少了个活模特儿,于是,就命令夫人在房里飞步行走。

    “光着身走吗?”她满怀希望地问道。

    沙伊斯科普夫少尉极为恼怒,两手啪地捂住了眼睛。他太太只晓得满足自己肮脏的肉欲,根本就无法理解高尚的人为实现无法达到的目标所做出的艰苦卓绝的伟大斗争。

    “你到底为啥不跟我做爱?”一天晚上,她撅着嘴问。

    “因为我没时间,”他很是不耐烦,冲着她厉声说道,“我没那工夫。难道你不知道在进行阅兵比赛吗?”

    他确实没时间。又到星期天了,只有七天的时间为下一次阅兵比赛做准备。他实在不明白,时间究竟是怎么过的。接连三次比赛,沙伊斯科普夫少尉的中队都是最后一名,搞得他名声极坏。为了改进目前的这种状况,他考虑了各种办法,甚至想到用一根长长的二英寸厚、四英寸宽且风干了的栎木桁,把每列的十二人一直线钉在上面。显然,这是行不通的,因为假如用这种办法,就必须在每个人的腰背部嵌入一个镍合金旋转轴承,不然,他们就无法作九十度转体。再说,能否从军需主任那里要到那么多镍合金旋转轴承,或者,能否争取医院外科医生的合作,对此,沙伊斯科普夫少尉实在没有丝毫把握。

    沙伊斯科普夫少尉采纳了克莱文杰的建议,让学员们选出了他们自己的学员军官。随后的那个星期,这个中队便夺得了那面黄色锦旗。这突如其来的胜利,让沙伊斯科普夫少尉心花怒放。当他妻子想拖他上床庆贺——以此表示他们蔑视西方文明中中产阶级下层的性风俗——时,他竟抡起旗杆,对着她的脑袋狠狠地打了下去。又过一个星期,中队夺得了那面红色锦旗。沙伊斯科普夫少尉简直是欣喜若狂。之后的又一个星期,他的中队创下了历史记录,连续两个星期夺得红色锦旗。现在,沙伊斯科普夫少尉坚信自己有能力一鸣惊人。经过广泛的研究,他发现,行进时,两只手不应像时下流行的那样自由摆动,而应该自始至终与大腿正中保持不超过三英寸的摆距,其实也就是说,两手几乎就不用摆动。
 


    沙伊斯科普夫少尉的准备工作周详充分,且又相当秘密。中队全体学员发誓保守秘密。夜深人静的时候,他们就在辅助阅兵场上进行演习。他们在漆黑的夜晚里行进,漫无目的地彼此瞎撞,但他们并不惊慌。他们是在练习不摆动双手行进。起初,沙伊斯科普夫少尉倒是考虑过让金属薄板店的一位朋友把镍合金钉嵌入每个学员的股骨,然后,再用恰好三英寸长的铜丝把钉子和手腕接起来,可是,时间来不及——时间老是不够用——再说,战争期间实在不大容易搞到手。他还考虑到,假如学员们受了这样的束缚,那么,齐步行进前,参加令人肃然的检阅仪式时,万一晕厥,他们便不能以规范的姿势倒下去,而昏倒的姿势若不合乎规范,便有可能影响中队的团体总分。

    整整一个星期,沙伊斯科普夫少尉强压住内心的喜悦,每次到了军官俱乐部,总是咯咯地欢笑。他的密友中便开始有了种种的猜测。

    “真不知那白痴在搞什么鬼,”恩格尔中尉说。

    每逢同事提问时,沙伊斯科普夫少尉总是会意地一笑。“到了星期日你们就会知道的。”他向大伙儿保证。“你们会知道的。”

    那个星期日,沙伊斯科普夫少尉以一名经验丰富的乐队指挥所特有的沉着自信,向公众揭露了他的划时代的惊人秘密。他一声不吭地目睹着其他中队用惯常的轻松步伐,从容却颇别扭地走过检阅台。即便当自己中队的前几排学员手臂一动不动地齐步走入视线,先是让他那些受惊的同僚个个吁吁地倒抽气,直为他担心,沙伊斯科普夫少尉依旧镇定得很。就是在那种时候,他也还是声色不露。后来,那名留了粗浓八字须的傲气十足的上校,猛地转过身来,恶狠狠地对着他,脸色铁青,这时,他才作出了解释——致使他名垂千古的解释。

    “您瞧,上校,”他说,“不用动手。”

    随后,他把自己那套费解的行进规则——他取得这令人难忘的成功,便是以此作为基础——的直接影印件,散发给了在场的观众——惊愕得鸦雀无声。这可是沙伊斯科普夫少尉生平最荣耀的时刻。他取得了阅兵比赛的胜利,自然是轻而易举的,从此便永久保持了那面红色锦旗,也就彻底结束了每星期日必定举行的阅兵比赛,因为优质的红色绵旗和优质铜丝一样,在战时都是极难到手的。沙伊斯科普夫少尉当即晋升为中尉,自此,便平步青云。因为他的重大发现,差不多每个人都把他视为真正的军事天才。

    “那个沙伊斯科普夫中尉,”特拉弗斯中尉说,“他可是个军事天才。”

    “没错,的确是个天才。”恩格尔中尉表示赞同。“可惜的是,这蠢驴不愿鞭打自己的老婆。”

    “我看不出这两者之间有什么关系,”特拉弗斯中尉很冷淡他说,“比米斯中尉每次跟太太做爱,总要狠狠地给她一顿鞭打,可在阅兵比赛中,他却是一点都不中用。”

    “我说的是鞭打自己的老婆,”恩格尔中尉反驳道,“谁在乎什么阅兵比赛?”

    说实话,除沙伊斯科普夫中尉之外,根本就没人真把阅兵比赛这事放在心上,那个留两撇浓粗八字须的上校更不用说了。这家伙是裁定委员会主席,克莱文杰刚战战兢兢地跨进委员会办公室,准备替自己申辩,不承认沙伊斯科普夫中尉对他提出的指控,他便对着他大声咆哮。上校握着拳头,猛击桌面,反倒痛了自己的手,于是,对克莱文杰更是暴怒,再又狠狠地捶了一下桌子,这次使的劲更猛,手也因此就更痛得厉害。克莱文杰留下了极坏的印象,这很让沙伊斯科普夫中尉丢脸,他恶狠狠地朝克莱文杰直瞪眼。

    “再过六十天,你就要跟意大利人打仗了,”留着粗浓八字胡的上校大声吼道,“可你还以为这是个天大的玩笑呢。”

    “我没这么想,长官,”克莱文杰答道。

    “别插嘴。”

    “是,长官。”

    “说话时得叫一声‘长官’,”梅特卡夫少校下令道。

    “是,长官。”

    “刚才不是让你别插嘴吗?”梅特卡夫少校冷冷地问了一句。

    “可是我没插嘴,长官,”克莱文杰抗辩道。

    “不错,你没插嘴,但你也没叫一声‘长官’。对他的指控加上这一条。”梅特卡夫少校命令那个会速记的下士。“尽管没有打断上级军官的说话,但没能向他们报告一声‘长官’。”

    “梅特卡夫,”上校说,“你真是头讨厌的蠢驴。你自己知道吗?”

    梅特卡夫少校好不容易把这口怨气咽了下去。“知道,长官。”

    “那就闭上你那张该死的嘴。老是胡说八道。”

    裁定委员会由三人组成,他们是,留着粗浓八字胡的傲气十足的上校,沙伊斯科普夫中尉和梅特卡夫少校。梅特卡夫少校正设法用冷冰冰的目光来审视别人。沙伊斯科普夫中尉身为裁定委员会的一名成员,同时也是其中的一个法官,必须对起诉人控告克莱文杰一案的是非曲直,进行认真的考虑。而沙伊斯科普夫中尉本人又是起诉人。克莱文杰有一名军官替他辩护,那个军官便是沙伊斯科普夫中尉。

    这一切把克莱文杰弄得实在是稀里糊涂。当上校猛地跳起身——酷似放肆地大声打嗝,扬言要肢解他那具散发恶臭的卑怯的躯体时,克莱文杰害怕得浑身直打战。一天,在列队齐步走去上课途中,克莱文杰绊了一跤。第二天,他便正式受到指控:“编队行进时打乱队形、行凶殴打、行为失检、吊儿郎当、叛国、煽动闹事、自作聪明、听古典音乐,等等。”一句话,他们一古脑儿把各种罪名加到他身上,于是,他便来到了裁定委员会,胆战心惊地站在这位傲气十足的上校跟前。上校又一次大声吼着,说再过六十天,他就要去跟意大利人打仗了,接着又问他,假如开除他,送他去所罗门群岛埋尸体,他究竟是否愿意。克莱文杰极是恭敬地回答说,他不愿意;他是个笨蛋,宁愿是一具尸体,也不甘埋一具尸体。上校坐了下去,身体往后一靠,态度一下子镇静了下来,变得谨小慎微,且又献殷勤一般地客气了起来。

    “你说我们不能惩罚你,这是什么意思?”上校慢悠悠地问道。

    “我什么时候说过这话,长官?”

    “是我在问你,你回答。”

    “是,长官。我——”

    “你以为我们带你来这里,是请你提问题,叫我来回答吗?”

    “不是的,长官。我一”“我们干吗带你来这儿?”

    “让我回答问题。”

    “你说得千真万确,”上校大声吼道,“好,你就先回答几个问题吧,免得我砸了你的狗头。你说我们不能惩罚你,你这狗杂种,究竟是什么意思?”

    “我想我从来就没有说过这样的话,长官。”

    “请你说得响一些,行不行?我听不见你的话。”

    “是,长官。我——”

    “梅特卡夫?”

    “什么事,长官?”

    “我刚才不是让你闭上你那张笨嘴吗?”

    “是,长官。”

    “我让你闭上你那张笨嘴,你就给我闭起来。明白没有,请你说得响一些,好不好?我听不见你的话。”

    “是,长官。我——”

    “梅特卡夫,是不是我踩了你的脚?”

    “不是,长官。一定是沙伊斯科普夫中尉的脚。”

    “不是我的脚,”沙伊斯科普夫中尉说。

    “那或许还是我的脚吧,”梅特卡夫少校说。

    “挪开点。”

    “是,长官。您得先把您的脚挪开,上校。您的脚踩在了我的脚上面。”

    “你让我把我的脚挪开?”

    “不是,长官。嗬,不是,长官。”

    “那就把你的脚挪开,然后,闭上你那张笨嘴。请你说响一些,好吗?我听不见你说的话。”

    “是,长官。我说了,我没说你们不能惩罚我。”

    “你到底在说什么?”

    “我在回答您的问题,长官?”

    “什么问题?”

    “‘你说我们不能惩罚你,你这狗杂种,究竟是什么意思?’”那个会速记的下士看着速记本读了一遍。

    “没错,”上校说,“你说这话究竟是什么意思?”

    “我没说你们不能惩罚我,长官。”

    “什么时候?”上校问。

    “什么什么时候,长官?”

    “嗨,你又在向我提问了。”

    “对不起,长官。恐怕我没听懂您提的问题。”

    “你什么时候没说过我们不能惩罚你?我的问题难道你听不懂?”

    “不懂,长官。我听不懂。”

    “你才跟我们说过。好,你就回答我的问题吧。”

    “可是这个问题我该怎么答呢?”

    “你这又是在问我一个问题了。”

    “对不起,长官。可我实在是不知道该怎么回答您的问题。我绝对没说过你们不能惩罚我。”

    “现在你告诉我们,你什么时候的确说过这话。我是在请你告诉我们,你什么时候没说过这话。”

    克莱文杰深吸了一口气。“我一直就没说过你们不能惩罚我,长官。”

    “这样回答可是好多了,克莱文杰先生,尽管你是在当面撒谎。

    昨天晚上在厕所里。难道你没悄声跟我们讨厌的另一个狗杂种说过,我们不能惩罚你吗?那家伙叫什么来着?”

    “约塞连,长官。”沙伊斯科普夫中尉说。

    “没错,是约塞连。一点没错。约塞连。约塞连?他是叫约塞连吗?约塞连究竟算是什么样的名字?”

    对所有的实情,沙伊斯科普夫中尉可是了如指掌。“这是约塞连的名字,长官。”他给上校作了解释。

    “没错,我猜想是这么回事儿。难道你私下没跟约塞连说,我们不能惩罚你?”

    “嗬,没有,长官。我私下跟他说过,你们不能裁决我有罪——”

    “或许我很笨。”上校打断了他的话。“不过,我怎么也看不出这两句话究竟有什么不同。我想我确实很笨,因为我怎么也看不出这两句话究竟有什么不同。”

    “我——”

    “你是个喜欢信口开河的狗杂种,是不是?没人请你作解释,你倒先跟我辩白起来了。我只是在说说自己的想法,不是请你作什么解释。你这杂种,就喜欢信口开河,是不是?”

    “不是,长官。”

    “不是,长官?你的意思是我在说谎咯?”

    “嗬,不是,长官。”

    “那么说,你是个喜欢信口开河的狗杂种,是不是?”

    “不是,长官。”

    “你是存心想跟我吵架咯?”

    “不是,长官。”

    “你是个喜欢信口开河的狗杂种,是不是?”

    “不是,长官。”

    “你他妈的,存心想跟我吵架。谁要是肯出两分臭钱,我就从这张大桌子上跳过去,把你那发恶臭的、卑怯的身体撕碎。”

    “太棒啦!太棒啦!”梅特卡夫少校大声叫道。

    “梅特卡夫,你这讨厌的狗杂种。我不是让你闭上你那张懦怯愚蠢的臭嘴吗?”

    “是,长官。对不起,长官。”

    “那你就给我闭嘴。”

    “我只是想试着学习学习,长官。一个人只有通过尝试,才有可能学到些东西。”

    “是谁这么说的?”

    “大伙儿都这么说,长官。就连沙伊斯科普夫中尉也这么说,”“你是这么说的吗?”

    “是的,长官,”沙伊斯科普夫中尉说,“不过,大伙儿都是这么说的。”

    “好吧,梅特卡夫,你就试试闭上你那张笨嘴。这或许是让你学会闭嘴的一个好办法。哎,我们刚才说到哪儿了?把最后一行记录再念给我听听。”

    “‘把最后一行记录再念给我听听。’”会速记的下士照本念了一遍。

    “没让你念我说的最后一句话,蠢货!”上校大叫道,“念别的最后那句话。”

    “‘把最后一行记录再念给我听听。’”下士念了一遍。

    “你念的还是我说的最后那句话!”上校气得脸色铁青,尖声叫道。

    “哦,不,长官,”下士纠正道,“那是我记下的最后一句话。我刚才给您念过了。难道您忘了,长官?就是刚才。”

    “哦,天哪!把他的最后一句话念给我听听,蠢货。哎,你究竟叫什么名字?”

    “波平杰,长官。”

    “好吧,下一个就该你了,波平杰。他一审讯完,就开始审问你。

    听到没有?”

    “听到了,长官。我犯了什么罪?”

    “那有什么两样?你们听见他问我的话吗?你会明白的,波平杰——我们一结束克莱文杰的审讯,你就会明白的。克莱文杰学员,你刚才——你是军校学员克莱文杰,不是波平杰,是不是?

    “我是克莱文杰,长官。”

    “很好。刚才——”

    “我是波平杰,长官。”

    “波平杰,你父亲是百万富翁,还是参议员?”

    “都不是,长官。”

    “这么说来,你的境遇相当糟糕罗,波平杰,连个靠山都没有。

    你父亲不是将军,也不是政府高级官员,是不是?”

    “不是,长官。”

    “很好。你父亲是干什么的?”

    “他早死了,长官。”

    “那实在是好极了。你的境遇的确很糟糕,波平杰。你真的是叫波平杰?波平杰究竟是什么样的名字?我很不喜欢这个名字。”

    “这是波平杰的名字,长官,”沙伊斯科普夫中尉解释道。

    “嗯,不过,我不喜欢这个名字,波平杰。我恨不得现在就肢解了你发恶臭的、卑怯的身体。克莱文杰学员,请你把昨天深夜你在厕所里悄悄对约塞连说过或者没说过的话,再重复一遍,行吗?”

    “是,长官。我说你们不能裁决我有罪——”

    “我们就从这儿接着问下去。克莱文杰学员,你说我们不能裁决你有罪,到底是什么意思?”

    “我没说过你们不能裁决我有罪,长官。”

    “什么时候?”

    “什么什么时候,长官?”

    “你他妈的,是不是又要追问我起来了?”

    “不是,长官。对不起,长官。”

    “那就回答我刚才的问题。你什么时候没说过我们不能裁决你有罪?”

    “昨天深夜在厕所里,长官。”

    “就只有这一次你没说过那句活?”

    “不是,长官。我一直就没说过你们不能裁决我有罪,长官。我真正对约塞连说的是——”

    “没人问你你真正对约塞连说的是什么。我们问你的是,你没跟他说的是什么。至于你真正对约塞连说些什么,我们一点都不感兴趣。明白了吗?”

    “明白了,长官。”

    “那么我们继续问下去。你跟约塞连说了些什么?”

    “我跟他说,长官,你们不能裁决我犯了你们指控我的那条罪行,同时还忠于——事业。”

    “什么事业?你说话含含糊糊的。”

    “说话别含含糊糊的。”

    “是,长官。”

    “含含糊糊说话时,也得含含糊糊地叫一声‘长官’。”

    “梅特卡夫,你这狗娘养的。”

    “是,长官,”克莱文杰含糊地说,“是正义事业,长官。你们不能裁决——”

    “正义?”上校很是愕然。“什么是正义?”

    “正义,长官——”

    “那可不是正义,”上校讥笑道,一边说一边又用粗壮的大手膨膨地擂桌子。“那是卡尔-马克思。我来告诉你什么是正义。正义就是半夜里从地板上用膝盖顶着别人的肚皮用手按着别人的下巴手里拿着一把刀偷偷摸摸地摸到一艘战列舰的弹药舱里事先不给任何警告在黑暗中秘密地用沙袋把别人打昏。正义就是勒杀抢劫。一旦我们大家都得残酷无情地去跟意大利人打仗,那就是正义。要凶残。懂吗?”

    “不懂,长官。”

    “别老是长官长官地叫我!”

    “是,长官。”

    “不叫‘长官’时,也得喊一声‘长官’,”梅待卡夫少校命令道。

    克莱文杰自然是有罪的,要不然他就不会受指控了。要想裁决他有罪,唯一的办法就是得证明他的确犯了罪,而裁决克莱文杰有罪,则是上校一帮人必须尽到的爱国义务。于是,克莱文杰被判了五十六次惩罚性值勤。波平杰则被禁闭了起来,以此作为对他的教训。梅特卡夫少校被运送到所罗门群岛,负责埋尸体。至于克莱文杰,所谓惩罚性值勤,就是每到周未,肩背一支沉重的没装子弹的步枪,在宪兵司令大楼前来回走上五十分钟。

    这一切都把克莱文杰搞得稀里糊涂。出了许多稀奇古怪的事情,可在克莱文杰看来,最怪的是裁定委员会三个人流露出的那种仇恨——那种赤裸裸的残酷无情的仇恨。那仇恨就像是不能扑灭的煤块,在三双眯缝了的眼睛里恶狠狠地燃烧着,又使他们本来便已凶险的面目,更添了冷酷蛮横的气势。克莱文杰察觉到了这种仇恨,简直惊呆了。假如可能,他们会用私刑把他处死。他们三个都是成年人,可他自己却还是小伙子。他们仇恨他,恨不得他快死。在他来军校之前,他们就仇恨他;他在军校时,他们也仇恨他;他离开军校后,他们还是仇恨他。日后,他们三个人分了手,都过上了独居的生活,但却还是恶狠狠地带走了对克莱文杰的仇恨,仿佛带走的是什么稀世珍宝。

    头天晚上,约塞连就好好地给了克莱文杰一番告诫。“你是不会有什么希望的,”他很愁闷地跟克莱文杰说,“他们仇恨犹太人。”

    “可我又不是犹大人,”克莱文杰回答说。

    “这没什么两样,”约塞连说,而约塞连的确没有说错。“他们是不会放过任何一个人的。”

    克莱文杰躲开了他们的仇恨,就像是避开耀眼的亮光一样。这三个仇视他的人,跟他说同一种语言,穿同样的制服,但他见到的这三张冷冰冰的脸,却自始至终密布着令人极不舒适且又深含敌意的皱纹。他顿时觉悟了:这世上随便什么地方,无论是在所有法西斯的坦克或飞机或潜艇里,还是在机关枪或迫击炮或吐着火焰的喷火器后面的掩体里,甚至在精锐的赫尔曼-戈林高射炮师的所有神炮手当中,或是在慕尼黑所有啤酒馆里的那些恐怖的密谋分子中间,以及任何别的地方,再也不会有谁比他们三个人更仇恨他了
