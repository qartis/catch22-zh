\chapter{佩克姆}
 
    第二天仍然没有奥尔的消息。惠特科姆下士迫不及待地在他的备忘夹里做了一个记号,满怀希望地等着九天过后给奥尔的亲属寄上一封由卡思卡特上校签名的通函。然而,佩克姆将军的司令部发布了一张告示,就贴在传达室外面的告示栏里。一群穿着短裤和游泳裤的军官和士兵围在告示前,吵吵嚷嚷地发牢骚,闹得乱哄哄的,约塞连也给吸引了过去。

    “我倒想知道这个星期天有什么特别?”亨格利-乔正大叫大嚷地质问一级准尉怀特-哈尔福特。“既然我们并不是每一个星期天都举行阅兵,那为什么这一个星期天就不能举行一次呢?嗯?”

    约塞连费了好大的劲才挤到告示栏前,他读了一遍那张简短扼要的告示,不禁发出一声痛苦的长叹。那告示是这样写的:

    由于我无法控制的情况,本星期天下午将不举行大阅兵。

    沙伊斯科普夫上校

    多布斯是对的。他们的确正在把国内的每个人派到海外,就连沙伊斯科普夫上校也不例外。他曾经绞尽脑汁竭尽全力反对这一调动,结果还是不得不带着强烈的不满情绪到佩克姆将军的办公室报到就职。

    佩克姆将军热情洋溢地欢迎了沙伊斯科普夫上校。他说,上校能到他这儿来工作真叫他高兴。在他的司令部班子里新增加一名上校就意味着他现在可以向上级要求再增加两名少校、四名上尉、十六名中尉和许许多多的士兵、打字机、办公桌、档案柜、汽车以及大量的装备给养。所有这些将会大大提高他的地位和声望,增强他在这场针对德里德尔将军的战争中的攻击能力。目前,他有两名上校了,而德里德尔将军只有五名上校,且其中四名是战地指挥官。

    佩克姆将军略施小计就成功地实施了一项将会使他的实力增加一倍的策略,而且,德里德尔将军喝醉酒的次数越来越多了。看来,前途十分美妙。佩克姆将军满脸堆笑,上下打量着这位新来的生气勃勃的上校,越看越喜欢。

 


    佩克姆将军准备公开批评他身边某个下属的工作时,常常发议论说自己在所有重大问题上都是一个现实主义者。佩克姆将军现年五十三岁,皮肤红润,相貌堂堂。他一向从容潇洒,极有风度;

    他总是身着制作考究的制服,一头银发,轻微近视的眼睛,两片向外突出的肉感的薄嘴唇,佩克姆将军是个感觉敏锐、斯文大方、稳重老练的人。他对任何人的缺点都十分敏感,对他自己的缺点却视而不见;他觉得所有人都愚蠢透顶,只有他自己是个例外。佩克姆将军尤其重视情趣和仪表,在这类小事情上十分挑剔。他用词总喜欢夸张。谈到快要发生的事件时,他从来不说正在来临,而总是用即将来临这个词,如果说他写了许多报告,在上面自吹自擂,并要求把他的权力扩展到能涵盖所有的作战行动,那是不真实的,他写的那些东西叫呈文,其他军官的呈文总是写得夸张、做作、含糊其辞。别人的错误从来都是可悲可叹的。规章制度则是不容通融的。

    他的资料从来都不是有可靠出处,却总是源自可靠出处。佩克姆将军常常迫于无奈,许多任务常常义不容辞地落到他的肩上,他行动起来常常是万分勉强,他永远记得黑和白都不是颜色,当地想表达口述这个意思时,他绝不用口头这个词,他善于引用柏拉图、尼采、蒙田、西奥多。罗斯福、萨德侯爵和沃伦-加-哈定的名言。一个像沙伊斯科普夫这样思想单纯的听众对佩克姆将军再合适不过了。他的到来使将军兴奋不已,因为他给将军提供了一个大展身手的机会。将军可以向他打开自己那令人眼花燎乱的知识宝手,尽情地运用双关语、俏皮活、诽谤、说教、轶事、谚语、警句、格言、隽语以及其它尖酸刻薄的俗语。佩克姆将军彬彬有礼地微笑着,着手帮助沙伊斯科普夫上校适应新环境。

    “我唯一的缺点,”他以他那种长期练就的诙谐口吻说道,同时密切注意着自己这句话的效果。“就是我没有缺点。”

 


    沙伊斯科普夫上校一点没笑,佩克姆将军不禁大吃一惊。深深的疑虑一下子打消了他的热情。他刚一说出这个他最拿手的悖论,就惊恐地注意到对方那张毫无表情的脸上没有流露出任何反应。

    这张脸的皮肤和肌理突然使他联想起一把没有用过的肥皂擦子。

    佩克姆将军宽容地想,沙伊斯科普夫上校也许是累了,他千里迢迢才来到这里,而这里的一切又都是那么陌生。对他手下的所有人员,无论是军官还是士兵,佩克姆将军的态度一向是随和、宽容、忍让的。他常说,如果为他工作的人迎合他的活,他将会更加主动地迎合他们。并且,他总是狡猾地笑着补充道,这样做的结果就是大家彼此间永远都不会做到心心相印。佩克姆将军认为自己是个美学家,是个知识分子。每当别人与他发生意见分歧时,他总是劝告他们要客观一些。

    此时,这位非常客观的佩克姆将军用鼓励的目光盯着沙伊斯科普夫上校,以一种宽容大度的态度继续对他进行教导。“你到我们这儿来得正是时候,沙伊斯科普夫。由于我们部队中指挥人员的无能,夏季攻势已告瓦解。我眼下急需一位像你这样肯吃苦、有经验、有能力的军官来帮我写呈文。这些呈文对我们非常重要,它们将告诉大家我们干得如何出色、我们做了多少工作。我希望你是个高产的文书。”

    “我对文书工作一窍不通,”沙伊斯科普夫闷闷不乐地回答道。

    “好吧,别为这件事烦恼了,”佩克姆将军随便地甩了甩手腕继续说,“去把我派给你的任务转派给别的人,看你的运气怎么样吧。

    我们把这叫做分工负责。在我掌管的这个协作机构中,在较下层的部门里,倒是有一些来了任务就认真完成的人,那些地方的工作样样都进行得很顺利,不需要我操多少心。我想,这是因为我是个优秀的行政官员。在我们这个大部门里,我们所干的工作实际上全都不怎么重要,也不需要赶任务。另一方面,重要的是我们要让人家知道我们做了大量的工作。你要是发现自己缺人手就告诉我。我已经正式提出申请,要求增加两名少校、四名上尉和十六名中尉来给你帮忙。我们做的工作全都不怎么重要,但重要的是我们做了大量的工作。你同意吗?”

    “阅兵的事怎么说?”沙伊斯科普夫上校插嘴问道。

    “什么阅兵?”佩克姆将军问,他感到自己的潇洒风度对这位上校一点不起作用。:=>“我可不可以每星期天下午主持一次阅兵?”沙伊斯科普夫上校气哼哼地问。

    “不可以,当然不可以。你怎么会有这个念头的?”

    “但他们说我可以的。”

    “谁说你可以?”

    “派我来海外的军官。他们告诉我,我只要愿意,就可以指挥部队进行阅兵。”

    “他们对你说谎。”

    “这不公平,长官。”

    “我很遗憾,沙伊斯科普夫。我愿意尽我所能使你在这里感到愉快,可是阅兵一事是不可能的。我们司令部本身人员不足,没法举行阅兵。要是我们让战斗部队参加阅兵,他们就会起来公开造反。你这件事恐怕得搁一搁,等我们控制住局面再说。到那时你想叫部队干什么就干什么。”、“那我的太太怎么办?”沙伊斯科普夫上校怀疑地问,他看上去非常不满意。“我仍然可以把她接来,对不对?”

    “你的太太?你为什么非把她接来不可呢?”

    “丈夫和妻子应该呆在一起。”

    “这件事也不可能。”

    “可他们说我可以把她接来。”

    “他们又对你说谎了。”

    “他们没有权利对我说谎!”沙伊斯科普夫上校抗议道。他气得眼泪都要流出来了。

    “他们当然有权利,”佩克姆将军厉声说道。他决定当场用批评指责来考验一下他这位新上校的勇气,于是故意摆出一副冷峻严厉的样子。“你别做傻瓜了,沙伊斯科普夫。人们有权利做任何不违犯法律的事情。而法律又没有规定不准对你说谎。听着,别再用你这些伤感的陈词滥调来浪费我的时间了。你听见了吗?”

    “听见了,长官,”沙伊斯科普夫上校唯唯诺诺地答道。

    沙伊斯科普夫上校垂头丧气,一副可怜相。佩克姆将军暗暗感谢上天给他派来这么一个懦弱的下属。如果派来的是个胆量十足的男子汉,后果就难以想象了。佩克姆将军制服了沙伊斯科普夫上校,又转而可怜起他来。他并不喜欢令他的手下人难堪。“如果你的太太是陆军妇女队队员,我也许可以把她调到这里来。不过,我只能帮这一点忙。”

    “她有个朋友是陆军妇女队队员,”沙伊斯科普夫上校满怀希望地建议道。

    “这恐怕还不够。要是沙伊斯科普夫太太愿意,就让她参加陆军妇女队吧,那样我就可以把她调到这儿来。不过现在,我亲爱的上校,如果可以的话,我们还是回到我们小小的战争上来吧。简单地说,这儿是我们目前所面临的军事形势。”佩克姆将军站起身,朝挂在旋转支架上的巨幅彩色地图走过去。

    沙伊斯科普夫顿时脸色苍白。“我们不会去打仗吧。”他惊恐万分地脱口问道。

    “噢,不,当然不,”佩克姆将军友好而宽容地笑着向他保证道,“相信我的话,好吗?这就是我们至今仍然驻扎在罗马的原因。当然,我也很想到佛罗伦萨去,在那儿我可以跟前一等兵温特格林保持更紧密的联系。但是,佛罗伦萨离实战区域太近了点,不适合我。”佩克姆将军兴致勃勃地举起一根木制指示棒,用它的橡皮头从意大利的一侧海岸划向另一侧海岸。“沙伊斯科普夫,这些就是德国人。他们在这些山里挖筑了坚固的哥特防线,估计明年夏天以前是赶不走他们的。当然,我们派去的那些乡巴佬会不断地向他们发起进攻的。这样一来,我们特种任务兵团就有大约九个月的时间实现我们的目标。这个目标就是夺取美国空军中的全部轰炸机大队。说到底,”佩克姆将军有节奏地低声窃笑道,“要是往敌人的头上扔炸弹不算是特种任务的话,那世界上还有什么特种任务呢?你同意吗?”沙伊斯科普夫上校没有作出任何同意的表示。然而,佩克姆将军正沉浸在自己的长篇大论之中,根本没有去注意他。“我们目前的情况好极了。像你这样的增援力量正源源不断地到达,我们有充裕的时间精心制订我们的整体战略。我们的直接目标,”他说,“就在这儿。”佩克姆将军把他的指示棒向南部的皮亚诺萨岛一挥,意味深长地用橡皮头敲了敲用黑色油彩笔写在那儿的一个大字。

    那个字是德里德尔。”沙伊斯科普夫上校眯缝起眼睛,走到地图跟前。自从他走进这个房间以来,他那张愚钝的脸上第一次闪现出一丝领悟的光。“我想我明白了,”他叫道,“是的,我知道我明白了。我们的头一项任务就是把德里德尔从敌人那边俘虏过来,对吗?”
 


    佩克姆将军宽厚地笑了笑。“不,沙伊斯科普夫。德里德尔是我们这边的,但德里德尔是敌人。德里德尔将军指挥着四个轰炸机大队,我们只有把这四个轰炸机大队夺过来,才能继续我们的进攻。战胜德里德尔将军将会给我们提供我们所急需的飞机和重要基地,这样我们就可以把我们的攻击扩展到其它地区。顺便说一句,这场战斗,我们就要赢了。”佩克姆将军慢慢地走到窗前,又平静地笑了笑,双臂合抱在胸前,背靠窗台站定。他对自己的才智,对自己的见多识广和讲究实际,对自己的厚颜无耻感到洋洋自得。他讲话时遣词造句的高超本领实在令人赞叹不已,佩克姆将军喜欢听自己讲话,而且特别喜欢听自己讲自己。“德里德尔将军根本不知道如何对付我,”他幸灾乐祸地说,“我一直在越权议论批评他管辖范围内的事情,这些事情我本来根本不该管的,他却不知道该怎么办才好。当他指责我企图削弱他的力量时,我仅仅回答他说,我揭露他缺点的唯一目的就是要消灭不称职现象,增强我军的战斗力,接着,我直截了当地问他是不是反对增强我军的战斗力。嘿,他发牢骚,他发脾气,他狂吼乱叫,可他就是拿我毫无办法。他实在是落伍了。你知道吗,他变得越来越像个大傻瓜。这个可怜的傻瓜真不应该当将军的。他没有一点将军的风度,一点都没有。感谢上帝,他撑不了多久了。”佩克姆将军得意洋洋地窃笑着,随口引用了一个他特别喜爱的文学典故。“我有时把自己当成了福丁布拉斯——哈,哈——在威廉-莎士比亚的《哈姆莱特》中,他一直在剧情之外兜圈子,直到一切都土崩瓦解了,他才悠闲地走进来为自己捞取好处。莎士比亚是——”

    “我对戏剧一窍不通,”沙伊斯科普夫上校生硬地插嘴说道。

    佩克姆惊奇地望着他。以前他引用莎士比亚神圣的剧本《哈姆莱特》时,从来没有遭受到如此冷漠而粗暴的蔑视和凌辱。他不由得认真寻思起来,五角大楼硬塞给他的究竟是一个什么样的笨蛋。

    “那你到底知道些什么?”他讥讽地问道。

    “阅兵,”沙伊斯科普夫急切地答道,“我可以把阅兵报告发送出去吗?”

    “只要你不定下阅兵的具体时间就行,”佩克姆将军回到椅子上坐下来,眉头依然皱着。“只要准备这些报告不妨碍你的主要任务就行。你的主要任务是呈文建议把特种任务部队的权力扩大到指挥所有的战斗活动。”

    “我能不能先定下阅兵时间,然后再取消呢?”

    佩克姆将军顿时眉开眼笑,“嘿,这是个多么绝妙的主意!不过,根本不必费心去安排阅兵的时间,只要每星期发布一个延期阅兵的告示就行。要是把时间定下来,麻烦可就太多了。”佩克姆将军又一次迅速露出一个热诚的笑脸。“不错,沙伊斯科普夫,”他说,“我认为你的确出了个好点子。说到底,哪个战斗指挥官会因为我们通知他的士兵下星期天取消阅兵而来找我们大吵大闹呢?我们只不过是公布一个众所周知的事实罢了。但是,这其中的寓意妙极了,是的,真是妙极了。我们是在暗示,如果我们愿意的话,我们是能够安排一次阅兵的。我开始喜欢你了,沙伊斯科普夫。你去见见卡吉尔上校,告诉他你打算做些什么。我知道你们两个会互相喜欢上的。”

    一分钟之后,卡吉尔上校旋风般地冲进佩克姆将军的办公室。

    他满腔怨愤,却又不敢肆意发作。“我在这儿工作的时间比沙伊斯科普夫长,”他抱怨道,“为什么不能由我来取消阅兵呢?”

    “因为沙伊斯科普夫对阅兵有经验,而你没有。如果你愿意,你可以取消劳军联合组织的演出。实际上,你为什么不这样做呢?想想看,不论在哪儿,不论在什么时候,都不会有什么劳军联合组织的演出的。想想看,不论是哪儿,也不会有什么名演员愿意来的。是的,卡吉尔,我认为你的确出了个好点子。我认为你给我们开辟出了一个全新的活动领域。告诉沙伊斯科普夫上校,我叫他在你的指导下干这项工作。你给他作完指示之后,叫他来见我。”

    “卡吉尔上校说你告诉他叫我在他的指导下负责劳军联合组织的活动计划,”沙伊斯科普夫上校抱怨说。

    “我根本没对他这样说过,”佩克姆将军回答道,“沙伊斯科普夫,对你说句心里话吧,我对卡吉尔上校有点反感。他专横霸道,反应迟钝。我希望你密切注意他的一举一动,并且想办法把他手里的工作再多接过来一些。”

    “他总是跟我对着干,”卡吉尔上校抗议说,“他搅得我什么工作都干不成。”

    “沙伊斯科普夫确实有点滑稽可笑。”佩克姆将军若有所思地表示同意。“你要密切注意他,设法发现他在干些什么。”

    “哼,他老是来干涉我的事情!”沙伊斯科普夫上校叫嚷道。

    “别为这个担心,沙伊斯科普夫,”佩克姆将军说。他在心里暗暗庆幸,自己已经十分巧妙地引导沙伊斯科普夫上校适应了自己那种标准作战方法。现在,他的两个上校几乎已经互相不理睬了。

    “卡吉尔上校嫉妒你,因为你把阅兵这项工作干得十分出色。他担心我会把炸弹散布面这项工作交给你负责。”

    沙伊斯科普夫竖起耳朵听着。“什么炸弹散布面?”

    “炸弹散布面?”佩克姆将军自鸣得意地眨眨眼睛重复道,“炸弹散布面是我几星期前创造出来的一个术语。这术语没有什么意思,可奇怪的是它这么快就流行起来了。嘿,我已经使各种各样的人相信,我认为重要的是把炸弹密集地投向地面,然后从空中拍一张清晰的照片。在皮亚诺萨岛上有一个上校,他一点也不关心自己是否击中了目标。今天咱们就飞到那儿去跟他开个玩笑。卡吉尔上校会因此而嫉妒的。今天早上我从温特格林那儿打听到,德里德尔将军要去撒丁岛。等到他发现我趁着他外出视察他的一个基地时去检查了他的另一个基地,他准会气得发疯的。我们甚至来得及赶到那儿去听他们下达简令。他们要去轰炸一个小小的不设防的村庄,他们打算把整个村子炸成废墟。我是听温特格林说的——顺便告诉你,温特格林原先是个中士——这次任务完全没有必要。它唯一的目的不过是拖延德国人的增援,可眼下我们甚至还没有准备发动进攻呢。不过,当你让平庸的人登上权力高位,事情就会这样。”他朝着那边的巨幅意大利地图做了个懒洋洋的手势。“喏,这个小山村太无足轻重了,地图上甚至都没标出来。”

    他们到达卡思卡特上校的轰炸机大队时,已经太晚了。他们没能赶上下达预备性简令,也没能听到丹比少校所做的一遍遍的说服和解释。“可它就在这儿,我告诉你们,它就在这儿,它就在这儿。”

    “它在哪儿?”邓巴装作没有看见,挑衅地问。

    “它就在地图上这条路稍稍拐弯的地方。你难道看不见你地图上的那个小弯吗?”

    “不,我看不见。”

    “我能看见,”哈弗迈耶凑上前说。他在邓巴的地图上把那个地方标了出来。“这些照片中有一张是那个小村子,拍得很好。这个任务我已经完全清楚了。它的目的就是把整个村庄从山坡上炸坍下去,从而堆积起一个路障。德国人不清除这个路障就无法进兵。

    对不对?”

    “对极了,”丹比少校说。他用手帕擦拭着前额上的汗水。“我很高兴,我们这儿终于有人开始明白这一点了。德国人的两个装甲师将会沿着这条路从奥地利开进意大利。这个村庄坐落在非常陡的山坡上,你们炸毁的房子和其它建筑物的瓦砾肯定全会直接滚落下来堆积在路上。”

    “见鬼,这又能有什么区别呢?”邓巴追问道。约塞连激动地望着他,目光中既有敬畏也有谄媚。“只要两三天,他们就能清除干净。”叫丹比少校竭力避免引起争论。“不过,对司令部来说,这还是有些区别的,”他语气缓和地回答说,“我想这大概就是他们为什么要布置这次任务的原因。”

    “是不是已经把这次轰炸通知村里的人了?”麦克沃特问。

    丹比少校有点惊慌,连麦克沃特这样的人也敢站出来表示反对意见了。“不,我想还没有。”

    “我们是不是已经撒传单告诉他们这一回我们的飞机要去轰炸他们了?”约塞连问,“难道我们就不能向他们暗示一下,叫他们躲出去吗?”

    “不行,我看不行。”丹比少校不安地转动着眼珠,他的汗越出越多。“德国人也许会发现的,那样他们就会改变路线,对于这一我不敢肯定,我只不过是假设而已。”
 


    “他们甚至不会隐蔽起来,”约塞连愤愤不平地争辩说,“当他们看见我们的飞机飞过来时,他们会连小孩带老人还有狗一起涌上街头冲着飞机挥手。天哪,我们为什么不能放过他们呢?”

    “我们为什么不能在别处设置路障呢?”麦克沃特问,“为什么非在这儿不可呢?”

    “我不知道,”丹比少校不高兴地回答说,“我不知道。听着,弟兄们,我们对向我们下达命令的上级应该有信心。他们知道他们自己在干些什么。”

    “他们知道个鬼,”邓巴说。

    “出了什么麻烦事?”科恩中校问。他穿着一件棕黄色的宽松衫,双手插在口袋里,悠闲自得地踱进简令下达室。

    “噢,没出什么麻烦事,中校,”丹比少校神情紧张地掩饰道,“我们正在讨论这次任务呢。”

    “他们不想轰炸那个村庄,”哈弗迈耶窃笑着说。他把丹比少校给出卖了。

    “你这个混蛋!”约塞连冲着哈弗迈耶呵斥道。

    “你离哈弗迈耶远点。”科恩中校粗暴地命令约塞连。他认出来了,约塞连就是第一次飞往博洛尼亚执行任务的前一天晚上在军官俱乐部里对他出言不逊的那个醉汉。他压制着自己的不满,转向邓巴问道:“你们为什么不想去轰炸那个村庄呢?”

    “这太残忍了,就因为这个。”

    “残忍?”科恩中校语调冷淡地问。邓巴毫无顾忌发作出来的敌对情绪使他心头一震。“让德国人的两个师开过来打我们的部队不是同样残忍吗?你当然知道,美国人的生命也处在危险之中。你愿意看到美国人流血吗?”

    “美国人是在流血。可那村庄里的老百姓正生活在和平之中呢。我们究竟为什么要去找他们的麻烦呢?”

    “不错,你这样讲倒挺容易,”科恩中校讥笑道,“你呆在皮亚诺萨岛上当然是很安全的。那些德国人的增援部队来与不来对你都没有关系,是吗?”

    邓已窘得满脸通红。他突然以一种自我辩解的口吻反问道:

    “我们为什么不能在别处设置路障呢?我们就不能把哪座山的山坡炸坍下来或者直接去轰炸那条路吗?”

    “你是不是宁愿回博洛尼亚去呢?”这个问题虽然是平静地提出来的,却像一发子弹似的飞了出去。屋子里顿时静了下来,大家面面相觑,神色紧张,约塞连又急又愧,暗暗祈求邓巴不要再开口说话了,邓巴垂下了眼睛。科恩中校知道自己赢了。“不,我想你不愿意,”他带着露骨的轻蔑目光继续说道,“你知道吗,卡思卡特上校和我本人费了多大的力气才给你们争来这么一个没有危险的飞行任务?要是你们宁愿飞到博洛尼亚、斯培西亚和弗拉拉执行任务的话,我们不费吹灰之力就可以把这些目标派给你们。”他的眼睛在无框镜片后面威胁性地闪着光,宽大的下巴黑不溜秋的,显得冷酷无情。“只要告诉我一声就行。”

    “我愿意去,”哈弗迈耶急忙答应道,发出一阵自高自大的窃笑声。“我愿意直接飞到博洛尼亚上空,把脑袋平对着轰炸瞄准器,听着那些高射炮弹在我四周呼啸爆炸。等到我完成任务回来,人们围过来指责我,咒骂我时,我会感到格外地开心。甚至连那些当兵的也气得骂我,恨不得揍我一顿。”
 


    科恩中校愉快地拍了拍哈弗迈耶的下巴,却没有跟他说话。他转而干巴巴地对邓巴和约塞连说:“我郑重地告诉你们,说到为山上那些意大利乡巴佬伤心难过,谁也比不上卡思卡特上校和我本人。战争就是这个样子。你们一定要记住,发动战争的不是我们而是意大利人,侵略者不是我们而是意大利人。这些意大利人、德国人、俄国人,他们自己对待自己已经够残忍的啦,我们怎么残忍也比不过他们。”科恩中校友好地捏了捏丹比少校的肩膀,可是他脸上的不友好表情却没有改变。“继续下达简令吧,丹比。一定要让他们理解密集的炸弹散布面的重要性。”

    “不,不,中校,”丹比少校眨眨眼脱口说道,“这个目标不采用这种方式,我已经告诉他们,每颗炸弹的落点间距为六十英尺。这样一来,路障就不是只集中在一个地点而是和整个村庄一样长了。

    疏散的炸弹散布面会形成更有效的路障。”

    “我们关心的不是路障,”科恩中校开导他说,“卡思卡特上校想借这次任务拍出一张高清晰度的空中照片,这张照片他可以自豪地通过各种渠道散发出去。别忘了,佩克姆将军要来这里听取下达正式简令。他对炸弹散布面的看法如何,你是知道的。顺便说一句,趁他还没来,你最好抓紧时间布置完这些细节,赶快离开。佩克姆将军受不了你。”

    “噢,不,中校,”丹比少校诚恳地纠正他说,“是德里德尔将军受不了我。”

    “佩克姆将军也受不了你。事实上,谁都受不了你。把你正在讲的讲完,丹比,然后就走吧。我来主持下达简令。”

    “丹比少校在哪儿?”卡思卡特上校驾车陪着佩克姆将军和沙伊斯科普夫前来听取下达正式简令,一下车便问道。

    “他一看到你开车来了,就请假离开了,”科恩中校回答说,“他担心佩克姆将军不喜欢他。本来也是准备由我主持下达简令的。我会干得比他好得多。”

    “好极了!”卡思卡特上校叫道。可一转眼,他想起第一次下达轰炸阿维尼翁的简令时,科恩中校在德里德尔将军面前干的好事,便急忙收回刚才的话。“不,我自己来主持吧。”

    卡思卡特上校精神抖擞地站起来主持会议。他心里想着自己是德里德尔将军的一个心腹,便学着德里德尔将军的样子,摆出一副粗鲁直率强硬的架势,对着那些凝神静听的下级军官斩钉截铁地厉声训话。他觉得,自己敞开着衬衫领口,手握着烟嘴,加上那一头剪得短短的花白卷发,站在讲台上的样子一定很威风。他口若悬河,滔滔不绝,讲得妙极了,甚至把德里德尔将军特有的某几个不正确发音都模仿得维妙维肖。后来,他突然记起来,佩克姆将军很厌恶德里德尔将军,于是便对佩克姆将军手下这位新来的上校生出几分惧怕来。他的嗓音变得沙哑了。他的自信心一下子全没了。

    他结结巴巴地往下讲,不由得满面羞惭,脸红耳热。突然间,沙伊斯科普夫上校使他惊恐万分起来。这个地区多了一个上校就意味着多了一个对手,多了一个敌人,多了一个恨他的人。而且,这个家伙不好对付!卡思卡特上校忽然产生了一个可怕的念头:要是沙伊斯科普夫上校已经贿赂了这会场里所有的人,叫他们起来抱怨,就像他们第一次执行轰炸阿维尼翁的任务前那样,他怎么做才能使他们安静下来呢:那他可就丢尽脸了!卡思卡特上校吓得都快撑不住了,差一点招手叫科恩中校过来接替他。他费了好大劲才使自己镇定下来,和大家对了对手表。对完表,他知道自己总算应付过去了,因为他现在可以随时结束会议。他已经顺利地渡过了危机。他真想以胜利者的姿态当面嘲笑挖苦沙伊斯科普夫上校一通。事实证明,他在压力下表现得很出色。他以一番鼓舞人心的演说结束了简令的下达。他的直觉告诉他,这番演说淋漓尽致地展现了他的雄辩口才和机智敏锐。

    “喂,弟兄们,”他鼓动地叫道,“今天到场的有一位贵宾,这就是来自特种任务部队的佩克姆将军,他给我们带来了垒球的球棒。

    连环漫画和劳军联合组织的演出。我要用这次任务向他献礼。出发到那儿去扔炸弹吧——为了我,为了你们的国家,为了上帝,为了这位伟大的美国人佩克姆将军。让我们看到你们把所有的炸弹全部扔到那一丁点大的地方上去吧!”
