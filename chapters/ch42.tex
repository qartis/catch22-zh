\chapter{约塞连}
 
    “科恩中校说,”丹比少校既谨慎又满意地笑着告诉约塞连,“那笔交易仍然有效。一切都正在顺利进展之中。”

    “不,不是的。”

    “噢,是的,的确是的,”丹比少校关切地坚持道,“事实上,一切都比以前好多了。你真是交了好运,差一点就叫那个女人给杀死了。现在,这笔交易可以顺利进行了。”

    “我没跟科恩中校做任何交易。”

    丹比少校兴致勃勃的乐观劲头突然间全没了,顿时冒出一身冷汗。“可你确实跟他谈过一笔交易,不是吗?”他苦恼而困惑地问道,“你们难道没有达成协议吗?”

    “我撕毁了协议。”

    “可你们达成协议时是握了手的,不是吗?你像个正人君子那样答应了他。”

    “现在我改主意了。”

    “哦,唉。”丹比少校叹了口气。他用一块折叠起来的白手帕徒劳无益地擦拭着他那忧郁的前额。“可为什么呢,约塞连?他们向你提出的是一笔很好的交易。”

    “是一笔卑鄙下流的交易,丹比。是一笔令人作呕的交易。”

    “哦,唉,”丹比少校烦躁地叹气道。他抬起一只光溜溜的手,抹了抹自己金属丝般的黑头发,他那一头又粗又短的卷发早已让汗水给浸透了。“哦,唉。”

    “丹比,你难道不认为这笔交易令人作呕吗?”

    丹比少校思索了一下。“是的,我是觉得它令人作呕,”他勉勉强强地承认道。他那双眼球突出的圆眼睛里流露出困惑不解的神情。“可既然你不喜欢,那又为什么要做这笔交易呢?”

    “我是一时软弱才这样做的,”约塞连阴郁地、嘲讽地打趣道,“我是想救自己的命。”

    “难道你现在就不想救自己的命了吗?”

    “正是为了这个,我才不让他们派我去执行更多的飞行任务。”

    “那么,让他们送你回国,你就不会再有任何危险了。”

    “我让他们送我回国,是因为我已经执行了五十次以上的飞行任务,”约塞连说,“并不是因为我被那个姑娘捅了一刀,也不是因为我变成了这么个顽固不化的狗杂种。”

    戴着眼镜的丹比少校使劲摇了摇头,一脸诚恳的苦恼神情。

    “那样一来,他们就不得不把几乎所有人送回国去。大多数人都已经执行了五十次以上的飞行任务。如果卡思卡特上校一下子要求增派这么多毫无经验的补充机组人员的话,上头不可能不派人来调查的:那样一来,他就掉进他自己设的陷阱里去了。”

    “那是他的问题。”

    “不,不不,约塞连,”丹比少校焦虑地反对道,“这是你的问题。

    因为;如果你不履行这笔交易的话,只要你办好手续出了医院,他们马上就会对你进行军法审判。”

    约塞连把大拇指搁在鼻尖上朝丹比少校做了个蔑视的手势,沾沾自喜;洋洋得意地哈哈一笑。“叫他们见鬼去吧:别骗我啦,丹比、他们根本不会这样做。”

    “可他们为什么不会?”丹比少校惊奇地眨着眼睛问道。

    “因为我眼下已经把他们握在手心里了。有份官方报告说,我是被一个前来暗杀他们的纳粹刺客刺伤的。在这种情况下,他们要是再对我进行军法审判的话,那不是出他们自己的洋相嘛。”

    “可是,约塞连!”丹比少校叫道,“还有另一份官方报告说,你是在从事黑市交易时被一个单纯的姑娘刺伤的。那上面说,你参与的黑市交易范围广泛,你甚至还卷入了破坏活动以及向敌方出售军事秘密的勾当。”

    约塞连不由得大吃一惊,又是诧异又是失望,“另一份官方报告?”

    “约塞连,他们想准备多少份官方报告就可以准备多少份,这样一来,在任何一种特定情况下,他们需要哪人份就可以选用哪一份;这儿点你难道不知道吗?”

    “哦,唉,”约塞连垂头丧气地嘟哝着,脸上一点血色都没有了。

    “哦,唉。”

    丹比少校露出一副出于好意的急切神情,热心地劝说者他。

    “约塞连,他们叫你做什么你就做什么,让他们送你回国吧,这样做对每个人都有好处。”

    “是对卡思卡特、科恩和我有好处,并不是对每个人。”

    “是对每个人。”丹比少校坚持道,“这样做整个问题全都解决了。”

    “对大队里那些将不得不执行更多飞行任务的人也有好处吗?”

    丹比少校畏缩了一下,不安地把脸转过去了一会儿。“约塞连,”他回答道,“如果你逼得卡思卡特上校对你进行军法审判,并证明你犯有他们指控你的所有罪行的话,那对任何人都没有好处,你会坐很长一段时间牢的,你的一生就全给毁了。”

    约塞连越往下听心里越着急。“他们会指控我犯了什么罪呢?”

    “在弗拉拉上空作战失利;违抗上级,拒绝执行与敌方交战的命令,以及开小差等等。”

    约塞连严肃地吸了吸两颊,“他们能指控我犯了这么一大堆罪状吗?在弗拉拉的那场空战后,他们还发给我一枚勋章呢。现在他们又怎么能够指控我作战失利呢?”

    “阿费将宣誓作证,说你和麦克沃特在你们给上级的报告中说了假话。”

    “我敢打赌,那个杂种准会这么干的。”

    “他们还将证明你犯有下列罪行,”丹比少校一件一件地列举着,“强xx,参与范围广泛的黑市交易,从事破坏活动,以及向敌方出售军事秘密等等。”

    “他们将如何证明这些呢?这些事情我一样也没有干过。”

    “可是他们手里有证人,那些人会宣誓作证说你干过。他们只需说服人家相信,除掉你对国家有好处,就可以找到他们所需要的全部证人。从某一方面说,除掉你对国家会有好处的。”

    “从哪方面呢?”约塞连追问道。他强压住心头的敌意,用一只胳膊肘撑着慢慢抬起身子来。

    丹比少校往后缩了缩身体,又擦拭起额头来。“唉,约塞连,”他结结巴巴地争辩道,“在目前这个时候,把卡思卡特上校和科恩中校搞得声名狼藉,对我们的作战行动是没有好处的。让我们面对现实,约塞连——不管怎么说,我们大队的战绩确实出色。如果对你进行军法审判而最后又证实你无罪的话,其他人很可能也会拒绝执行更多的飞行任务,卡思卡特上校就会当众丢脸,部队的作战能力也许就全部丧失了。所以,从这方面讲,证明你有罪并把你关进监狱,对国家是会有好处的,即使你没罪也得这样做。”

    “你把事情说得多么动听啊!”约塞连刻薄而怨恨地厉声说道。

    丹比少校的脸红了。他局促不安地扭动着身体,不敢正眼看约塞连。“请不要怪我,”他带着焦虑而诚恳的神情恳求道,“你也知道这不是我的过错。我现在所做的不过是试图客观地看问题,并且找出办法来解决一个极为困难的局面。”

    “这个局面又不是我造成的。”

    “可你能够解决它。要不你还能干些什么呢?你又不愿意执行更多的飞行任务。”

    “我可以逃走。”

    “逃走?”

    “开小差,溜之大吉。我可以甩开眼前这个乌七八糟的局面,掉头就跑。”

    丹比少校大吃一惊。“往哪儿跑?你能去哪儿呢?”

    “我可以轻而易举地跑到罗马去,在那儿藏起来。”

    “那样你的生命就无时无刻不处在危险之中,他们随时会找到你的。不,不,不,不,约塞连。那样做是卑鄙可耻的,会带来灾难。

    逃避问题是永远解决不了问题的。请相信我,我是想尽力帮助你的。”

    “那个好心的密探把大拇指戳进我的伤口之前就是这么说的,”约塞连嘲讽地反驳道。

    “我不是密探,”丹比少校愤怒地回答道。他的双颊又涨红了。

    “我是个大学教授,我具有极强的是非感,我决不会欺骗你,也决不会对任何人撒谎。”

    “要是大队里有谁向你问起我们的这次谈话,那你怎么办?”

    “那我就对他撒个谎。”

    约塞连嘲讽地大笑起来。丹比少校虽然面红耳赤,浑身不自在,却也松了口气,靠坐到椅背上。约塞连情绪上的变化预示着短暂的缓和气氛的出现,这似乎正是丹比少校希望看见的,约塞连凝视着丹比少校,神情中既流露出淡淡的怜悯又包含着轻蔑。他背靠着床头坐了起来,点燃一支香烟,露出一副苦中取乐的神情微笑着,怀着一种奇特的同情盯着丹比少校的脸。自从执行轰炸阿维尼翁的任务那一天德里德尔将军下令把丹比少校拖出去枪毙时起,丹比少校的脸上就流露出一种强烈的惊恐表情来,而且再也无法抹去。那些给惊吓出来的皱纹也像深深的黑色伤疤一样永久地留在了他的脸上。约塞连为这位文雅正派的中年理想主义者感到惋惜,正像他总是为许多有着这样或那样的小毛病、遇到这种或那种小麻烦的人感到惋惜一样。

    他故作亲热地说:“丹比,你怎么能够跟卡思卡特和科恩这样的人一块共事呢?这难道不使你倒胃口吗?”

    约塞连的这个问题似乎使丹比少校感到惊奇。“我跟他们共事是为了帮助我的祖国,”他回答说,好像这个回答是不言而喻的。

    “卡思卡特上校和科恩中校是我的上级,执行他们的命令是我能对我们所进行的这场战争作出的唯一贡献。我和他们共事,是因为这是我的职责,而且,”他垂下眼睛,压低嗓门补充说,也因为我不是个富于进取心的人。”
 


    “你的祖国已经不再需要你的帮助了,”约塞连心平气和地开导他说,“所以你现在所做的一切只不过是在帮助他们。”

    “我尽量不这么考虑问题,”丹比少校坦率地承认道,“我极力把注意力只集中在已取得的巨大成果上,极力忘掉他们也在获得成功这一事实。我极力骗自己说,他们不过是些微不足道的小人物而已。”

    “你知道,我的麻烦也就在这里,”约塞连抱拢双臂,摆出一副沉思的模样说道,“在我和我的全部理想之间,我总是发现许多个沙伊斯科普夫、佩克姆、科恩、卡思卡特那样的人,而这种人又多多少少改变了我的理想。”

    “你应当尽量不去想他们,”丹比少校口气肯定地劝告说,“你决不能让他们改变你的行为准则。理想是美好的,但人有时却不是那么美好、你应当尽量抬起头来看大局。”

    约塞连怀疑地摇了摇头,拒绝接受丹比的劝告。“当我抬起头来时,我看到人们全在设法赚钱。我看不见天堂,看不见圣人,也看不见天使。我只看见人们利用每一次正当的冲动和每一场人类的悲剧大把大把地捞钱。”

    “可你应当尽量不去想这类事情。”丹比少校坚持道,“你应当尽量不让这类事情弄得你心烦意乱。”

    “噢,我倒也没有真的心烦意乱。不过,叫我心烦意乱的是,他们把我当成了傻瓜。他们以为自己很聪明,而我们其余的人都笨得很,你知道,丹比,我刚才突然头一回冒出这么个念头,也许他们是对的。”

    “可你也应当尽量不去想这种事。”丹比少校争辩道,“你应当只考虑国家的利益和人类的尊严。”

    “是啊,”约塞连说。

    “我真的是这个意思,约塞连。这不是第一次世界大战。你千万不要忘了,我们现在是在跟侵略者作战。如果他们打赢了,他们不会让我们俩中的任何一个活下去。”

    “这我知道,”约塞连硬邦邦地回答道。他突然恼怒地板起了脸。“哼,丹比,无论他们发给我那枚勋章的理由是什么,那勋章反正是我自己挣来的。我已经执行了七十次该死的飞行任务,别再对我讲那些为拯救祖国而战斗的废话啦。我一直在为拯救祖国而战斗,现在我要为救我自己而战斗一下。祖国已经没有什么危险了,而我却正处在危险之中呢。”

    “战争还没有结束呢。德国人正朝安特卫普推进。”

    “几个月之内,德国人就会被打败。那之后再过几个月,日本人也会被打败。如果我现在战死了,那就不是为国捐躯,而是替卡思卡特和科恩送死。所以,在此期间,我要交回我的轰炸瞄准器。从现在起,我只考虑我自己。”

    丹比少校高傲地笑笑,颇为宽容地反问道,“可是,约塞连,要是每个人都这么想呢?”

    “要是那样,如果我不这么想,我不就成了个头号大傻瓜了吗?”约塞连露出一副嘲讽的表情,身体坐得更直了。“你知道吗?我有一种奇怪的感觉,好像我以前也和什么人进行过一次跟这次一模一样的谈话。这跟牧师的感觉一样,他觉得每件事他都经历过两次。”

    “牧师希望你让他们把你送回国去。”

    “牧师希望什么,我才不在乎呢。”

    “哦,唉。”丹比少校叹了口气,遗憾而失望地摇了摇头,“他担心自己可能影响了你。”

    “他没有影响我。你知道我可能会干什么吗?我可能会一直呆在医院的这张病床上,像株植物那样生活。我在这儿可以舒舒服服地过植物般的生活,让别人去拿主意吧。”

    “你必须自己拿主意,”丹比少校反驳道,“一个人不能像一株植物那样生活。”

    “为什么不能?”

    丹比少校眼中出现了一丝淡淡的热情。“像一株植物那样生活必定是很愉快的,”他若有所思地承认道。

    “是糟糕透顶的,”约塞连说。

    “不,摆脱了所有这些疑虑和压力的生活必定是非常舒适的,”丹比少校坚持道,“我觉得我很愿意像一株植物那样生活,那样就不必为大事情操心拿主意了。”

    “什么样的植物呢,丹比?”

    “黄瓜,或者胡萝卜。”

    “什么样的黄瓜?是好黄瓜还是坏黄瓜?”

    “噢,当然是好黄瓜咯。”

    “那么,你只要一成熟,他们就会把你摘下来,切成片做色拉。”

    丹比少校沉下脸来。“那只能是坏黄瓜啦。”

    “那么,他们会让你腐烂掉,把你拿去给好黄瓜当肥料,好让它们快些成熟。”

    “要是那样的话,恐怕我不会愿意像一株植物那样生活的,”丹比少校无可奈何地微微一笑,伤感地说。

    “丹比,我真的必须让他们送我回国吗?”约塞连严肃地问他。

    丹比少校耸了耸肩。“这是救你自己的一种方法。”

    “这是毁掉我自己的一种方法,丹比。这个道理你应该明白的。”

    “你可以得到许多你想要的东西。”

    “没有多少我想要的东西,”约塞连回答道。他内心突然涌起一股愤怒和失望,举起拳头狠狠地捶着床垫。“真他妈的,丹比!我有不少朋友在这场战争中送了命。这笔交易我不能做。让那个娼妇捅了一刀,这算是我所经历过的最好的事情了。”

    “那你宁愿进监狱吗?”

    “你会愿意让他们送你回国吗?”

    “我当然愿意!”丹比少校斩钉截铁地说,“我肯定愿意。”过了一会,他又用不那么肯定的口气加上了一句。“不错,要是我处在你的地位,我想我会让他们送我回国的。”他忧虑不安地思索了片刻之后,很不自在地拿定了主意。接着,他流露出极为痛苦的神情,厌恶地猛然把脸扭向一边,脱口叫道,“噢,是的,当然啦,我会让他们送我回国的!可我是一个最最胆小的人,我根本不可能处在你的位置上。”

    “可假如你不是个胆小的人呢?”约塞连目不转睛地打量着他问道,“假如你的确有勇气跟某个人作对呢?”

    “要真是那样,我是不会让他们送我回国的,”丹比少校断然发誓说。他的声音强劲有力,欢快热情。“可我肯定不会让他们对我进行军法审判的。”

    “你愿意执行更多的飞行任务吗?”

    “不,当然不愿意。那样做无异于全面投降。再说,我可能会送命的。”

    “那你会逃走吗?”

    丹比少校露出高傲的神色,刚要反驳,又突然停住了,他那半张开的嘴巴也默默地闭上了。他厌烦地噘起了嘴唇。“我想,我根本就没有什么希望,不是吗?”

    不一会,他的前额和暴出的白眼球又显出了紧张不安。他把两只软绵绵的手腕交叉着放在膝盖上,坐在那儿屏住呼吸,垂下眼睛盯着地板,默默地承认了自己的失败。陡斜的暗影从窗外映了进来。约塞连神情严肃地看着他。一辆疾驶而来的汽车在外面猛然刹住,发出一阵嘎的声响。随后,传来了什么人匆匆跑进大楼的咯咯脚步声。可是他们俩谁也没有动一动。

    “不,你还有希望。”约塞连愣了好一会,才想出一个主意来。

    “米洛也许会帮助你。他比卡思卡特上校有来头,他还欠我几桩人情呢。”

    丹比摇了摇头,语调平淡地回答道:“米洛和卡思卡特上校现在是伙伴啦。他让卡思卡特上校当上了副总裁,还答应他战争结束后给他安排一个重要的职务。”

    “那么,前一等兵温特格林会帮助我们的,”约塞连叫道。“他恨他们两个,这件事准会把他惹火的。”

    丹比少校又一次悲哀地摇了摇头。“米洛和前一等兵温特格林上个星期合伙了,他们现在全都是MM辛迪加联合体的合伙人了。”

    “这么说我们没有希望了,是吗?”

    “没有希望了。”

    “没有一点希望了,是吗?”

    “没有,没有一点希望了,”丹比少校承认道。过了一会,他抬起脸,说出一个尚未成熟的想法来。“如果他们能够像使其他人失踪那样让我们失踪,使我们摆脱这些沉重的负担,那不是件好事情吗?”

    约塞连认为那不是好事。丹比少校忧郁地点点头,表示同意,随后便又垂下了眼睛。两个人全都觉得毫无希望了。突然,走廊里传来一阵很响的脚步声,牧师可着嗓门嚷嚷着冲进门来。他带来了一个令人振奋的消息,是关于奥尔的。他又高兴又激动、有那么一两分钟连话都说不成句了。他的眼睛里闪动着喜悦的泪花、当约塞连终于听明白牧师的话时,他不敢相信地大叫一声,抬腿从床上跳了下来。

    “瑞典?”他大声问。

    “奥尔!”牧师大声说。

    “奥尔?”约塞连大声问。

    “瑞典!”牧师叫道。他兴高采烈地不住地点着头,开心地、兴奋地咧嘴笑着,得意洋洋地满屋子走个不停。“我告诉你,这是个奇迹!奇迹,我又信仰上帝啦!真的。在海上漂了这么多个星期,最后竟被冲到瑞典海岸上去啦!这是个奇迹!”

    “冲到岸上去的?见鬼!”约塞连大声说,他在屋里蹦来蹦去,欣喜若狂地冲着墙壁、冲着天花板、冲着牧师和丹比少校吼叫着。

    “他不是被冲到瑞典海岸上去的。他是划到那儿去的。他是划到那儿去的,牧师,他是划到那儿去的。”

    “划到那儿去的?”

    “他预先就这么计划好的!他是存心去瑞典的。”

    “噢,这我不管。”牧师依旧热情洋溢地回答说,“这仍然是个奇迹,这是人类智慧和忍耐力所创造的奇迹;瞧瞧,他干出了什么事情来!”牧师伸出双手捂往脑袋,笑得弯下了腰,“你们难道想象不出来他的样子吗?”他惊奇地叫道,“你们难道想象不出来他的样子?坐在黄色的救生艇里,握着那把小小的蓝色船桨,趁着黑夜划过直布罗陀海峡——”

    “身后拖着那根钓鱼线,一路上吃着生鳕鱼划到瑞典,每天下午还给自己泡茶喝。”

    “我甚至能看见他的样子!”牧师大叫道,他停了一下,趁机喘了口气,接着又赞叹下去。“我告诉你们,这是人类不屈不挠的毅力所创造的奇迹;这也正是我从现在起要做的事情。我也要不屈不挠,是的,我要不屈不挠。”

    “奥尔自始至终都知道自己在干什么!”约塞连欣喜若狂地叫道;他得意洋洋地高高举起两个拳头,似乎想从拳头里面挤压出什么启示来。他猛地转过身面对着丹比少校。“丹比,你这个笨蛋,到底还是有希望的、你难道没看出来吗?甚至克莱文杰也可能还活在那片云彩里面呢,他就藏在那里面一个什么地方,要一直等到安全了才出来。”

    “你们在说些什么呀?”丹比少校困惑地问,“你们两个在说些什么呀?”

    “给我弄些酸苹果来,丹比,还有坚果。快去呀,丹比,快去呀。

    趁着这会儿还来得及,给我弄些酸苹果和七叶树坚果来,给你自己也弄一些。”

    “七叶树坚果?酸苹果?要这些做什么?”

    “当然是塞到我们的腮帮子里去咯。”约塞连自责而又绝望地高高扬起两只手臂。“唉,我为什么不听他的呢?我为什么就没有信心呢?”

    “你疯了吗?”丹比少校惊恐而困惑地问道,“约塞连,请你告诉我你们在讲些什么,好吗?”

    “丹比,奥尔预先就这么计划好的。你难道不明白吗?他从一开始就是这么打算的。他甚至演习过如何让自己的飞机被击落下来。每次执行飞行任务时,他都要演习一遍。可我竟然不愿意跟他一起飞!唉,我为什么不听他的呢?他叫我跟他一起飞,可我竟然不愿意!丹比,再给我弄些龅牙来,还有装牙的牙套。只要装成一副愚蠢无知的傻瓜模样,就没有人会怀疑你其实是个机灵鬼。所有这些东西我都需要。唉,我为什么不听他的话呢?现在我明白他一直想跟我说什么了,我甚至明白了那个姑娘为什么拿鞋砸他的脑袋。”

    “为什么?”牧师追问道。

    约塞连猛地转过身,一把抓住牧师衬衣的前襟,恳求道:“牧师,帮帮我吧!请帮帮我。把我的衣服找来。赶快去找,行吗?我现在就需要它们。”

    牧师抬起腿就往外走。“好吧,约塞连,我去找。可你的衣服在哪儿呢?我怎么才能拿到它们呢?”

    “谁要是拦住你不让拿,你就吓唬他们,对他们吹胡子瞪眼睛。

    牧师,给我把制服拿来!我的衣服肯定在这医院里的某个地方。你这辈子就这么一次,干成件事情吧。”

    牧师坚定地挺了挺肩膀,又咬了咬牙。“别着急,约塞连。我会给你把制服拿来的。可那个姑娘为什么拿她的鞋砸奥尔的脑袋呢?

    求你告诉我吧。”

    “因为是他出钱叫她干的,就为这个!可她打得还不够狠,所以他只好划到瑞典去了。牧师,给我把制服找来,我好离开这个地方。

    问问达克特护士吧,她会帮你找到的。只要能甩开我,她什么都愿意干的。”

    “你要去哪儿呀?”牧师冲出房间后,丹比少校担心地问道,“你打算干什么呀?”

    “我打算逃走,”约塞连用欢快而清晰的嗓音宣布道。他已经拉开了睡衣领口处的扣子。

    “噢,不。”丹比少校叹息了一声,用两只手掌来来口口地轻轻拍着自己那张汗淋淋的脸。“你不能逃走。你能逃到哪儿去?你能到哪儿去呢?”

    “去瑞典。”

    “去瑞典?”丹比少校惊奇地叫道,“你要跑到瑞典去?你疯了吗?”

    “奥尔已经去了。”

    “噢,不不,不不,不,”丹比少校恳求道,“不,约塞连,你永远也到不了那儿。你不能跑到瑞典去。你连船都不会划。”

    “可是,只要你离开这儿后闭上嘴不吭气,找个机会让我搭上一架飞机,我就可以到罗马去。”

    “可他们会找到你的,”丹比少校固执地争辩道,“会把你抓回来,会更加严厉地惩罚你的。”

    “这一回,他们要想抓住我可得使出吃奶的力气来。”

    “他们会使出吃奶的力气来的。就算他们找不到你,你过的将会是一种什么样的日子呀?你永远只能孤零零地一个人呆着,没有任何人会跟你在一起,而且,你随时随地可能会被人出卖。”

    “我现在就是过的这种日子。”

    “可你不能就这么背弃你的职责一走了之,”丹比坚持道,“这是一种十分消极的行为,是逃避现实。”

    约塞连轻快而蔑视地哈哈一笑,又摇了摇头。“我并没有逃离我的职责,我正冲着它跑过去呢,为了救自己的性命而逃走,这根本算不上消极。你当然知道是谁在逃避现实,丹比,对吗?不是我,也不是奥尔。”

    “牧师,请你跟他谈谈,好吗?他要开小差,他想逃到瑞典去。”

    “太棒了!”牧师欢呼起来。他得意地把一个装满约塞连衣服的枕套扔到床上。“逃到瑞典去吧,约塞连。我要留在这儿,不屈不挠地坚持下去,是的,我要不屈不挠地坚持下去。每次我遇到卡思卡特上校和科恩中校时,我都要找他们的碴儿,跟他们胡搅蛮缠。我不怕他们,就连德里德尔将军我也敢找他闹事。”

    “德里德尔将军调走了。”约塞连一边提醒他,一边套上裤子;

    匆匆忙忙地把衬衣下摆塞进裤腰里。“现在是佩克姆将军当指挥官了。”

    牧师依旧信心十足地唠叨着,“那么,我就找佩克姆将军闹事,甚至找沙伊斯科普夫将军闹事。你知道我还要于什么吗?我下回见到布莱克上尉时要朝他的鼻子狠揍一拳。是的,我要朝他的鼻子狠揍一拳。我要找个周围有许多人的时候揍他,这样他就没有机会还手了。”

    “你们两个都疯了吗?”丹比少校抗议道。他内心充满了痛苦、敬畏和恼怒,两只突出的眼球楞睁着。“你们两个是不是都失去理智了?约塞连,听着——”
 


    “我告诉你,这是个奇迹,”牧师宣布道,他一手抓住丹比少校的手腕,拾起胳膊肘,拖着他转着圈子跳起华尔兹舞来。“一个真正的奇迹。如果奥尔能划到瑞典去,那我只要不屈不挠地坚持下去、就一定能战胜卡思卡特上校和科恩中校。”

    “牧师,请你住嘴好吗?”丹比少校一边有礼貌地恳求着,一边从牧师手里挣脱出来,焦虑不安地轻轻拍了几下自己那汗淋淋的前额。随后,他俯下身去对正在伸手拿鞋子的约塞连说,“可上校那儿——”

    “他那儿怎么样我才不管呢。”

    “但这实际上可能会——”

    “叫他们两人全都见鬼去吧!”

    “但这实际上可能会帮他们的忙,”丹比少校固执地坚持道,“你想过这一点没有?”

    “让这两个杂种升官发财去吧,我才不管呢。既然我没有办法阻止他们,我就只能靠开小差来给他们捣捣乱了。现在我有我自己的职责,丹比、我一定要到瑞典去。”

    “你绝不会成功的,这是不可能的。从这儿跑到瑞典,单从地理上讲,就几乎是不可能的。”

    “见鬼,这我知道,丹比。可我至少得试一试。在罗马有个小女孩,要是我能找到她、我想把她救出来。要是我能找到她,我就把她带到瑞典去。所以、这并不完全是为了我自己,不是吗?”

    “你绝对是疯了。你的良心将使你永远不得安宁。”

    “上帝保佑我的良心吧。”约塞连哈哈大笑。“我要是没有什么担惊受怕的事情就觉得活不下去了。对吗,牧师?”

    “我下回见到布莱克上尉时要朝他的鼻子狠揍一拳,”牧师得意地说。他先伸出左臂往空中打了两拳,又像翻晒干草一样笨拙地挥了挥右臂。“就像这样。”

    “可这不是丢脸的事情吗?”

    “什么丢脸的事情?我现在这个样子才更丢人现眼呢。”约塞连把第二根鞋带结结实实地系好后,一下子跳了起来。“喂,丹比,我准备走啦。你看怎么样?请你闭上嘴不吭气,让我搭上一架飞机好吗?”

    丹比少校默默地打量着约塞连,他的脸上浮现出一丝奇怪而凄惨的微笑。他已经不再出汗了,显得十分镇定。“要是我真的阻拦你,你会怎么办?”他用悲哀的嘲弄口吻问道,“狠狠揍我一顿吗?”

    听到这句问话,约塞连吃了一惊,觉得自尊心受到了伤害。

    “不,当然不。你为什么这样说呢?”

    “我要狠狠揍你一顿,”牧师夸耀地说。他一步跳到丹比少校跟前,摆出挥拳格斗的架势。“我要狠狠地揍你和布莱克上尉一顿,可能还要揍惠特科姆中士一顿。如果我发现我再也不必害怕惠特科姆中士了,那不是太妙了吗?”

    “你打算阻拦我吗?”约塞连紧紧盯住丹比少校问。

    丹比少校从牧师面前跳到一旁,犹豫了片刻之后脱口说道:

    “不,当然不!”他突然急切而有力地朝着门口的方向挥了挥两只手臂。“我当然不会阻拦你。走吧,看在上帝的分上,赶快走吧!你需要钱吗?”

    “我有点钱。”

    “喏,我这儿还有些钱,”丹比少校热情洋溢,激动万分。他掏出厚厚一叠意大利钞票塞给约塞连,又用双手紧紧握住约塞连的一只手,既是为了给约塞连鼓劲,也是为了使自己的手指不再颤抖。

    “这个时候住在瑞典一定是很惬意的,”他羡慕地说,“那儿的姑娘非常可爱,那儿的人们非常开明。”

    “再见,约塞连,”牧师告别说,“祝你好运。我要在这儿不屈不挠地坚持下去,战争结束后我们会再见面的。”

    “再见,牧师。谢谢你,丹比。”

    “你觉得怎么样,约塞连?”

    “很好,不,我很害怕。”

    “这才对头,”丹比少校说,“那说明你还活着,因为那不会是什么好玩的事。”

    约塞连往外走去。“不,是挺好玩的。”

    “我说的是真话,约塞连。你每天每时每刻都要保持警惕。他们会撒下天罗地网抓你的。”

    “我时时刻刻都会保待警惕的。”

    “你得赶快跑。”

    “我是要赶快跑的。”

    “赶快跑吧!”丹比少校叫道。

    约塞连跑了出去。内特利的妓女就藏在门外。她举刀砍了下去,差一点砍到他。约塞连跑走了——

    完
