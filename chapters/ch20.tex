\chapter{惠特科姆下士}
 
    八月下旬的朝阳热烘烘的,晒得大地水汽腾腾,阳台上一丝风也没有。随军牧师慢吞吞地走着。当他穿着那双棕色的胶底胶跟鞋静悄悄地从上校的办公室里出来的时候,他垂头丧气,不停地责备自己。他恨自己胆小怕事。他原先打算就六十次飞行任务一事对卡思卡特上校采取较为强硬的立场,对一个自己已开始深为关切的问题大胆地进行一番有条有理的雄辩。可事实却相反,在一个更加强硬的人的反对下,他一败涂地,又一次语塞了。这是一次司空见惯了的、不光彩的经历,他实在是很瞧不起自己。

    片刻之后,当他发现科恩中校那矮胖的、单色的身影正无精打采地急匆匆地快步登上用黄色石块砌成的宽阔的弧形楼梯向他走过来时,他语塞得就更厉害了。科恩中校从下面那个高大、破败的门厅里走上来。门厅高高的黑色大理石墙壁上满是裂痕,圆形地面上的砖也已破裂,积满污垢。随军牧师虽害怕卡思卡特上校,但更怕科恩中校。这个皮肤黝黑的中年中校戴着一副寒气逼人的无边眼镜,总是不停地张开手用指尖敏感地摸摸他那个凸凹不平的、像个圆形大屋顶似的光脑袋。他不喜欢牧师,常常对他不礼貌。他用粗率无礼、冷嘲热讽的言词和洞悉一切、似笑非笑的目光使牧师常处于一种担惊受怕的状态,除了偶尔刹那间的目光相遇之外,牧师从没有足够的勇气去正视中校片刻。由于牧师在中校面前总是战战兢兢、低头哈腰,因此他的目光总是不可避免地落在科恩中校的腰部,看见他的衬衫下摆从凹陷下去的皮带里皱巴巴地鼓出来,像只气球似的垂挂在腰间,使他的腰部显得臃肿、邋遢,因此他虽是中等身材,但看起来比实际身高要矮几英寸。科恩中校是个不修边幅、傲慢无礼的人,皮肤油光光的,几道又深又粗的皱纹几乎一直从鼻子下延伸到灰暗的两颊下的垂肉和似刀削的方下巴之间。他脸色阴沉,当他们两人在楼梯上走近,将要擦肩而过时,他朝牧师扫了一眼,没有显示出任何认出他的神情。

    “你好,神父,”他用平板的声调问候说,连看都没看牧师一眼。

    “过得好吗?”

    “早晨好,长官,”牧师答道,他明白地看出来科恩中校只不过是要他回问一声好。

    科恩中校没有放慢脚步,继续朝楼梯上方走,牧师真想再次提醒他,他不是天主教教徒而是再洗礼教教徒,因此没有必要叫他神父,而且这样称呼也不正确,但他忍住了。他几乎可以肯定科恩中校是记得这一点的,他带着一种如此无动于衷的无知神情叫他神父只不过是他嘲弄他的另一种方法,因为他只是一名再洗礼教教徒。

    科恩中校几乎已经走过去了,突然又冷不防地停了下来,转过身一阵风似地朝牧师冲过来,眼里露出愤怒、怀疑的目光。牧师吓呆了。

    “你拿着那只红番茄做什么,牧师?”科恩中校态度粗暴地问道。

    牧师惊讶地低头看了看手里那只卡思卡特上校叫他拿的红番茄。“我是在卡思卡特上校办公室里拿的,长官,”他费了很大劲才回答出来。

    “上校知道你拿吗?”

    “知道,长官。是他送给我的。”

    “哦,既是这样,我想那就没关系了,”科恩中校说,态度缓和了下来。他毫无热情地笑了笑,一面用大拇指把皱巴巴的衬衫下摆重又塞进裤子里去。他两只眼睛闪烁着刺人的光,流露出一种暗自得意的恶作剧的神色。“卡思卡特上校召你去干什么,神父?”他突然问。

    牧师结结巴巴,一时不知该如何回答。“我想我不该——”

    “做祷告给《星期六晚邮报》的编辑们看?”

    牧师差点笑出来。“是的,长官。”

    科恩中校为自己的直觉感到高兴。他轻蔑地大笑起来。“你知道,我担心他一看到这个星期的《星期六晚邮报》,就会开始考虑如此荒唐可笑的事。我希望你成功地向他表明了这是一个多么糟糕的主意。”

    “他已经决定不这么干了,长官。”

    “那就好。我很高兴你使他确信《星期六晚邮报》的编辑们不可能重复登载那种相同的故事,去宣传某个不出名的上校。在野地里过得怎么样,神父?还能对付吧?”

    “能,长官。没什么问题。”

    “很好。我很高兴听到你说没什么问题。如果你需要点什么让自己过得舒服些,就告诉我们。我们大家都想让你在野外过得愉快。”

    “谢谢你,长官。我会的。”

    从下面门厅那边传来一阵越来越大的喧闹声。快到吃午餐的时间了,最先到的人正走进大队部的食堂。士兵和军官分别进入了不同的餐厅,餐厅就设在那个具有古代建筑风格的圆形大厅的四周。科恩中校收住了微笑。

    “你一二天前曾在这儿和我们共进过午餐,对吗,神父?”他意味深长地问道。

    “是的,长官。是前天。”

    “我想也是前天,”科恩中校说,然后停了一下,让牧师慢慢领会他的意思。“那么,放心好了,神父。当到了你再到这儿来吃饭的时候,我会考虑你的。”

    “谢谢长官。”

    军官餐厅和士兵餐厅各有五个,牧师不清楚哪天他被安排在哪个餐厅吃午餐,因为科恩中校为他制定的轮流就餐制度十分复杂,而他又把记录本遗忘在帐篷里了。随军牧师是唯一一位隶属于大队部编制而不住在那幢破旧的、红石头砌的大队指挥部大楼里的军官,他也不住在大楼四周那些独立的、较小的卫星式建筑物里。牧师住在大约四英里外一块介于军官俱乐部和四个中队营区中第一个中队营区之间的林间空地上。这四个中队的营区排成一线,从大队部所在地一直延伸到很远的地方。牧师独自一人住在一顶宽大的方形帐篷里,那也是他的办公室。夜晚,从军官俱乐部那边传来的狂欢声常常使这位过着半是被迫半是自愿的流放生活的随军牧师躺在帆布行军床上翻来覆去难以入眠。他偶尔吃几片药性温和的药丸助他入睡,可那些药丸对他没有什么作用,而且事后他还要内疚好几天。

    唯一和随军牧师一起住在林间空地上的是他的助手惠特科姆下士。惠特科姆下士是个无神论者、也是个心怀不满的部下,因为他觉得他做随军牧师的工作能比牧师本人做得好得多,因此他把自己看做是被剥夺了基本权利的社会不公正现象的受害者。他住在一顶同牧师的帐篷一样宽敞的方形帐篷里。自从有一次他发现自己做了错事牧师竟没有惩罚他之后,他便公开地对牧师采取粗暴、蔑视的态度。空地上的两顶帐敞间至多不过四五英尺。

 


    是科恩中校为牧师安排了这种生活方式。科恩中校认为,有一条很好的理由让随军牧师住在大队部大楼之外,那就是,牧师像他的大多数教徒那样住在帐篷里能使他与教徒之间保持更密切的联系。另一条重要的理由是,让牧师一天到晚呆在大队部周围会使其他军官感到不自在。同上帝保持联系是一码事,他们都赞同这一点,但让上帝一天二十四小时都呆在身边就是另一码事了。总之,正如科恩中校向那个极度紧张不安、眼珠突出的大队作战参谋丹比少校所描绘的那样,牧师的日子过得很轻松,他只要听听别人诉说烦恼,举行葬礼,看望卧床不起的伤病员和主持宗教仪式。科恩中校指出,现在已不再有多少死人需要他去举行葬礼,因为德国战斗机的反击基本上已经停止,还因为,据他估计,将近百分之九十的现有阵亡人员不是死在敌军防线之后就是在云层中失踪了,因此牧师根本用不着去处理尸体。再说,主持宗教仪式也不是什么太劳累的事,因为每周只在大队部大楼里举行一次,而且参加的人也很少。

    事实上,牧师正努力使自己喜欢在这片林间空地上生活。人们为他和惠特科姆下士两人提供了一切便利措施,因此他俩谁也不可能以生活不便为依据,要求允许他们回到大队部大楼里去。牧师轮流到八个飞行中队的食堂去和不同的人吃早餐、中餐和晚餐,每五餐最后一餐去大队部的士兵食堂吃,每十餐最后一餐去那儿的军官食堂吃。还在威斯康星州家中的时候,牧师非常喜欢栽培花木。每当他陷入沉思,想起那些小树的低矮、多刺的树枝和几乎把他围起来的、齐腰深的野草和灌木丛的时候,一种土地肥沃、果实累累的美好印象便涌上心头。春天,他很想在帐篷四周种上窄窄的一条秋海棠和百日草,但又害怕惠特科姆下士有怨气而未种。牧师非常欣赏自己住在这青枝绿叶的环境中才会有的幽静和与世隔绝的气氛,以及生活在那儿所引起的种种遐想和幽思。现在来找他倾吐苦恼的人比以前少多了,他对此也表示几分感谢,牧师不善与人相处,与人谈话也不大自在。他很想念妻子和三个幼小的孩子,他的妻子也想念他。

    除了牧师相信上帝这一点之外,惠特科姆下上最讨厌牧师的就是他缺乏主动性,做事缩手缩脚。惠特科姆下士认为,这么少的人参加宗教仪式令人伤心地反映了牧师本人所处的地位。为点燃伟大的精神复兴运动之火,他把自己想象成这一运动的缔造者,他头脑里狂热地想出种种具有挑战性的新主意——午餐盒饭、教堂联欢会、给战斗伤亡人员家属的通函、信件审查、宾戈赌博游戏。

 


    但牧师阻止了他。惠特科姆下士对牧师的管束很恼火,因为他发现到处都有改进的余地。他断定,正是像牧师这佯的人才使宗教有了那么一个坏名声,使他们两人均沦为被社会遗弃的流浪汉。和牧师不同的是,惠特科姆下士极为讨厌在林中空地上的隐居生活。等他让牧师免了职之后,他想做的第一件事就是搬回到大队部大楼里去,过上热热闹闹的生活。

    当牧师离开科恩中校,开车回到那块空地的时候,惠特科姆下士正站在外面闷热的薄雾里,用密谋似的声调同一个圆脸的陌生人在谈着什么。那个陌生人穿着一件栗色的灯芯绒浴衣和灰色的法兰绒睡衣。牧师认出那浴衣和睡衣是医院的统一服装。那两个人谁也没有以任何形式跟他打招呼。那陌生人的齿龈被涂成了紫色;

    他的灯芯绒浴衣后面有一幅画,画着一架B-25轰炸机正穿过桔红色的高射炮火,浴衣的前面画上了整整齐齐的六排小炸弹,表示飞满了六十次战斗任务。牧师被这两幅图深深吸引住了,他停住脚步目不转睛地看着。那两个人停止了谈话,默不作声地等着他走开。

    牧师匆匆走进他的帐篷。他听见,或者说他想象着他听见他们在窃笑。

    过了一会儿,惠特科姆下士走进来问道:“情况怎么样?”

    “没什么新闻,”牧师回答说,眼睛看着其他地方。“刚才有人来这儿找我吗?”

    “还不是那个怪人约塞连。他真是个惹事生非的家伙,不是吗?”

    “我倒不那么肯定他是个怪人,”牧师评论说。

    “说得对,你和他站在一边,”惠特科姆下士用受到伤害的口气说,然后跺着脚走了出去。

    牧师难以相信惠特科姆下士又被惹气并真的走出去了。刚等他弄明白,惠特科姆下士又走了进来。

    “你总是支持别人,”惠特科姆下士指责他说,“可你不支持你手下的人。这就是你的过错之一。”

    “我并不是想支持他,”牧师抱歉地说,“我只是表明一下态度。”

    “卡思卡特上校想要干什么?”

    “不是什么重要的事。他只是想商量一下每次飞行任务前是否有可能在简令下达室里做一下祷告。”

    “好吧,不告诉我就算了。”惠特科姆下士怒气冲冲地说完,就又走了出去。

    牧师非常难过。他想方设法,但无论他考虑得多么周到,却总好像是在设法伤害惠特科姆下士的感情。他懊恼地向下凝视着,发现科恩中校硬派来替他打扫帐篷、看管物品的勤务兵又忘了给他擦皮鞋了。

    惠特科姆下士又回来了。“你从来不把重要的消息告诉我,”他刻薄地抱怨说,“你不信任你手下的人。这是你的又一个过错。”

    “不对,我信任,”牧师内疚地向他保证说,“我非常非常信任你。”

    “那么,那些信怎么办?”

    “不发,现在不发,”牧师畏畏缩缩地恳求说,“别提信的事。请别再提这件事了;如果我改变了主意,我会告诉你的。”

    惠特科姆下士大发雷霆。“是这样吗?好吧,你倒轻松,往那儿一坐,摇摇头说不行,而所有的工作全得由我去做。你没看见外面那个浴衣上画上了那些图画的家伙吗?”

    “他来这儿是找我的吗?”

    “不是,”惠特科姆下士说,然后走了出去。

    帐篷里闷热、潮湿,牧师觉得自己浑身湿滴滴的。他像个极不情愿的偷听者,听着帐篷外面的人压低嗓门窃窃私语,声音沉闷低沉,嗡嗡的听不清楚。他有气无力地坐在那张作为办公桌用的摇摇晃晃的正方形桥牌桌前,双唇紧闭,两眼露出茫然若失的神色,脸色蜡黄。他脸上长着好几块很小的粉刺窝,已有不少年头了,上面的颜色和表面纹理就像完整的杏仁壳。他绞尽脑汁想理出一些头绪,找到惠特科姆下士怨恨他的根源。他无论如何想不出是什么问题,于是他确信自己对他犯下了不可饶恕的错误。如果说惠特科姆下士的那种长期的愤恨是由于牧师拒绝了他的宾戈赌博游戏和给在战斗中阵亡的将士家属寄通函的主意而产生的,这似乎令人难以置信。牧师垂头丧气,自认自己无能。几个星期以来,他一直打算和惠特科姆下士开诚布公地谈一次,以便弄清到底是什么使他烦恼,但现在他已对自己有可能弄清楚的事情感到害臊了。

    帐篷外面,惠特科姆下士在窃笑,另一个人也在抿着嘴轻声地笑。有那么几秒钟,牧师头脑里迷迷糊糊的,突然产生了一种神秘、离奇的感觉,仿佛以前在生活中曾经历过这一完全相同的情景。他竭力想抓牢并留住这一印象,以便预测,也许甚至能控制下面将会发生的事情,但正如他事先已知道的那样,这一灵感没给他留下什么印象便消失了。这种微妙的在幻想与现实之间反复出现的内心混乱是典型的错构症;牧师被这种症状迷住了,他对此还颇有了解,比如说,他知道这种症状叫做错构症,他对这种推论性的视觉现象很感兴趣。
 


    有些时候,牧师突然感到惊惴失措,那些伴随他度过了几乎大半生的事物、想法,甚至人莫名其妙地呈现出一种他以前从未见过的、陌生而又反常的样子,这种样子使这些事物、想法或人显得似乎是完全陌生的。他脑里几乎闪过一些十分清晰的景象,他在其中几乎见过绝对真理。在斯诺登的葬礼上有个赤条条的人在树上,这个插曲使他迷惑不解,因为当时他没有以前在斯诺登的葬礼上看见一个赤条条的人在树上时曾有过的那种感觉。因为那个幽灵不是以一种陌生的外表出现在他面前的熟悉的人或事。因为牧师确确实实看见了他。

    一辆吉普车在帐篷外面用回火发动起来,然后轰轰地开走了。

    在斯诺登葬礼上看见的那个赤条条地呆在树上的人仅仅是个幻觉呢?还是一件真实的事?牧师一想到这个问题就直打哆嗦。他极想把这个秘密告诉约塞连,然而每当他想起那件事的时候,他就决定不再去回想它了,尽管此刻他的的确确在回想这件事,但他不能肯定他以前是否真的想到过这件事。

    惠特科姆下士喜眉笑眼地闲荡着走了进来,一只胳膊肘很不礼貌地靠在牧师住的帐篷的中央支柱上。

    “你知道那个穿红浴衣的家伙是谁吗?”他虚张声势地问,“那是鼻梁骨折了的刑事调查部的工作人员。他是因公事从医院到这儿来的。他正在进行一项调查。”

    牧师飞快地扬起双眼,露出一副讨好、同情的神情。“我希望你没遇到什么麻烦。有什么事需要我帮忙的吗?”

    “不是,我没有什么麻烦,”惠特科姆下士答道,笑得合不拢嘴。

    “是你有麻烦啦。由于你在所有那些你一直在签华盛顿-欧文的名字的信上签上了华盛顿-欧文的名字,他们准备对你采取严厉的措施。你觉得这事怎么样?”

    “我从没有在任何信上签过华盛顿-欧文的名字,”牧师说。

    “你不必对我说谎,”惠特科姆下士回答说,“我不是你要说服的人。”

    “但是我没在说谎。”

    “你在不在说谎不关我的事。他们还因为你截取梅杰少校的信函要惩办你呢。他的信函里有许多东西都是机密情报。”

    “什么信函?”牧师越来越气愤,满肚子冤屈地问道,“我连看都没看到过梅杰少校的任何信函。”

    “你用不着对我说谎,”惠特科姆下士回答说,“我不是你要说服的人。”

    “但是我没在说谎!”牧师抗议说。

    “我不明白你干吗非得向我喊叫,”惠特科姆下士带着受到伤害的表情反击说。他离开了帐篷中央的那根柱子,朝牧师摇晃着一根手指表示强调。“我刚才帮了你这一辈子最大的忙,而你甚至没有意识到。每次他企图向上级打你的小报告时,医院里总有人把那些具体内容删除掉。几个星期来,他发了疯似地想告发你。我甚至连看都没看就在他的信上签上“已经检查”的字样,并签上保密检查员的名字。那样将会为你在刑事调查部总部里留下个非常好的印象。让他们知道我们丝毫不害怕把有关你的全部事实真相公布于众。”

    牧师头脑里一团乱麻,被搞得晕头转向。“可是没有人授权让你去检查信件啊,是吗?”

    “当然没有,”惠特科姆下士回答说,“只有军官才有权做那种工作。我是用你的名义去检查的。”

    “但是我也没被授权去检查信件啊,是吧?”

    “我也替你想到那一点了,”惠特科姆下士宽慰他说,“我代你签的是其他人的名字。”

    “这不是伪造吗?”

    “哦,这也不必担心。唯一可能控告你犯伪造罪的人就是那个你伪造他的签名的人,于是我为你着想挑了一个死人。我用了华盛顿-欧文的名字。”惠特科姆下士仔细打量着牧师的脸,想看看有没有反对的迹象,然后隐隐带着讽刺的口吻轻快而自信地说下去。

    “我的脑筋转得快吧,不是吗?”

    “我不知道。”牧师声音颤抖地轻轻哀叹了一声,又痛苦又不明白,蹩眉皱眼,一副怪相。“我想我没弄明白你说的这一切。如果你签的是华盛顿-欧文的名字而不是我的名字,那怎么会为我留个好印象呢?”

    “因为他们确信你就是华盛顿-欧文。你明白吗?他们会知道那就是你。”

    “但是我们不正是要让他们不相信那一点吗?这样不是帮助他们相信了吗?”

    “要是我早知道你对这事会这么呆板教条,我压根儿就不会试着去帮你了,”惠特科姆下士气愤地说。然后他走了出去。一秒钟后他又走了进来。“我刚才帮了你这辈子中最大的一个忙,而你甚至不知道。你不知道怎样表示感谢。这是你的又一个过错。”

    “我很抱歉,”牧师后悔地道歉说,“我真的很抱歉。你跟我说的那一切把我彻底吓糊涂了,我也搞不清自己在说些什么。我真的十分感激你。”

    “那么让我寄那些通函怎么样?”惠特科姆下士立即要求说,“我可以开始写初稿吗?”

    牧师惊愕得嘴都合不拢了。“不,不,”他呻吟着说,“现在不要。”

    惠特科姆下士被激怒了。“我是你最好的朋友,而你却不知道,”他咄咄逼人地说,然后走出了牧师的帐篷。他又走了进来。“我在支持你,你甚至不知道。你不知道你遇到多大的麻烦了吗?刑事调查部的那个人已经赶回医院去写一份新的报告,揭发你拿那只番茄的事。”

    “什么番茄?”牧师眨着眼睛问。

    “就是你刚回到这里时藏在手里的那只红色梨形番茄。这不是吗!这只番茄你直到这一刻还拿在手里呢!”

    牧师吃惊地松开了手,发现自己还拿着那只从卡思卡特上校的办公室里得到的红色梨形番茄。他赶忙把它放在牌桌上。“我是从卡思卡特上校那儿弄到这只番茄的,”他说,突然惑到自己的解释听起来是多么荒唐可笑。“他非要让我拿一只。”

    “你用不着对我说谎,”惠特科姆下士回答说,“你是不是从他那儿偷的不关我的事。”

    “偷的?”牧师惊诧地叫道,“我于吗要偷一只红色梨形番茄?”

    “这正是使我们两人都迷惑不解的问题,”惠特科姆下士说,“那时,刑事调查部的那个人断定你也许把什么重要的秘密文件藏在里面了。”

    牧师绝望了,在这山一般重的心理重压下、他整个人都瘫软了。“我没有什么重要的秘密文件藏在里面,”他坦白地陈述道,“我开始甚至都不想要。喏,你可以拿去。你自己拿去看看吧。”

    “我不要。”

    “请把它拿走吧,”牧师恳求说,声音低得几乎听不见。“我想摆脱它。”

    “我不要,”惠特科姆下士气冲冲地又说了一遍,怒容满面地走了出去、他内心里却高兴无比,只是忍着没笑出来,因为他与刑事调查部的那个人结成了新的强大的联盟,并且又一次成功地使牧师相信他真的生气了。

    可怜的惠特科姆,牧师叹息道,他为助手心情阴郁而责备自己。他一声不吭地坐在那里,傻乎乎地陷入了沉思,满怀期望地等待着惠特科姆下士走回来。当他听见惠特科姆下士那高傲的步伐声慢慢消逝在远方时,他失望了。他接下来什么事也不想做。他决定不用午餐了,从床脚柜里各拿出一块银河牌和鲁丝宝贝牌巧克力糖吃了,喝了几白水壶里的温水。他觉得自己像是被笼罩一切的大雾包围了,看不见一星半点的光,随时有可能发生什么事情。他担心,一旦有人把他被怀疑成是华盛顿-欧文的消息汇报给卡思卡特上校,上校会怎么想呢?然后又想到卡思卡特上校曾因他提过六十次飞行任务的事已经对他有看法了,因而忧心忡忡。世界上竟有这么多不幸的事,他思忖着,想到这件令人伤心的事情、他心情忧郁地低下了头。他对任何人的不幸都无能为力,尤其是对他自己的不幸更是如此
