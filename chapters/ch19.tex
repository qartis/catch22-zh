\chapter{卡思卡特上校}
 
    卡思卡特上校聪明圆滑,事业一帆风顺,但却衣着邋遢,满腹忧愁。他三十六岁,走起路来步伐沉重,一心想当将军。他有股子冲劲,但又容易泄气;他处事泰然自若,但又时常懊恼;他自鸣得意,但对自己的前程又没有把握;他无所顾忌地采用各种行政计谋以博取上级的青睐,但又害怕自己的计谋会弄巧成拙。他长相不错,但缺乏魁力;他强壮如牛,但又有些虚张声势,而且还很自负。他已经开始发胖,为此他时常感到担忧,想挥也挥不去,所以,长期以来他一直受着它的折磨。卡思卡特上校很自负,因为他才三十六岁就成了一名带领一支战斗部队的上校军官;但他又感到沮丧,因为他虽然已经三十六岁了还只不过是个上校。

    卡思卡特上校不是个绝对主义者。他衡量自己的进步的唯一的方法就是拿自己同别人比较。他认为,所谓优秀,就是同样做一件事情,至少能同与他年龄相仿但做事却更高明的人做得一样好。

    一方面,有成千上万和他年龄相同或者比他大的人还没爬到少校这一级,这一事实使他对自己的超人的才能和价值沾沾自喜;而另一方面,有不少同他一般年纪甚至比他年轻的人已经成了将军,这又使他产生一种失败感,使他痛心疾首,直咬指甲,那种难以抑制的急切心情甚至比亨格利-乔还要强烈。

    卡思卡特上校身材高大,虎背熊腰,卷曲的黑发剪得短短的,发尖已开始发白,嘴里常叼着他来皮亚诺萨指挥飞行大队前一天购买的那个装饰精美的烟嘴。他一有机会就要把那烟嘴炫耀一番,而且他还学会了熟练地摆弄烟嘴的手段。他无意中发现,在他身体内部有一种生来就有的使用烟嘴抽烟的本领。据他所知,他的这个烟嘴在整个地中海战区是独一无二的。这一想法既使他喜形于色,又使他忧虑不安。他相信,像佩克姆将军那样又有教养又有知识的人肯定会赞同他用烟嘴抽烟的,尽管他与佩克姆将军很少见面。不过从另一个方面看,他们难得见面也不是什么坏事,卡思卡特上校欣慰地认识到这一点,因为佩克姆将军也有可能压根就不赞同他使用烟嘴。当这样的烦恼困扰他时,卡思卡特上校总强忍住呜咽,真想把这个该死的东西扔掉。但是他那种不可动摇的信念使他始终未能这么做,那就是:这个烟嘴一定会为他那副充满阳刚之气的军人体魄增色,使他显得老练、威武、卓越超群,明显胜过美军中所有其他与他竞争的上校军官。不过他到底有多大把握呢?

 


    卡思卡特上校就是这么一个不知疲倦的人,一个不分昼夜地为了自己而不住地盘算着的勤劳、紧张、全身心投入的战术家。同时,他又是自己的掘墓人,既是一位颇具胆识的、一贯正确的外交家,又总是为自己失去了众多良机而责骂自己,或为自己所犯的所有错误而自怨自艾,懊悔不已。他神经紧张,性情急躁,言语尖刻,可又自鸣得意。他是个英勇无畏的机会主义者,贪婪地扑向科恩中校为他提供的每一个机会,可事后对自己可能遭受的不良后果又马上吓得浑身发抖,冷汗直冒。他极爱搜集谣言传闻,十分喜欢流言蜚语。他不管听到什么消息都信以为真,但对每一则消息又都不相信。他高度警觉,时刻准备应付每一个信号,即使对那些根本不存在的关系和情况也极其敏感。他是个了解内幕消息的人,总是可怜巴巴地想弄清正在发生什么事情。他是个狂暴、凶猛、欺软怕硬的恶棍。他记得他曾不断地给那些大人物留下了可怕的不可磨灭的印象,每想到这些他就伤心不已,可实际上,那些大人物几乎根本不知道有他这么个人活在世上。

    每个人都在迫害他。卡思卡特上校凭他的才智生活在一个有时受到羞辱、有时得到荣誉、动荡不定、斤斤计较的社会里。他想象着,在这个社会里他有时得到了绝对的胜利,有时又遭到了灭顶的惨败。他时时刻刻都在极度的痛苦与极度的欢乐之间徘徊,一会儿将胜利的辉煌业绩扩大到了令人难以置信的程度,一会儿又把失败的严重性夸大到了惨绝人衰的地步。从未有人发现他对任何事情有过疏忽。如果他听说有人看见德里德尔将军或佩克姆将军微笑或皱眉头,或既不笑也不皱眉头,他不找到一个可以接受的解释是决不会使自己平静的,而且还老是唠叨个没完,直到科恩中校来劝他不要那么紧张,劝他把事情想开些为止。

    科恩中校是个忠实且不可缺少的助手,可他总使卡思卡特上校心烦。卡思卡特上校对科恩中校提出的一些具有独创性的建议十分感激,并发誓说这种感激是永久不变的,可后来当他觉得这些建议行不通时,便对他大发雷霆。卡思卡特上校非常感激科恩中校的帮助,但根本就不喜欢他。这两个人只是关系很近而已。卡思卡特上校妒忌科恩中校的聪明才智,只得常常提醒自己科恩中校还只是个中校,而且还比自己大将近十岁,又是个州立大学的毕业生,卡思卡特上校悲叹命运不公,他需要一个得力的助手,可命运却给了他一个像科恩这样平庸的人。得完全依靠一个州立大学毕业的人,真是有失身份。卡思卡特上校伤心地感叹道:要是有人真的要成为他的必不可少的助手的话,他得是个富有、有教养、出身名门的人,要比科恩中校成熟得多,而且不会把他一心想当将军的强烈愿望看做是毫无意义的妄想。卡思卡特上校内心里怀疑科恩中校私下里就是这么看待他的。

 


    卡思卡特上校一心渴望当将军,以至于他宁愿尝试任何手段,甚至不惜利用宗教来达到目的。在他下令把战斗飞行的次数提高到六十次的那个星期的某天上午的后半晌,他把随军牧师叫到他的办公室里,突然朝下指着他办公桌上那份《星期六晚邮报》。上校穿着卡其布衬衫,领口大敞着,短而硬的黑须茬子映在雪白的颈子上,富有弹性的下唇下垂着。他是个从未被晒黑过的人,他总是尽可能地避开阳光,免得皮肤被晒黑。上校比牧师高出一个头还要多,身体宽出一倍,因此,在他那副趾高气扬的官架子面前,牧师感到弱不禁风,苍白无力。

    “看看这个,牧师,”卡思卡特上校吩咐道,一边把一支香烟塞进烟嘴里,一边满满当当地坐在他办公桌后的转椅里。“告诉我你是怎么认为的。”

    牧师顺从地低下头看了看那份打开着的杂志,看见是满满一页社论,内容是关于美国驻英格兰的一支轰炸机大队的随军牧师在每次战斗任务前都要在简令下达室里做祷告:当牧师意识到上校并不准备训斥他时,他高兴得几乎要哭起来。自从那个闹哄哄的夜晚,一级准尉怀特-哈尔福特朝穆达士上校的鼻子揍了一拳之后,卡思卡特上校遵照德里德尔将军的吩咐把他扔出军官俱乐部以来,他俩几乎还没说过话。牧师起初担心的是,他前天晚上未经允许又去了军官俱乐部,上校因此要训斥他。他是同约塞连和邓巴一道去的。那天晚上,这两个人突然来到林中空地上他的帐篷里要他同他们一起去,虽然他受到卡思卡特上校的威胁,但他觉得他宁愿冒惹卡思卡特上校生气的危险,也不愿谢绝这两位新朋友的盛情邀请。这两位新朋友是他几星期前去医院的一次访问中刚刚结识的。他的职责是同九百多名陌生的官兵生活在一起、并与他们保持最密切的关系,而这些官兵却认为他是个古怪的家伙,顺此,他势必会在人际交往中遇到不少令人意想不到的事情,而这两位朋友却卓有成效地帮他从其中解脱了出来。

 


    牧师眼睛盯着杂志,将每幅照片都看了两遍、并全神贯注地看了照片的说明,与此同时,他在反复思考如何回答上校的问题,并在头脑里组织好正确、完整的句子;默念了好几遍,最终才鼓起勇气开口回答。

    “我认为在每次飞行任务前做祷告是非常道德,且又十分值得赞美的做法,长官。”他胆怯地提出了自己的看法,然后等待着。

    “是的,”上校说,“不过我想知道,你是否认为做祷告在这儿会起作用。”

    “会的,长官,”牧师停了一会儿回答说,“我想一定会起作用的。”

    “那么,我倒想试一试。”上校那阴沉沉的、像淀粉做成的雪白的双颊突然泛起两片热情的红晕。他站起身来,激动地走来走去。

    “瞧,做祷告给在英国的这些人带来了多大的好处。《星期六晚邮报》上登了一幅上校的照片,每次执行任务前,他的随军牧师都要做祷告。如果祷告对他有作用,那对我们也应该有作用。假如我们也做祷告,他们也许会把我的照片也登在《星期六晚邮报》上。”

    上校又坐下来,脸上带着茫然的微笑想入非非起来。牧师感到不得要领,不知接下去该说什么才好。他那长方形的、苍白的脸上带着忧郁的表情,目光渐渐落在那几只装满了红色梨形番茄的大筐上。像这样的筐屋里有许多,里面装满了红色梨形番茄,沿墙四周摆了一排又一排。他假装在考虑问题。过了一会儿,他才意识到自己正凝视着一排排装在筐里的红色梨形番茄,注意力完全转移到了这个问题上:这一筐筐装得满满的红色梨形番茄摆在大队指挥官的办公室里干什么?他把做祷告的话题忘得一干二净。这时,卡思卡特上校也离开了话题,用温和的语调问道:

    “你想买一点吗,牧师?它们是从我和科恩中校在山上的农场里刚摘下来的。我可以优惠卖一筐给你。”

    “噢,不要,长官。我不想买。”

    “不买也没关系,”上校大度地安慰他说,“你不一定非要买。不管我们收多少米洛都乐意要。这些番茄是昨天刚刚摘下来的。你瞧,它们是多么结实饱满,和大姑娘的Rx房一样。”

    牧师脸红了,上校马上明白自己说错了话。他羞愧地低下头,臃肿的脸上热辣辣的。他的手指都变得迟顿、笨拙、不听使唤了。他恨透了牧师,就因为他是个牧师,才使他铸成说话粗俗的大错。他明白,他那个比喻若在其他任何情况下,都会被认为是趣味横生、温文尔雅的连珠妙语。他绞尽脑汁想找个办法让他们两人从这极为尴尬的场面中摆脱出来。办法他没想出来,却记起牧师只不过是个上尉而已。于是,他立刻挺直了身子,既像吃惊又像受到侮辱似的喘了口粗气。想到刚才一个年纪与自己差不多、军衔不过是上尉的人竟使自己蒙受羞辱,上校气得绷紧了脸,用杀气腾腾的眼神复仇似地扫了牧师一眼,吓得牧师哆嗦了起来。上校用愤怒、恶意和仇恨的目光,长时间一言不发地瞪着牧师,像个虐待狂似的以此来惩罚他。
 


    “我们刚才在谈另外一件事,”他最终尖刻地提醒牧师说,“我们刚才谈的事情不是漂亮姑娘的成熟、丰满的Rx房,而是另一件与此完全不相干的事。我们谈的是每次飞行任务前在简令下达室里举行宗教仪式的事。难道有理由说我们不能这么做?”

    “没有,长官,”牧师嘟哝着说。

    “那么,我们就从今天下午的飞行任务开始。”当上校谈起细节问题时,他原先那种敌意的态度也渐渐变得温和起来。“现在,我要你仔细考虑一下我们要说的祷告词。我不喜欢令人忧郁、悲伤的话。我想要你念些轻松愉快的祈祷文,让那些小伙子出去飞行时感觉良好。你明白我的意思吗?我不想听那种‘上帝的国度’或‘死亡的幽谷’之类的废话。那些话太消极。你干吗这样愁眉苦脸的?”

    “对不起,长官,”牧师结结巴巴地说,“就在你说刚才那些话时,我恰好想到了第二十三首赞美诗。”

    “那诗是怎么说的?”

    “就是你刚才提到的那首,长官。‘基督是我的牧羊人,我——’”“那是我刚才提到的一首。这首不要。你还有别的什么吗?”

    “‘啊,上帝,拯救我;洪水漫进了——’”。

    “洪水也不要,”上校断言道,一面把烟头轻弹进他那精制的黄铜烟灰缸里,然后对着烟嘴吹得呜呜响。“咱们为什么不试试跟音乐有关的祈祷文呢?柳树上的竖琴那首怎么样?”

    “那首诗里提到了巴比伦的河,长官,”牧师回答说,“……我等坐于彼处,当我等忆及郇山,就哭泣了。’”“郇山?咱们忘掉这段吧。我倒想知道那首诗是怎么被收进去的。你就不记得什么有趣的诗,文中没有洪水、幽谷和上帝吗?如果可能,我倒想完全避开宗教不谈。”

    牧师感到抱歉。“对不起,长官,但我所知道的所有祈祷文调子都相当低沉,而且至少要顺带提到上帝。”

    “那让咱们找些新的祷告词。那些家伙的埋怨已经够多的了,说我派遣他们执行任务前没有布道,没谈上帝、死亡或天堂什么的。咱们为什么不能采取一种更积极的方法?为什么不能祈祷一些美好的事情,比如说,把炸弹投得更密集些?难道咱们不能祈祷把炸弹投得更密集些吗?”

    “这个,可以,长官,我想可以,”牧师犹豫不决地答道,“假如那是您想做的一切,您甚至都用不着我。您自己就可以做。”
 


    “我知道我可以做,”上校尖刻地答道,“但你认为你在这儿是干什么的?我也可以为自己购买食物,但那是米洛的工作,那就是他为什么要为本地区每一个飞行大队购买食物的道理,你的工作是带领我们做祈祷。从现在起,每次执行飞行任务前,你将带领我们祈祷把炸弹投得更密集些。明白吗?我认为把炸弹投得更密集些倒的确是件值得祈祷的事。那样,佩克姆将军将会给我们所有的人嘉奖。佩克姆将军认为,当炸弹紧挨在一起爆炸时,从空中看到的景观就更漂亮。”

    “佩克姆将军,长官?”

    “是的,牧师,”上校回答说,看着牧师那副迷惑不解的神情,他像父亲似的咯咯地笑了起来。“我不想让这事传出去,但看来德里德尔将军最终要调走了,而佩克姆将军已被提名来接替他。坦率地说,我对发生这样的事情并不感到难过。佩克姆将军是个非常好的人,我相信我们大家在他的领导下处境会好得多。但另一方面,这种情况也许决不会发生,我们继续在德里德尔将军手下工作。坦率地说,我对此也不会感到难受,因为德里德尔将军也是个非常好的人。我想,我们大家在他的手下干,处境也将会好得多。我希望对这一切你能守口如瓶,牧师。我不想让他们两人中任何一位知道我在支持另一位。”

    “是,长官。”

    “那就好,”上校大声说道,然后快活地站起身来。“不过,这些闲谈是不可能让我们上《星期六晚邮报》的,不是吗,牧师?让我们看看还能想出什么办法来。顺便说一下,牧师,关于这事,事先一个字也不要透露给科恩中校。明白吗?”

    “明白,长官。”

    卡思卡特上校开始在那一筐筐红色梨形番茄与屋子中央的办公桌和木椅子之间留出来的那些狭窄的空道里来回走动着,一边走一边思考着。“我想我们得让你在门外等到作战命令下达完毕,因为一切消息都是保密的;等到丹比少校给大家对表时,我们再让你悄悄地进来。我想校对时间没什么可保密的。我们在日程安排上可以留一分半钟。一分半钟够了吗?”

    “够了,长官;如果不包括让那些无神论者从房间里出去并让士兵进来的时间。”

    卡思卡特上校停住了脚步。“什么无神论者?”他自卫似地吼道,一眨眼换了个人似的,摆出一副德行高尚、要与无神论者决斗的架势。“我的部队里决没有无神论者!无神论是违法的,不是吗?”

    “不是,长官。”

    “不违法?”上校吃惊地问,“那么,它就是非美活动,不是吗?”

    “我不太清楚,长官,”牧师回答说。

    “哼,我清楚!”上校断言说,“我不会为了迁就一小撮无耻的无神论者而毁掉我们的宗教仪式;他们不可能从我这儿得到任何特权。他们可以呆在原地和我们一同祈祷。怎么又冒出士兵的事?他妈的真见鬼,他们干吗要参加这个活动?”

    牧师感到脸红了。“对不起,长官。我刚才以为既然士兵将一同执行作战任务,您一定也想让他们一同参加祈祷。”

    “嗯,我可没这样想。他们有自己的上帝和牧师,不是吗?”

    “没有,长官。”

    “你说什么?你的意思是他们与我们向同一个上帝祈祷?”

    “是的,长官。”

    “那么上帝也听?”

    “我想是的,长官。”

    “呸,真见鬼,”上校评论说。他觉得荒唐可笑,暗自哼了一声。

    过了一会儿,他的情绪突然低落下去。他心神不安地用手抹了抹他那又短又黑的、有点灰白的卷发,关切地问道:“你真的认为让士兵进来是个好主意吗?”

    “我倒是认为只有这样才妥当,长官。”

    “我想把他们拒之门外。”上校说出了心里话。他一边来回走动,一边把指关节弄得啪啪响。“哦,别误解了我的意思,牧师。那并不是说我认为士兵卑微、平庸、低人一等,而是我们没有足够大的房间。不过,说实话,我不大希望当官的和当兵的在简令下达室里称兄道弟。我觉得他们在执行任务过程中见面的机会已经够多的了。你是了解的,我最要好的朋友中有几个就是士兵,但我跟他们要好也是有限度的。说真心话,牧师,你不会愿意你的妹妹嫁给一个士兵吧?”

    “我妹妹本人就是个士兵,长官,”牧师回答说。

    上校再次停住脚步,目光锐利地盯着牧师,想搞清楚牧师是不是在嘲弄他。“你那么说是什么意思,牧师?你是想开个玩笑?”

    “哦,不是,长官,”牧师带着极其不安的神色急忙解释说,“她是海军陆战队的一名军士长。”

    上校从未喜欢过牧师,现在就更讨厌他,不信任他了。他突然产生了一种强烈的可能遭到危险的预感。他怀疑牧师也在阴谋反对他,怀疑牧师那沉默寡言、平平淡淡的举止实际上是一种险恶的伪装,掩藏着内心深处熊熊燃烧着的、狡猾而肆无忌惮的野心。此时牧师有什么地方让人觉得可笑,上校很快就发现是什么问题了。

    牧师一直直挺挺地立正站在那里,原来上校忘了让他“稍息”了。就让他那么站着好了,上校带着报复的心理作出了决定,让他看看谁是长官,再说向他承认疏忽难免不丢架子。

    卡思卡特上校昏昏沉沉地走向窗前,他目光忧郁、呆滞,内心正在进行反省。他断定,士兵总是有叛逆之心的。他满面愁容地俯视着那个根据他的命令为他的司令部里的参谋们修建的飞靶射击场,想起了那个使他蒙受耻辱的下午。那天下午,德里德尔将军当着科恩中校和丹比少校的面毫不留情地把他训斥了一顿,并命令他把射击场对所有执行战斗任务的官兵开放。这个飞靶射击场对他来说真是件丑事,卡思卡特上校不能不得出这样的结论。他确信德里德尔将军从未忘掉这件事,不过他也确信德里德尔将军甚至根本就记不得这件事了。这件事的确很不公平,卡思卡特上校为此感到痛心,因为即便这件事如此使他丢人现眼,但修建一个飞靶射击场这个主意本身应该是他的荣耀。这个该死的射击场使他得到了多大好处,或是蒙受了多大损失,卡思卡特上校无法准确地估量出来。他希望科恩中校此时此刻就在他的办公室里,再帮他估量一下这件事的整个得失,减轻他的担忧。

    一切都使人不知所措,令人泄气。卡思卡特上校把烟嘴从嘴上拿下来,竖着放进了衬衫口袋里,然后开始难过地咬起两只手的指甲来。每个人都反对他,而使他伤心透顶的是科恩中校在这关键时刻也不在他身边,就祈祷的事帮他决定该怎么办。他对牧师几乎毫无信赖感,而且牧师只是个上尉。“你认为,”上校问道,“把士兵排除在外会不会影响我们取得成效的机会呢?”

    牧师犹豫起来,觉得这对自己又是个陌生的问题。“会的,长官,”他最后答道,“我认为,既然你们要祈祷把炸弹投得更密集些,那么这种做法可能会影响你们取得成效的机会。”

    “我根本没有考虑这个问题!”上校喊道,两只眼睛像两个小水坑似的闪动着。“你是说上帝甚至会决定惩罚我们,让我们把炸弹投得更加稀稀拉拉的?”

    “是的,长官,”牧师说,“有可能上帝会这样决定。”

    “那就见它的鬼去吧,”上校断言说,怒气冲冲地不想依赖任何人。“我搞这些该死的祈祷并不是要把事情搞得更糟。”他冷笑了一声,在办公桌后坐下来,然后把空烟嘴重又叼在嘴上,有好长时间一言不发地坐在那儿沉思苦想。“现在我考虑清楚了,”他既像是对牧师也像是对自己表白说,“不管怎样,让官兵向上帝祈祷可能不是好主意。《星期六晚邮报》的编辑们也许不会与我们合作。”

    上校懊悔地放弃了他的这个计划,因为这个计划是他独自一人设想出来的,他曾希望把它作为一个引人注目的例证拿出来给众人看一看,他并不真正需要科恩中校。既然现在这个计划不行了,他很乐意舍弃它,因为他制定这个计划时没有事先同科恩中校商量,因此他从一开始就担心这个计划有风险。他满意地长舒了一口气;现在既然他放弃了这个计划,他对自己的评价就更高了,因为他觉得他作出了一个非常明智的决定,而且最重要的是,他没有同科恩中校商量就作出了这一明智的决定。

    “还有其他事吗,长官?”牧师问道。

    “没啦,”卡思卡特上校回答说,“除非你还有什么别的建议。”

    “没有,长官。只是……”

    上校像是受到冒犯似的抬起头,带着冷淡而不信任的表情看着牧师。“只是什么,牧师?”

    “长官,”牧师说,“因为您把飞行任务增加到了六十次,有些官兵感到非常不安。他们要我把这件事向您反映一下。”

    上校缄口不语。牧师等在那儿,脸一直红到沙色的头发根旁;

    上校脸上毫无表情,用冷冷的目光死死地盯着牧师,使牧师长时间不安地扭动着身体。

    “告诉他们现在正在打仗,”他最后用平淡的语气劝告他说。

    “谢谢长官,我一定照办,”牧师极为感激地答道,因为上校终于开口说话了。“他们感到纳闷,你为什么不调一些正在非洲待命的预备机组人员来接替他们,然后让他们回家。”

    “那是个行政问题,”上校说,“不关他们的事。”他无精打采地指了指墙那边。“吃个红色梨形番茄吧,牧师。吃吧,我付钱。”

    “谢谢长官。长官——”

    “别客气。你住在外面林子里还喜欢吧,牧师?一切都挺不错吧?”

    “是的,长官。”

    “那就好。如果你需要什么,来找我们好了。”

    “是,长官。谢谢长官。长官——”

    “谢谢你来这儿,牧师,我现在有些工作要处理一下。如果你想到什么好主意能让我们的名字上《星期六晚邮报》的话,请告诉我,行吗?”

    “行,长官,我会的,”牧师用惊人的毅力和勇气打起精神,厚着脸说道,“我特别担心一名投弹手的情形,长官,他叫约塞连。”

    上校觉得这名字有些耳熟,吃惊地匆匆向上扫了一眼。“谁?”

    他惊恐地问道。

    “约塞连,长官。”

    “约塞连?”

    “是的,长官。是叫约塞连。他的情形很不好,长官。我担心他忍受不了多久,会挺而走险地做出一些出格的事来。”

    “这事确实吗,牧师?”

    “是的,长官。恐怕是的。”

    上校默默地考虑了一会。“告诉他应该相信上帝,”他最后劝告说。

    “谢谢长官,”牧师说,“我一定照办。”
