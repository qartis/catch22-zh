\chapter{亨格利·乔}
 
    亨格利-乔的确早已完成了五十次飞行任务,但这于他实在是毫无益处,他把行装打点好了,又等着回家。到了晚上,他就做可怖的噩梦,乱叫乱吼,闹得中队全体官兵无法入眠,只有赫普尔除外。

    赫普尔才十五岁,是个飞行员,当初是虚报了年龄才入伍的。他和自己那只宝贝猫跟亨格利-乔合住一顶帐篷。赫普尔睡觉一向容易惊醒,但他声称自己从未听见亨格利-乔惊叫过。亨格利-乔心里觉得难受。

    “那又怎么样呢?”丹尼卡医生满是怨恨地吼叫道,“不瞒你说,我以前可有钱啦,一年净赚五万美元,而且差不多是免税的,因为我要求来就诊的病人一概支付现金。此外,我还有世界上最有实力的同业协会做后盾。可你瞧瞧,后来出了什么事。就在我做好准备,开始积攒一笔钱的当儿,他们却炮制出什么法西斯主义,发动了一场令人悚然的战争,竟连我也没逃脱这场灾难。每天晚上听见亨格利-乔这样的家伙歇斯底里地喊叫,我就憋不住想大笑。我实在是憋不住想大笑。他觉得难受?我心里啥感受,他哪里晓得?”

    亨格利-乔自己多灾多难,实在是管不了丹尼卡医生心里究竟是什么感受。就拿那些噪声来说吧,即便是些很轻的噪声,也会让他勃然大怒。每当阿费口含唾沫,咂咂地一口一口抽烟斗,或是奥尔丁丁当当做些修补活计,或是麦克沃特玩二十一点或扑克牌时,每出一张牌总会摔得劈啪直响,或是多布斯一边笨手笨脚、跌跌撞撞四处乱跑,一边喀塔地牙齿直打战,这种时候,亨格利-乔便会直冲着他们吼叫,直到把嗓门吼哑了为止。亨格利-乔患的是运动表象型兴奋增盛症,性情激动暴躁。静静的房间里,手表有规律的嘀嗒声,似酷刑一般,猛击着他全无保护的脑袋。

    “听着,小家伙,”一天深夜,亨格利-乔没好气地跟赫普尔说,“假如你想在这顶帐篷里住下去,我喜欢怎么做,你就得怎么做:每天晚上,你必须得用羊毛袜裹好你自己的手表,然后把它放在帐篷那头你自己的床脚柜的最底层。”

    赫普尔很不服气地猛抬起下巴,让亨格利-乔明白,他可不是任人摆布的,于是,便不折不扣地依亨格利-乔的吩咐去做了。

 


    亨格利-乔是很神经质的,长得极瘦削,一副可怜相,脸色憔悴泛黄,两侧黑黢黢的太阳穴上,一根根抽搐着的青筋,似被切成若干的蛇段,在皮下蠕动。那张脸瘦得两颊凹陷,透着孤独凄凉,因久虑而显得阴沉,全无了光泽,恰似一座废弃的矿工城。亨格利-乔吃起来狼吞虎咽,总是不停地啃手指尖,说话结巴,有时又会因情绪激动而哽得说不出半句活来,身上处处发痒,又好出汗,嘴角常挂着口水。他时常背着一架复杂精密的黑色照相机,着了魔似地东奔西颠,一直想拍些女人的裸体照片。可是从未拍出一张照片。他总是忘记装胶卷、打灯光,或是忘记打开镜头盖。说服裸体女人摆各种姿势,这实在不是桩容易的事,不过,亨格利-乔在这方面倒是颇有些诀窍。

    “我可是个大名人,”他总会这么大声说道,“我是《生活》杂志大名鼎鼎的摄影记者,想给杂志的大封面拍张顶刮刮的照片。没错,没错,没错!好莱坞大明星。用不完的钞票,离不完的婚,整天跟男人寻欢作乐。”

    这世上,恐怕很少有女人能抵挡住这种甜言蜜语的劝诱。妓女总会急不可耐地一跃而起,只要是亨格利-乔的吩咐,不管摆的姿势有多怪,她们必定会全身心地投入。女人简直让亨格利-乔神魂颠倒。女性是他狂热崇拜的偶像。女人于他,是人间奇迹,美丽动人,令人赏心悦目,心醉神迷;是取乐的工具,威力之巨实在难以估量,欲望之强令人无法招架,造就得又是这般精美,不足道的卑劣男人是没资格享用的。在他看来,女人赤裸了玉体任他摆弄,只是一个天大的疏忽——终究会迅速得到纠正。因此,他总是不得不赶在别人获悉内情匆匆把她们带走之前,尽一切可能以极短的时间,充分利用她们的肉体。究竟是玩弄她们,还是给她们拍照,他一直举棋不定,因为他发觉这两件事实在无法同时进行。其实,他开始觉得,这两桩事体他几乎一桩也干不了。原因是,他自始至终摆脱不了行事匆忙草率的积习,结果导致了他的办事能力极度低下,老是东一郎头,西一棒子。照片是一张也没拍成,到了手的女人一个也没玩成。令人奇怪的是,亨格利-乔服役前确曾当过《生活》杂志的摄影记者。

 


    如今,他可是位英雄。在约塞连眼里,他是最了不起的空军英雄,因为他完成作战飞行任务的次数超过了空军里的其他英雄。他已经完成了六次作战飞行任务。亨格利-乔完成第一次作战飞行任务时,那时的规定要求每人必须完成二十五次飞行任务。只要完成了这二十五次飞行任务,他便可以打点好行装,喜滋滋地给家里写信报喜讯,然后开始兴致勃勃地缠住陶塞军士,探问让他轮换调防回美国的命令是否下达。待命期间,他每天在作战指挥室门口周围,极有节奏地跳着曳步舞。每每有人路过,他便扯大了嗓门,没完没了地说俏皮话;每次见到陶塞军士匆匆走出中队办公室,就打趣地骂他是讨厌的狗杂种。

    驻屯萨莱诺滩头堡的一周内,亨格利-乔就完成了最初规定的二十五次飞行任务。当时,约塞连因染上了淋病住在医院治疗。

    这种花柳病,是一次——他正在执行前往马拉喀什空运补给的低空飞行任务——他跟一名陆军妇女队队员在灌木丛里野合时传染上的。后来,约塞连全力以赴,拼命追赶亨格利-乔,结果几乎就让他赶上了,六天里,他完成了六次飞行任务。可是,他的第二十三次任务是飞往阿雷佐,内弗斯上校便是在那儿阵亡的。那次任务完成以后,再飞两次,他就可以回家了。可是到了第二天,卡思卡特上校着一身崭新的制服来到中队,摆出一副傲慢专横不可一世的模样。他将规定的飞行次数从二十五提高到三十,以此来庆贺自己接任大队指挥官的职位。亨格利-乔解开行装,把写给家里的报喜信重新又写了一遍。他不再兴致勃勃地缠住陶塞军士。他开始仇恨陶塞军士,极凶狠地将一切归罪于陶塞军士,即便他心里很清楚,卡思卡特上校的到任,或是遣送他们回国的命令一直搁着不下达——本来完全可以让他提早七天回家,逃掉后来新增的五次飞行任务,这一切跟陶塞军士实在是毫不相干的。

 


    亨格利-乔再也经受不住等待回国命令时的极度紧张,每每完成又一次飞行任务,他的身心健康便迅速崩溃。每次被撤下不执行作战任务,他就举行一个规模不小的酒会,请上自己那一小帮朋友聚一聚。他打开一瓶瓶波旁威士忌——是他每周四天驾驶军邮班机巡回递送邮件时想了法子才买到的——以飨朋友。随后,他又是笑又是唱,还跳起曳步舞,大声喊叫,似过节一般陶醉,欣喜若狂,直到后来睡意袭来,再也支撑不住,方才安静入睡。待约塞连、内特利和邓巴刚安顿好他上床,他就开始尖声叫喊。第二天上午,他走出帐篷,形容枯槁,流出恐惧和负疚的神情,整个人看似一座蛀空的建筑物,只剩下个空骨架,摇摇欲坠,一触便会倒坍。

    每当亨格利-乔不再执行作战飞行任务,再次等待永远等不来的回国命令,他便受尽了痛苦的折磨。期间,他在中队度过的每一个晚上,那一个个噩梦总是准时出现在他的梦乡,就同天体的运行一样正点,不差分秒。亨格利-乔每做噩梦,必定歇斯底里地尖叫,扰得中队里像多布斯和弗卢姆上尉那些神经过敏的人心绪不宁,结果,他们也开始做噩梦,歇斯底里地尖叫。于是,每天晚上,他们便从中队各个不同的角落把各种尖厉的下流话吐入空中,在黑夜里回响着,颇有些趣味,仿佛发情的鸟交尾时的欢叫。在科恩中校看来,这是梅杰少校的中队里露出的不良倾向,于是,他便采取了果断行动,决定杜绝这股苗头。他的措施是,下令亨格利-乔每周驾驶一次军邮班机巡回递送邮件,这样,有四个晚上他就没法在中队过夜了。这一补救办法同科恩中校采取的所有补救办法一样,的确很奏效。

    每次卡思卡特上校增加飞行任务的次数并让亨格利-乔重返战斗岗位时,亨格利-乔便不再梦魇。他只是宽心地微微一笑,又恢复了平常的恐惧状态。约塞连琢磨亨格利-乔那张皱缩的脸,就像是在读报纸上的一条大标题。每当亨格利-乔神情阴郁,表明一切正常,可一旦他兴致勃勃,那就说明出了什么麻烦事。亨格利-乔这种阴阳错乱的反应,在大伙看来,确实是个怪现象,只有他本人对此断然否认。

    “谁做梦?”当约塞连问他都做些什么梦时,亨格利-乔反问道。

    “乔,你干吗不去丹尼卡医生那里看看?”约塞连劝说道。

    “我干吗非得去看丹尼卡医生?我又没病。”

    “你不是老做噩梦吗?”

    “我可没做噩梦。”亨格利-乔说了个谎。

    “或许丹尼卡医生有办法治那些噩梦。”

    “做噩梦又不是什么病,”亨格利-乔答道,“哪个不做噩梦?”

    约塞连心想,这下他可上了圈套。“你是不是每天晚上做噩梦?”他问。

    “难道每天晚上做噩梦就不成吗?”亨格利-乔反诘道。

    亨格利-乔这一反诘,突然让约塞连茅塞顿开。他问得没错,为什么就不能天天晚上做噩梦?这样,每天晚上梦魇时痛苦地狂叫,也就可以理解了。比起阿普尔比来,这就更容易理解了。阿普尔比一向严守规章制度。在一次前往海外执行飞行任务途中,他曾授命克拉夫特,下令约塞连吞服阿的平药片,尽管当时他和约塞连彼此早已不再搭腔。亨格利-乔比克拉夫特要懂道理得多。克拉夫特已经不在人世。当时在弗拉拉,约塞连再一次把自己小队的六架飞机导入目标上空,一台发动机爆炸了,克拉夫特就这样死于非命。飞行大队连续轰炸了七天,还是没有炸悼弗拉拉的那座桥梁,尽管他们使用的轰炸瞄准器十分精密,可以在四万英尺的高空把一枚枚炸弹扔进一只腌菜桶。早一个星期前,卡思卡特上校可是自告奋勇要部下在二十四小时内炸毁那座桥。克拉夫特是宾夕法尼亚州人,小伙子长得极瘦弱,没丝毫要害人的坏心眼。他唯一的希望就是讨人喜欢,然而,就连这一点点有辱人格的卑贱的愿望,也终究注定要破灭的。他死了,没有受到别人的怜爱,就像熊熊燃烧的烈火堆上的一块血淋淋的炭渣,无声无息地离开了人世。就在那架只剩一片机翼的飞机快速坠落的当儿,谁也不曾听见他在生命最后的宝贵瞬间里说了些什么。克拉夫特与世靡争地生活了一小段时间,然后到了第七天,在弗拉拉上空随烈火一起消逝。当时,上帝正在安息,麦克沃特将飞机调了头,约塞连引导他飞至目标上空,作又一轮轰炸飞行,因为第一轮轰炸飞行时,阿费慌了手脚,结果,约塞连没能扔下炸弹。

    “我想我们只得再往回飞了,是不是?”麦克沃特通过对讲机闷闷不乐地说了一句。

    “我想是吧,”约塞连说。

    “是吗?”麦克沃特问道。

    “是的。”

    “那好吧,”麦克沃特说,“只好如此了。”

    他俩重新飞回目标上空,而其他小队的飞机在远处盘旋了一圈后,便安全飞走了。这时,地面上赫尔曼-戈林师的每一门火炮,便都一齐对准他俩猛烈开炮。

    卡思卡待上校是个极果敢的人。只要有什么现成的轰炸目标,他向来毫不迟疑地主动提出请求,让自己的部下前去摧毁。在他的飞行大队看来,任何一个目标,不管有多危险,都是攻无不克的,正如对阿普尔比来说,在乒乓球台上没有什么险球是救不起的。阿普尔比是位很出色的飞行员,又是一名球艺超绝的乒乓球选手,尽管眼睛里有苍蝇,却从未失过一球。对阿普尔比来说,要让对手输得丢尽脸面,发二十一次球便足够了。他的乒乓球球技实在是高超非凡。只要举行球赛,他必定是场场都赢。后来,有一天晚上,奥尔喝过杜松子酒和威士忌后,醉醺醺地跑去找阿普尔比打乒乓球。开局时,他接连发的头五个球,全让阿普尔比给猛抽了回去,于是,他便拿起球拍,把阿普尔比的前额砸了个口子。奥尔扔掉球拍,纵身一跃,跳到乒乓球台上,紧接着一个急行跳远,从台子的另一端猛跳了下去;两脚恰好踩在了阿普尔比的脸上,立时一片混乱。阿普尔比差不多花了足足一分钟,才好不容易挣脱掉奥尔的拳打脚踢,摸索着爬了起来,一手揪住奥尔的衬衣前胸,把他提了起来,另一手握成拳头缩回去,正欲猛力击去,把他打死。就在这当儿,约塞连跨步上前,把奥尔从他身边拉走。这一夜对阿普尔比来说,是充满意外的一夜。阿普尔比和约塞连一样魁梧粗壮,他挥起拳,狠狠地打了约塞连一拳。这一拳打得一级准尉怀特-哈尔福特乐不可支,于是,他转过身,照准穆达士上校的鼻子也重重击了一拳。德里德尔将军可高兴极了,便让卡思卡特上校把随军牧师逐出军官俱乐部,又命令一级准尉怀特-哈尔福特搬进丹尼卡医生的帐篷,这样,每天二十四小时他就可以得到医生的照料,身体健康也有了保障,这样,德里德尔将军什么时候要他拳打穆达士上校的鼻子,他便可以再应付了。有的时候,德里德尔将军带着穆达士上校和护士,特地从联队司令部下来,只是想让一级准尉怀特-哈尔福特在他女婿的鼻子上狠狠打一拳。
 


    一级准尉怀特-哈尔福特是极愿意留在他跟弗卢姆上尉合住的那间活动房里的。弗卢姆上尉是中队的新闻发布官,不爱说笑,性情烦闷。每天晚上,他总要花上一大半时间冲洗白天拍摄的照片,然后跟他的宣传稿一同发出去。他每天晚上尽量留在暗房工作,之后,便躺在自己的帆布床上,交叉着食指和中指,脖子上缠了只兔子的后足,想足了法子不让自己睡着。跟一级准尉怀特-哈尔福特合住,他始终处于极度的恐惧之中。他脑子里老是困扰着一个念头:说不定哪个晚上,一级准尉怀特-哈尔福特会趁他酣睡之际,悄悄走到他的床前,一刀切开他的咽喉。他之所以生出这么个念头,也全因一级准尉怀特-哈尔福特本人。有天晚上,弗卢姆上尉正打着盹儿,一级准尉怀特-哈尔福特确实蹑手蹑脚地走到他的床前,极凶险地用尖利的嘘声威胁道:总有一天晚上,趁他,弗卢姆上尉,熟睡的时候,他,一级准尉怀特-哈尔福特,会一刀割开他的咽喉。弗卢姆上尉吓得浑身直冒冷汗,睁大了双眼,抬起头,直愣愣地注视着一级准尉怀特-哈尔福特那双离他仅几英寸远的闪闪发亮的醉眼。

    “为什么?”弗卢姆上尉最终用低沉而沙哑的声音总算问了一句。

    “为什么不?”一级准尉怀特-哈尔福特的答复倒是极干脆。

    此后的每个晚上,弗卢姆上尉尽量迫使自己不睡着。亨格利-乔的噩梦着实给他帮了极大的忙。他一夜夜专注地倾听亨格利-乔疯狂般的号叫,渐渐地仇恨起他来了,真希望哪天晚上,一级准尉怀特-哈尔福特会悄悄地走到他的床前,一刀割开他的咽喉。其实,大多数晚上,弗卢姆上尉睡得很沉,只是梦见自己醒着。这些梦极其真实,结果,每天早晨他从睡梦中醒来时,已是筋疲力尽,顷刻又复睡去。

    自弗卢姆上尉发生惊人的巨变后,一级准尉怀特-哈尔福特渐渐地喜欢上他了。那天晚上,弗卢姆上尉上床时,还相当活泼开朗,可第二天上午起身时,却变得阴郁寡欢,性格内向。一级准尉怀特-哈尔福特很自豪地视这个新的弗卢姆上尉为自己创造的作品。他从未打算要割断弗卢姆上尉的咽喉。他扬言这么做,就如同他说要死于肺炎、要给穆达士上校的鼻子狠狠一拳或者要同丹尼卡医生比角力,全都只是想开个玩笑而已。每天晚上,他醉醺醺地蹒跚着走进帐篷,想做的头一桩事,便是即刻睡觉,可亨格利-乔经常让他入睡不得。亨格利-乔梦魇时歇斯底里地狂叫,吵得他烦躁不安。于是,他便经常希望有人悄悄溜进亨格利-乔的帐篷,从他脸上把赫普尔的猫拎走,再一刀割开他的咽喉。这样,中队上下除弗卢姆上尉外,就可以好好睡一个安稳觉了。
 


    一级准尉怀特-哈尔福特不时地替德里德尔将军重重拳击穆达士上校的鼻子,纵然如此,他依旧还是个局外人。中队长梅杰少校也是个局外人。梅杰少校在从卡思卡特上校那里得知自己晋升中队长的同时,发现自己本是个局外人。杜鲁斯少校于佩鲁贾上空阵亡后的第二天,卡思卡特上校坐了他那辆特大马力的吉普车,飞速驶进中队驻地。卡思卡特上校在离那条铁路壕沟几英寸的地方,嘎然把车刹住。壕沟就横在吉普车和那片倾斜的篮球场之间。

    卡思卡特上校一到,梅杰少校便遭到那些球友——几乎和他交上了朋友——的拳打脚踢,左推右搡,还有乱石的袭击,最终,被逐出了球场;

    “你现在是新任的中队长,”卡思卡特上校隔着壕沟朝梅杰少校高声喊道,“不过,别以为这有什么了不起,因为这算不得什么。

    只不过是由你来担任新的中队长罢了。”

    卡思卡特上校来得突然,去得也同样突然。说罢,他就猛地掉转车头,车轮一阵飞转,扬起一片细砂砾,吹了梅杰少校一脸,于是,车便轰隆隆地开走了。这个消息把梅杰少校惊呆了。他呆呆地站在那儿,一句话也说不出来,瘦长的身体愈发显得难看,两只长手捧着一只磨损了的破篮球,看着卡思卡特上校如此迅速播下的仇恨的种子在他身边的士兵们心中扎了根。而这些弟兄一直跟他打篮球,又允许他像先前谁都乐意的那样跟他们交朋友。梅杰少校两眼毫无光泽,眼白增大,模糊不清,嘴巴翕动着,极想说些什么,可就是出不了声,那种熟悉的、驱赶不了的孤寂,再一次飘来,似令人窒息的烟雾,将他团团困住。

    像大队司令部的其他所有军官——丹比少校除外——一样,卡思卡特上校亦极具民主精神:他认为,人生来是平等的。所以,他便以同样的热情,一脚踢开了大队司令部以外的所有官兵。不过,他信任自己的部下。正如他在简令下达室常跟他们说的那样,他相信,同其他任何部队相比,他们要强得多,至少可以多完成十次飞行任务。同时,他还认为,谁要是对部下没有这样的信心,他就可以滚出去。不过,他们要滚出去,唯一的办法,就像约塞连飞去见前一等兵温特格林时探听到的那样,便是完成这另增的十次飞行任务。

    “我还是搞不明白,”约塞连抗辩道,“丹尼卡医生究竟是错还是对?”

    “他说是多少次?”

    “四十次。”

    “丹尼卡说的没错,”前一等兵温特格林认可道,“就第二十六空军司令部来说,只要完成四十次飞行任务就可以了。”

    约塞连听了心花怒放。“这么说,我可以回家咯?我已经飞了四十八次。”

    “不行,你还不能回家,”前一等兵温特格林纠正道,“你不会是疯了吧?”

    “为什么不能回家?”

    “第二十二条军规规定这样。”

    “第二十二条军规?”约塞连很感吃惊。“第二十二条军规跟回家到底有什么关系?”

    “第二十二条军规规定,”亨格利-乔开飞机送约塞连回皮亚诺萨岛后,丹尼卡医生极耐心地答复他说,“你自始至终得服从指挥官的命令。”

    “但第二十六空军司令部说,我完成四十次飞行任务就可以回家了。”

    “可他们没说你必须回家。军规明文规定,你必须服从每一个命令。圈套便在这里。即便上校违反了第二十六空军司令部的命令,非要你继续飞行不可,你还是得执行任务,否则,你违抗他的命令,便是犯罪。而且第二十七空军司令部必定会问你的罪。”

    约塞连彻底灰了心。“这么说,我必须完成规定的五十次飞行任务咯?”他极伤心地问。

    “是五十五次,”丹尼卡医生纠正道。

    “什么五十五次?”

    “上校现在要求你们大家完成五十五次飞行任务。”

    亨格利-乔听了丹尼卡医生的后,如释重负地深叹了一口气,咧嘴笑了笑。约塞连一把揪住亨格利-乔的脖子;迫使他立刻开飞机跟他一块回去见前一等兵温特格林。

    “要是我拒飞的话,”约塞连极信任地问道,“他们会怎么对待我?”

    “我们或许会毙了你,”前一等兵温特格林回答他说。

    “我们?”约塞连吃惊地大声叫道,“你说我们是什么意思?你什么时候站在他们一边了?”

    “要是你给毙了,你指望我跟谁站在一边。”前一等兵温特格林反驳道。

    约塞连畏缩了。卡思卡特上校又一次让他上了圈套
