\chapter{米洛市长}
 
    就是在执行那次飞行任务时,约塞连被吓得惊慌失措。约塞连之所以会在执行轰炸阿维尼翁的任务时吓得惊慌失措,是因为斯诺登被吓破了胆,而斯诺登之所以吓破了胆,是因为那天他们的驾驶员是赫普尔,而赫普尔的年纪只有十五岁。他们的副驾驶是多布斯,而多布斯这人则更糟糕,他竟要约塞连同他一起去谋杀卡思卡特上校。约塞连知道赫普尔是个优秀的驾驶员,但他还只是个孩子,并且多布斯对他也毫无信心。于是,当他们扔完炸弹之后,多布斯一声不吭地一把夺过了操纵杆。他就这么着在半空中突然发起疯来,使飞机向下栽去,那震耳欲聋的声音和快得难以描绘的速度令人心惊肉跳,丧魂落魄。这不要命的俯冲把约塞连的耳机连接线扯断了,使他的头抵在了机头的舱顶,无能为力地悬挂着那儿。

    哦,上帝!当约塞连感到他们都在向下坠落时,他尖叫起来,可却发不出声音。哦,上帝!哦,上帝!哦,上帝!哦,上帝!他尖声哀求着,可因飞机急速下坠,他连嘴都张不开。他头抵着舱顶,身体处于失重状态,晃来晃去。后来,赫普尔设法夺回了操纵杆,在一片疯狂猛烈的高射炮的火网中拉平了飞机。那高射炮火组成了一个两边是悬崖峭壁的大峡谷,他们刚刚从里面爬出来,此刻又得逃命了。几乎就是同时,砰的一声,飞机舱盖上的有机玻璃被打了一个拳头那么大的洞。只见闪闪发光的碎片四下飞溅,约塞连的两颊一阵刺痛。没有出血。

    “怎么回事?怎么回事?”他喊了起来,可却听不见自己的声音,禁不住浑身剧烈地颤抖起来。他的对讲机里寂静无声,他被这吓得要死。他趴跪在地上,害怕得要命,一动也不敢动,活像一只中了圈套的老鼠,呆在那里,大气不敢出一下。后来,他终于瞥见自己耳机上那圆柱形的插头一闪一闪地在眼前晃荡,于是赶紧用颤抖的手指将其重新插回到插孔里,此时高射炮火在他四周砰砰作响,并形成了一朵朵蘑菇状的云烟,他惊恐万状地一再尖叫着:“啊,上帝!

    啊,上帝!”

    当约塞连把插头插回到对讲机的插孔后,他又能听见声音了。

    他听到多布斯正在哭泣。

    “救救他,救救他吧,”多布斯呜咽着喊道,“救救他,救救他。”

    “救救谁、救救谁呀?”约塞连朝他回叫着,“救谁呀?”

    “轰炸员,轰炸员,”多布斯喊道,“他那里没有回答。快救轰炸员,快救轰炸员吧。”

    “我就是轰炸员,”约塞连大叫着口答道,“我就是轰炸员。我没事,我没事。”

    “那就快救救他,救救他吧,”多布斯哭喊道,“救救他,救救他吧。”

    “救谁呀,救谁?”

    “救那个报务员兼炮手,”多布斯哀求道,“快救救咱们的报务灵兼炮手吧。”

    “我冷。”斯诺登在对讲机里用微弱的声音啜泣着,接着又发出一阵痛苦的哀怨声,“请救救我吧,我好冷啊。”

    约塞连匍匐着通过了爬行通道,爬上了弹舱,然后爬进飞机的尾舱,斯诺登就躺在那儿的地板上。他受了伤,躺在一片黄色的日光中,冻得快要死了。在他身旁,那个新来的尾炮手直挺挺地躺在那里,已经昏死过去。

    多布斯是世界上最差劲的飞行员,这点他自己也知道。他本是一个身强力壮的小伙子,可现在身体却全垮了。他总是千方百计地想说服他的上司,让他们相信他已不再适合驾驶飞机了,可是他的上司都不听他的。就在宣布飞行次数提高到六十次的那天,多布斯偷偷地溜进了约塞连的帐篷。当时奥尔正好出去找垫圈了,他就向约塞连吐露了他制定的暗杀卡思卡特上校的阴谋。他说他需要约塞连的协助。

    “你想让咱俩把他给蓄意谋杀掉?”约塞连可不赞成这主意。

    “没错。”多布斯十分同意他的说法,脸上挂着乐观的微笑。约塞连这么快就领会了他的意图,他更是受到了鼓舞。“咱们就用那枝卢格尔手枪把他给毙了。这枪是我从西西里带回来的,谁也不知道我有这家伙。”

    “我想我不能这么干。”约塞连在心里将这主意默默地掂量了一番,得出了这一结论。

    多布斯大感惊讶:“为什么不能?”

    “你瞧,对我来说,最能让我开心的事就是有一天这个狗娘养的会赶上飞机坠毁的事故,让他跌断脖子,或跌死掉。要不就是能看到另外的什么人把他一枪给毙了。可我想我是不能去杀他。”

    “可他会杀你,”多布斯争辩道,“其实,这都是你告诉我的,说他老是不停地让咱们去作战,就是想让咱们统统去死。”

    “可我想我不能也这么去对待他。我认为他也有活的权利。”

    “可他老想剥夺你我的生存权利,只要他这么做,那他就无权再活下去。你这是怎么了?”多布斯感到大惑不解。“我以前老是听到你和克莱文杰为这事争个不歇。可现在你瞧瞧克莱文杰怎么了。

    他就死在了那块云团里。”

    “你别嚷好不好?”约塞连嘴里发着“嘘——”的声音,示意他小声点。

    “我没嚷!”多布斯喊的声音更高了,他心里充满了希望进行一场革命的狂热。此时他已是一把眼泪一把鼻涕的了,他那颤动不已的深红色的下唇上溅满了起沫的泪水和鼻涕。“在咱们这个大队里,肯定有将近一百个人已经完成五十五次飞行任务了,可到了这时卡思卡特却又把这数目提高到了六十。像你这样还要再飞上几次才满五十五次的人至少还有一百个。要是我们让他一直这样干下去,他就会把咱们全部给害死掉。我们一定得先把他给干掉才行。”

    约塞连毫无表情地点了点头,根本没有明确表态。“你认为咱们干了这事以后能逃脱?”

    “我已把一切都计划好了。我——”

    “看在基督的分上,别这么大声嚷嚷。”

    “我没嚷,我已经——”

    “你别嚷了,好不好?”

    “我已经把一切都计划好了,”多布斯小声地说,一面用手紧紧地抓住奥尔的吊床边,不让两手晃动,由于用力,他的指关节都发白了。“星期四早上,当他从山上他的那所该死的农舍返回的时候,我就悄悄地穿过树林,溜到公路的那个急转弯处,在树丛中藏起来。他的车到了那儿非减速不可,而我呆在那里能清楚地看到公路两头的动静,以弄清确实没有其他人在附近。等看到他的车子过来了,我就把一根大木头推到公路上去,让他的吉普车停下来。那时我就端着我的那枝卢格尔手枪从树丛里走出来,对着他的脑袋开火,直到把他打死为止。然后我就把枪埋起来,再穿过树林返回中队,像其他人一样,去忙活我自己的事。这样干能出什么差错呢?”

    约塞连聚精会神地听着他讲的每一个步骤。“我打哪儿能插得上手呢?”他迷惑不解地问。

    “这事没你的帮助我干不了,”多布斯解释道,“我需要你对我说声‘就这么干吧’。”

    约塞连觉得他的话简直难以置信。“你要我做的就是这个?就要我对你说声‘干吧’?”

    “我只需要你做这个,”多布斯回答,“你只要说声干,那后天我就独自一人把他的脑浆给打出来。”由于感情激动,他的声音越来越急,此时又变得响亮起来。“既然咱们干了,那我也想在科恩中校的脑袋上也来上一枪。不过如果你不反对的话,我倒想饶了丹比少校。这以后我还想杀掉阿普尔比和哈弗迈耶。干掉阿普尔比和哈弗迈耶之后,我还要杀麦克沃特。”

    “麦克沃特?”约塞连叫道,吓得几乎跳起来。“麦克沃特是我的朋友。你干吗要对麦克沃特下手?”

    “我不知道,”多布斯坦白说,一脸的慌乱和尬尴。“我只是想既然咱们要干掉阿普尔比和哈弗迈耶,那咱们不妨也把麦克沃特给干掉。你不想杀麦克沃特,是吗?”

    约塞连采取了坚定的立场。“你瞧,假如你不再将这事在这整个岛上乱嚷嚷,假如你坚持只干掉卡思卡特上校,那我还可能对这事感兴趣。可如果你想把这事搞成一场屠杀,那你还是把我忘掉的好。”

    “好吧,好吧。”多布斯竭力想安抚约塞连。“只杀卡思卡特上校一人。我应该去干吗?对我说声‘干吧’。”

    约塞连摇了摇头。“我想我不能叫你去干。”

    多布斯激动得像要发狂。“我愿意做点让步,”他强烈地恳求道,“你不必对我说‘干’。你只要对我说一声这是个好主意就行了。

    行吗?这是个好主意吗?”

    约塞连还是摇头。“要是你根本不告诉我就直接动手,把这事给干了,那倒是个极好的主意。可现在太晚了。有关这事我对你没什么好说的。给我点时间,没准我会改主意的。”

    “那会来不及的。”

    约塞连仍一个劲地摇头,多布斯不禁大为失望。他在那里坐了一会,一脸的沮丧,然后突然跳了起来,踏着重重的脚步走了出去。

    他又起了一阵冲动,想去说服丹尼卡医生支持自己。在他转身时,他的臀部把约塞连的脸盆架给撞翻了,脚又绊在了奥尔还没做好的电炉丝上。丹尼卡医生不耐烦地连连点头,以此抵挡住了多布斯的咆哮和指手划脚的指责,然后打发他到医务室去把他的症状说给格斯和韦斯听。到了那里,他刚一开口说话,格斯和韦斯就立即在他的牙床上涂满了龙胆紫溶液。接着他俩又将他的脚趾也涂紫了。当他再次张嘴想要抗议时,他们又将一粒轻度腹泻药片塞进了他的喉咙,然后便把他打发走了。
 


    多布斯的情况比亨格利-乔要糟。亨格利-乔不做噩梦的时候,至少还可以执行飞行任务。多布斯几乎和奥尔一样糟糕。奥尔看上去总是乐呵呵的,时常像发神经似的咯咯地傻笑,那长得歪歪扭扭的龅牙不住地颤动着,活像一只发育不全、龇牙裂嘴的云雀。

    上级已准许他前往开罗休假,同去的还有米洛和约塞连。他们去那里是为了采购鸡蛋,可是米洛却买了棉花。米洛在黎明时分起飞赶往伊斯但布尔,飞机里装满了具有异国情调的有柄带脚的煎锅和青里透红的香蕉,连飞机的炮塔里都塞得满满的。奥尔是约塞连遇到过的最难看的怪人之一,可他也挺吸引人的。他的脸粗糙且凸凹不平,淡褐色的眼睛从眼眶中暴出来,活像一对褐色的半粒子弹头。他那一头杂色相间的浓密头发是波浪式的,倾斜向上直到头顶心,就像一顶上过油的小帐篷。他几乎每次上了天都要出事,不是被击落坠入水中,就是一个引擎被人打中失灵。那天他们的飞机起飞后是向着那不勒斯出发的,可不曾想到却在西西里降落了。一路上奥尔像个疯子似的使劲地拉约塞连的胳臂,要他在那里降落。

    他们上那儿是为了找那个鬼精的、会抽雪茄的年仅十岁的皮条客。

    这小子有两个十二岁的处女姐姐,她们在市区的一家旅馆门口等候着他们。那家旅馆有一间房专供米洛使用。约塞连毅然地从奥尔身边走开,独自向远方眺望着。此时他眺望到的不是维苏威火山,而是埃特纳火山,眼神里既透着几分关注,也透着几分迷茫。

    他心里纳闷,他们不去那不勒斯而到西西里来干什么。与此同时,奥尔简直是欲火难熬。他一个劲地傻笑着,结结巴已地吵个不歇,恳求约塞连同他一道跟着那个一肚子鬼主意、年仅十岁的皮条客去找他那两个十二岁的处女姐姐。其实,她们既不是处女,也不是他姐姐。她们实际上已有二十八岁了。

    “同他去吧。”米洛简洁地给约塞连下达了指令。“别忘了你的使命。”

    “好吧。”想到自己的使命,约塞连叹了口气,终于让了步。“可至少先让我试试找间旅馆,这样在完事之后我就可以好好地睡上一夜了。”

    “你可以和那些姑娘好好地睡上一夜,”米洛用同样狡黠的语气答道,“只要别把你的使命给忘了就行了。”

    可那一夜约塞连和奥尔根本就没睡。他们发现自己和那两个自称十二岁实际上已二十八岁的妓女同挤在一张床上。弄了半天那两个妓女原来是两个油腻腻、长着一身肥肉的女人。她俩夜里就是不让他们睡觉,吵着要交换搭档。约塞连不一会就迷迷糊糊的了,根本没注意到那个挤在他身上的胖女人整整一夜头上都裹着一条米色头巾。第二天早上很晚的时候,那个一肚子鬼心眼、嘴里总叼着古巴雪茄的十岁皮条客突然像个畜牲似的说翻脸就翻脸,一把扯下了那条头巾。顿时,这个女人那颗丑陋的奇形怪状的光秃秃的头颅就一览无遗地暴露在了西西里的光天化日之下。她曾陪德国人睡过觉,为此她的那些复仇心重的邻居将她的头给剃得亮光光的,几乎要露出了骨头。那姑娘带着女性特有的愤怒,一面用尖厉刺耳的声音大叫着,一面拖着肥胖的身子摇摇摆摆地追赶着那个十岁的一肚子坏水的皮条客,那情形甚是滑稽。她那吓人的、颜色苍白且受到了极大冒犯的头皮,环绕着她那张同样古怪的黑肉瘤似的脸,十分可笑地上下滑动着,活像一块经过漂白但却仍然污秽不堪的东西。约塞连以前从未见过如此光秃秃的脑袋。那个小皮条客用一根手指高高地挑着那块头巾,让它转个不停,像举着一件战利品似的。他始终在离她的手指头几英寸的地方蹦着,跳着,让她够不着,引得她在广场上团团转,干着急,把挤在广场上看热闹的人逗得大笑不止,有人还指着约塞连嘲笑他。这时米洛挂着一脸的严厉急匆匆地大步走来。他咂起嘴唇,对眼前这个伤风败俗、轻薄无聊、不成体统的场面深表不满。米洛坚持立即离开这里前往马耳他。

    “可我们困得要命,”奥尔嘀咕道。

    “那只能怪你们自己。”米洛自认自己很有道德,故而这样训斥他俩。“要是你们呆在旅馆里过夜,不和这些淫荡的女人鬼混,那么你们今天就会和我一样有精神了。”

    “是你要我们跟她们走的,”,约塞连用责备的口气反驳道,“而且我们也找不到旅馆房间。只有你一人能弄到房间。”

    “那也不能怪我呀,”米洛傲慢地解释说,“我哪里知道鹰嘴豆上市时,会有那么多的买主涌到这城里来呀?”

    “你当然知道,”,约塞连指责道,“这就是为什么我们不去西西里,而跑到那不勒斯来的原因。你他妈可能已经把整架飞机都塞满了鹰嘴豆。”

    “嘘嘘嘘——!”米洛神情严厉地向他发出警告,一面意味深长地朝奥尔瞥了一眼。“别忘了你的使命。”

    当他们来到机场准备飞往马耳他时,飞机的弹舱、后舱和尾舱,以及炮塔射手座舱的大部分地方已统统塞满了鹰嘴豆。

    约塞连这趟飞行的使命就是分散奥尔的注意力,不让他知道米洛在哪儿买鸡蛋,尽管奥尔也是米洛的辛迪加联合体的成员之一,而且同别的成员一样,他也拥有一份股份。约塞连感到自己的这一使命很可笑,因为人人都知道,米洛在马耳他用七分钱一个的价格买下鸡蛋,然后再以五分钱一个的价钱卖给辛迪加联合体的食堂。
 


    “我就是不信任他。”米洛像母鸡抱窝似的一动不动地坐在飞机里,一面冲着坐在后面的奥尔点了点头,奥尔则像一根缠结在一起的绳子,蜷缩着躺在下面那排装满了鹰嘴豆的筐子上,竭力想使自己睡着,那样子受罪得要命。“我情愿在我买鸡蛋时他不要在边上转悠,将我的生意秘密全打听去。你还有什么不明白的吗?”

    约塞连坐在他身旁副驾驶的坐位上。“我不明白,你在马耳他花七分钱买来的一个鸡蛋,为什么又用五分一个的价卖掉呢?”

    “我这样做是为了弄点赚头。”

    “可你怎样才能有赚头呢?你每个鸡蛋反倒要赔二分钱呢。”

    “我在马耳他按每个四分二厘五的价将鸡蛋卖给那儿的人,然后再按每个七分钱的价将鸡蛋从那些人的手中买进,这样我就赚了三分二厘五。当然,我是不赚钱的,赚钱的是咱们的联合体。大伙人人有份。”

    约塞连觉得自己开始有点明白了。“你按每个四分二厘五的价将鸡蛋卖给那些人,而他们再按每个七分钱的价把鸡蛋卖给你,这样他们每个鸡蛋就净赚二分七厘五。是这样吗?你干吗不把鸡蛋直接卖给你自己,省得再经他人之手买回这道手续呢?”

    “因为这个‘他人’就是我自己,”米洛解释说,“我将鸡蛋卖给我自己时,我每个蛋可赚三分二厘五。我再把蛋从我的手里买回时,我每个又可赚到二分七厘五。这样每个鸡蛋一共可赚到六分钱。我把它们照每个五分钱的价卖给食堂时,每只蛋只不过少赚二分钱而已。这就是我如何以七分钱一只买进,五分钱一个卖出还能赚到钱的原因。我在西西里收购鸡蛋时,每只蛋只要付老母鸡一分钱就行了。”

    “在马耳他,”约塞连纠正道,“你是在马耳他买的鸡蛋,而不是在西西里。”

    米洛得意洋洋地哈哈大笑起来。“我可不是在马耳他买的鸡蛋,”他带着一种暗自得意的神态承认道,这可同他平日显出的那副既勤奋又清醒的样子相违背,约塞连还是第一次看到他的这种神态。“我在西西里一分钱一个买来,然后在马耳他悄悄地以每个四分五厘的价格转手,为的是别人到马耳他来买鸡蛋时,蛋价能上扬到七分钱一个。”

    “既然马耳他的蛋价这么贵,那人们干吗要上那儿去买蛋?”

    “因为他们总是这么干。”

    “他们为什么不去西西里买鸡蛋呢?”

    “因为他们从来没有那么干过。”

    “我实在不懂,你为什么要将鸡蛋按五分一个的价卖给食堂,而不卖七分一个呢?”

    “因为要是这样一来,我的食堂就不需要我了。七分钱一个的鸡蛋任何人都能买到。”

    “他们为什么不越过你,而直接去马耳他以每个四分二厘五的价格从你的手里将鸡蛋买下呢?”

    “因为我不会将蛋卖给他们的。”

    “你为什么不卖给他们?”

    “因为那样的话就没有什么赚头了。作为中间商,我这样做至少能让我自己能有点赚头。”

    “这么说,你的确为你自己赚了钱,”约塞连断言道。

    “我当然赚了。不过赚到的钱全归咱们的辛迪加联合体。人人部有份。你难道不明白?我卖给卡思卡特上校的红色梨形番茄也正是这么回事。”

    “你是买,不是卖,”约塞连纠正道,“你不是将红色梨形番茄卖给卡思卡特上校和科恩中校。你是从他们的手上买番茄。”
 


    “不对,是卖,”米洛纠正约塞连道,“我用了个假名字,在皮亚诺萨岛所有的市场上抛售番茄,这样卡思卡特上校和科恩中校各自也用了个假名,以每个四分的价钱将番茄全部买进,第二天我再以辛迪加的名义按每个五分的价格将番茄买回来。他们每个番茄赚一分钱,而我每个赚三分五厘钱,这样每人都有了赚头。”

    “你们每人都赚了,只有辛迪加不赚。”约塞连对此嗤之以鼻。

    “辛迪加出五分钱买进一个番茄,而你每个只花了五厘钱。这样辛迪加怎么能赢利?”

    “只要我能赚到钱,辛迪加也就赚到了钱,”米洛解释说,“因为人人有份。只要咱们的辛迪加能得到卡思卡特上校和科恩中校的支持,那他们就会像这次这样派我出差。再过大约十五分钟,当我们在巴勒莫降落时,你就会看到咱们能赚到多少钱了。”

    “在马耳他,”约塞连纠正他说,“我们正在往马耳他飞,而不是朝巴勒莫。”

    “不对,我们是在朝巴勒莫飞,”米洛回答道,“在巴勒莫有一个苣菜出口商,我要和他谈几分钟,因为我有一批发了霉的蘑菇要运到伯尔尼去。”“米洛,你是怎么干的?”约塞连面带既惊讶又钦佩的笑容问,“你的飞行计划单上填的是一个地方,可后来你却飞到另外一个地方去了。指挥塔上的人就从不找你的麻烦?”

    “他们都属于咱们的联合体,”米洛说,“他们都明白凡是对咱们联合体有利的事,对国家也是有利的,因为只有这样才会让美国大兵们卖力气。再说指挥塔上的那些人也是有份子的,这就是他们为什么要千方百计地给咱辛迪加联合体帮助的缘故。”

    “我也有份吗?”

    “人人都有份。”

    “奥尔也有份?”

    “人人都有份。”

    “亨格利-乔呢?他也有份吗?”

    “人人都有份。”

    “呸,活见鬼。”约塞连心里在骂,有生以来,有关股份的主意还是第一次在他的脑子里留下了深刻的印象。

    米洛将脸转向约塞连,眼睛里隐约闪出一丝图谋不轨的神色。

    “我有一个主意,可以稳稳当当地从联邦政府那里骗得六千美元。

    到时咱俩平分,各得三千元,并用不着担任何风险。你有兴趣吗?”

    “没兴趣。”

    米洛十分激动地望着约塞连。“这就是我喜欢你的原因,”他大声地说,“你很诚实!在我认识的人中间你是唯一能让我信赖的人。

    也就是这个原因,我希望你能给我更多的帮助。昨天在卡塔尼亚大街,当你同那两个荡妇一起溜走的时候,我真感到失望。”

    约塞连盯住米洛,感到大惑不解,简直不敢相信他的话。“米洛,可是你叫我同她们走的呀。难道你不记得了?”

    “那不是我的过错,”米洛一本正经他说,“以往是在我们进城后,我才设法将奥尔给甩掉。而这次到巴勒莫,情况就大不一样了。

    当我们在巴勒莫着陆后,我要你同奥尔立即就跟着姑娘离开机场。”

    “跟着什么姑娘?”

    “我事先已发过无线电报,同一个四岁的小皮条客安排好了,为你和奥尔找了两个八岁大的、有着一半西班牙血统的处女。他将在机场的一辆交通车上等你们。你俩一下飞机就立即上那辆车。”

    “不行,”约塞连说,“我只想去个地方睡上一觉。”

    米洛立刻发火了,脸都涨成了猪肝色,细长的鼻子在两道黑眉毛之间痉孪地颤动着,唇上那抹不对称的赤黄色的小胡子像一根蜡烛发出的暗淡、细弱的火焰。“约塞连,别忘了你的使命。”他提醒约塞连,那口气还算恭敬。

    “让使命见鬼吧!”约塞连满不在乎地答道,“让辛迪加也见鬼去吧,管它有没有我一份呢。我也不想要什么八岁大的处女,哪怕她们有一半的西班牙血统。”
 


    “这我不怪你。不过这些所谓的八岁大的处女实际上是三十二岁。她们并不是真的有一半西班牙血统,只不过是有三分之一的爱沙尼亚血统。”

    “我一点也不稀罕什么处女。”

    “她们其实连处女也不是,”米洛用劝告的口气继续说道,“我为你选定的那个女人曾嫁过一个上了年纪的教师,不过时间不长,那男的只在星期天才同她睡觉,所以她几乎就同一个没破了身子的姑娘差不多。”

    然而,奥尔也同样瞌睡得要命,所以当他们驱车离开机场驶进巴勒莫时,约塞连和奥尔仍一边一个坐在米洛的身旁。他们发现在巴勒莫的旅馆里仍然没有他俩的房间。更重要的是,他们还发现米洛竟是那里的市长。

    对米洛的古怪的、令人难以置信的欢迎从机场就开始了。在机场上忙碌着的平民百姓们认出了米洛,都恭恭敬敬地停下手上的工作,目不转睛地看着他,一边还做着颇有节制的动作,嘴里还说着奉承话。米洛要来的消息已先于他本人传到了城里,所以当他们乘坐着敞篷小卡车疾驶而来时,城郊早已挤满了欢呼的人群。约塞连和奥尔大惑不解,所以作声不得,只好紧紧地挤在米洛的身边以求平安无事。

    卡车进城后放慢了速度,朝着市中心缓缓驶去,这期间,人们的欢呼声越来越响。男童女童们都用不着上学了,而是穿着新衣,排列在大街的人行道两旁,手里不住地挥舞着小旗子。对此,约塞连和奥尔惊讶得一句话也说不出来。大街上人山人海,欢声雷动,空中到处悬挂着绘有米洛肖像的旗帜。米洛在肖像上的样子是穿着当地农民常穿的那种黄褐色的圆领衬衫,唇上蓄着一抹不齐整的小胡子,两只眼睛一大一小,正用一种无所不知、无所不晓的目光凝视着人群。他那审慎而又慈祥的脸上露出一副宽厚、睿智、严谨而又刚毅的神色。体弱无力的病人从窗口向他送来一个又一个的飞吻。围着围裙的店主们站在狭窄的店堂门口欣喜若狂地欢呼不已。无数大号嘀嘀嗒嗒地吹得震天响。到处都有人给挤倒,被踩死。一些抽抽噎噎的老妇女围着缓缓而行的卡车拼命地你推我搡,竞相去摸米洛的肩膀,或握他的手。米洛和善而又不失风度地接受着这场喧闹的庆祝。他用很优美的动作朝每一个人挥手作答,并且还很慷慨地大把大把地朝着欢乐的人群抛去飞吻,就像在散发包着锡纸的赫尔希牌巧克力一样,一排排朝气蓬勃的少男少女臂挽着臂,蹦蹦跳跳地跟在他的后面,一面扯着嘶哑的嗓门,直瞪着两眼,极敬慕地一遍又一遍地喊着:“米一洛!米一洛!米一洛!”
 


    现在既然自己的秘密已被人知道了,米洛也同约塞连和奥尔一样松弛下来了,他不禁显得洋洋得意,感到无比的自豪,同时也显得有点羞答答的。他的双颊也变得红润起来。米洛早被选为巴勒莫的市长——同时也是附近的卡里尼、蒙雷阿莱、巴盖里亚、泰尔米尼、伊梅雷塞、切法利、米斯特雷塔和尼科西亚的市长——因为是他给西西里岛带来了苏格兰威士忌。

    约塞连感到很惊奇。“难道这儿的人就这么喜欢喝苏格兰威士忌?”

    “他们连一滴都不喝,”米洛解释道,“苏格兰威士忌可贵了,而这里的人都很穷。”

    “既然没人喝,那你为什么要将酒运到西西里来?”

    “为的是定出一个价钱来。我把酒从马耳他运到这里来,然后经我转手再替别人卖给我,这样赚头就大了。我在这里开创了一个新兴行业。今天,西西里已是世界上第三大苏格兰威士忌酒的出口基地了。这就是他们为什么要选我当市长的原因。”

    “既然你是这么一个大人物,那你给我们在旅馆里弄间房怎么样?”奥尔用疲倦、含糊的声音十分不恭地咕哝道。

    米洛很歉疚地作出了反应。“我正打算办这件事呢,”他允诺道,“实在抱歉,我忘了事先应用无线电替你俩在旅馆里订两个房间。随我来办公室吧,我马上就跟我的副市长说一声。”

    米洛的办公室是一家理发店,他的副市长是一个矮胖的理发师。他一张嘴就是满口的奉迎,亲热的问候,两片嘴皮子上挂满了白沫,就像他在杯子里搅个不停的肥皂沫——他这是在准备替米洛刮脸。

    “嗬,维托里奥,”米洛懒洋洋地仰面躺在维托里奥的一张理发椅上问,“我不在的这阵子情况怎么样啊?”

    “大伙很难过,米洛先生,很难过。不过现在你回来了,大伙就都又开心了。”

    “我在纳闷呢,怎么有这么大群大群的人。这旅馆怎么都住满了?”

    “米洛先生,这一来是因为有那么多的人从别的城市赶来看您,二来是因为所有朝鲜蓟的买主都到咱们城来参加拍卖。”

    米洛的一只手像只老鹰似的笔直地腾空而起,一把抓住维托里奥的修面刷。“朝鲜蓟是什么东西?”他问。

    “朝鲜蓟,米洛先生?朝鲜蓟是一种非常好吃的蔬菜,不管在哪儿都受欢迎。趁您在这儿的期间,您真该尝尝它的味道,米洛先生。

    我们这儿种的朝鲜蓟是世界上最好的。”

    “真的?”米洛问,“今年朝鲜蓟卖什么价?”

    “看样子它今年能卖个好价钱。因为收成很不好。”

    “这是真的吗?”米洛若有所思地问,突然就走得不见人影了。

    他从椅子上溜下来的动作是那么快,以至于他刚才围在身上的条纹围布在他离开了一两秒钟后才落地。等约塞连和奥尔跟在他的后面冲到理发店门口时,米洛已消失得无影无踪了。

    “下一位?”米洛的副市长殷勤地嚷嚷道,“下一位谁来?”

    约塞连和奥尔垂头丧气地从理发店走了出来。他俩被米洛抛弃了,无家可归,只得艰难地在狂欢的人群里穿行着,徒劳地寻找着一个能睡觉的地方。约塞连已是精疲力竭了。他的脑袋一阵一阵地隐隐作痛,浑身乏力。他对奥尔很恼火,那家伙不知在哪里找到了两只山楂果,在走路的当儿一直塞在腮帮子里。后来约塞连发现了,硬是让他吐了出来。后来奥尔又找到两颗七叶树果子,又偷偷地将它们塞到嘴巴里,结果又一次被约塞连察觉了。约塞连再次抓住他,要他把山楂果从嘴里弄出来。奥尔咧嘴笑着,回答说那不是山楂果而是七叶树果,并且它们不是在他的嘴里,而是在他的手上。可是,因为他嘴里含着七叶树果,他说的话约塞连连一个字也没听懂,约塞连却死活要他将果子吐出来。此时奥尔的眼中闪出了狡猾的光芒。他用指关节使劲地磨擦着脑门,就像个醉鬼一样,一面样子下流地嘿嘿笑个不停。

    “你还记得那个姑娘吗——?”他止住笑问,紧接着又下流地嘿嘿地笑了起来。“有一次在罗马的那个公寓里,那个姑娘用鞋子揍我的脑袋,当时我和她都一丝不挂,你还记得吗?”他脸上带着狡猾的期待神情问道。他等待着,直到约塞连戒备地点了点头。“如果你让我把七叶树果放回嘴里,我就告诉你她为什么要揍我。这个交易怎么样?”

    约塞连点了点头,于是奥尔便源源本本地给他讲了那个离奇故事,告诉他在内特利的妓女的公寓里,那个赤身裸体的妓女为什么要用鞋子揍他的脑袋。可是约塞连还是一个字没听懂,因为那两颗七叶树果又回到了奥尔的嘴里。约塞连被他的这一诡计气得大笑了起来。然而,当黑夜降临时他俩实在无计可施,只好去了一家肮脏的小饭馆,吃了一顿乏味的晚饭,然后搭上一辆便车回到了机场。他们就睡在机舱内凉冰冰的金属地板上,辗转反侧,哼个不停,受罪得要命。这样过了还不到两个小时,他们就听到了卡车司机冲着他们大喊大叫的声音,原来他们运来了许多箱朝鲜蓟。那些司机将他俩从飞机上赶到地面,以便让他们往飞机上装货。这时天又下起了大雨,等到卡车开走时,约塞连和奥尔已被淋得透湿,浑身的雨水直往下滴。两人无奈,只好又重新挤进机舱,将身子缩成一团,像两条正在发抖的鱼那样挤在装满了朝鲜蓟的摇摇晃晃的板条箱的角落里。黎明时分,米洛将这些朝鲜蓟空运到了那不勒斯,将其换成了肉桂、丁香、香草豆和胡椒荚,当天又把这些东西赶运回南方的马耳他。结果到了马耳他,他们又发现米洛原来还是那里的副总督。在马耳他,约塞连和奥尔仍然弄不到房间。米洛在马耳他成了米洛-明德宾德少校爵士,并在总督府里有一间极大的办公室。

    他的那张桃花心木的办公桌也是硕大无比的。在橡木板壁的一块嵌板上两面交叉的英国国旗下,悬挂着一张极其醒目的米洛-明德宾德少校爵士身穿英国威尔士皇家明火枪手制服的大幅照片。

    照片上,米洛唇上的小胡子经过了修剪,细细的一抹,他的下巴像是经刀刻斧凿过的一样,双眼像利刺那样尖锐,米洛已受封为爵士,并被委任为威尔士皇家明火枪团的少校,还被任命为马耳他的副总督,因为他在马耳他开创了鸡蛋生意。米洛慷慨地表示让约塞连和奥尔睡在他的办公室里厚厚的地毯上过夜。可是他刚离开不久,就来了一个全副武装的警卫,用刺刀顶着他们,将他俩赶出了这座大楼。这时他俩已是筋疲力尽,只得乘出租车回到机场。那司机脾气大得要命,在车钱上还宰了他们一刀。他俩又钻进机舱去睡觉,这一次机舱里到处塞的都是黄麻袋,里面装满了可可和新磨的咖啡,只只麻袋都被撑漏了,散发出一股股浓烈的气味,以至两人不得不跑出机舱,趴在飞机的起落架上大吐特吐起来。第二天一大早,米洛就乘专车来到机场,整个人显得精神焕发,立即就起飞前往奥兰,到了奥兰,约塞连和奥尔还是找不到旅馆房间,而米洛又摇身一变成了那儿的代理国君。在那座橙红色的王宫里,有一处专供米洛支配的住所,可是约塞连和奥尔却不能随同他进宫,因为他俩是信仰基督教的异教徒。在王宫门口,他俩被手持弯刀、身材魁梧的柏柏尔族警卫给拦住,被赶走了。奥尔患了重感冒,又流鼻涕又打喷嚏。约塞连那宽阔的脊背也弯了下来,疼得要命。他真想把米洛的脖子给拧断,可怎奈他是奥兰的代理国君,他的身体是神圣不可侵犯的。事实还表明:米洛不仅是奥兰的代理国君,他同时还是巴格达的哈里发,大马士革的伊玛目和阿拉伯的酋长。在那些落后的地区,米洛既是谷物之神,也是雨神和稻米之神,因为在那些地方,这些神灵仍受到愚昧而又迷信的人们的崇拜。说起在非洲丛林深处,米洛突然变得很谦虚起来了,他暗示说在那里到处都可见到他那留着小胡子的巨大的脸部石雕,那些石雕的面孔俯视着无数个被人血染红了的原始的石头祭坛。他们一行的足迹所到之处,人们都要朝着米洛热烈欢呼。他去了一个又一个城市,每到一处都要受到英雄凯旋式的欢迎。最后他们来到了开罗,就是在那里,米洛垄断了市场上所有的棉花,可这时世界上谁也不需要棉花,这使得他一下子就濒于破产的边缘。事情的起因是这样的,那天在开罗,约塞连和奥尔终于在旅馆里找到了房间。他们终于有了柔软的床铺、蓬松的枕头、浆洗干净的被单,也有了盥洗室,里面还有供他们挂衣服的衣架,另外还有水可以洗澡。约塞连和奥尔将他门那散发着难闻的恶臭的身体浸泡在一只盛满了滚烫的热水的大盆里,直到将浑身的皮肤泡得通红。洗完澡,他俩随着米洛出了旅馆,来到一家很讲究的饭馆,先是吃了鲜虾开胃口,然后又吃了些切得小小的肉片。饭馆的前厅有一架可自动记录证券行市的收报机,当米洛向侍者领班打听它是啥机器时,它恰好在劈劈啪啪地打出埃及棉花的最新行情。米洛从来连想都没想过,世上竟有证券行情自动收报机这种奇妙无比的机器。

    “真的?”当侍者领班结束了他的解释时,米洛不禁叫出了声。

    “现在埃及棉花卖什么价?”侍者领班告诉了他,米洛立即就将市场上的原棉统统买了下来。

    然而米洛买下的埃及棉花倒并不怎么让约塞连感到害怕,真正让他感到担心的是当地市场上的一串串青里透红的香蕉。米洛是在他们驱车进城时发现这些香蕉的。事实证明他的担心是有道理的,因为当夜十二点以后,米洛将他从熟睡中摇醒了,将一个剥了一半皮的香蕉硬塞到他的嘴里。约塞连给噎得差点没哭出来。

    “尝一尝。”米洛催促着,一面拿着那根香蕉紧跟着约塞连那张痛苦不堪的脸转来转去。

    “米洛,你这个杂种,”约塞连用呻吟般的声音说道,“我要睡觉。”

    “把它吃了,然后告诉我好不好吃,”米洛坚持道,“别告诉奥尔,这是我送给你的。我刚才也给他吃了一根,收了他两个皮阿斯特。”

    约塞连只好顺着他,吃了那根香蕉,告诉他味道不错,便又合上了双眼。然而米洛却又把他摇醒了,要他立刻以最快的速度穿好衣服,因为他们马上就要飞离这里到皮亚诺萨岛去。

    “你和奥尔必须立即把香蕉装上飞机,”米洛解释说,“那人说在搬弄这一串串香蕉时得留神,别让蜘蛛钻进去。”

    “米洛,我们不能等天亮再飞吗?”约塞连恳求说,“我得睡一会才行。”

    “它们烂起来可快啦,”米洛回答说,“我们一分钟也耽搁不起。

    想想吧,咱们中队在家的那些人要是吃到这些香蕉,该有多高兴啊。”

    然而,中队在家的那些人却连香蕉的影子也没见着。这是因为在伊斯坦布尔,香蕉是卖方的市场,而在贝鲁特茴香籽却是买方市场,所以米洛抛售了香蕉,买下茴香籽,将其运往班加西。六天以后,他们又马不停蹄地赶回皮亚诺萨岛,这时奥尔的假期也结束了。这一次他们的飞机上装满了从西西里购来的上好的白皮鸡蛋,可米洛却说这些鸡蛋是从埃及买来的,并且仅以四分一个的价钱卖给了食堂。这一来那些已加入辛迪加联合体的指挥官全都恳求米洛立即赶回开罗,再多弄些青里透红的香蕉到土耳其卖掉,在那里再多买些班加西急需的茴香籽。这样,人人都得到了一份好处
