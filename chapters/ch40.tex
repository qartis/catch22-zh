\chapter{第二十二条军规}
 
    当然,这里面有个圈套。

    “第二十二条军规?”约塞连问。

    “当然。”科恩中校轻轻挥了挥手,又略带轻蔑的神情点了点头,便把那帮押送约塞连的膀大腰圆的宪兵赶了出去,随后,他愉快地回答了约塞连的问话——和往常一样,他最轻松的时候也就是他最刻薄的时候。“毕竟,我们不能因为你拒绝执行更多的飞行任务就把你送回国去,而让其余的人留在这儿,对吧?那样对他们很难说是公平的。”

    “你说得太正确了!”卡思卡特上校突然说道。他像一头气喘吁吁的公牛那样来来回回定着,生气地板着面孔,不停地喘粗气。“我真想每回执行任务时都把他手脚捆起来扔到机舱里去。这就是我想做的事。”

    科恩中校示意卡思卡特上校保持沉默,然后又对约塞连笑了笑。“你知道,你把事情弄成这个样子,的确使卡思卡特上校感到十分难办,”他漫不经心他说,好像这件事一点也不惹他生气似的。

    “官兵们都很不乐意,士气越来越低落。这全都是你的过错。”

    “这是你们的过错,”约塞连争辩道,“因为你们一再增加飞行任务的次数。”

    “不,这是你的过铬,因为你拒绝执行飞行任务,”科恩中校反驳道,“以前,当他们觉得自己别无选择的时候,不管我们要求他们执行多少次飞行任务,他们都心甘情愿地执行了。可现在,你使他们有了选择的希望,他们就开始不乐意了。所以,这全都怪你。”

    “难道他不知道眼下正在进行战争吗?”卡思卡特上校愤愤地质问道。他仍然跺着脚来回地走动着,看也不看约塞连一眼。

    “我敢肯定他是知道的,”科恩中校回答说,“也许这就是他拒绝执行飞行任务的原因。”

    “难道那对他有什么影响吗?”

    “知道现在正在进行战争会动摇你拒绝参战的决定吗?”科恩中校嘲弄地模仿着卡思卡特上校的口吻,严肃而讥讽地问道。

    “不会的,长官,”约塞连回答道。他差点冲着科恩中校笑起来。

    “我也担心这个,”科恩中校字斟句酌地说。他悠闲地抬起双手搁到他那光滑闪亮的褐色秃顶上,把十个手指头对插到一起。“你当然明白,公平他讲,我们待你还算不错,对吧?我们供给你吃的,并且按时发给你军饷。我们奖给你一枚勋章,甚至还提拔你当了上尉。”

    “我根本就不该提拔他当上尉,”卡思卡特上校抱怨地大声说,“那次执行轰炸弗拉拉的任务时,他竟然飞了两圈,结果把事情搞得一团糟。我真应该送他上军事法庭的。”

    “我告诉过你不要提拔他,”科恩中校说,“可你不肯听我的。”

    “不,你没说。是你叫我提拔他的,不是吗?”

    “我告诉你不要提拔他,可你就是不肯听。”

    “我真应该听你的。”

    “你从来也不听我的,”科恩中校意味深长地坚持道,“就因为这个,我们才落到这步田地。”

    “唉,行了,别磨牙了,好吗?”卡思卡特上校把两个拳头深深地插进衣袋里,懒洋洋地转过身去。“别老找我的碴了,你为什么不好好考虑一下我们该拿他怎么办呢?”

    “恐怕我们只能送他回国了,”科恩中校一边得意洋洋地窃笑道,一边从卡思卡特上校那边转过脸来对着约塞连。“约塞连,对你来说战争已经结束了。我们将要送你回国。你当然知道,你实在是不配被送回国的,可这正是我乐意送你回国的原因之一。既然眼下没有什么别的好办法可供我们一试,我们只好决定把你送回合众国去。我们已经盘算好了这笔交易——”

    “什么样的交易?”约塞连满腹狐疑,挑衅地质问道。

    科恩中校仰面大笑。“噢,是一笔不折不扣的卑鄙交易,这一点毫无疑问。绝对令人恶心。不过,你很快就会接受下来的。”

    “别那么有把握。”

    “即使这笔交易臭气熏天,你也会接受的,对此我没有丝毫的怀疑。哦,顺便问一句,你还没有告诉任何人你拒绝执行更多的飞行任务,是吗?”

    “没有,长官,”约塞连毫不迟疑地回答道。

    科恩中校赞许地点点头。“这很好,我喜欢你这种说谎的方式。

    如果你有几分雄心壮志的话,你在这个世界上一定会飞黄腾达的。”

    “难道他不知道眼下正在进行战争吗?”卡思卡特上校突然大叫起来,接着又满脸疑虑地对着烟嘴吹了一口气。

    “我敢肯定他是知道的,”科恩中校尖刻地回答道,“因为你刚才已经向他提出过这一问题了。”科恩中校不耐烦地皱起眉头帮约塞连讲话,他的黑眼睛里闪烁着狡黠而放肆的嘲弄目光。他用双手抓住卡思卡特上校的桌子边,抬起他那软绵绵的屁股从桌角往里坐去,只剩下两条短短的小腿悬垂着自由摆动。他用鞋跟轻轻踢着黄色的橡木桌子。他的脚上穿着上褐色的袜子,因为没系吊袜带,袜筒一圈一圈直褪落到异常苍白小巧的脚踝下面。“你知道,约塞连,”他和颜悦色地沉思片刻,流露出一种漫不经心的神情,看上去既像是嘲笑又显得非常真诚,“我真的有点佩服你。你是个道德高尚的聪明人,你采取了一种极为勇敢的立场。而我却是个毫无道德观念的人,因此,我正好处在评价你的道德品格的理想位置上。”

    “现在是关键时刻。”站在办公室一个角落里的卡思卡特上校气呼呼地插话说。他看也没看科恩中校一眼。

    “的确是关键时刻。”科恩中校心平气和地点点头表示同意。

    “我们刚刚换了指挥官。要是出现某种局面,使我们在沙伊斯科普夫将军或者佩克姆将军面前出丑的话,那我们可受不了。你是这个意思吧,上校?”

    “他难道就没有一点爱国精神吗?”

    “难道你不愿意为你的祖国而战吗?”科恩中校模仿着卡思卡特上校自以为是的刺耳腔调质问道,“难道你不愿意为卡思卡特上校和我而献出你的生命吗?”

    听到科恩中校这最后一句话,约塞连十分惊讶,不由得紧张起来。“这是什么意思?”他大叫道,“你和卡思卡特上校跟我的祖国有什么关系?你们完全是另一回事。”

    “你怎么能把我们和祖国分开呢?”科恩中校神色安祥,讥讽地反问道。

    “对啊,”卡思卡特上校使劲地喊道,“你要么为我们而战,要么对抗你的祖国,这两条路你只能选一条。”

    “恐怕这下子他把你难住了。”科恩中校加上一句。“你要么为我们而战,要么对抗你的祖国,事情就是这么简单。”

    “噢,得啦,中校,我可不吃这一套。”

    科恩中校依然很沉着。“坦率地说,我也不信这一套,可别人都会相信的。你瞧,事情就是这么简单。”

    “你真给这身军装丢脸!”卡思卡特上校怒气冲冲地喊叫着。他猛地转过身来,头一回正面对着约塞连。“我倒很想知道你究竟是怎么当上上尉的。”

    “是你提拔他的,”科恩中校强忍住笑,亲切地提醒道。

    “唉,我真不应该提拔他。”

    “我告诉过你别这么做,”科恩中校说,“可你就是不肯听我的。”

    “得啦,你别再跟我磨牙了,行吗?”卡思卡特上校叫了起来。他皱起眉头,怀疑地眯起眼睛盯着科恩中校,把两只握紧的拳头抵在后腰上。“你说,你究竟站在哪一边?”

    “站在你这一边呀,上校。我还能站在哪一边呢?”

    “那就别再老是找我的碴了,行吗?别再拿我开心了,行吗?”

    “我是站在你这一边的,上校。我满怀爱国热情。”

    “那么,你要保证不忘记这一点。”卡思卡特上校仍然没有完全放下心来。他停了一下才犹犹豫豫地转过身去,双手揉搓着长长的香烟烟嘴,重又开始踱起步来。他用一个大拇指朝约塞连猛地一指,说道:“让我们跟他了结了吧。我知道我应该怎么处置他。我想把他拉到外面去枪毙。我就打算这么处置他。德里德尔将军也准会这么处置他。”

    “可是德里德尔将军已经不再指挥我们了,”科恩中校说,“所以我们不能把他拉到外面去枪毙。”此时,科恩中校和卡思卡特上校之间的紧张时刻已经过去,他又变得轻松愉快起来,又开始拿脚轻轻踢着卡思卡特上校的桌子。“所以,我们不打算枪毙你而是打算送你回国。这事费了我们不少脑筋,可我们最后还是想出了这个小小的、糟透了的计划。这样一来,你的回国就不会在那些被你撇在身后的朋友当中引起太大的怨言。这难道不使你开心吗?”

    “这是个什么样的计划?我不能肯定我会喜欢它。”

    “我知道,你不会喜欢它的。”科恩中校哈哈一笑,重又心满意足地把双手举到头顶,手指对插到一起。“你会憎恨这个计划的。它的确令人作呕,而且肯定会使你良心不安。但是,你很快就会同意这个计划。你会同意的,不但因为这计划会在两周之内把你安全送回国去,而且因为你别无选择。你要么接受这个计划,要么接受军法审判。你可以接受,也可以不接受。”

    约塞连哼了一声。“别吓唬我了,中校。你们不会用在敌人面前临阵脱逃的罪名对我进行军法审判的。那样一来,你们的面子不好看,而且你们大概也没有办法证明我有罪。”
 


    “可是我们可以指控你擅离职守,根据这个罪名对你进行军法审判,因为你没有通行证就跑到罗马去了。我们可以使这一罪名成立。你只要稍微想一想就会明白的,你逼得我们没有别的路可走了。我们不能就这么眼睁睁地看着你违抗命令到处乱跑而不对你加以惩罚。要是那样,其他所有的人也都会拒绝执行飞行任务的。

    这样是不行的,这一点你相信我的话好啦。你要是拒绝我们提出的这笔交易,我们就要对你进行军法审判,哪怕这样一来会引起许多问题,会叫卡思卡特上校当众出丑,我们也顾不上了。”

    听到“出丑”这两个字,卡思卡特上校吓得一哆嗦。随后,他似乎想也没想便气势汹汹地把他那个镶有条纹玛瑚和象牙的细长烟嘴往办公桌的木制桌面上猛地一摔。“耶稣基督啊!”他出人意料地叫了一声。“我恨透了这个该死的烟嘴!”烟嘴在桌面上蹦了两下,弹到了墙壁上,接着又飞过窗台,落到地上,最后滚到卡思卡特上校的脚边上不动了。卡思卡特上校恶狠狠地低头怒视着烟嘴说:

    “我不知道这对我是不是真的有好处。”

    “这在佩克姆将军看来是你的荣耀,而在沙伊斯科普夫将军看来却是你的丑事,”科恩中校装出一副天真无邪的调皮模样对他说。

    “那么,我应该讨哪一个人的欢心呢?”

    “应该同时讨他们两个人的欢心。”

    “我怎么能够同时讨他们两个人的欢心呢?他们互相憎恨。我要怎么做才能既从沙伊斯科普夫将军那里获取荣耀,又不至于在佩克姆将军面前丢人现眼呢?”

    “操练。”

    “对啦,操练。这是唯一能讨他欢心的方法。操练,操练。”卡思卡特上校温怒地做了个鬼脸。“那些将军!他们真给那身军装丢脸。

    要是像这两个家伙这样的人都能当上将军的话,我看不出为什么我就当不上。”

    “你会飞黄腾达的,”科恩中校以一种毫无把握的语调安慰他说,说完就转脸对着约塞连格格笑了起来。当约塞连流露出敌视、怀疑的固执表情时,他越发轻蔑地开怀大笑起来。“现在你知道问题的关键了吧。卡思卡特上校想当将军,我想当上校,这就是我们必须送你回国的原因。”

    “他为什么想当将军呢?”

    “为什么?这跟我想当上校的原因是一样的。我们还能做什么呢?人人都教导我们要有更高的追求。将军比上校的地位高,上校又比中校的地位高,所以,我们俩都在往上爬。你知道,约塞连,我们的这种追求对你来说是件幸运的事情。你的时机选择得再恰当不过了,可我觉得,你事前策划时就把这一因素考虑进去了。”

    “我根本没策划什么,”约塞连反驳道。

    “是的,我的确欣赏你这种说谎的方式,”科恩中校说,“当你的指挥官被提拔为将军——当你知道你所在的部队平均每人完成的战斗飞行任务比任何别的部队都多时——难道你不为此而感到骄傲吗?难道你不愿意获得更多的通令嘉奖和更多的橡叶簇铜质奖章吗?你的集体主义精神哪儿去了?难道你不愿意执行更多的飞行任务以对这一伟大的纪录做出自己的贡献吗?说‘愿意’吧,这是你的最后一次机会了。”

    “不。”

    “要是这样的话,你可就逼得我们走投无路了——”科恩中校客客气气地说。

    “他应该为自己而感到惭愧!”

    “——我们只好送你回国啦。只是,你要为我们做几件小事情,而且——”

    “做什么事情?”约塞连以怀疑和敌对的态度打断了他的话。

    “噢,很小的事情,无关紧要的事情。真的,我们跟你做的这笔交易十分慷慨。我们将发布送你回国的命令——真的,我们会的——而作为报答,你得做的不过是……”

    “是什么,我得做什么?”

    科恩中校假惺惺地笑了笑。“喜欢我们。”

    约塞连惊愕地眨了眨眼睛。“喜欢你们?”

    “喜欢我们。”

    “喜欢你们?”

    “不错,”科恩中校点点头说。约塞连那副不加掩饰的惊奇神态和那种手足无措的样子使他十分得意。“喜欢我们,加入到我们中来,做我们的伙伴。不论是在这里,还是回国以后,都要替我们说好活,成为我们中的一员。怎么样,这个要求不算过分,是吧?”

    “你们只是要我喜欢你们,就这些吗?”

    “就这些。”

    “就这些。”

    “只要你从心眼里喜欢我们。”

    约塞连终于明白了,科恩中校讲的是实话,他大为惊奇,真想自信地放声大笑一通。“这并不是太容易,”他冷笑着说。

    “噢,这比你想象的要容易多了,”科恩中校反唇相讥道。约寒连这句讽刺的话并没有使他灰心丧气。“你只要开了头,准会吃惊地发现喜欢我们是件多么容易的事情。”科恩中校往上扯了扯他那宽松的裤腰。他露出一个讨人嫌的嘲讽笑容,他那方下巴和两颧骨之间的深深的黑色纹路又一次弯曲了起来。“你瞧,约塞连,我们打算让你过舒服日子,我们打算提拔你当少校,我们甚至打算再发给你一枚勋章。弗卢姆上尉正在构思几篇热情洋溢的通讯,打算把你在弗拉拉大桥上空的英勇事迹,你对自己部队的深厚持久的忠诚,以及你格尽职责的崇高献身精神大大描绘一番。顺便说一句,这些都是通讯里的原话。我们打算表彰你,把你作为英雄送回国去。我们就说是五角大楼为了鼓舞士气和协调与公众的关系而把你召回国的。你将像个百万富翁那样生活,你将成为所有人的宠儿。人们将列队欢迎你,你将发表演说号召大家筹款购买战争债券。只要你成为我们的伙伴,一个奢侈豪华的崭新世界就将出现在你的面前。这难道不迷人吗?”
 


    约塞连发现自己正聚精会神地倾听着这一番详尽而动听的长篇大论。“我可拿不准我想不想发表演说,”“那么我们就不提演说的事啦。重要的是你对这儿的人讲些什么。”科恩中校收敛笑容,满脸诚恳地往前探了探身体。“我们不想让大队里任何人知道,我们送你回国是因为你拒绝执行更多的飞行任务。我们也不想让佩克姆将军或者沙伊斯科普夫将军听到风声说,我们之间不和,就是为了这个,我们才打算跟你结成好伙伴的。”

    “要是有人问我为什么拒绝执行更多的飞行任务,我对他们说什么呢?”

    “告诉他们,有人已经私下向你透露就要送你回国了,所以你不愿意为了一两次飞行任务而去冒生命危险,只不过是好伙伴之间的一个小小分歧,就这么回事。”

    “他们会相信吗?”

    “等到他们看到我们成了多么亲密的朋友,读到那些通讯,读到那些你吹捧我和卡思卡特上校的话时,他们自然就会相信了。别为这些人操心。你走了以后,他们是很容易管教和控制的。只有当你仍然呆在这里时,他们才会惹事生非。你知道,一只坏苹果能毁了其它所有苹果。”科恩中校故意用讽刺的口气结束了他的这番话。“你知道——这办法真是太棒了——你也许能成为激励他们执行更多飞行任务的动力呢。”

    “要是我国国以后谴责你们呢?”

    “在你接受了我们的勋章、提拔和全部的吹捧之后吗?没有人会相信你的话的,军方不会允许你这样做。再说,你倒是为了什么竟想这样做呢?你将成为我们中的一员,记住了吗?你将过上富裕、豪华的生活,你将得到奖赏和特权。如果你仅仅为了某条道德准则而抛弃这一切的话,那你就是个大傻瓜,可你不是个傻瓜。成交吗?”

    “我不知道。”

    “要么接受这笔交易,要么接受军法审判。”

    “这样一来我就对中队里的弟兄们玩弄了一个极为卑鄙的骗局,不是吗?”

    “令人作呕的骗局。”科恩中校和蔼可亲地表示同意。他眼中闪烁着暗自高兴的微光,耐心地望着约塞连,等待着他的答复。

    “见鬼去吧!”约塞连大叫道,“如果他们不想执行更多的飞行任务,那就叫他们像我这样站出来采取行动,对吗?”

    “当然对,”科恩中校说。

    “我没有理由为了他们去冒生命危险,对吗?”

    “当然没有。”

    约塞连迅速地咧嘴一笑,做出了决定。“成交了!”他喜气洋洋地宣布。

    “好极了,”科恩中校说。他表现得并没有像约塞连指望的那么热情。他从卡思卡特上校的办公桌上滑下来站到地板上,先扯了扯裤子和衬裤裆部的皱纹,随后才伸出一只软绵绵的手来让约塞连握住。“欢迎你入伙。”

    “谢谢,中校。我——”

    “叫我布莱基,约翰。我们现在是伙伴了。”

    “当然啦,布莱基。我的朋友叫我约-约。布莱基,我——”

    “他的朋友叫他约-约,”科恩中校大声对卡思卡特上校说,“约-约迈出了多么明智的一步,你为什么不祝贺他呢?”

    “你迈出的这一步的确非常明智,约-约,”卡思卡特上校边说边笨拙而热情地使劲握住约塞连的手。

    “谢谢你,上校。我——”

    “叫他查克,”科恩中校说。

    “当然啦,叫我查克。”卡思卡特上校热诚而局促地哈哈一笑。

    “我们现在是伙伴了。”

    “当然啦,查克。”

    “笑着出门吧。”科恩中损说着把两只手分别搭在了他们两个人的肩膀上,三个人一起朝门口走去。

    “哪天晚上过来跟我们一块吃顿饭吧,约-约,”卡思卡特上校殷勤地邀请道,“今天晚上怎么样?就在大队部的餐厅里。”

    “我非常乐意,长官。”

    “叫查克,”科恩中校责备地纠正道。

    “对不起,布莱基。查克。我还没有叫习惯。”

    “这没关系,伙计。”

    “当然啦,伙计。”

    “谢谢,伙计。”

    “别客气,伙计。”

    “再见,伙计。”

    约塞连亲亲热热地挥手向他的新伙伴告别,溜达着朝楼厅走廊走过去。等到剩下他一个人时,他差一点高声唱了起来。他自由了,可以回国了。他达到了目的,他的反抗成功了,他平安无事了。

    再说,他并没有做任何对不起别人的事情。他逍遥自在、兴高采烈地朝楼梯走去。一个身穿绿色工作制服的士兵朝他行了个礼,约塞连快活地还了一个礼。出于好奇,他看了那个士兵一眼。他感到奇怪,这个士兵看上去十分面熟,就在约塞连还礼时,这个身穿绿色工作制服的士兵突然变成了内特利的妓女。她手里拿着一把骨柄厨刀凶神恶煞般地朝他劈了下来,一刀砍在他扬起的那只胳膊下面的腰上。约塞连尖叫一声,倒在了地上。他看到那女人又举刀朝他砍下来,便惊骇地闭上了很睛,就在这时,科恩中校和卡思卡特上校从办公室里冲了出来,把那个女人吓跑了,这才救了他的命。

    不过,他已经失去了知觉
TODO
