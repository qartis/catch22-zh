\chapter{德·科弗利少校}
 
    移动了轰炸路线,没有骗过德国人,反倒骗了德-科弗利少校。

    他打点好野战背包,调用了一架飞机。他有个印象,好像佛罗伦萨也让盟军给占领了,于是,便要人开飞机送他去佛罗伦萨,租两所公寓,好让中队官兵休假时有个安身的地方。等到约塞连向后跳出梅杰少校办公室,寻思着下面该求谁帮忙的时候,德-科弗利少校还没有从佛罗伦萨回来。

    德-科弗利少校不苟言笑,令人敬畏,却是一个极好的老头儿,长一颗硕大的狮子脑袋,一头松散杂乱的白发,仿佛一场大风雪,在他那张家长似的严峻的面孔四周肆虐。正如丹尼卡医生和梅杰少校所推测,他作为中队主任参谋的全部职责,实实在在就是掷马蹄铁,绑架意大利劳工,还有为中队官兵外出休假租借公寓。

    每当像那不勒斯、罗马或佛罗伦萨这样的城市即将陷落,德-科弗利少校便会打点好自己的野战背包,调用一架飞机和一名飞行员,把他送走。办妥这一切,他无需说一句话,仅凭藉他那张严厉专横的脸所具有的威力,以及他那根多皱的手指打出的武断手势。

    城市陷落后一两天,他便回到中队,同时带回两所豪华大公寓的租约,军官和士兵各占一所,且都已配备了成天乐呵呵的称职的厨师和女佣。几天之后,世界各地的报纸便会刊登出那些踩着瓦砾冒着烟雾最先攻进已炸成废墟的城市的美国士兵的照片。在这些士兵当中,必定会有德-科弗利少校。他像一根通条似的直挺挺地坐在一辆不知从什么地方弄来的吉普车里,目不斜视地盯着正前方,炮火在他那颗坚不可摧的脑袋四周爆炸。行动轻快敏捷的年轻的步兵们端着卡宾枪,或是在着了火的建筑物的掩蔽下,沿着人行道大步冲向前,或是在建筑物的出入口倒毙身亡。德-科弗利少校依旧端坐车上,四周处处是危险,可他好像是永远摧毁不了的,依旧毫不动摇地铁板着那张中队上下无人不识、无人不敬畏的面孔:凶险,威严,正直,严厉。

    对德国情报机构来说,德-科弗利少校是个令人伤透脑筋的谜。许许多多的美国战俘中,竟没有一个提供过有关这位白发老军官——一副饱经了风霜的面容令人生畏,两只炯炯的眼睛咄咄逼人,似乎每一次发动重大进攻,他都那么无所畏惧地冲锋在前,而且又是每战必胜——的任何具体的情报。对美国当局来说,他的身份也同样令人困惑;他们曾从刑事调查部派出了整整一个团的一流高手,前往各路前线,查明他的真实身份。同时,一大批久经沙场的新闻发布官,奉命一天二十四小时处于紧急状态,一旦打听到德-科弗利少校,就立即着手宣传他。

 


    在罗马,德-科弗利少校尽了最大的努力,替中队官兵安排度假公寓。军官们——通常是四五人一组来罗马的——住的是一幢崭新的白色的石砌公寓大楼,每人一间宽大的双人房。楼里有三间宽敞的浴室,墙壁贴的是闪亮的浅绿色瓷砖。大楼女仆名叫米恰拉,人瘦得皮包骨,见到什么事都傻笑,倒是把公寓整理得有条不紊,一尘不染。楼下住的是见人必阿谀奉承的房东;楼上住的是一位漂亮富有的黑发伯爵夫人和她那个同样漂亮富有的黑发媳妇,婆媳俩只愿意献身内特利和阿费。但,内特利太羞怯,没敢要她们;

    阿费则太古板,也没占有这婆媳俩的玉体,这家伙竟还想劝她们,除自己的丈夫——偏偏留在了北方,经营家族的生意,千万别献身其他任何一个男人。

    “这婆媳俩真是一对尤物。”阿费很认真地跟约塞连道出了自己的心里话。而约塞连朝思暮想的,正是希望这一对漂亮富有的黑发尤物一同赤裸了玉体,伸展四肢跟他躺在床上,调情做爱。

    士兵们通常是十二人左右结伙来罗马,带来的是特大的胃口,还有一只只塞满罐装食品的沉甸甸的柳条箱,好让女仆们烧了,给他们端到公寓餐厅,侍候他们进餐。士兵们住的公寓在一幢红色的砖砌楼房的六层楼上,上下楼由一部电梯运送,开起来老是丁零当啷作响。士兵们住的地方,总是要热闹得多。首先是士兵人数一向比较多,还有不少女人侍候他们,替他们做饭,收拾房间,擦洗地板。而且,总是不断有约塞连找来的淫荡却又傻里傻气的颇肉感的年轻女子。此外,还有士兵们自己带来的年轻姑娘,待他们精疲力竭地放纵了一个星期,困倦地返回皮亚诺萨岛时,便把姑娘们留了下来,供后来的士兵尽情享用。姑娘们有得住,有得吃,想呆多久就呆多久。她们唯一要做的,就是顺从任何一个想跟她们上床睡觉的士兵,以此作为报答。对她们来说,这样的安排似乎是再理想不过了。

 


    要是亨格利-乔不幸再次完成自己的飞行任务后,驾驶军邮班机,每隔四天左右,他便像备受了折磨一般,嘶哑了嗓音,发狂地闯来罗马。大多数时候,他住在士兵的公寓里。德-科弗利少校究竟租了多少房间,谁也说不准,就连住底层的那个穿黑色紧身胸衣的胖女人也搞不明白,虽说房间是她租给德-科弗利少校的。德-科弗利少校租下了顶层所有的房间,约塞连知道,一直到五楼还有他租的房间。轰炸博洛尼亚后的那天上午,亨格利-乔在军官公寓里发现约塞连跟露西安娜同床睡觉,竟着了魔似的跑去取自己的照相机,这后来,约塞连在五楼斯诺登的房间里最终找到了那个手持干拖把、身穿灰白色短裤的女佣人。

    那个身穿灰白色短裤的女佣人是个热心肠,生性快乐,年纪三十五岁左右,身材肥胖,那条灰白色的短裤紧裹着两条软绵绵的大腿,还有不停地左右扭动的屁股。只要有男人需要,不管是谁,她都会把这短裤脱了。她相貌极平常,一张宽宽的脸盘,尽管如此,却是世界上最公正的女人:她为每个男人躺下,不论种族、信仰、肤色,或是国籍,把自己当做社会性的财物贡献出去,以此表示自己的殷勤好客。一旦有人把她抱住,不管当时手里抓的是抹布,还是扫帚,或是干拖把,她也不会为了搁下这些东西而耽误片刻的时间。她的诱惑力也就在于她容易到手。她就像是埃佛勒斯特峰,始终耸立在那里,男人们一旦欲火中烧,使爬上她的身体。约塞连迷上了这个穿灰白色短裤的女佣人,因为她似乎是世上剩下的唯一的女人,他可以不动真情地跟她做爱。就连西西里岛那个秃顶姑娘也还唤起他内心强烈的情感:怜悯,温情,惋惜。

 


    德-科弗利少校每次租公寓,总会遇上不少危险,尽管如此,他唯一的一次受伤,竟出乎意料地发生在他率凯旋的队伍进入不设防的罗马城的时候。当时,一个衣衫褴褛的醉老头一个劲地格格直笑,站在近处,对着德-科弗利少校猛掷去一朵花,不料,伤了他的一只眼睛。紧接着,那个撒旦一般的老头,幸灾乐祸地跃上德-科弗利少校的汽车,粗暴而又轻蔑地抓住德-科弗利少校那颗令人敬重的白发苍苍的脑袋,在左右两颊上嘲弄地吻了吻——嘴里有股酒、奶酪和大蒜混合的酸臭气味。随后,老头发出一阵呵斥似的沉闷的干笑,便又从车上跳回到欢庆的人群里了。德-科弗利少校仿佛身陷逆境的斯巴达人,自始至终没有在这场可怕的磨难面前畏缩半步。直到了结了在罗马的公务,回到皮亚诺萨岛,他方才去找医生,治自己的眼伤。

    他打定了主意,还是用两只眼睛瞧世界,于是,便对丹尼卡医生明确要求,必须给他用透明眼罩,便于他继续以完好的视力投掷马蹄铁,绑架意大利劳工,以及租借公寓。对中队官兵来说,德-科弗利少校实在是个大人物,不过,他们从来就没敢当面跟他这么说。唯一敢跟他说话的,只有米洛-明德宾德。来中队后的第二个星期,米洛便来到马蹄铁投掷场,手拿一只煮鸡蛋,高高举起,让德-科弗利少校瞧。见米洛如此放肆,德-科弗利少校深感惊讶地直挺起了身体,满脸怒容,两眼瞪着他,布满深深皱纹的额头直凸向前,峭壁似的弓形大鼻子,仿佛一名十大学联合会的进攻后卫,愤然地猛冲前去。米洛丝毫不退却,防卫地高举了那只煮蛋,仿佛是具有魔力的护身符,挡在自己的面前。风暴最终平息了下去,危险也随之过去。

    “那是什么?”德-科弗利少校最终问道。

    “一只蛋,”米洛答道。

    “什么样的蛋?”德-科弗利少校问。

    “煮蛋,”米洛回答。

    “什么样的煮蛋?”德-科弗利少校问。

    “新鲜的煮蛋,”米洛回答。

    “哪来的新鲜蛋?”德-科弗利少校问。

    “鸡下的呗,”米洛回答。

    “鸡在哪儿?”德-科弗利少校问。

    “鸡在马耳他,”米洛回答。

    “马耳他有多少鸡?”

    “有足够的鸡给中队的每一位军官下新鲜鸡蛋吃,从食堂经费里拿出五分钱,就能买一只鸡蛋。”

    “我特爱吃新鲜鸡蛋,”德-科弗利少校坦白道。

    “要是中队里有人让一架飞机给我用,我就可以每星期飞一次去那里,把我们需要的所有新鲜鸡蛋全带回来,”米洛回答说,“毕竟,马耳他不算怎么太远。”

    “马耳他是不算怎么太远,”德-科弗利少校说,“你或许可以开一架中队的飞机,每星期飞一次去那里,把我们需要的新鲜鸡蛋全部带回来。”

    “行,”米洛一口答应,“只要有人让我去做,再给我一架飞机,我想我能办到。”

    “我喜欢煎新鲜鸡蛋吃。”德-科弗利少校想了起来。“用新鲜黄油煎。”

    “我可以在西西里买到我们需要的所有新鲜黄油,两毛五分钱一磅,”米洛回答说,“新鲜黄油两毛五分钱一磅,挺合算的。食堂经费里还有足够的钱买黄油,再说,我们或许可以卖一些给其他中队,赚些个钱,把我们自己买黄油的大部分钱给捞回来。”

    “你叫什么名字,孩子?”德-科弗利少校问。

    “我叫米洛-明德宾德,长官,今年二十七岁。”

    “你是个挺不错的司务长,米洛。”

    “我不是司务长,长官。”

    “你是个挺不错的司务长,米洛。”

    “谢谢您,长官。我一定尽自己的全力,做一名称职的司务长。”

    “愿上帝保佑你,我的孩子。拿一只马蹄铁。”

    “谢谢您,长官。我拿了它该怎么办?”

    “掷它。”

    “掷掉吗?”

    “对着那边的那根木桩掷过去,然后再去把它拣起来,对准这根木桩掷过去。这是一种游戏,明白吗?你把那只马蹄铁拣回来。”

    “是,长官。我明白了。马蹄铁卖多少价钱?”

    一只新鲜鸡蛋在一汪新鲜黄油里热腾腾地煎着,劈劈啪啪直响,香味随地中海信风飘去了很远的地方,馋得德里德尔将军胃口大增,飞速地赶了回来,随他一起来的,是形影不离地伴着他的那个护士和他的女婿穆达士上校。起初,德里德尔将军一日三餐都在米洛的食堂里吃得狼吞虎咽。后来,卡思卡特上校大队的其他三支中队亦把各自的食堂交托给了米洛,同时又各配给他一架飞机和一名飞行员,好让他也能替他们采购新鲜鸡蛋及新鲜黄油。于是,一周七天,米洛坐了飞机不停地来回奔波,而四支中队的每一位军官倒是在贪得无厌地吞食新鲜鸡蛋了。每天早中晚三餐,德里德尔将军都是狼吞虎咽地吃新鲜鸡蛋——正餐之间还要大吃好多新鲜鸡蛋。直到米洛采购来了大量新鲜小牛肉、牛肉、鸭肉、小羊排、蘑菇菌盖、花茎甘蓝、南非龙虾尾、小虾、火腿、布丁、葡萄、冰淇淋、草莓和朝鲜蓟,他这才不再大吃新鲜鸡蛋了。德里够尔将军的作战联队还有另外三支轰炸大队,他们因眼红,便都派了各自的飞机去马耳他购买新鲜鸡蛋,但却发现那里的鸡蛋卖七分钱一只。既然从米洛那里能五分钱买一只,那么,在他们,把各自的食堂也交托给米洛的辛迪加联合体,并给他配备所需的飞机和飞行员,空运来他曾答应供给的所有其他美味食品,这才是更为明智的选择。

    这一事态的发展,着实令大家兴高采烈,尤其是卡思卡特上校,更是兴奋至极,他确信自己赢得了荣誉。每次见到米洛,他总是乐呵呵地打招呼。同时,他又因抱愧而显出极度的慷慨,竟一时冲动、提议擢升梅杰少校。他的提议一到第二十七空军司令部,当即被前一等兵温待格林驳回。温特格林匆匆作了个批示,言辞简慢,且又无署名:陆军部只有一个梅杰-梅杰-梅杰少校,不打算只为了讨好卡思卡特上校就提升梅杰少校而最终失去他。这一番粗暴的叱责刺痛了卡思卡特上校。上校深感疚惭,躲在自己的房里,痛苦万分,拒不见人。他把这次出丑归咎于梅杰少校,于是决定当天便降他为尉官。
 


    “或许他们不允许你这么做的,”科恩中校很是傲慢地笑了笑说道,一面仔细琢磨着这桩事。“理由就跟他们不让你提升他完全一样。再说,你才想要把他升到跟我同军衔,这会儿却又要降他为尉官,你这么做,必定会让人觉得你实在是太愚蠢了。”

    卡思卡特上校感到束手无策。当初,弗拉拉一战大败后,他还那么轻而易举地让约塞连得了枚勋章。卡思卡特上校曾主动要求让自己的部下去炸毁波河大桥,可是七天过后,大桥依旧完好无损地横跨河上。六天的时间里,他的士兵们飞了九次去那里,但大桥终究没被摧毁。直到第七天,士兵们第十次去那里执行任务,才炸了那桥。约塞连引着他小队的六架飞机,第二次飞入目标上空,结果,让克拉夫特和他的机组人员全部丧了命。执行第二次轰炸时,约塞连很谨慎,因为当时他无所畏惧。他一直专注于轰炸瞄准器,待炸弹投放出,才抬起头;当他举起头来,便见机舱至弥漫了一种奇怪的桔黄色光。起先,他以为是自己的飞机着了火。紧接着,他便在自己头顶正上方发现了那架引擎着火的飞机,于是通过内部通话系统,高叫着让麦克沃特急速左转。片刻后,克拉夫特飞机的机翼断裂,燃烧着的飞机残骸往下坠落,先是机身,再是那旋转着的机翼,与此同时,阵雨般的金属小碎片啪喀啪喀地打在了约塞连自己的飞机顶上。一刻不绝的高射炮火依旧砰砰砰地在他的周围作响。

    待返回地面,约塞连便于众人阴冷的目光下,气急败坏地走到布莱克上尉——正站在绿色护墙楔形板搭建的简令下达室外面——身边,想向他汇报战况;于是便得知卡思卡特上校和科恩中校正在里边等着跟他谈话。丹比少校站在那儿,把守着门,脸色灰白,一语不发,挥挥手把其余的人一一支开了去。约塞连疲惫得不行,恨不得马上卸了这一身黏叽叽的衣服。他心绪不宁地走进简令下达室,实在不知道自己对克拉夫特和其他几个人该有什么样的感觉。因为他们当时是在远处默默忍受着孤立无援的痛苦中阵亡的,也就是在那一瞬间,他自己灾难临头,身陷同样令人苦恼、恶劣透顶的窘境:要么尽职,要么毁灭。

    卡思卡特上校同样也让这件事给搅得心神不安。“两次?”他问道。

    “要不然,我第一次或许炸不到目标,”约塞连垂下头,低声答道。

    他们的声音在狭长的平房里轻轻回响着。

    “可是轰炸了两次?”卡思卡特上校实在很是怀疑,便再又问了一遍。

    “要不然,我第一次或许炸不到目标。”约塞连重新答了一句。

    “可是克拉夫特或许就能活着回来。”

    “那么桥或许还是完好无损的。”

    “受过训练的轰炸员应该第一次就投放炸弹,”卡思卡特上校提醒他说,“其余五个轰炸员都是第一次就投放炸弹的。”

    “但都没有击中目标,”约塞连说,“我们就不得不再飞回去一次。”

    “或许你第一次就该炸了那桥的。”

    “或许我压根就炸不了它。”

    “但或许就不会有什么损失了。”

    “要是桥还没有炸毁,或许损失就会更大了。我想你要的是让人把桥炸掉。”

    “别跟我争辩,”卡思卡特上校说,“我们的麻烦已经够多的了。”

    “我不是在跟您争辩,长官。”

    “不,你是在跟我争辩。就连这句话也是在争辩。”

    “是,长官。实在是很抱歉。”

    卡思卡特上校使劲扼了指关节,格格地直响。五短身材的科恩中校,肤色黝黑,肌肉松弛,挺着个极不匀称的大肚子,很是悠闲自在地坐在前排的一张长椅上,两手舒坦地搭在他那黑不溜秋的秃顶上,一双眼睛躲在那副闪闪发亮的无边眼镜后面,流露出顽皮的神情。

    “我们尽力绝对客观地对待这件事。”他提醒卡思卡特上校。

    “我们尽力绝对客观地对待这件事,”卡思卡特上校突然计上心来,于是就热情地对约塞连说,“倒不是我感情用事或是别的什么原因。我压根就不在乎死那几个人或是损失那架飞机。只是写进报告太难看了。我在报告里该怎样掩饰这样的事呢?”

    “您何不给我一枚勋章呢?”

    “就因为你轰炸了两次?”

    “那次亨格利-乔因失误而撞毁了飞机,您就给了他一枚勋章。”

    卡思卡特上校很是悔恨地窃笑了一下。“不送你上军事法庭,就算你走运啦。”

    “可我第二次就炸了那座桥,”约塞连抗辩道,“我想您要的是让人把桥炸掉。”

    “哦,我也不清楚自己要什么,”卡思卡特上校恼羞成怒,大声说道,“哎,我要的当然是让人把桥炸了。自从我决定派你们出去炸毁那座桥以后,它就接连不断给我带来烦恼。你为什么就不能第一次把它炸了呢?”

    “我没有足够的时间。我的领航员当时也没法确定我们是否到了指定的城市。”

    “指定的城市?”卡思卡特上校困惑了。“你是想把所有责任推给阿费喽?”

    “不,长官。是我的过错,让他分散了我的思想。我想说的是,我不是绝对不犯错误的。”、“谁也不是绝对不犯错误的,”卡思卡特上校严厉他说。接着,他想了想,含糊其辞地又说道:“同样,谁也不是必不可少的。”

    约塞连不再反驳。科恩中校伸了个懒腰。“我们该作决定了。”

    他随口对卡思卡特上校说了一句。

    “我们该作决定了,”卡思卡特上校对约塞连说,“这一切全都是你的过错。你干吗要飞两次呢?你为什么就不能像所有别的人那样第一次就投炸弹?”

    “第一次我可能会炸不了那桥。”

    “我觉得好像我们这会儿的谈话是在转第二圈了,”科恩中校暗自笑了笑,插嘴道。

    “可是我们该怎么办?”卡思卡特上校极是苦恼地大声叫道,“其他人都在外面等着呢。”

    “我们何不给他一枚勋章呢?”科恩中校建议道。

    “就因为他飞了两次?我们给他一枚勋章,凭什么?”

    “就凭他飞了两次这一点,”科恩中校沉思片刻,自鸣得意地笑了笑,答道,“说实话,当时周围没有其他飞机帮着转移高射炮的人力,在那种情况下,要在目标上空再盘旋一次,我想这实在是需要足够的胆量。而且他确实炸了那座桥。你要知道,凡是碰上该让我们感到羞耻的事,我们反倒要自吹自擂——这或许是解决问题的办法。这是一门诀窍,好像从来就不会出什么差错似的。”

    “你觉得这样行吗?”

    “保证没问题。让我们再提升他为上尉,这样就万无一失了。”

    “难道你不觉得我们这么做有些过头了吗?”

    “不,我倒不这么看。办事最好是稳当一些。再说,一个上尉实在是没什么了不起的。”

    “好吧。”卡思卡特上校拿定了主意。“我们就给他发一枚勋章,嘉奖他两次勇敢地飞越轰炸目标上空。同时再提升他为上尉。”

    科恩中校伸手取过帽子。

    “出门时得面带笑容,”他开玩笑他说,一手搂住约塞连的肩膀,两人一同走出了门
