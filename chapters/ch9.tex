\chapter{梅杰·梅杰·梅杰少校}
 
    梅杰-梅杰-梅杰少校自呱呱坠地起,便是不很顺当的。

    他跟米尼弗-奇维一样,出娘胎那会儿拖的时间过长——足足拖了三十六个小时,结果,把他母亲的身体给拖垮了。她母亲是个温柔、多病的女人,临盆前足足痛了一天半,才把梅杰生下来,产后,便全没了心思去跟丈夫争执给新生婴儿取名。医院的过道里,她丈夫严肃而又果断地忙着该他做的一切,他是个极有主心骨的男人。梅杰少校的父亲是个瘦高个儿,着一套毛料服装和一双笨重的鞋子。他丝毫不迟疑地填写了婴儿出生证明书,之后,便很镇静地把填好了的出生证明书交给楼层主管护士。护士一声不吭地从他手中接了过去,于是就放轻脚步走开了。他目送着她离开,一边在纳闷,不知道她贴身穿的是什么内衣裤。

    他回到病房,见妻子软绵绵地躺在病床上,身上盖着毛毯,活像一棵失了水分的萎蔫的蔬菜,皱巴巴的面孔又干瘪又苍白,衰弱的躯体一动不动。她的床在病房最尽头,临近一扇尘封的破窗。大雨哗哗地从喧闹的天空瓢泼下来。天阴沉冷峭。医院的其他病房里,那些惨白得见不到一丝血色的病人,正等候着死神的最终降临。梅杰少校的父亲直挺挺地站立在病榻一旁,垂下头,久久地注视着自己的女人。

    “我给孩子取了个名,叫凯莱布,”临了他低声跟她说,“是照了你的意思取的。”女人没有答话,慢慢地,男人便笑了起来。这句话是他经过精心的考虑之后,才说出口的,因为他妻子睡着了,永远也不会知道,就在她躺在县医院这间破旧的病房里的病床上时,自己的丈夫竟对她说了谎。

    正是从这艰难的起点,走出了这位无能的中队长。眼下,他正在皮亚诺萨岛,每天的大部分工作时间全都用来在公文上假冒签华盛顿-欧文的名字。为了避免有人识别出他的笔迹,梅杰少校煞费了苦心,左手签名。他把自己隔离了起来,并利用自己不曾希图的职权,禁止任何人侵扰他。同时,他又用了假胡子和墨镜伪装自己,以防有人偶然从那扇尘封的赛璐珞窗户——有个小偷在上面挖了一道口子——外面往里张望,发现秘密。从最初卑贱的出身到取得如今不怎么起眼的成功,梅杰少校走过了三十一年的凄怆岁月,尝尽了孤寂和挫折。

 


    梅杰少校是姗姗来迟地来到这世上的,实在太缓慢,而且天生就是平庸透顶的人物。有些人是天生的庸才,有些人则是后天一番努力后才显出庸碌无能的,再有些人却是被迫平庸地过活的。至于梅杰少校,他是集三者于一身。即便是在平庸的人中间,他也毫无疑问要比所有其余的人来得平庸,因此反倒很突出了。只要是见过他的人,总有很深的印象,他这人实在是太平常太不起眼了。

    梅杰少校自一出世便背上了三个不利因素——他母亲、他父亲和亨利-方达。差不多从出娘胎的那一刻起,他就显出与亨利-方达有叫人受不了的酷肖相貌。还在他不清楚亨利-方达为何人之前,曾有很长一段时间,无论走到什么地方,他总是发现别人把他跟亨利-方达放一块,做些令他很难堪的比较。素不相识的人都觉得应该轻视他,结果,害得他自小就像犯了罪似地惧怕见人,而且还讨好地迫不及待地想跟人家道歉:他的确不是亨利-方达。生就了一副酷似亨利-方达的相貌,在他说来,要这样走完一生的路,实在不是桩容易的事。然而,他继承了父亲——极富幽默感的瘦高个儿——百折不回的品性,从来就不曾有过一丝逃避现实的念头。

    梅杰少校的父亲一向为人持重,又很敬畏上帝。依他看,谎报自己的年龄,是他最得意逗人的笑话。他是个农民,四肢细长,却能吃苦耐劳,同时,他又是个敬畏上帝、热爱自由、尊纪守法的个人主义者。他认为,如果联邦政府援助别人,而不援助农民,这便是奴性社会主义。他提倡勤俭,很讨厌那些曾拒绝过他的浪荡女人。种植苜蓿是他的专长,可他倒是因为没种一棵苜蓿而得到了不少利益。

    政府依据他没有种植的苜蓿的多少,以每一蒲式耳为单位,付给他一笔相当数量的钱。他没有种植的苜蓿的数量越大,政府给他的钱也就越多。于是,他便用这笔没出力而挣到手的钱,购置新的田产,以此来扩大自己没有种植的苜蓿的数额。为了不生产苜蓿,梅杰少校的父亲一刻都不曾停歇过。到了漫长的冬夜,他便待在屋里,搁着马具不修理。每天到了中午那一会儿,他就会跳下床来,只是为了查明的确没有人会把杂活做掉。他很聪明,知道该如何投资田产,不久,他没有种植的苜蓿的数量超过了县里的任何一个农民。于是,四邻的农民都跑来请教他方方面面的问题,因为他挣到了很多钱,所以必定是个聪明人。“种瓜得瓜,种豆得豆嘛。”他给大伙儿提了这么一条忠告。临了,大伙儿便道:“阿门。”

 


    梅杰少校的父亲直言不讳,力主政府厉行节约,但其前提是,丝毫不影响政府的神圣职责——以农民能接受的高价,收购他们生产却没人想要的全部苜蓿,或者支付他们一定数额的钱,作为对他们没有种植一棵苜蓿的酬劳。他这个人相当傲慢,而且极有主见。他反对失业保险,只要能够敲诈到大笔的钱财,无论是向谁,他部会毫不迟疑地使出各种着数,或是哼哼唧唧地诉苦,或是一把鼻涕一把泪地哭诉,或是甜言蜜语地哄骗。他是个很虔诚的人,不管走到什么地方,总是要做一番传道。

    “上帝赐给了我们这些善良的农民一双强有力的手,这样,我们就可以用这两只手尽量多捞多拿。”他时常满腔热情地布道,不是站在县政府大楼的台阶上,就是站在大西洋一太平洋食品商场的前面,一边等着他正在找的那个脾气暴躁、口嚼口香糖的年轻出纳员出来,狠狠地瞪自己一眼。“假如上帝不想让我们尽量多捞多拿的话,”他讲道,“那么,他就不会赐给我们这么好的一双手了。”

    其余的人便低声道:“阿门。”

    梅杰少校的父亲和加尔文教信徒一样,也信仰宿命论。他可以清楚地看到,不管是谁碰上了什么触楣头的事情,全都是上帝的意志的体现,不过,他自己的那些不幸却尽是例外。他抽烟,喝威士忌酒。靠了能说会道和振奋人心的机巧的谈话——尤其是他谎报自己年龄时,或是讲述有关上帝及他妻子难产生下梅杰少校的那段颇令人发噱的趣话时编造出的话,他腾达了。有关上帝及他妻子难产的那段趣话是这样说的:上帝创造整个世界,只用了六天的时间,而他妻子光为了生下梅杰少校,分娩期足足持续了一天半。那天,要是换了个不中用的家伙,或许会站在医院的过道里束手无策;要是换了个懦弱的家伙,或许会妥协了,给孩子取其他一些极好听的名字,但,梅杰少校的父亲熬了十四年,才等到这么一个机会,他是无论如何不愿错过的。

    关于机会,他说过一句颇有意味的笑话。“机不可失,时不再来。”这是他时常说的。这句颇有意味的笑话,梅杰少校的父亲只要有了机会,便会重复着说。

 


    梅杰少校没有欢乐的一生中,命运自始至终接二连三地对他进行恶作剧,使他成了不幸的牺牲品。这些恶作剧中,最早的便是让他生就一副叫人极不舒服的酷似亨利-方达的相貌。第二个恶作剧,是他一出世就给取了梅杰-梅杰-梅杰这么个名字。他一生下来就被取名梅杰-梅杰-梅杰,这件事是桩秘密,只有他父亲一人知晓。直到梅杰少校注册入幼儿园,人们才发现了他的真名,而且也因此造成了灾难性的后果。他母亲的性命给断送了,她不想再活下去,于是,日渐消瘦下去,最终离开了人世。然而,这在梅杰少校的父亲实在是桩好事,因为他早就决定,如果逼不得已,就跟大西洋一太平洋食品商场那个坏脾气姑娘结婚。再说,要是她不死,想不给她一笔钱,或是不给她一顿毒打,就休掉她,对这种可能性,他一向是不怎么乐观的。

    自己真名的发现,也影响到了梅杰少校本人,其严重的程度并不亚于她母亲所受的打击。以前,他一直误以为自己是卡莱勃-梅杰,可是在这么幼小的年纪,突然令人震惊地被迫承认,自己不是卡莱勃-梅杰,而是某个毫不相识的陌生人,叫什么梅杰-梅杰-梅杰,对这人,不仅他自己一无所知,而且也没有别的什么人听说过。

    无论如何,这是一件残酷的事。从此,曾跟他一起玩耍的同伴离开了他,而且再也没有来找过他,因为他们对所有陌生人一向是不信任的,尤其不信任一个因自称是他们相识多年的朋友而早让他们上了当的骗子。没人愿意跟他有什么来往。他开始丢三落四,说话结结巴巴。每次接触生人,他总显得很羞怯而又充满希望,但临了总是失望。他太需要有一个朋友了,结果一个也没找到。就这样,他不合时宜地长大长高了,变成了一个古里古怪的爱幻想的小伙子——一双脆弱的眼睛,一张极纤巧的嘴巴:每次遭到别人拒绝交往,那张嘴微露出的怯生生的试探性一笑,便即刻收敛起来,继而是受了伤害后的失态。

    于长辈,梅杰少校一向是很恭敬的,可长辈却讨厌他。只要是长辈的吩咐,他什么事都做。他们告诉他,遇事要谨慎,于是,不论遇到什么事情,他一向都很谨慎;他们告诉他,千万不要把当天能做的事情,拖到第二天,他也就做到了当日事当日毕;他们跟他说,要尊敬父母,他就尊敬父母;他们还跟他说,入伍前不应该杀人,他也的确做到了,一个人都没杀。于是,入伍服役了,长辈们便要他杀人,他就此开了杀戒。无论什么时候,他一贯逆来顺受。他一向以诚待人,就像他觉得别人也会这么待他一样。他一旦做善事,从来都是慷慨大度。他从不滥用上帝的名义,从不与人通奸,或是垂涎邻居的老婆。其实,他很爱他的邻居,从来就没有作过不利于邻居的伪证。梅杰少校的长辈们都讨厌他,因为他竟如此明目张胆地置约定俗成的传统规范于不顾。
 


    既然没有什么地方可以让他显身手,梅杰少校便在学校里出尽风头。在州立大学学习期间,他相当认真,结果,同性恋者怀疑他是共产主义者,而共产主义者则怀疑他是同性恋者。他主修的是英国历史,这本身就是个错误。

    “英国历史!”来自梅杰少校同一州的那位白发的资深参议员大发脾气,怒声训斥道,“美国历史怎么了?美国历史一点都不比世界上其他任何国家的历史逊色!”

    于是,梅杰少校即刻改学美国历史,但事不凑巧,这时,联邦调查局已经开始对他立案调查了。有六个人和一条苏格兰狗,住在那个梅杰少校称之为家的偏远的农舍里,而其中的五个人和那条苏格兰狗,原来竟是联邦调查局的探子。没过多久,他们便已掌握了大量不利于梅杰少校的材料,他们可以随意处置他。然而,他们能找到的唯一的处置办法,便是送他进陆军部队,当一名二等兵,四天后升他为少校,这样,议员们因为没有别的什么重重心事,就可以匆匆忙忙地来回走过华盛顿特区的一条条大街,边走边反复念叨:“是谁提升梅杰-梅杰的?是谁提升梅杰-梅杰的?”

    其实,是IBM公司的一台机器提升梅杰-梅杰的。这台机器跟梅杰少校的父亲一样,也是极幽默的。战争爆发时,梅杰-梅杰还是很顺从听话的。他们让他当兵,他就当了兵;他们让他申请到航空军校接受训练,他便顺从地照办了。可是,入伍的第二天凌晨三点,他和其他新兵竟光着脚,站在冰冷的烂泥里,面前是一个来自美国西南部的中士,这家伙蛮横霸道,又好斗成性。他告诉他们说,他可以痛打自己中队里的任何一个士兵,并且随时准备证实自己说的这句话。刚几分钟前,中士手下的几个下士极粗暴地摇醒了中队的所有新兵,命令他们到行政处的帐篷前集合。当时,天还在下雨,雨水直往梅杰-梅杰身上浇。新兵们穿着便服——是三天前入伍时随身带的——站好了队。那些因为穿鞋子和袜子而磨蹭了老半天才赶去集合的,结果又被命令回到各自阴冷潮湿、黑乎乎的帐篷里,脱掉鞋袜。新兵全都光了脚,站在烂泥里,中士用冷冰冰的目光,一一扫视了他们的脸,于是,告诉他们说,他可以痛打中队里的任何一个士兵。新兵呢,一个个懒得跟他争辩。
 


    第二天,梅杰-梅杰竟意外地晋升少校,一下子把那位好斗的中士打入灰心失望的无底深渊,因为他从此再也没法吹嘘什么他可以痛打中队里的任何一个士兵了。他躲在自己的帐篷里,跟扫罗一样,苦思冥想,不见任何来客,由下士组成的精锐警卫队垂头丧气地在门口替他站岗。次日凌晨三点,他想出了一条对策。梅杰少校和其他新兵再次被粗暴地摇醒,奉命冒着耀眼的蒙蒙细雨,光着脚赶往行政处的帐篷前集合。中士早就等候在那里,双拳紧握着叉在胯部两侧,一副盛气凌人的样子,很是急不可待地想训话,几乎等不及全体新兵集合完毕。

    “我和梅杰少校,”他夸口道,语调还是跟前一天晚上发话时一样:强硬、清脆、快速。“可以痛打中队里的任何一个士兵。”

    同一天晚些时候,基地的军官们就梅杰少校一事采取了行动。

    他们该如何对待梅杰少校这样的少校呢?要是当面羞辱他,那就等于贬损与他同军衔或是军衔比他低的所有军官。但要是很恭敬地待他,那实在是不可思议的事情。幸亏梅杰少校早就申请到航空军校接受训练。当天傍晚,梅杰少校的调令送到了油印室。次日凌晨三点,梅杰少校再次被粗暴地摇醒,中士向他道了声“一路平安”,于是,他便被送上了一架西去的飞机。

    当梅杰少校飞抵加利福尼亚,向沙伊斯科普夫少尉报到时,他依旧是光着一副脚板,脚趾沾满了烂泥,沙伊斯科普夫少尉一见,脸色顿时刷白。至于梅杰少校,当有人再次粗暴地把他摇醒时,他便想当然地以为,肯定又是光着脚站在烂泥里,因此就把鞋子和袜子留在了帐篷里。向沙伊斯科普夫少尉报到时,他还是穿了那身便服,皱皱巴巴、脏不拉叽的。当时,沙伊斯科普夫少尉还没有在阅兵比赛中扬名,一想到下星期天梅杰少校光着脚和他中队的全体士兵一起接受检阅时的那副模样,他便不由得浑身一阵剧烈的战栗。

    “赶快去医院”,当他彻底缓过神来,可以说话时,沙伊斯科普夫少尉咕哝道,“告诉他们说,你身体不舒服。你就留在那儿,等拿到制服津贴,有钱买几件衣服后,你再回来。还有几双鞋子。买几双鞋子。”

    “是,长官。”

    “我想你没必要喊我‘长官’,长官,”沙伊斯科普夫少尉向他指出,“你的军衔比我高。”

    “是,长官。我的军衔或许是比你高,长官,可你毕竟还是我的指挥官。”

    “是,长官,你说的没错。”沙伊斯科普夫少尉表示同意。“你的军衔或许是比我高,但我毕竟还是你的指军官。因此,你最好按我的吩咐去做,长官,不然你会倒霉的。到医院去,告诉他们说,你身体不舒服,长官。你就留在那儿,等拿到制服津贴,有钱买几件制服后,你再回来。”

    “是,长官。”

    “还有几双鞋子,长官。一有机会,你就先买几双鞋子,长官。”

    “是,长官。我一定买,长官。”

    “谢谢你,长官。”

    在梅杰少校,军校生活和以前那么多年的生活没有什么差别。

    不管他跟谁呆在一块儿,那人总想把他撵走,希望他跟别的什么人呆在一起。每到一个阶段,教官们就给他优待,为的是让他赶快结束训练期,好尽早打发他离开军校。梅杰少校几乎没用多长时间,便训练合格,获得了空军飞行胸章,于是,即刻被遣往海外。到了海外,一切突然好转了起来。对梅杰少校来说,被别人当做自己人,是他这辈子梦寐以求的事情。到了皮亚诺萨岛,没过多久,他的愿望最终成了现实。军衔,在投身作战行动的军人眼里,实在是毫无半点价值,军官和兵士间的关系,无拘无束,轻松自在。有些人,尽管梅杰少校连名字都不知道,却跟他招呼一声“喂”,邀请他一起游泳,或是打篮球。他每天最畅快的时刻,便是耗在一场场从早到晚的篮球比赛上,谁都不在乎输赢,也从不记录比分,每场球赛的人数不等,多则三十五人,少则一人。梅杰少校先前从未打过篮球,也不曾玩过别的什么球,不过,他身材高大,上窜下跳,再加上着了魔似的勃勃兴致,倒是弥补了他天生的笨拙和缺乏经验的不足。在那方倾斜的篮球场地上,和那些差不多成了他朋友的官兵一起玩球,梅杰少校寻到了真正的快乐。赛球既然没有赢家,自然也就无所谓输家了。梅杰少校又是蹦又是跳,每一刻他玩得都十分尽兴。直到杜鲁斯少校死后的一天,卡思卡特上校坐了吉普车轰隆隆地开进营地,从此,梅杰少校便再也不可能在篮球场上尽情地打篮球了。

    “你现在是新任的中队长啦,”卡思卡特上校隔着铁路壕沟,冲着梅杰少校很粗鲁地喊道,“不过,别以为这有什么了不起,因为这根本就算不得什么,只不过表明你是新任的中队长而已。”

    好长一段时间来,卡思卡特上校对梅杰少校一直抱有很深的积怨。梅杰少校是他花名册上一个多余的少校,这意味着人员编制相当混乱,无疑成了第二十七空军司令部的那些人——卡思卡特上校坚信是他的敌人和竞争对手——攻击自己的把柄。卡思卡特上校一直在祈祷,希望能碰上像杜鲁斯少校的死这样的好运。花名册上多余了一名少校,实在令他很苦恼。可这会儿他又有了个少校的空缺。他任命了梅杰少校为中队长,于是,便坐上吉普车,来也突然去也突然地在马达的吼叫声中开走了。

    这在梅杰少校便是就此结束球赛。他满脸通红,感觉很不自在,两腿像生了根似地一动不动。这时,雨云又在他头顶上方集结起来。他朝球友们转过身去,一个个脸上挂着好奇的思索神色,又用含着沮丧和深不可测的敌意的眼神,木然地注视着他。他深感羞耻,浑身禁不住一阵寒战。球赛继续进行,可是不再有任何的趣味。

    他运球时,没人想上前阻拦;他一喊传球,不管谁掌握着球,必定把球传给他;即便他投篮不中,也没人上前跟他争抢篮板球。球场上只听得见他一个人的声音。第二天还是这样,第三天他便不再来球场打球了。

    差不多就在这个时候,全中队上下不再有人跟他说话,每个人都盯着他看。梅杰少校天天都低垂双眼,两颊热辣辣的,在忐忑不安之中度日。所到之处,他便是众矢之的,受人蔑视、嫉妒、猜疑、怨恨,以及含沙射影地恶意诽谤。有些人先前不曾怎么注意他酷像亨利-方达,这下可好,竟没完没了地议论起这事来了。甚至有人心怀叵测地暗示,梅杰少校所以被提升为中队长,就是因为他长得像亨利-方达。就说布莱克上尉吧,他本人便一向觊觎中队长这个职位,因此,他坚信,梅杰少校的确是亨利-方达;可他实在是没有种,不敢启口承认。

    接任中队长后,梅杰少校在昏乱中接二连三地遇上了令人难堪的倒霉事。陶塞军士事前没征得他的同意,便擅自差人把他的东西搬进了杜鲁斯少校生前独自占用的那间宽敞的拖车式活动房里。当梅杰少校一路急跑,上气不接下气地冲进中队办公室,报告自己的东西遭窃一事时,里边的那个年轻下士一见他进来,忙跳起身,大喊道:“立正!”险些没把他吓死。梅杰少校同办公室里所有的人一起啪的一声立正,心想不知是哪个要人跟在他身后走了进来。

    好几分钟过去了,房间里鸦雀无声。要不是二十分钟后丹比少校从大队部顺道过来向梅杰少校贺喜,让他们放松下来,或许他们全都得在那儿毕恭毕敬地直站到世界未日。

    在食堂,梅杰少校遭遇的事更令人心酸。米洛满面笑容地在食堂恭候梅杰少校的光临,巴望着洋洋自得地领他到前面一张由他亲自摆好的小餐桌旁。桌上铺一方绣花台布,搁一只粉红色雕花玻璃花瓶,里边插了一束鲜花。梅杰少校畏缩不前,可众目睽睽之下,他又不敢拒绝入座。甚至连哈弗迈耶也抬起头,离开正在用餐的盘子,昂起松垂的大下巴,吃惊地盯着他。米洛又拖又拉,梅杰少校只得乖乖就范,深感耻辱地蜷缩在自己私用的餐桌旁,好不容易才把这顿饭吃完。饭到嘴里,像是灰末,无滋无味,可他还是一口一口地咽了下去,他生怕得罪了那些为他准备这顿饭的人。后来,跟米洛单独在一起的时候,梅杰少校第一次觉得该说说自己的意见了。他告诉米洛说,他还是喜欢像往常一样,跟其他军官一起就餐。米洛对他说,这无论如何不行。

    “我看不出有什么不行的,”梅杰少校争辩道,“以前可从未出过这种事。”

    “以前您可从未做过中队长。”

    “以前杜鲁斯少校是中队长,可他一直是跟其他军官同桌就餐的。”

    “这跟杜鲁斯少校可不同,长官。”

    “跟杜鲁斯少校有什么不同?”

    “我希望您别问我这个问题,长官,”米洛说。

    “是不是因为我像亨利-方达?”梅杰少校鼓足了勇气问道。

    “有人说,您就是亨利-方达,”米洛回答说。

    “哎,我不是亨利-方达,”梅杰少校大声嚷道,气得连说话的声音都发抖了。“我跟他没一点相像。即便我的确长得很像亨利-方达,这又有什么关系呢?”

    “什么关系也没有。我想跟您说的也就是这个,长官。只是您跟杜鲁斯的情况不一样。”

    确实就是不一样。下一顿用餐时,梅杰少校取了饭菜离开食品柜台,走过去准备跟其他人一起坐在普通餐桌旁就餐。不料,他们一个个猛抬起头,满脸敌意,仿佛有一道不可越过的屏障,梅杰少校当即给吓呆了,僵尸般地站在那里,手里的托盘抖个不停。直到米洛悄悄地走过去,引他乖乖地到他独用的餐桌旁,这才替他解了围。此后,梅杰少校便断了和其他军官同桌用餐的念头,一直是一个人背对着大伙坐在自己的餐桌旁,独自用膳。他很清楚,他们恨他,就因为他是中队长了,似乎高人一等,不便跟他们同桌就餐。只要有梅杰少校在,食堂里就从来没有人说话聊天。他意识到,其他军官都想方设法避开跟他在同一个时间吃饭。后来,梅杰少校再也不上食堂了,就在自己的活动房里用餐,大伙这才感觉到了彻底的解脱。

    一天,中队第一次来了个刑事调查部的工作人员,讯问梅杰少校有关医院里有人在公文上假冒签华盛顿-欧文的姓名一事。这下,那个假冒签名的家伙反倒提醒了梅杰少校。于是,他第二天就开始在公文上假冒签上了华盛顿-欧文的姓名。对自己刚接替的新职位,他实在是厌倦透顶,极为不满。他被任命为中队长,但作为中队长,该做些什么,他一无所知。他只晓得自己该做的事情,就是躲在中队办公室帐篷后面自己的那间小办公室里,在公文上假冒签上华盛顿-欧文的姓名,谛听窗外德-科弗利少校掷马蹄铁落地时发出孤寂的丁当声和嘭嘭声。他老是心神不宁,总觉得有什么极其重要的任务还没完成,于是便整天无所事事,空等着任务哪一天突然从天而降。非万不得已,他极少出门,因为他受不了众人瞪眼看他。间或,这种乏味的生活也会被打断。陶塞军士因为解决不了某桩事情,就让某个军官或士兵来找梅杰少校,请示该作何处理,可梅杰少校也无能为力,便又马上让来人回去见陶塞军士,由他妥善处理。他身为中队长,该由他做的事情全都给办妥了,但显然他没有派上丝毫用场。他变得郁郁寡欢,沮丧消沉。有时,他经过一番认真考虑,准备去拜见随军牧师,倾吐自己满腹的苦水,但随军牧师自己似乎也是苦难重重,所以,梅杰少校又不愿给他再添什么烦恼。再说,他也实在没什么把握,随军牧师是不是也替中队长服务。

    对德-科弗利少校,他也向来没什么把握。德-科弗利少校不是出去租借公寓,或诱拐外国劳工,就是掷马蹄铁,除此之外,便再没什么更要紧的事情可做了。梅杰少校经常细心观察马蹄铁如何轻声坠地,或边滚边碰撞地上的小钢桩。他又时常一连好几个小时朝外偷看德-科弗利少校,心中不由惊奇,这么威风的一个人竟没有什么更重要的事情可做。他常常极想跟德-科弗利少校一块掷马蹄铁、可一天到晚掷马蹄铁,差不多跟在公文上签署“梅杰-梅杰-梅杰”一样,乏味无聊。而且,德-科利弗少校面容严峻,实在令梅杰少献望而生畏,不敢接近。

    梅杰少校颇是怀疑自己跟德-科弗利少校的关系,或是德-科弗利少校跟自己的关系。他知道,德-科弗利少校是他的主任参谋,可他不清楚这主任参谋究竟是怎么回事。有德-科弗利少校在身边,他是有幸得到了一位宽厚的上司,还是不幸碰上了一个失职的部下,对此,他实在无法断定。他不想问陶塞军士,因为心里惧怕他,此外,也就没有别的什么人可以问了,德-科弗利少校更是不用说了。不管出什么事,几乎没人敢去请教德-科弗利少校。唯独一个军官很蠢,竟敢掷了德-科弗利少校的一块马蹄铁,不料,第二天便染上了最奇怪的皮亚诺萨怪病,就连格斯和韦斯,甚至丹尼卡医生,都不曾见过或听说过。所有的人都断定,是德-科弗利少校为了报复,才让那可怜的军官染上这种怪病的,可是究竟怎么让他染上的,谁也说不准。

    送至梅杰少校案头的公文,多数与他无关。其中的绝大部分公文内容涉及他接任前的一些文牍,是他从未见过听过的。这些文牍根本就无需查阅,因为每一份的批示总是老一套,否定前一份的内容。因此,梅杰少校每一分钟的效率都极高,可以签署二十份公文——每一份都建议他丝毫不必理会其他公文。每天都要接到由设在大陆的佩克姆将军办公室发送来的冗长简报,标题通常是一些乐观的道德说教,诸如“因循拖延即是偷盗时间的窃贼”,“爱清洁仅次于爱上帝”。

    读了佩克姆将军那些关于清洁和因循拖延的公文,梅杰少校深觉自己就像是一个既啊邋遢又拖拉的家伙。因此,他总是尽快地送走那些公文。唯一能提起他兴趣的,就是偶尔送来的有关一个少尉的那些公文。这家伙实在是倒霉透顶,来皮亚诺萨岛还不足两个小时,就在奥尔维耶托上空送了命,才打开了一半的行李包至今还留在约塞连的帐篷里。由于那个倒霉的少尉没去中队办公室报到,而是去作战室报到,所以,陶塞军士决定,万无一失的办法就是向上级报告说,他根本没到中队报到。偶尔发送来的涉及这个少尉的那些公文,都谈到了一个事实,即,他似乎已消失得无影无踪。这,就某种意义而言,也正是他的结局。至于梅杰少校,他对送至自己案头的那些公文颇为感激,因为终日坐办公室签署公文,较之一天到晚闲坐办公室,实在要强得多。有了那些公文,他也就有了事情可做。

    梅杰少校签署的每一份公文,照例过了二至十天的时间,必定退还给他,不过附上了一页空白纸,要求他再签个字。退还的公文总比原来厚了许多,因为他上次签字的纸和供他再签字的附加纸中间,添进了不少张纸,全都是散驻各处的所有其他军官新近才签的字。那些军官也是一天到晚忙着在同一份公文上签字。看着简单的公文愈积愈厚,最终积成大本大本的手稿,梅杰少校好不失望。

    他在同一份公文上签字,不管签了多少回,总要返回,还让他签一次。他渐渐明白,要想摆脱其中任何一份公文,都是白费心机。一天——就是刑事调查部那名工作人员初次来访后的第二天——梅杰少校在一份公文上签上了华盛顿-欧文的姓名,没签自己的名字,他只是想看看会有什么效果。他挺喜欢这个签名,实在是非常喜欢,于是,这之后,他整个下午都在所有公文上签华盛顿-欧文的名字。这纯粹是他一时无聊所为,自然也是一种反抗行为,他知道事后必定会因此而受到严惩。翌日上午,他胆战心惊地走进办公室,却巴望着看看会发生什么事。结果,啥事儿也没有。

    他犯了罪,但反倒是桩好事,原因是,凡经他签上华盛顿-欧文姓名的公文,再没有一份退还!最终取得了进展,于是,梅杰少校便以全身心的热情,投入新的事业,往公文上签署华盛顿-欧文的姓名,这或许算不得是什么了不起的活动,但总要比签“梅杰-梅杰-梅杰”有些趣味。一旦华盛顿-欧文实在乏味了,他就倒个个儿,写成欧文-华盛顿,直签到再无趣味为止。他终究是了结了一桩事情,因为凡是签上华盛顿-欧文或欧文-华盛顿的公文,再没有一份返回中队。

    最终真正返回中队的,倒是假扮成了飞行员的另一名刑事调查部工作人员。中队上下全都知道他是刑事调查部的,因为他向他们吐露了自己的真实身份,并恳求每个人别告诉其他任何人,可其实呢,他早就跟其他人说了,自己是刑事调查部派来的。

    “中队里知道我是刑事调查部派来的只有你一个人,”他向梅莎少校吐露说,“你要绝对保守秘密,以免影响我的工作效率。你明白吗?”

    “陶塞军士也知道你是谁。”

    “是的,我知道。我想进来见你,只得告诉他。不过,我知道他是无论如何不会跟谁说的。”

    “他跟我说了,”梅杰少校说,“他告诉我说,外面有个刑事调查部的人想见我。”

    “这杂种。我得对他进行安全审查。如果我是你,我不会把任何绝密文件摊在这儿。至少在我汇报之前得把它们收起来。”

    “我这儿没什么绝密文件,”梅杰少校说。

    “我说的就是这类文件。把它们锁进你的公文柜,这样,陶塞军士也就没法拿到了。”

    “公文柜唯一的一把钥匙就在陶塞军士手里。”

    “恐怕我们这是在浪费时间,”刑事调查部的来人说,语气颇为生硬。这家伙身量矮胖,极有朝气,却好激动,动作敏捷果断。他从一只特大的红色信封里抽出许多份直接影印件。“你见过这些吗?”——那只信封一直醒目地藏在一件皮制的飞行短上衣里边,衣服上画得花里胡哨——飞机穿越滚滚的橘黄色高射炮火,以及标志完成五十五次作战飞行任务的一排排整齐的小炸弹。

    梅杰少校木然地看着一份份寄自医院的私人函件的直接影印件,上面均有审查官签署的“华盛顿-欧文”或“欧文-华盛顿”。

    “没见过。”

    “这些呢?”

    梅杰少校继而又盯着一份份寄给他的公文,上面是他签署的相同的姓名。

    “没见过。”

    “签这些姓名的人是不是在你的中队?”

    “哪一个?这上边有两个姓名。”

    “随便哪一个。据我们估计,华盛顿-欧文和欧文-华盛顿是同一个人,他用两个姓名,只不过是想迷惑我们。你知道,经常有人耍这种把戏。”

    “我想我中队里没这两个姓名的人。”

    刑事调查部的那名工作人员面露失望。“他可比我们想的要聪明得多,”他说,“他在用第三个姓名,又要冒充别的什么人了。我想……没错,我想我知道这第三个姓名是什么。”他灵机一动,极兴奋地又抽出一份直接影印件,让梅杰少校看个仔细。“这个见过没有?”

    梅杰少校略微前倾了一下身体,见到的是那份V式航空信函的直接影印件,上面除玛丽这个名字外,所有内容都让约塞连给涂掉了,不过,约塞连还写上了:“我苦苦地思念着你。美国随军牧师A-T-塔普曼。”梅杰少校摇了摇头。

    “我以前可从未见过。”

    “你知道谁是A-T-塔普曼吗?”

    “是飞行大队的随军牧师。”

    “这事总算真相大白了,”刑事调查部的来人说,“华盛顿-欧文就是飞行大队的随军牧师。”

    梅杰少校一阵惊恐。“A-T-塔普曼是飞行大队的随军牧师。”

    他纣正道。

    “你能肯定吗?”

    “当然。”

    “飞行大队的随军牧师怎么会在一封信上写这样的话呢?”

    “也许是别人写的,冒用他的姓名。”

    “别人怎么会想冒用随军牧师的姓名呢?”

    “想不被人发现。”

    “你说的或许有些道理,”刑事调查部的人迟疑片刻后断言道,接着很清脆地咂了咂嘴。“也许我们面对的是一帮人,有两人的姓名恰好可以相互调换,就串通一气。没错,我敢肯定是这样。其中一个就在你的中队里,一个在医院里,再有一个就是跟随军牧师在一块儿。这么说来,一共有三个人,是不是?你是不是绝对肯定以前从未见过这些公文?”

    “要是见过,我就会在上面签名了。”

    “签谁的名?”刑事调查部的人问得很狡猾。“你的还是华盛顿-欧文的?”

    “签我自己的名字,”梅杰少校对他说,“我连华盛顿-欧文的姓名还不知道呢。”

    刑事调查部的人绽开了笑脸。

    “少校,我很高兴你跟这事无关。也就是说,我们俩能够合作。

    只要是能合作的,不管是谁我都需要。欧洲战区某个地方,正有人在设法把发送给你的公文弄到手。你是否清楚究竟是谁?”

    “不清楚。”

    “嗯,我倒有个挺不错的主意,”刑事调查部的人说,接着又俯身向前,很隐秘地低语道,“很可能是陶塞那个杂种。不然的话,他又何必到处泄露我的身份呢?好,从今后你多留点神,一听到有人谈起华盛顿-欧文,就告诉我。我要对随军牧师和这里所有其余的人进行安全审查。”

    那家伙刚走,刑事调查部派遣来的第一个工作人员便从窗外跳进梅杰少校的办公室,想知道刚才那人是谁。梅杰少校几乎没认出他来。

    “是刑事调查部的工作人员,”梅杰少校告诉他说。

    “他绝对不是,”那人说,“这一带只有我才是刑事调查部的人。”

    那人穿一件褪了色的褐紫红色灯芯绒睡袍——夹肢窝的线缝都已绽开来了,一条棉法兰绒睡裤,一双破旧的室内便鞋——其中一只鞋底裂了开来,走起路来啪喀啪塔直响。梅杰少校差点没认出他来,接着便想了起来,这是住院病人规定穿的衣服。这人体重增加了二十磅左右,看上去身体极健壮。

    “我的确病得很厉害,”他哀叹道,“我在医院里从一个战斗机飞行员那里染上了感冒,最后却得了相当严重的肺炎。”

    “我很难过,”梅杰少校说。

    “不过,这场病对我很有好处,”那个刑事调查部的人抽了下鼻子说,“我用不着你同情。我只是想让你知道我在调查什么。我来这里提醒你,华盛顿-欧文似乎把他的作战基地从医院转到了你的中队。难道你没听见周围有什么人谈起过华盛顿-欧文吗?”

    “说实话,我听见过,”梅杰少校回答说,“刚才在这里的那个人,他正谈着华盛顿-欧文呢。”

    “是吗?”刑事调查部的人高兴地叫道,“也许这是我们破案的关键所在!我这就赶回医院,给上司写份报告,请求进一步的指示,你每天二十四小时监视他。”说罢,他便越窗跳出了梅杰少校的办公室,消失得无影无踪。

    片刻后,梅杰少校办公室和中队办公室之间的帐篷门帘给挑了开来,刑事调查部的第二个工作人员又回来了,一边不停地喘着气。他上气不接下气地叫道:“我刚才看见一个穿红睡衣的家伙从你的窗子跳了出去,沿大路跑了!你没看见吗?”

    “他在这里跟我谈话哩,”梅杰少校答道。

    “我刚才想,有人穿红睡衣跳窗逃跑,这事看来一定很可疑。”

    那人绕着窄小的办公室一圈圈地踱着有力的方步。“起先我以为是你,急急忙忙逃往墨西哥呢。不过现在我明白了,不是你。他没提起华盛顿-欧文,是不是?”

    “说实话,”梅杰少校说,“他提过。”

    “真的?”那人叫了起来。“太好了!或许这是我们破案的关键所在。你知道在哪儿能找到他吗?”

    “在医院里。他病得相当厉害。”

    “好极了!”那人惊叫道,“我马上去医院找他。最好是隐匿了身份去。我这就去医务室说明情况,让他们把我当做病人送医院。”

    “除非我的确有病,他们是不肯把我当做病人送医院的,”从医务室回来后,他跟梅杰少校说,“其实,我病得不轻。我一直想去医院做一次体格检查,这一次倒是个极好的机会。我再跑一趟医院,跟他们说我病了,这么一来,他们就会送我去医院的。”

    “瞧瞧,他们对我干的好事,”从医务室回来后,他就跟梅杰少校汇报说,满嘴齿龈都变成了紫色,神情极度痛苦。他双手提着鞋子和袜子,脚趾也给涂上了龙胆紫溶液。“有谁听说过刑事调查部的人牙龈是紫色的?”他哀叹道。

    他低着头离开了中队办公室,跌进一条狭长掩壕,摔破了鼻子。他的体温依旧正常,不过,格斯和韦斯把他当做例外,用救护车送他进了医院。

    梅杰少校撒了谎,但一切正常。对此,他实在是没有丝毫惊讶的感觉,因为他早就发现,真正说谎的人,总体上说,较不说谎的人来得机敏,有抱负,也更容易达到目的。要是跟刑事调查部的第二个工作人员说了实活,他就会给自己惹一身麻烦的。相反,他说了个谎,反倒可以无忧无虑地继续做自己的事情了。

    自刑事调查部派第二个工作人员来中队暗查以后,梅杰少校工作时变得越发慎重。所有签字他一律改用左手,并且得戴上墨镜和假胡子——他曾用了这两样东西做掩护,想再上球场打篮球,但结果失败了。为了做进一步的防备,他巧妙地把华盛顿-欧文改成了约翰-弥尔顿。约翰-弥尔顿灵活性强,且又简洁。跟华盛顿-欧文一样,一旦写腻了,也可以倒过来写,而且效果同样不错。此外,还能使梅杰少校签字的效率提高一倍,因为比起自己的姓名或是华盛顿-欧文的姓名,约翰-弥尔顿要简短得多,写起来也就省了不少时间。另外还有一个方面,约翰-弥尔顿也极有成效。约翰-弥尔顿具有极广泛的用途,于是,梅杰少校没多久就把签名写进了假想的对话片断。这样,公文上便有可能见到一些典型的批注:“约翰-弥尔顿是个性虐待狂”,或是“你见过弥尔顿吗,约翰?”其中有一条他是最为感到自豪的:“约翰中有人吗,弥尔顿?”约翰-弥尔顿展现了一个个崭新的前景,处处是使之不尽的妙计,为永远消灭令人厌倦的单调提供了保障。一旦写烦了约翰-弥尔顿,梅杰少校便又改写华盛顿-欧文。

    那副墨镜和假胡子,梅杰少校是在罗马买的。那时,他正日渐陷入困境,无以摆脱,为了解救自己,他便买了这两样东西,算是作最后一番徒然的努力。首先是伟大的效忠宣誓运动让他蒙受了奇耻大辱。当时,有三四十人四处跑动,相互竞争着找人签字效忠,但居然没一个人肯让他签名。接着,那件事刚过,又出了克莱文杰的飞机及全体机组人员在空中神秘失踪一事。别人又阴毒地把造成这场离奇灾难的责任一古脑儿推给了梅杰少校,原因是,他从来没有签过字,进行效忠宣誓。

    那副墨镜镶的是品红色宽边镜架。那副假胡子则是身着鲜艳服装的街头手摇风琴艺人用的那种。一天,梅杰少校觉着自己再也耐不得孤独了,于是,便戴上墨镜和假胡子,前去球场打篮球。他装出一副轻松随便的模样,漫步走向球场,暗地里则在默默祈祷,可千万别让人给认出来。其余的人全都装作没认出他,于是,他便来了兴头。他很为自己这无害的计策感到庆幸,正当他暗自得意时,对方一名队员突然猛撞了他一下,把他撞倒在地。不一会儿,又有人狠狠撞了他一下,他顿时反应了过来,他们全都认出了他,正利用他的伪装,不是用肘挤他,就是用脚绊他,或是使足了劲把他推来搡去。他们压根就不希望他在这里。他刚意识到这一点,自己的队员便本能地跟对方的队员联合了起来,仿佛一群凶暴的乱民,围住他狂叫乱吼,恶语咒骂,又拳脚相加。他们把他打倒在地,趁他还没来得及爬起身,便对着他猛踢。当他盲目地挣扎着站起身之后,他们对他又是拳打脚踢。他双手捂住眼睛,什么也看不见。他们一个个你拥我挤,发了狂一般,身不由己地涌上去,狠狠地对着他拳打脚踢,用手指扣挖他的眼睛,又用乱脚踩他。他给打得天旋地转,直至壕沟边,一头栽了下去。在沟底,他站住了脚,沿另一侧爬了上去,摇摇晃晃地走开了,身后那伙人冲着他大声吼叫,乱掷石块,直到他踉跄地拐过中队办公室帐篷一角,方才躲了过去。遭围攻时,梅杰少校自始至终最关心的是,千万别让墨镜和假胡子掉落下来,如此,他或许能伪装下去,也就没必要再以中队长的身份出现跟他们冲撞了——这可是最让他害怕的事。

    回到办公室,他哭了;哭完,他便洗净嘴上和鼻子上的血迹,擦去脸颊和前额上抓伤处的泥垢,于是,把陶塞军士召了进去。

    “从现在起,”他说,“只要我在这儿,任何人不得进来见我。听明白了没有?”

    “明白了,长官,”陶塞军士说,“包括我吗?”

    “是的。”

    “我知道了。就这些吗?”

    “就这些。”

    “要是您在的时候,有人真的要来见您,我该怎么跟他们说?”

    “告诉他们我就在里边,让他们等着。”

    “是的,长官。等多长时间?”

    “等到我离开。”

    “那么,之后我该怎么应付他们?”

    “这我就管不着了。”

    “您离开后,我可以让他们进去见您吗?”

    “可以。”

    “可您早就不在这儿了,是不是?”

    “是的。”

    “明白了,长官。就这些吗?”

    “就这些。”

    “是,长官。”

    “从现在起,”梅杰少校对那个替他收拾屋子的中年士兵说,“我在这儿的时候,你别进来问我是否有什么吩咐。听明白了吗?”

    “听明白了,长官,”勤务兵说,“我该什么时候进来问您是否有什么吩咐?”

    “我不在的时候。”

    “是,长官。那我该做什么?”

    “我吩咐你做什么,你就做什么。”

    “可是您不在的话,就没法吩咐我了。您会在这里吗?”

    “不会”“那我该怎么办?”

    “该办的事,就办。”

    “是,长官。”

    “就这些,”梅杰少校说。

    “是,长官,”勤务兵说,“就这些吗?”

    “不,还有,”梅杰少校说,“你也别进来打扫。只要你不知道我是否在这里,千万别进来。”

    “是,长官。可是我没法一直知道你究竟是否在里边。”

    “假如你不知道,你就只当我在这里,你自己就走开,等弄明白了再说。知道了吗?”

    “知道了,长官。”

    “很抱歉,不得不跟你这么说话,可我实在是迫不得已。再见。”

    “再见,长官。”

    “谢谢你。谢谢你替我做的一切。”

    “是,长官。”

    “从现在起,”梅杰少校对米洛-明德宾德说,“我不再上食堂吃饭。我要人把每顿饭都送到我的活动房去。”

    “我想这主意倒是挺不错,长官,”米洛答道,“这样,我就可以另外给您做些菜,其他人绝对不知道。我保证您一定喜欢吃。卡思卡特上校一直就很喜欢吃。”

    “我不需要什么特别的菜。其他军官吃什么,我就吃什么。只要让送饭的人在我的门上敲一下,把托盘搁在台阶上,就可以了。听明白了没有?”

    “听明白了,长官,”米洛说,“十分明白。我让人藏了些缅因活龙虾,今天晚上我就烧给您吃,另外再给您来一盘鲜美可口的罗克福尔干酪色拉和两块冰冻巧克力奶油小蛋糕。这种蛋糕是昨天跟法国地下组织的一名重要成员一块从巴黎偷运出来的。开始先这么吃,行吗?”

    “不行”“是,长官。我明白了。”

    当晚用餐时,米洛给梅杰少校送去了烤缅因龙虾,鲜美可口的罗克福尔干酪色拉和两块冰冻巧克力奶油小蛋糕。梅杰少校颇为恼火。不过,要是让人送回去,只会白白浪费,或者由别的什么人吃掉。梅杰少校可是酷爱吃烤龙虾的。他便很内疚地把这顿饭吃了下去。第二天中午,送来的是马里兰水龟和整一夸脱一九三七年酿制的佩里尼翁酒。梅杰少校连想都没想,便三口两口地吃了个精光。

    米洛之后,便只剩下中队办公室里的那帮人了。梅杰少校一直避着他们,为此,他每回进出都是从自己办公室那扇尘封的窗户经过。窗户从不上销,开得极低,很大,因此,跳进跳出相当的便利。每次离开中队办公室回自己的活动房屋,他总是等四周围没有人的时候,一个箭步冲过帐篷的拐角,紧接着纵身跃进铁路壕沟,低着头一直往前直奔进那片森林。及至与活动房屋成一直线,他便爬出壕沟,飞速地从茂密的矮树丛里穿来穿去,直奔回家。穿越矮树丛时,他只碰到过一个人,就是弗卢姆上尉。某日黄昏,脸色憔悴苍白的弗卢姆上尉,冷不丁地从一块露莓灌木地里冒了出来,把梅杰少校吓了个半死。他向梅杰少校诉说,一级准尉怀特-哈尔福特曾扬言要切断他的喉管。

    “假如以后你再这么吓我,”梅杰少校对他说,“我会切断你的喉管。”

    弗卢姆上尉倒抽了一口冷气,立刻躲进了那块露莓灌木地。从此,梅杰少校便再也没有见到过他。

    当回头看看自己所做的一切,梅杰少校不由得深感欣慰。就在这几英亩的外国土地上,满满挤了两百多人,可他竟然成功地做上了隐士。他用了一点计谋和想象,就让中队全体官兵几乎再也没法跟他说话了。不过,他察觉到,这也正合了他们的意,因为没人想跟他搭讪。事实也的确如此,只有那个疯子约塞连除外。一天,梅杰少校正沿沟底急匆匆奔回活动房屋用午餐,约塞连突然一个鱼跃,把他撞倒在地。

    全中队上下,只有约塞连一人鱼跃把他撞倒时,是最让梅杰少校感到厌恶的。约塞连从来都是臭名在外,总是逢人便唠叨个没完——实在是把个脸丢尽了——抱怨自己帐篷里的那个死人——

    其实压根就没在他的帐篷里;阿维尼翁飞行任务完成后归来,他竟脱光了衣服,四处溜达,德里德尔将军上前给他别一枚勋章——以嘉奖他在弗拉拉上空执行任务时的英勇善战——的那天,他还是赤条条地站在队伍里。

    那个死人的遗物杂乱地堆放在约塞连的帐篷里,天底下谁都没这份权力把它们清理出去。由于梅杰少校准许陶塞军士汇报上级说,到中队后还不足两个小时就战死奥尔维那托上空的那名少尉根本就没来中队报到,因此,他也就不再有这种权力。真正有权力把少尉的遗物清理出约塞连帐篷的,在梅杰少校看来,只有一个人,就是约塞连自己,不过,梅杰少校似乎又觉得,约塞连实在是没这个权力。

    梅杰少校让约塞连一个鱼跃给撞倒之后,不停地呻吟,扭动着身子想站立起来。约塞连却不让。

    “约塞连上尉请求立刻和少校面谈,”约塞连说,“有一桩生死攸关的大事。”

    “请让我站起来,”梅杰少校浑身难受,便没好气地命令道,“我的手臂撑在地上,没法回礼。”

    约塞连放开了他。两个人慢慢地站直了身子。约塞连再行了个军礼,复述了自己的请求。

    “到我办公室吧,”梅杰少校说,“我想这里可不是谈话的地方。”

    “是,长官,”约塞连答道。

    他们拍打掉身上的砂土,于是,默不作声极不自在地朝中队办公室的门口走去。

    “等我一两分钟,先让我在这些伤口上涂些红药水。然后再让陶塞军士送你进来。”

    “是,长官。”

    那些办事员和打字员正在办公桌和文件柜旁忙着,梅杰少校连瞧都没瞧他们一眼,便庄严地大步向办公室的后面走去。他随手放下了自己办公室的门帘。一进自己的办公室,趁没人在,他便快步穿过房间,走到窗口,跳了出去,拔腿就跑,却发现约塞连挡了他的去路。约塞连立正守候着,又行了个军礼。

    “约塞连上尉请求立刻和少校面谈,因为有一桩生死攸关的大事。“他很坚定地复述了一遍。

    “拒绝你的请求,”梅杰少校厉声说。

    “那可不行。”

    梅杰少校作了让步。“好吧,”他极不耐烦他说,“我就跟你谈谈。请跳进我的办公室去。”

    “您先请。”

    他们跳进了办公室。梅杰少校坐了下来,约塞连在办公桌前不停地走动,告诉少校说,他不想再执行作战飞行任务了。他又能怎么办?梅杰少校暗暗问自己。他只能按科恩中校的指示办事,只能希望一切顺利。

    “为什么?”梅杰少校问道。

    “我害怕。”

    “这不是什么羞耻。”梅杰少校很亲切地安慰他。“我们大家都害怕。”

    “我不是觉得羞耻,”约塞连说,“我只是害怕。”

    “要是你从来不害怕,那才不正常呢。即便是最有胆量的人也会有害怕的时候。作战中,我们所有人都面临不少最为重要的任务,其中之一就是战胜恐惧。”

    “哦,得了吧,少校。我们就不能不说这些屁话吗?”

    梅杰少校极是窘迫地垂下了目光,不住地拨弄手指。“那你要我跟你说些什么呢?”

    “就说我完成的飞行任务次数已经足够了,可以回国了。”

    “你飞过多少次?”

    “五十一次。”

    “那你只要再飞四次就行了。”

    “他又会增加飞行次数的。每次我快要飞满的时候,他就又增加了。”

    “这一次他或许不会这么做。”

    “不管怎么说,他从来就不让一个人回国。他只是把大伙儿留在这里,等候命令轮换调防,待到人手不足时,他便又增加每个人的飞行次数,迫使大家重返战场。自从他来这里以后,他一直是这么做的。”

    “你不该责怪卡思卡特上校,轮换调防回国的命令一再延缓,根本就不是他的过错,”梅杰少校告诉他说,“这完全是第二十六空军司令部的责任,一接到我们的轮换调防命令,他们就应该马上处理。”

    “尽管如此,他还是可以请求补充兵员,一旦命令下达,就能让我们回国。不管怎样,反正有人告诉我说,第二十七空军司令部只规定每人完成四十次飞行任务,只有他一个人要我们飞五十五次。”

    “这事我倒是不太清楚,”梅杰少校回答说,“卡思卡特上校是我们的指挥官,我们必须服从他。你何不飞完最后四次,看看会有什么结果。”

    “我不想这么做。”

    你又能怎么办?梅杰少校又暗暗问自己。这么一个人正直视你的眼睛,说他宁死也不愿在战场上送命;在行事方面,他至少跟你一样明理,机敏——可你却不得不装着他根本就不如你,对于他,你能奈何呢?又能跟他说些什么呢?

    “假如我们让你自己挑选任务,执行例外的飞行,”梅杰少校说,“那样的话,你就可以完成最后的四次飞行任务,而且又不冒一点风险。”

    “我不想执行例外的飞行任务。我不想再卷进这场战争。”

    “难道你愿意亲眼看见我们的国家战败?”梅杰少校问。

    “我们不会战败的。我们有充足的人力、财力和物力。我们有一千万军人,他们可以替代我。有些人正战死疆场,而更多的人却在捞钱,花天酒地。就让别的人去战场送死吧。”

    “但要是我们所有的人都像你这么想,那还了得?”

    “这么说来,假如我不这么想,就必定是个十足的笨蛋。难道不是吗?”

    你究竟能跟他说些什么呢?梅杰少校满脸愁苦,实在是疑惑不解。有一句话他是万万说不得的:他毫无办法。跟人说他毫无办法,这便有了某种暗示:要是他有法子,他会尽一份力的;同时又让人觉出了言外之意:科恩中校的政策不是有错,就是有欠公允。科恩中校对这件事向来是没有半点含糊。

    “对不起,”他说,“可我实在毫无办法。”
