\chapter{浑身雪白的士兵}
 
    约塞连直接跑进了医院,决心永远呆在那儿。他已完成了三十二次飞行任务,他决定不再多飞一次。当他改变了主意从医院出来后的第十天,上校又把飞行任务提高到四十五次,于是约塞连又跑回医院,决定永远呆在医院里,除了他刚刚又多飞的六次之外,不再多飞一次。

    由于他的肝脏和眼睛的缘故,约塞连只要愿意,随时都可以住进医院;那些医生由于不能确诊他的肝病,因此每次约塞连跟他们说他的肝有毛病时,他们都不敢正视他的目光。只要他的病房里没有人真的病得很厉害,他在医院里就能自得其乐。他的身体还真够结实,别人得疟疾或流感,他几乎连一点不舒服的感觉都没有。他能忍受别人进行扁桃体切除术,并且他们手术后他也不会有任何苦恼。他甚至能忍受他们的疝气和痔疮,只是稍有点作呕和厌恶。

    不过,他也只能到这个地步而不生病。超过这个地步,他随时要逃走。他可以在医院里休息,因为在那儿没有人指望他做什么。人们期望他在医院里不是死掉就是好起来。既然他一开始就没病,好起来是很容易的。

    呆在医院里要比在博洛尼亚上空或飞越阿维尼翁上空时的情景好多了,当时赫普尔和多布斯在操纵飞机,斯诺登奄奄一息地躺在后面。

    通常,医院里面的病人没有约塞连在医院外面见到的多,而且医院里一般很少有人是病得很严重的。医院里的死亡率远比医院外的低,是一种健康得多的死亡率。很少有人死得没有必要。人们对死在医院里这种事知道得要多得多,因而死得更加干净,更加井然有序。他们虽然在医院里还无法支配死神,但却肯定可以让她乖乖听话。他们教她举止得体。他们虽不能把死神挡在医院之外,但当她进来时,她得像位贵妇人一样温文尔雅。在医院里,人们死得文雅而得体。这儿没有医院外边十分常见的那种耸人听闻、野蛮丑陋的死法。他们不会像克拉夫特那样在半空中被炸得身首异处,不会像约塞连帐篷里的那个死人,也不会像斯诺登那样在飞机的后舱里向约塞连吐露了他的秘密之后,在骄阳似火的夏季被活活冻死。

    “我冷。”斯诺登当时低声呻吟着。“我冷。”

    “好了,好了。”约塞连极力安慰他。“好了,好了。”

    他们没有像克莱文杰那样神奇地逃入一片云层。他们没有被炸成血乎乎的肉块。他们没有被淹死,没有遭到雷击,没有被机器轧得血肉模糊或在山崩中被砸得粉身碎骨。他们没有在拦路抢劫中被击毙,没有在强xx中被扼死,没有在酒吧里被捅死,没有被父母和孩子用斧头劈死,或遭上帝的某个天条的惩罚而一命呜呼。没有人窒息而死。人们因流血过多在手术室里像绅士一般死去,或者在氧气帐里断了气而未吭一声。完全没有医院外边流行的那种“这会儿你见到我过会儿就见不到我”的变戏法似的事情,也没有“这会儿我还在过会儿就完蛋”那种事情。这里没有饥荒或洪水。孩子们不会闷死在摇篮里或冰箱里,也不会跌倒在卡车轮下。没有人被活活打死。没有人把他们的脑袋伸进开着煤气的烤箱里,或跳到疾驶的地铁列车前方,或像大铅锤似的带着呼呼声从旅馆窗户里骤然跌落,以每秒三十二英尺的加速度垂直向下,最后令人胆寒地扑通一声,像只装满草莓冰淇淋的羊驼呢口袋摔在人行道上,鲜血淋淋,粉红色的脚趾还在抽动,令人恶心地死于众目睽睽之下。

    权衡再三,约塞连常常还是宁愿呆在医院里,尽管医院有医院的毛病。那里的护士往往好管闲事,那里的规定,如果执行的话,很有约束性,那里的管理也常常干预病人的事情。由于病人随时有可能住进来,他也不能总指望有一群活泼的年轻人跟他住在同一间病房里,而且,文娱活动也常常没什么意思。他不得不承认,随着战争的继续,人们越来越靠近战场,医院的情况已在逐步变坏。在战区内住院的病员情况恶化得十分明显,这立即说明了战争变得越来越激烈。他越深入到战斗中心去,那儿病员的情况也就越糟,直到最后医院里来了那位浑身雪白的士兵,除了死之外,他不可能病得再厉害了,而他很快就死了。

 


    那个浑身雪白的士兵全身上下缠着纱布,绑着石膏,外加一只体温表。那体温表只不过是件装饰品,每天清晨和傍晚由克拉默护士和达克特护士平稳地放在他嘴巴上缠着的绷带中一个小黑洞里,直到那天下午克拉默护士来看体温表时才发现他已经死了。此刻约塞连回想起来,觉得好橡是克拉默护士而不是那个得克萨斯人谋害了那个浑身雪白的士兵。假如她那天没来察看体温表并报告她发现的情况,那个浑身雪白的士兵也许还像往常那样一直活着躺在那儿,从头到脚裹在石膏和纱布里,两条奇形怪状的僵硬的腿从臀部被吊起来,两只奇形怪状的膀子也笔直地吊在那里,四肢都绑着石膏,又粗又大,这些奇形怪状的、无用的四肢用拉紧的电缆线吊在半空中,一些长得出奇的铅块黑乎乎地悬在他上方。那个样子躺在那儿说明他的性命也许不多了,不过那可是他最后的全部生命,因此约塞连觉得似乎不应该由克拉默护士来作出结束他的性命的决定。

    那个浑身雪白的士兵像块展开的、上面有个洞的绷带,或者像港口里一块破碎的石块,上面有一根扭曲了的锌管突出来,除了那个得克萨斯人之外,病房里其他的病人都是软心肠。他是那天晚上被悄悄送进病房里来的,从第二天早晨他门看见他那一刻起,大家就厌恶地避开他。他们神情庄重地聚集在病房的另一角,用恶毒的话语和受到冒犯的口吻低声议论着他;他们反对硬把他这令人恐怖的模样塞到他们面前,怨恨他那极为醒目的模样,活生生地向他们提醒了那令人作呕的现实,他们都害怕同一件事情:他将开始呻吟。

    “如果他真的开始呻吟,我不知道我该怎么办,”那个打扮漂亮的、留着金黄色小胡子的年轻的战斗机飞行员可怜兮兮地哀叹道,“那意味着他晚上也要呻吟啦,因为他辨不出白天黑夜。”

 


    那个浑身雪白的士兵一直躺在那儿,没有一点声音。他嘴巴上方那个边缘参差不齐的圆洞又深又黑,一点没露出嘴唇、牙齿、上腭或舌头的迹象。唯一走到足够近的地方去看他的人就是那个和蔼可亲的得克萨斯人。他每天好几次走到离他比较近的地方,同他闲谈关于多给那些正派的人投票的事。他每次开始谈话都这么一成不变地先打招呼:“你说什么,伙计?感觉怎么样?”其他病人都穿着规定的栗色灯芯绒浴衣和敞开着的法兰绒睡衣,避开他俩呆在一旁,神情优郁地在猜想那个浑身雪白的士兵到底是谁,他为什么会在这儿,那纱布和石膏里面的他到底是个什么样子。

    “我跟你们说,他没问题。”每次结束他的社交访问之后,那个得克萨斯人总是这样鼓舞人心地向他们汇报。“他内部完全是个正常的家伙。只不过是他现在还有点儿怯生,有点儿不踏实,因为他不认识我们这儿的任何人,而且也不能说话。你们干吗不都走到他面前去介绍一下自己?他不会把你们吃掉的。”

    “你他妈的到底在说些什么?”邓巴问道,“他知道你在说些什么吗?”

    “他肯定知道我在说什么。他并不傻。他没什么问题。”

    “他能听得见你说话吗?”

    “嗯,我不清楚他能不能听见我说话,但我肯定他知道我在说什么。”

    “他嘴巴上的那个洞有没有动过?”

    “咳,这是个什么怪问题啊?”那个得克萨斯人不大自在地问道。

    “如果那个洞根本不动,你怎么知道他在呼吸呢?”

    “你怎么知道那是个男的?”

    “他脸上的绷带下有没有纱布块盖在眼睛上?”

    “他有没有动过脚趾头或手指尖?”

    那个得克萨斯人退却了,自己也越来越糊涂了。“好了,这是些什么怪问题啊。你们这些家伙肯定都疯了或傻了。你们为什么不走到他跟前和他认识一下?他真的是个挺好的家伙,我跟你们说。”

    那个浑身雪白的士兵与其说是个活生生的人,还不如说更像个已制成标本、消过毒的木乃伊。达克特护士和克拉默护士使他保持得干干净净。她们常用一只短柄小刷轻刷他的绷带,用肥皂水擦洗他手臂上、腿上、肩膀上、胸脯上和骨盆上的石膏。她们用装在一个圆听里的金属抛光剂,给一根从他的腹股沟处的石膏板上伸出来的暗淡的锌管涂上淡淡的一层光。她们还用湿抹布每天几次擦去两条细细的黑橡胶管上的灰尘。这两条管子从他身上一进一出,连着两只塞住的大口瓶,其中一只吊在他床旁边的一根柱子上,瓶中的药液通过他手臂上的绷带中的一个缝隙不断地滴进他的体内;另一只瓶则放在地板上几乎看不见的地方,通过那根从他腹股沟处伸出来的锌管把液体排掉。这两个年轻的护士一刻不停地擦着那两只玻璃瓶。她俩为自己所做的杂务活而感到自豪。在她们两人中,克拉默护士更为细心。她是位身材修长的姑娘,漂亮但不性感,长着一张健康却不迷人的脸庞。克拉默护士的鼻子娇小可爱,脸上的皮肤光泽耀人,透露出青春的气息,脸上星星点点地生着一些动人、然而却让约塞连讨厌的小雀斑。她被那个浑身雪白的士兵深深打动了。她那双善良的、淡蓝色的、又大又圆的眼睛常在意想不到的时候涌出巨大的泪珠,那眼睛真让约塞连受不了。

    “你怎么知道他在那里面?”他问她。

    “你怎么敢这样跟我说话!”她气冲冲地回答。

    “嗯,你怎么知道,你甚至不知道那是不是真的是他。”

    “谁?”

    “谁在那些绷带里就是谁。你也许真的在哭其他什么人。你怎么知道他还活着。”

    “你怎么能说出这么可怕的话来!”克拉默护士嚷道,“好了,快回到床上去,别再拿他开玩笑啦。”

    “我可不是在开玩笑。任何人都可能在那里面。因为我都知道,那甚至有可能是马德。”

    “你在说什么呀?”克拉默护士声音颤抖地恳求他说。

    “也许那就是死人呆的地方。”

    “什么死人?”

    “我的帐篷里就有个死人,没有人能把他扔出去。他的名字叫马德。”

    克拉默护士的脸一下子变得苍白,眼巴巴地转向邓巴求助。

    “叫他不要再说这样的话吧,”她乞求道。

    “也许里面没有人,”邓已帮腔似地暗示说,“也许他们只是把这些绷带送到这儿来开个玩笑。”

    她惊恐地从邓巴身边退开。“你疯了,”她一边喊着,一边用哀求的目光四下张望。“你们两个都疯了。”

    这时达克特护士出现了,把他们都赶回到他们自己的床上去,而克拉默护士则为那个浑身雪白的士兵更换了塞住口的瓶子。为那个浑身雪白的士兵换瓶子是件毫不费力的事,因为那些相同的、清澈的液体一遍又一遍地滴进他的体内,没有明显的损耗。当那只盛着滴入他手臂内的液体的瓶子差不多要空了的时候,那只放在地板上的瓶子就快要满了,只要把那两只瓶子从它们各自的管子上拿开并很快换个位置,这样液体就又能滴入他的体内。换瓶子这件事对其他人来说并没有什么,但却使那些看着这些瓶子大约每小时被更换一次的人受不了,他们对这一程序感到迷惑不解。

    “他们干吗不把两只瓶子连起来,去掉那个中间的人呢?”那个刚同约塞连下完棋的炮兵上尉问,“他们到底需要他干什么?”
 


    “我不晓得他做了些什么要受这份罪,”那个得了疟疾、屁股上曾被蚊子叮过一口的二级准尉,在克拉默护士察看过体温表并发现那个浑身雪白的士兵已经死了之后这样哀叹道。

    “他打过仗,”那个留着金黄色小胡子的战斗机飞行员猜测说。

    “我们都打过仗,”邓巴反驳说。

    “我就是那个意思,”那个得疟疾的二级准尉继续说,“为什么是他?这种奖惩制度好像没什么逻辑。看看我的遭遇。要是我那次在海滩上放纵五分钟之后得了梅毒或淋病而不是被那该死的蚊子叮了一口,我倒觉得还有点公平。可怎么会得疟疾?疟疾?谁能解释私通的结果会是疟疾?”那个二级准尉摇了摇头,惊讶得无话可说。

    “我的情况怎么样呢?”约塞连说,“在马拉喀什,我有天晚上从帐篷里出来去买块糖,不想那个我以前从未见过的陆军妇女队队员悄悄把我引进树丛里,于是就得了该你得的那种淋病。我的的确确是想去买块糖,但谁能拒绝那种事呢?”

    “那听起来是像该我得的淋病,不错,”那准尉赞同他说,“可是我还是得了别人的疟疾。就这一次,我真想看到所有这些事情都能改正过来,每个人该得到什么就得到什么。这也许能使我对这个世界有几分信心。”

    “我得到了别人的三十万元钱,”那个留着金黄色小胡子的年轻、漂亮的上尉战斗机飞行员承认说,“我从生下来的那天起就开始混日子。我靠欺骗的方法从预备学校一直混到大学毕业;从那以后我所做的一切就是跟漂亮妞睡觉,她们还以为我会做个好丈夫呢。我压根儿就没什么雄心大志。战争结束之后我想做的唯一的一件事就是找个比我还有钱的姑娘结婚,同更多的漂亮妞睡觉。那三十万块钱是在我出生前由我的一个祖父辈的亲戚留给我的,他做国际生意发了财。我知道我不配得到这笔钱,但我要是不拿,我就不是人。我不知道这钱真正该归谁。”

    “也许该归我父亲,”邓巴推测说,“他辛辛苦苦干了一辈子,也没有挣到足够的钱来送我姐姐和我上大学。他现在已经死了,所以你完全可以留着这笔钱啦。”
 


    “现在只要我们能找到我得的疟疾应当归谁,我们的问题就都解决了;这并不是因为我要跟疟疾作对,只要能尽快逃避工作,得疟疾跟得其他病都一样。只是我觉得这事不公平。干吗要我患上别人的疟疾,而你又染上我的淋病呢?”

    “我还不止得了该你得的淋病呢,”约塞连跟他说,“由于你那个淋病,我不得不一直执行战斗飞行任务,直到他们把我打死为止。”

    “那这事就更糟了。这件事情里有什么公正可言?”

    “两个半星期之前,我有个朋友叫克莱文杰,他总认为这事挺公正的。”

    “这是最公正的事啦。”克莱文杰当时得意扬扬地拍着手,高兴地笑着。“我不禁想起欧里庇得斯的《希波吕托斯》。在那个剧里,由于忒修斯早年生活放荡,他儿子便信奉禁欲主义,这便导致了把他们都毁灭掉了的悲剧。即使没有别的事,那件与陆军妇女队员的插曲也该让你知道风流好色的恶果。”

    “它让我知道了糖果的恶果。”

    “你难道看不出,你现在处境尴尬,你自己并非完全没有责任吗?”克莱文杰接着说,一点也不掩盖他的兴致。“如果不是你染上花柳病在非洲那边的医院里躺了十天的话,你也许在内弗斯上校被打死之前,也就是说在卡思卡特上校来接替他之前就按时完成了你的二十五次飞行任务,现在已被送回家了。”

    “你怎么样?”约塞连以问代答,“你在马拉喀什从未染上淋病,而你也一样处境尴尬嘛。”

    “我不知道,”克莱文杰假装有点关切地招认说,“我想我这一生中一定干了什么非常坏的事。”

    “你真的相信那种事情吗?”

    克莱文杰笑了起来。“不,当然不相信。我只是想和你逗逗乐。”

    对约塞连来说,危险多得数不胜数。比如说,有希特勒、墨索里尼和东条,他们都极力想杀掉他;还有那个队列狂沙伊斯科普夫少尉和那个留着两撇粗大的八字胡、狂热地盲目相信因果报应的胖上校,他们也都想弄死他;还有阿普尔比、哈弗迈耶、布莱克和科恩;还有克拉默护士和达克特护士,他几乎可以肯定她们都盼他死;还有那个得克萨斯人和那个罪犯调查部的官员,对这两人他也毫无疑问;还有世界各地的酒吧招待、砖瓦匠和公共汽车售票员,他们也都希望他死;还有那些房东和房客、叛徒和爱国者、行私刑的人、吸血鬼和走狗,他们全部一心想谋害他。就是在执行飞往阿维尼翁的任务时斯诺登向他泄露了秘密——他们千方百计想杀死他:而斯诺登当时是在飞机的后舱里把这个秘密泄露出来的。

    还有淋巴腺也有可能要他的命;还有肾脏、神经束膜和神经膜细胞;还有脑瘤;还有何杰金氏病、白血病、肌萎缩性侧索硬化;还有上皮组织再生性红斑滋生癌细胞;还有皮肤病、骨科病、肺病、胃病、心脏病、血液病和动脉血管病;还有头部疾病、颈部疾病、胸部疾病、大小肠疾病、胯部疾病,甚至还有脚病;还有几十亿个勤劳的人体细胞,在维持他的生命和庭康的复杂的工作中,像默默无闻的牲口一样不分昼夜地进行氧化作用,而它们中任何一个都是潜在的叛徒和敌人。疾病是如此之多,如果有谁像他和亨格利-乔那样经常去考虑它们,那这个人的脑袋瓜一定是有毛病了。
 


    亨格利-乔搜集了一大堆不治之症的名称,并把它们按字母顺序排列起来,这样他就能很快找到他想要担心的任何疾病。每当他把某种疾病的名称摆错了位置或当他无法把它加进他的疾病名单里去时,他就会变得心神不安,浑身冷汗地跑去向丹尼卡医生求援。

    丹尼卡医生在处理亨格利-乔的事情时总会来向约塞连求援。

    “说他得了尤因氏瘤,”约塞连向医生建议说,“还说他得了黑素瘤。

    亨格利-乔喜欢旷日持久的病,不过他更喜欢暴发性疾病。”

    丹尼卡医生从未听说过这两种病。“你怎么能记得住这么多那样的病?”他带着职业性的崇高的敬慕问道。

    “我在医院里读《读者文摘》知道的。”

    约塞连有那么多疾病要担心,有时他真想永远呆在医院里度过余生:四肢平展地躺在氧气帐里,一群专家和护士一天二十四小时坐在他的病床的一边,等待着病情发生恶化;在病床的另一边至少有一名外科医生拿着刀,做好了准备,一旦需要随时准备冲上前来开始手术。比如说动脉瘤,要是他得了主动脉瘤,不采取这样的措施,他们又怎能及时医治他呢?尽管约塞连像讨厌任何人一样讨厌外科医生和他的手术刀,他还是觉得呆在医院里面要比呆在医院外面安全得多。在医院里,他可以随时大声叫喊,人们至少会跑过来想办法帮他;而在医院外面,如果他对所有他认为每个人都该大声叫喊的事情大叫大喊,人们会把他关进监狱或者把他送进医院。他想对其大声叫喊的东西之一就是外科医生的手术刀,那刀几乎肯定在等待着他和其他所有活得够长的、可以死去的人。他常常想弄明白他怎样才能辨认出初起的风寒、发烧、剧痛、隐痛、打嗝、打喷嚏、色斑、嗜眠症、失语、失去平衡或者记忆力衰退,那预示着不可避免的结局的不可避免的开始。

    他还担心当他跳出梅杰少校的办公室再去找丹尼卡医生时,丹尼卡医生仍旧拒绝帮助他。他的担心是对的。

    “你以为你得了什么可以担心的病了吗?”丹尼卡医生问道,说话间抬起他那低垂在胸前、黑发梳得一尘不染的头,两只满是泪水的眼睛愤怒地盯了约塞连一会儿。“我怎么样呢?我的宝贵的医疗技术在这个该死的岛上白白地荒废了,而其他的医生却在挣大钱。

    你以为我喜欢日复一日地坐在这儿拒绝帮助你吗?如果我是在国内或在像罗马这样的地方拒绝帮助你,我倒不特别在乎。但在这儿向你说不,对我来说也不是件容易的事。”

    “那么就别说不。让我停止飞行。”

    “我不能让你停飞,”丹尼卡医生嘟嚷道,“这话得告诉你多少遍?”

    “你能。梅杰少校跟我说你是飞行中队里唯一能让我停飞的人。”

    丹尼卡医生惊得瞠目结舌。“梅杰少校跟你那么说的?什么时”候?”

    “我在壕沟里同他交涉的时候。”

    “梅杰少校是那么跟你说的?在一个壕沟里?”

    “他是在我们离开壕沟,跳进他的办公室后跟我说的。他叫我不要跟任何人说是他告诉我的,所以请你不要乱嚷嚷。”

    “为什么是那个卑鄙、诡计多端的骗子!”丹尼卡医生喊道,“他不应该告诉任何人。他有没有告诉你我怎样才能让你停飞?”

    “只要填写一张小纸条,说我已处于精神崩溃的边缘,把它送到大队部就行了。斯塔布斯医生一直让他的中队里的人停飞,你为什么不能呢?”

    “斯塔布斯让那些人停飞之后,他们的情况又怎么样呢?”丹尼卡医生冷笑着反驳说,“他们马上被恢复战斗状态,不是吗?而他也发现他自己处于困境。当然,我也可以填写一张说你不适合飞行的纸条,让你停飞。但是有一条规定。”

    “第二十二条军规?”

    “是的。假如我取消你的战斗任务,还得大队部批准,而大队部是不会批准的。他们会立即让你回到战斗岗位上去。那么,我又会在什么地方呢?也许在去太平洋的路上,不行,多谢你啦,我不想为你去冒险。”

    “难道这不值得一试吗?”约塞连争辩道,“皮亚诺萨岛有什么好呢?”

    “皮亚诺萨岛糟透了,但它却比太平洋好。要是用船把我运到某个文明发达的地方,在那儿我时不时可以赚一二块打胎的钱,我倒不会在乎。然而在太平洋却只有丛林和季风。我在那儿会烂掉的。”

    “你在这儿也会烂掉的。”

    丹尼卡医生突然发起怒来。“是吗?不过,至少我会活着走出这场战争,这比你所要做的一切都强。”

    “那正是我想跟你说的,嘿。我求你救我一命。”

    “救命不是我的职责,”丹尼卡医生绷着脸驳斥道。

    “什么是你的职责?”

    “我不知道我的职责是什么。他们告诉我的就是要坚持我的职业道德,决不作证去反对另一个医生。听着,你以为你是唯一有生命危险的人吗?我怎么样呢?医疗帐篷里那两个为我工作的庸医至今还查不出我有什么病。”

    “可能是尤因氏瘤,”约塞连嘲讽地咕哝说。

    “你真的那么认为?”丹尼卡医生害怕得嚷起来。

    “噢,我不知道,”约塞连不耐烦地回答,“我只知道我不想再执行任务了。他们不会真的枪毙我吧,是吗,我已经飞了五十一次。”

    “你为什么不至少完成五十五次飞行任务再做决定呢?”丹尼卡医生劝告说,“你成天抱怨,可你一次也未完成过任务。”

    “我怎么能完成呢?每次我快要完成的时候,上校又把飞行次数提高了。”

    “你从未完成任务,是因为你老是不断地进医院或者离队去罗马。假如你完成了五十五次飞行任务,然后再拒绝飞行,你的处境就会有利得多。那样,我也许会考虑我能做点什么。”

    “你能保证吗?”

    “我保证。”

    “你保证什么呢?”

    “如果你完成你的五十五次飞行任务,再让麦克沃特把我的名字登入他的飞行日志中,让我不用上飞机就可以拿到我的飞行津贴,我保证我也许会考虑做点什么帮助你。我害怕飞机。你有没有看到三周前发生在爱达荷州的那次飞机坠毁的报道,六个人送了命。太可怕了。我不知道他们为什么非要我每月飞行四小时才能拿到飞行津贴。难道用不着担心死在飞机坠毁中,我要担忧的事就不够多吗?”

    “我也担心飞机坠毁事故,”约塞连跟他说,“你不是唯一担忧的人。”

    “是啊,不过我还很担心那个尤因氏瘤,”丹尼卡医生虚夸道,“你看我的鼻子一直不通,身体总觉得冷,是不是就是这个原因?搭搭我的脉。”

    约塞连也担心尤因氏瘤和黑素瘤。到处都潜伏着灾难,多得数不胜数。当他想到有那么多疾病和可能发生的事故时刻威胁着他,而他却能安然无恙地活到今天,他着实吃惊不小。每一天他所面临的都是新的一次战胜死亡的危险使命。他已经这样活了二十八年了
