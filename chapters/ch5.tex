\chapter{一级准尉怀特·哈尔福特}

 
    丹尼卡医生和一级准尉怀特-哈尔福特合住一顶污渍斑斑的灰色帐篷;对哈尔福特,丹尼卡医生极害怕,可又很鄙视。

    “我能想象得出他的肝长得什么样,”丹尼卡医生咕哝道。

    “那你说说我的肝怎么样,”约塞连跟他说。

    “你的肝没什么不好。”

    “这说明你真是太无知了。”约塞连故意虚张声势。他告诉丹尼卡医生说,他的肝曾痛得让他大受折磨,再者,这肝痛又没转成黄疸病,也没消失,让达克特护士、克莱默护士和医院里所有的医生着实苦恼了一阵子。

    丹尼卡医生毫无兴趣。“你以为自己得了病?”他问了一句,“那我呢?那天,那对新婚夫妇走进我诊所的时候,你应该在场的。”

    “什么新婚夫妇?”

    “有一天走进我诊所的那对新婚夫妇。难道我从未跟你提起过?那新娘可真漂亮。”

    丹尼卡医生的诊所也极漂亮。候诊室里陈放着金鱼,还有一套算是上品的廉价家具。只要可能,他买东西向来是赊帐的,即便是买金鱼,也是如此。至于无法赊购的东西,他便以分享诊所的收益为条件,从那些贪心的亲戚处换取些许现钱。他的诊所设在斯塔腾岛,是一座两户合用的简易房,没有任何消防设施。诊所离渡口只四条马路,往北仅隔一条马路,便是一家超级市场,三家美容院和两家非法药铺。诊所正好处在街角,但无甚益处。此地人口流动量极小,居民出于习惯,看病总是找打了多年交道的医生。帐单迅速堆积了起来,丹尼卡医生丢失了自己最心爱的医疗器械:加法机被收口,随后是打字机,也让人取了回去。金鱼全都死了。幸运的是,就在他感到暗无天日的时候,战争爆发了。

    “真是天赐良机,”丹尼卡医生很认真地坦言道,“其他医生当中,有大多数人很快服了役,事情一夜间便大有转机。我诊所的地理位置,这下可真开始发挥作用了。不久,来诊所的病人越来越多,忙得我应接不暇。我便加倍付酬金给那两家药铺。那几家美容院也挺不错,每星期介绍两三个人来我这儿做人工流产。生意实在是好得不能再好了。可你瞧,后来竟出了件事。他们派了征兵局的一个家伙来替我做体格检查。我是4-F体位者。先前,我早就给自己做了相当全面的体格检查,发现自己的身体不宜服兵役。你大概会想,只要我说出实情,就能免去一切麻烦,因为在我们县医务界和本地商业信用局,我一向是口碑极好的医生。然而,事实并非如此。

 


    他们派那家伙来,目的只是想查实:我是否确实齐髋切除了一条腿,是否确实患了不治的风湿性关节炎,终日缠绵病榻,连生活都无法自理。约塞连,我们生活在一个相互猜疑、精神准则日趋堕落的时代。这实在是大可怕了,”丹尼卡医生断言道。他情绪极为激动,说话时,连声音都颤抖了。“就连自己心爱的祖国,也怀疑起一个领有开业执照的医生所说的话,这实在是太可怕了。”

    丹尼卡医生应征入伍,被运送到皮亚诺萨岛,当上了一名航空军医,尽管他惧怕飞行。

    “坐在飞机上,我倒是用不着自找麻烦,”丹尼卡医生说,一边眨着那对棕色的、亮晶晶的小近视眼,两眼满是气恼。“麻烦会自己找上门来的。就跟我同你说起过的那个生不了孩子的处女一样。”

    “什么处女?”约塞连问,“我还以为你是在说那对新婚夫妇。”

    “我说的处女,就是那个新娘。他俩其实年纪还很小。那天来我诊所,两人事先没预定。当时,他们结婚才不过一年多一点。真可惜,你没眼福。那姑娘长得极甜,人年轻,实在是很漂亮。我问她经期是否正常,她竟羞得脸绯红。我想我今生今世是会永远喜爱那姑娘的。她就像是梦中的美女,脖子上挂了条项链,项链下端是一枚圣安东尼像章,垂在里面的胸脯前。那胸脯真是美妙绝伦,是我先前从未见过的。‘这对圣安东尼来说,实在是个可怕的诱惑。’我开了个玩笑——只是想让她放松些。‘圣安东尼?’,她丈夫说,‘谁是圣安东尼?’‘问你妻子,’我对他说,‘她可以告诉你谁是圣安东尼。’‘谁是圣安东尼?’他问她。‘谁?’她问。‘圣安东尼,’他对她说。‘圣安东尼?’她说,‘谁是圣安东尼?’在诊察室里,我替她做了详细检查,发现她还是个处女。趁她重新穿上紧身褡,把它钩在长统袜上的当儿,我跟她丈夫单独谈了一会,‘每天晚上,’他夸口道。你要知道,他实在是个自作聪明的家伙。‘我从来不错过一个晚上,’他夸口道,像是真有那么回事儿。‘每天早晨上班前,她给我准备早餐,用餐前,我还要跟她作爱,’”他向我夸口说。只有一个办法可以跟他们解释清楚。过后,我把他俩重新叫到一起,用诊所的橡胶模特儿,给他们表演性交的示范动作。这些橡胶模特儿都在我的诊所里,此外,还有男女生殖器官的各种模型,我都分别锁在几个柜子里,免得人家说三道四。我的意思是,我曾经有过这些东西,可现在,一无所有,连诊所都没了。有的只是这低体温,真让我担心。在医务所给我当助手的那两个家伙,简直是蠢猪,连看病都不会。他们只知道发牢骚。他们以为自己有难言之苦?那我呢?那天,在诊所给那对新婚夫妇做性交示范时,那两个家伙要是在场就好了。当时,那对新婚夫妇望着我,好像我是在跟他们说以前从未有人听说过的事。你从未见过有谁会如此兴致勃勃。‘你是说这样?’男的问我,且动手演示了一番。你要知道,我清楚什么人在这种演示过程中到了什么时候兴趣最大。‘没错,’我跟他说,‘行了,你们这就回家去,按我的方法试几个月,看是否有效。怎么样?’‘好吧。’说罢,他们便很爽快地付了钱。‘祝你们快乐,’我对他们说。他们向我道了谢,于是便一同走了出去。他伸手搂住她的腰,仿佛等不及带她回家作爱了。几天后,他一个人跑到我的诊所,告诉护士说,他得马上见我。一旦我俩单独见了面,他便对着我的鼻子狠狠一拳。”

    “他怎么着?”

    “他骂我是个自命不凡的混蛋,对着我的鼻子狠狠一拳。‘你是个啥东西,一个自命不凡的混蛋?’刚说完,他便把我打得仰面倒在了地上。砰!就像这样。我骗你不是人。”

    “我知道你没骗我,”约塞连说,“可他干吗要那么做?”

    “这我怎么知道?”丹尼卡医生反问了一句,显得很是恼怒。

    “也许跟圣安东尼有关吧?”

    丹尼卡医生木然地望着约塞连。“圣安东尼?”他吃惊地问道,“谁是圣安东尼?”

    “我怎么知道?”一级准尉怀特-哈尔福特回答道,这时,他正巧蹒跚着走进帐篷,一手捧了瓶威士忌,在他俩中间坐了下来,一副咄咄逼人的模样。

    丹尼卡医生一声不吭地站了起来,驼着背——长年来,生活中的种种不公平,始终是沉重的负担,压弯了他的腰——把椅子挪到了帐篷外面。他实在是讨厌跟自己同帐篷的人聚在一块。

 


    一级准尉怀特-哈尔福特以为他疯了。“真不晓得这家伙是怎么回事,”他说,颇有些责备的口气。“他是头蠢驴,就这么回事。假如他聪明的话,他就会抓过一把铁锹,动手挖掘。就在这顶帐篷里动手挖,就在我床底下。他马上就能挖到石油。那个士兵在美国用铁锹挖到了石油,这事难道他不知道?那家伙后来发生的事,难道他也从未耳闻?就是科罗拉多州那个拉皮条的卑鄙无耻的孬种,叫什么来着?”

    “温特格林。”

    “温特格林。”

    “他很怕,”约塞连解释道。

    “哦,没那回事。温特格林可是啥都不怕的。”一级准尉怀特-哈尔福特摇了摇头,对温特格林的钦佩之情溢于言表。“那个讨厌的小流氓,自命不凡的杂种,是谁都不怕的。”

    “丹尼卡医生可是很害怕。他就是这么一回事。”

    “他怕什么?”

    “他怕你,”约塞连说,“他怕你会得肺炎死。”

    “他怕,反倒是桩好事,”一级准尉怀特-哈尔福特说,结实的胸腔里发出一阵低沉的笑声。“一有机会,我也很乐意这么个死法。你等着瞧吧。”

    一级准尉怀特-哈尔福特,来自俄克拉何马州的伊尼德,是个印第安人,克里克混血儿。哈尔福特肤色黝黑、长得倒是相当英俊:粗眉大眼、高高的颧骨、一头蓬乱的乌发,出于某些只有他自己知道的原因,他已经打定主意,要得了肺炎死去。他报复心极强,见到任何人都是怒目相待,对一切早已不抱丝毫幻想。他憎恨那些取名卡思卡特、科恩、布莱克和哈弗迈耶的外国人;希望他们全都滚回自己讨厌的祖先原来生活的地方。

 


    “你是不会信的,约塞连,”他深思后说道,同时,故意提高了嗓门,引诱丹尼卡医生。“不过,先前这地方让人住着,确实感到挺舒畅,但后来,他们带来了该死的虔诚,把这儿搞成一团糟。”

    一级准尉怀特-哈尔福特一心想报复白人。他差不多是个文盲,不识一字,也不会写字,却被委派担任布莱克上尉的助理情报官。

    “我哪有条件读书认字?”一级准尉怀特-哈尔福特用假装寻衅的口吻问道,且又提高了嗓门,好让丹尼卡医生听见。“我们每到一处搭起帐篷,他们使钻一口油井。每次钻井,他们又总是找到石油。

    每次找到了石油,他们便逼迫我们收起帐篷,去别的地方。我们成了活的探矿杖。我们全家生来就踉石油矿有缘分。不久,世界上所有的石油公司都派了技术人员,处处跟踪我们。我们常年四处奔波。跟你说吧,抚养一个孩子,不知要费多大的劲。我想,我在一个地方住的时间,从未超过一个星期。”

    他最早的记忆,是一位地质学家。

    “每次我们家生了个小孩,”他接着说,“股票行情便上涨。不久,所有钻井工人便带上全部设备,随我们东奔西跑,谁都想捷足先登。一家家公司开始合并,以便削减为追踪我们而派出的人员。

    然而,跟在我们身后的人,数量一天天上升。我们一家人从未睡过一个安稳觉。我们歇腿,他们也歇腿;我们上路,他们也上路,随身还带了流动炊事车、推土机、井架和发电机。我们一家成了活财神,走到哪里,哪里便是一片繁荣。于是,我们开始接到一些一流旅馆的请柬,原因便是我们能使他们的生意兴盛。有些旅馆在请柬上提出了相当优厚的条件。但我们无法接受任何一家旅馆的邀请,因为我们是印第安人,而给我们发出邀请的那些一流旅馆,是不会接纳印第安人的。种族偏见,实在令人可怕,约塞连。确实很可怕。把体面忠诚的印第安人看做黑鬼、犹太佬、意大利人,或是西班牙人,这的确是件可怕的事。”一级准尉怀特-哈尔福特慢悠悠地点了点头,显得极有自信。

    “后来,约塞连,终于出了事儿——也就是结局的开始。他们走到前面跟着我们转。他们会想法子猜测,接下来我们在哪里歇息,于是,趁我们还没赶到,他们便开始钻井,结果,我们就无法停下来歇息。我们刚想铺开毯子,他们就赶我们走。他们很信任我们。他们甚至等不及把我们赶走,就急不可耐地挖井钻油。我们给折腾得精疲力竭,即便是死,也毫不畏惧。一天早晨,我们发现四周给钻井工人团团围住,他们都等着我们朝他们各自的方向走去,然后把我们赶走。我们环顾四周,见到每一处山脊上都有一个钻井工人守候着,犹如印第安人随时准备发起进攻。我们的未日到来了。我们无法在原地停留,因为他们才把我们赶走。我们走投无路。最终,倒是军队救了我。正当紧要关头,战争爆发了。征兵局把我救了出来,又把我安全送到了科罗拉多州的洛厄里基地。我们全家只有我一个人活了下来。”
 


    约塞连知道他是在撤谎,但没有打断他,因为一级准尉怀特-哈尔福特接着又说了下去。他说,此后他再也没有父母的任何消息。不过,他不怎么担心,因为他只是听他们说,他是他们的儿子。

    以前有不少事他们都没跟他说实话,那么,至于这件事,他们也完全可能是在说假话;他倒是很清楚自己一帮表堂兄弟的命运。他们曾分散了目标,往北走,因一时大意,竟闯入了加拿大境内。就在他们想法子返回时,美国移民局把他们挡在了边界上,不允许他们回国。他们回不了国,就因为他们是红种人。

    这笑话实在是骇人听闻。丹尼卡医生没有笑。直到后来,约塞连执行一次飞行任务返回,又一次恳请丹尼卡医生准许他停飞——自然,他去见丹尼卡医生,实在是不抱任何希望的,这时,丹尼卡医生才窃笑了一下,但没一会儿,他便沉思起自己的种种棘手事来。其中就有与一级准尉怀特-哈尔福特之间的纠葛。那天整整一个上午,一级准尉怀特-哈尔福特一直向他挑战,要跟他角力,决一雌雄。此外,还有约塞连,这家伙竟当即拿定主意,要装疯卖傻。

    “你是在浪费时间,”丹尼卡医生不得不跟他这么说。

    “难道你就不能让一个疯子停飞?”

    “哦,当然可以。再说,我必须那么做。有一条军规明文规定,我必须禁止任何一个疯子执行飞行任务。”

    “那你为什么不让我停飞?我真是疯了。不信,你去问克莱文杰。”

    “克莱文杰?克莱文杰在哪儿?你把克莱文杰找来,我来问他。”

    “那你去问问其他什么人。他们会告诉你,我究竟疯到了什么程度。”

    “他们一个个都是疯子。”

    “那你干吗不让他们停飞?”

    “他们干吗不来找我提这个要求?”

    “因为他们都是疯子,原因就在这里。”

    “他们当然都是疯子,”丹尼卡医生回答道。

    “我刚跟你说过,他们一个个都是疯子,是不是?

    你总不至于让疯子来判定,你究竟是不是疯子,对不?”

    约塞连极严肃地看着他,想用另一种方式试试。“奥尔是不是疯子?”

    “他当然是疯子,”丹尼卡医生说。

    “你能让他停飞吗?”

    “当然可以。不过,先得由他自己来向我提这个要求。规定中有这一条。”

    “那他干吗不来找你?”

    “因为他是疯子,”丹尼卡医生说,“他好多次死里逃生,可还是一个劲地上天执行作战任务,他要不是疯子,那才怪呢。当然,我可以让奥尔停飞。但,他首先得自己来找我提这个要求。”

    “难道他只要跟你提出要求,就可以停飞?”

    “没错。让他来找我。”

    “这样你就能让他停飞?”约塞连问。

    “不能。这样我就不能让他停飞。”

    “你是说这其中有个圈套?”

    “那当然,”丹尼卡医生答道,“这就是第二十二条军规。凡是想逃脱作战任务的人,绝对不会是真正的疯子。”

    这其中只有一个圈套,那便是第二十二条军规。军规规定,凡在面对迫在眉睫的、实实在在的危险时,对自身的安危所表现出的关切,是大脑的理性活动过程。奥尔是疯了,可以获准停止飞行。他必须做的事,就是提出要求,然而,一旦他提出要求,他便不再是疯子,必须继续执行飞行任务。如果奥尔继续执行飞行任务,他便是疯子,但假如他就此停止飞行,那说明他神志完全正常,然而,要是他神志正常,那么他就必须去执行飞行任务。假如他执行飞行任务,他便是疯子,所以就不必去飞行;但如果他不想去飞行,那么他就不是疯子,于是便不得不去。第二十二条军规这一条款,实在是再简洁不过,约塞连深受感动,于是,很肃然地吹了声口哨。

    “这第二十二条军规,实在是个了不起的圈套,”他说。

    “绝妙无比。”丹尼卡医生表示赞同。

    约塞连很清楚,第二十二条军规用的是螺旋式的诡辩。其中各个组成部分,配合得相当完美。这种配合极是简洁精确——优雅得体却又令人惊异,与优秀的现代艺术相仿。但有时,约塞连又没什么把握,究竟自己是否通晓这第二十二条军规,就像他从来没有真正理解优秀的现代艺术一样,也如同他从来就不怎么相信奥尔在阿普尔比的眼睛里见到苍蝇一般。他听了奥尔说的话,竟信了阿普尔比的眼睛里有苍蝇。

    “噢,他的眼睛里的确有苍蝇,”一次,约塞连和阿普尔比在军官俱乐部打架之后,奥尔深信不疑地对约塞连说,“或许连他自己还不知道。他之所以总不识事物的真面目,其原因也就在这里。”

    “他怎么会不知道?”约塞连问。

    “因为他眼睛里有了苍蝇,”奥尔异常耐心地解释道,“假如他眼睛里有苍蝇,他又怎么能看见自己眼睛里有苍蝇呢?”

    这话没太多的道理,但在没有取得相反的论据之前,约塞连倒是愿意暂且相信奥尔说得挺在理的,因为奥尔来自纽约市外的荒郊,对野生生物的了解,无疑要比他约塞连深得多。再者,奥尔以前从未在关键性问题上跟他说过假话,这一点便不同于约塞连的父母亲、兄弟姊妹、伯父伯母、姻亲、师长、宗教领袖、议员、邻居和报纸。约塞连曾用了一两天的时间,独自反复考虑了新近听到的这件关于阿普尔比的事,于是,决定做桩好事,把传闻告诉阿普尔比本人。
 


    “阿普尔比,你眼睛里有苍蝇,”约塞连好心地跟阿普尔比低语道。那天,他俩恰巧在降落伞室门口碰面,正准备去执行每周一次的飞往帕尔马的例行任务。

    “什么?”阿普尔比迅速做出反应,约塞连竟会跟他说话,这实在很让他惊慌失措。

    “你眼睛里有苍蝇。”约塞连重复说了一遍。“你自己看不见,原因很可能就在这里。”

    阿普尔比一脸反感和困惑地离开了约塞连,独自生着闷气。直到后来,坐进吉普车,跟哈弗迈耶一同沿着长长的笔直的公路,驱车前往简令下达室,他这才把脸舒展了开来。大队作战处长丹比少校正焦躁不安地等候在简令下达室,准备给全体领队飞行员、轰炸员和领航员做飞行前的预先指示。阿普尔比说话时声音极低,以免司机和布莱克上尉听见,布莱克上尉闭着双眼,舒展了肢体,躺坐在吉普车前排座上。

    “哈弗迈耶,”阿普尔比言语支吾地问道,“我眼睛里有苍蝇吗?”

    哈弗迈耶极是疑惑地眨了眨眼,问道:“睑腺炎?”

    “不,我是问你我眼睛里有没有苍蝇。”

    哈弗迈耶又眨了眨眼。“苍蝇?”

    “在我的眼睛里。”

    “你一定是疯了,”哈弗迈耶说。

    “不,我没疯。疯的是约塞连。你只要告诉我,我眼睛里到底有没有苍蝇。你快说,我是不会介意的。”

    哈弗迈耶又往嘴里塞进一块花生薄脆糖,于是,凑近了过去,极仔细地看了看阿普尔比的眼睛。

    “我没见到一只苍蝇,”他说。

    阿普尔比深叹了一口气,如释重负。哈弗迈耶把一片片花生薄脆糖碎屑粘在嘴唇、下巴和面颊上。

    “花生薄脆糖碎屑都粘到你脸上了,”阿普尔比提醒他说。

    “与其让苍蝇钻进眼睛里,倒不如往脸上粘花生薄脆糖碎屑呢,”哈弗迈耶反击道。

    每一小队其他五架飞机的军官坐了卡车来到简令下达室,准备听取半小时后所做的全面指示。每一机组有三名士兵,飞行前的指示他们是听不到一点的。他们被直接送往机场上预定那天执行飞行任务的一架架飞机旁,和地勤人员一同在那里等候,直等到预定和他们一起飞行的军官坐卡车到来,纵身跳下格格作响的卡车后拦板。于是,便登机,启动引擎。引擎在冰棍形的停机坪上极不情愿地启动了起来,先是怎么也转不起来,接着,便平稳地空转了片刻。随后,所有飞机隆隆地绕了一圈,像一个个笨拙的瘸腿瞎子,沿着铺满卵石的地面一瘸一拐,小心翼翼地往前滑行而去,待上了机场尽头的跑道,在一阵震耳欲聋的轰呜声中,一架紧接一架,迅捷腾空而起,继而慢慢倾斜飞行,编成队形,掠过斑驳陆离的树高线,随即又平稳地绕机场飞了一圈。待由六架飞机组成的各小队均已编好队形,机群遂调转了航向,掠过蔚蓝色的水面,朝意大利北部或是法国的目标飞去。机群渐渐爬高,等到飞入敌国领空时,已升至九千多英尺的高空。每次出航总有不少令人惊奇的事,其中之一便是自觉镇定,四周极度静谧,唯一的声响是机关枪的试射,以及对讲机偶尔传出的单调生硬的一句话,最终便是每架飞机上的轰炸员提醒全体机组人员,宣布飞机已进入轰炸点,准备飞往目标。

    天气又是每次晴和,由于空气稀薄,总有些许黏糊的异物卡在喉咙口。

    他们驾驶的是B25型暗绿色飞机,性能平稳可靠,装有两只方向舵,两只引擎,两片宽机翼。唯一的不足之处——就轰炸员约塞连所坐的位置来看,便是那条狭窄的爬行通道——把设在有机玻璃机头里的轰炸员舱内最近的应急离机口隔了开来。爬行通道是一个正方形长孔,狭小、冰凉,上面是飞行控制系统。像约塞连这样的彪形大汉,只有费了劲才能勉强挤身通过。有一个圆脸的矮胖领航员——长一对奸诈的小眼,身上揣一只与阿费相同的烟斗——也很难从这个孔过去。每当他们飞往目标——相距仅几分钟,约塞连便会把他逐出机头。紧接着是一段时间的紧张不安,默默地等待,什么也听不见,什么也看不见,什么也做不了,只有默默地等待。此时,下面的高射炮已瞄准了他们,假如可能,随时准备把他们彻底击落,坠入长眠之谷。

    一旦飞机即将坠落,这条通道,对约塞连来说,就是通向机外的生命线,可约塞连竟诅咒它,对它恨之入骨,辱骂它是老天故意设置的一道障碍,是欲置他于死地的阴谋的一部分。按说,B25型飞机还有地方可再开一个应急离机口,而且就在机头,但他们却没有一个应急离机口,替而代之的是这条通道,自那次在阿维尼翁上空执行任务时发生混乱以后,他便开始憎恨这条通道的每一英寸空间,因为它把他和降落伞——太是笨重,无法随身携带——之间的距离延长了若干秒钟;又使他取了降落伞后赶往应急离机口——设在立架式驾驶舱的后部和顶炮塔射击手(高高在上,因而遮没了脸面)两脚之间的地板上——的时间延宕得更长。约塞连一旦把阿费逐出机头,自己便极迫切地想坐到阿费的位置上;他还很想在应急离机口顶端的地板上,用自己乐意多带的防弹衣筑一个拱形掩体,然后蜷缩了身体躲在里面,降落伞早已用钩固定在相应的安全带上,一手紧紧握住红柄开伞索,一手死死抓牢应急开盖开关——一旦听到飞机遭击毁的可怕声响,打开开关,他便坠入空中,朝地面落下去。假如他必须得留在机头的话,他就想占据这个位置。他可不愿守在前面,像一条该死的金鱼,给死死地困在一只该死的动不了的金鱼缸里。原因是,一旦战火起,那该死的高射炮火便喷出一团团发恶臭的黑色浓烟,在他的四周上下急速地翻腾,恰似变幻无常、硕大无朋的邪魔,时而徐徐上升、僻啪作响,时而摇荡不定、砰然爆裂,震得飞机格格直响、上下颠簸、左右晃悠,又一个劲地往机内直穿进去,威胁着要在瞬息间将他们全都湮灭在一片火海之中。

    阿费无论充当领航员,抑或承担别的什么职责,于约塞连全无益处。约塞连每回都是极没好气地把他逐出机头,这样,假若他俩突然要仓皇逃命,也就不会相互碍事。一旦让约塞连逐出机头,阿费就可以蜷缩在约塞连迫切地想躲身的那块地方,但他没那么做,却是直挺挺地立着,两只又粗又短的胳臂极适意地搁放在驾驶员和副驾驶员座位的靠背上,一手端了烟斗,跟麦克沃特和当班的副驾驶员轻快地聊着夭,同时又指出天空出现的有趣味的东西,让他俩瞧。可是,麦克沃特和副驾驶员实在大忙,没有丝毫的兴致。麦克沃特守在控制系统一侧,忙于执行约塞连尖声喊出的命令。约塞连让飞机侧滑进入轰炸航路,接着,又尖起嗓门,以极粗鲁的口吻满嘴脏话地给麦克沃特下命令——酷似亨格利-乔在黑夜里梦魇时叫出的痛苦的哀求声,要大伙儿迅速绕过炸弹爆炸溅起的一根根饿虎似的火柱,离开轰炸航路。混战中,阿费自始至终很沉静地抽着烟斗,透过麦克沃特一侧的窗户,满心好奇地在一旁观战,颇显得泰然自若,仿佛这场战争发生在千里之外,于他无丝毫的影响。

    阿费对联谊会活动一向是很热衷的,什么事都喜欢领个头,对校友联欢活动从来都是尽心尽力。他头脑极单纯,因此,无所畏惧。约塞连倒是极有头脑,所以就顾虑重重。遭炮火袭击时,约塞连并没有像胆小的耗子那样,擅自离弃岗位,急匆匆地从爬行过道逃出去。

    他之所以没这么做,唯一的原因就是他不愿把飞离目标区时采取的规避动作托付给别的什么人。这世上还没有别的什么人可以让他放心地委以如此的重任。而在他的熟人当中,没有哪一个人会像他那么胆小。约塞连是飞行大队最出色的规避动作能手,但这一点就连他自己也说不清究竟是什么原因。

    规避动作,并没有一套固定的程序。要的便是恐惧。这种恐惧心理在约塞连身上算是发挥到了极点。较之奥尔或亨格利-乔,他的胆量要小得多,甚至比邓巴还要小。邓巴早已是听天由命,觉得自己总有一天非死不可。约塞连并没有那么悲观,每次执行任务,只要一扔完炸弹,他便疯狂逃命,一边对麦克沃特死命吼叫:“使劲!使劲!使劲!使劲!你这狗狼养的,快使劲!”而且对麦克沃特他一向是恨之入骨,好像他们在空中执行任务,遭陌生人的轰炸,全都是麦克沃特的过错。飞机上,除他俩之外,其他任何人都禁用对讲机,只有那次去阿维尼翁执行任务是个例外。当时,一片混乱,着实让人痛心,多布斯在半空中发了疯,哭得很伤心,一个劲地喊救命。

    “救救他,救救他,”多布斯哭着说,“救救他,救救他。”

    “救救谁?救救谁?”约塞连把耳机插头重新插入内部通话系统后,高声问道。这之前,多布斯抢过赫普尔手里的操纵杆,随着一阵震耳欲聋的响声,飞机突然俯冲下去,大伙儿全部给吓傻了,一个个呆若木鸡。约塞连的耳机插头由于剧震脱离了内部通话系统,他自己的头像是被什么东西死死粘贴在机舱的顶端,无法动弹。赫普尔又及时救了他们。他拼命夺回了多布斯手里的操纵杆,飞机几乎又是突然进入了平飞,重新飞回到他们刚刚逃脱的那一片猛烈的震耳欲聋的高射炮火之中。啊,上帝!啊,上帝!啊,上帝!约塞连默默地祈祷,他依旧头贴在机头的顶端,像是悬在空中,无法动弹。

    “轰炸员,轰炸员,”约塞连通过对讲机问话时,多布斯哭着答道,“他没有回话,他没有回话;快救救轰炸员,快救救轰炸员。”

    “我就是轰炸员,”约塞连叫喊着答道,“我就是轰炸员。我一切正常。我一切正常。”

    “那就快救救他,快救救他,”多布斯哀求道。

    这时,斯诺登正奄奄一息地躺在尾舱里
