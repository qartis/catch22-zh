\chapter{看什么都是两个图像的士兵}
 
    约塞连身体非常健康,这得归功于体育锻炼、新鲜空气、伙伴的精诚合作以及他所具有的良好的运动家的道德风范。可是自从他想到进医院这一主意以后,那就意味着他得远离这一切。一天下午,当洛厄里基地的体育教官命令所有人员原地解散做健美体操的时候,士兵约塞连却去了医疗所,他报告说他的右腹部位有些疼痛。

    “拍拍它,”正在玩纵横填字游戏的值班医生对他说。

    “我们不能叫他拍,”一名下士说,“对于腹部疾病刚刚出台了一条新规定。我们得把病人留下来观察五天,因为他们其中有许多人在我们叫他们拍打过腹部之后正慢慢地死去。”

    “好吧,”医生咕哝道,“把他留下来观察五天,然后再让他拍。”

    他们把约塞连的衣服拿走了,让他住进一间病房。病房里没有人在他附近打呼噜,他很高兴。第二天早晨,一位年轻的英国实习医生匆匆走进来询问他的肝脏情况,他实际上给了约塞连很大的帮助。

    “我想是我的阑尾疼,”约塞连对他说。

    “阑尾疼有什么用,”那英国人洋洋自得地以专家的口气断言道,“如果是你的阑尾出了毛病,我们可以把它割了,很快就可以让你回到战斗岗位上去。但是要是你来跟我们说肝有问题,那倒可以糊弄我们几个星期。你知道,肝对我们来说可是个摸不着边际的、令人讨厌的神密玩意儿。你如果吃过动物肝脏,就明白我的意思了。我们今天已经相当肯定,肝是存在的,而且当它按照正常的情况运行时,我们对它的功能也比较了解。超出这一范围,我们真的是一无所知了。说到底,肝究竟是怎么回事?比如说,我的父亲死于肝癌,可直到临死前,他一生中从未生过一天病,从未感到过有半点的疼痛。从某种意义上说,那太便宜他了,因为我恨我的父亲。要知道,他把我母亲当成了泄欲工具。”

    “一个英国医官来这儿值勤做什么?”约塞连想弄明白。

    那个医官笑了起来。“我明天早晨来看你时把一切都告诉你。

    把那个该死的冰袋扔掉,要不你会得肺炎死掉的。”

    约塞连再也没见到他。那是有关这所医院里所有医生的有趣的事情之一。他再也没有见过他们中间的任何一个。他们来去匆匆,从此消失了。第二天代替那个英国实习医生的是一组他以前从未见过的医生,他们问他有关他阑尾的情况。

    “我的阑尾没有问题,”约塞连告诉他们说,“昨天的医生说我的肝脏有问题。”

    “也许是他的肝脏有问题,”那个负责的白头发的医官答道,“他的血球指数多少?”

    “他还没有做过血球计算。”

    “立即给他做一个。像他这种情形的病人我们不能冒险。万一他死掉了,我们得有理由为自己辩护。”他在带夹子的书写板上做了个记号,然后对约塞连说:“在此期间,把那个冰袋一直放在上面,这很重要。”

    “我没有冰袋好放在上面。”

    “那么,找一个吧。这附近什么地方一定有个冰袋。假如疼痛变得不能忍受,告诉我们。”

    到第十天时,又来了一组医生,他们给约塞连带来了坏消息:

    他身体极为健康,必须出院。在此关键时刻,走道对面的一个病人开始看什么东西都是两个图像,这可救了约塞连。那个病人未作任何说明,突然坐在床上大叫起来。

    “我看什么东西都是两个图像。”

    一名护士尖叫起来,还有一名护理员晕了过去。医生从四面八方跑过来,有的拿着针,有的拿着灯,还有的拿着试管、橡皮槌和振动金属叉。他们又陆续用车子推来了更多的精密而又复杂的器械。

    就这一个病号,不够大伙分的,于是那些专家便排成一行,一个接一个地轮着给他诊治。一个个火气还大得很,常常是站在后面的人不客气地大声朝前面的人嚷嚷,催他们快点,给排在后面的人也留点机会。不久,一个长着大脑门,眼睛上戴着一副角质边框眼镜的上校得出了诊断结论。

    “是脑膜炎,”他以强调的语气喊道,一边挥手让别人回去。“虽然天晓得没有丝毫的理由这么认为。”

    “那你为什么说是脑膜炎?”一个少校带着讥笑的口吻问道。

    “为什么不是,比如说,急性肾炎。”

    “因为我是个脑膜炎医生,而不是个急性肾炎医生,这就是原因,”上校反驳说,“我可不打算就这么一声不响地将他拱手送给你们这些摆弄肾脏的家伙。我可是第一个到的。”

    最后,所有的医生意见都一致了。他们一致认为他们不清楚那个看见重影的士兵出了什么毛病,于是,他们顺走廊把他推进了一间病房,并将原病房里的其他人隔离十四天。

 


    感恩节到了,约塞连仍呆在医院里。感恩节过得很平静,没有出任何乱子。唯一不好的事情是晚餐火鸡,甚至火鸡也相当不错。

    这是他过过的最平静的感恩节,于是他立下了神圣的誓言:以后每年都要在与世隔绝的医院病房里过感恩节。他第二年就打破了他的神圣誓言,这一年他是在一家旅馆的客房里过的节。那天,他与沙伊斯科普夫中尉的太太进行了学者式的谈话。沙伊斯科普夫中尉太太戴着多丽-达兹的身份识别牌。尽管她同约塞连一样不太相信上帝,但却像老婆教训丈夫似的口口声声责怪他对感恩节玩世不恭、毫无感情。

    “我可能和你一样是个无神论者,”她以自夸的口气推测道,“但即便如此,我也感到我们都有许多事情需要感谢上帝,而且我们表现这一点也不应该感到羞耻。”

    “你举个例子,说说有什么事情值得我表示感谢,”约塞连兴趣索然地以挑战的口气说道。

    “这个——”沙伊斯科普夫中尉太太一时语塞,停了一会儿,犹豫不决地陷入了沉思。“为我。”

    “咳,得了吧,”他嘲弄道。

    她惊讶地扬起了双眉,问道:“你难道不为我而感谢上帝吗?”

    她气冲冲地皱起眉头,自尊心受到了伤害。“我并不是非要跟你过夜不可,这你知道,”她摆出一副高贵的神气冷冰冰地对他说,“我丈夫有整整一中队的航空军校学员,他们就算是为了增加一点刺激也会非常高兴同他们队长的太太过夜的。”

    约塞连决定换个话题。“你在变换话题嘛,”他很策略地指出来。“我可以打睹说,对于你能列出的需要感谢的每一件事,我都能举出两件使人感到痛苦的事情。”

    “你得到了我应该表示感谢,”她坚持说。

    “是的,宝贝。可是我又非常难过,因为我再也不能跟多丽-达兹好了,也不能跟我这短短的一生中将遇见并想要的成百上千的其他姑娘和女人好了,就连跟她们睡一觉都不可能。”

    “你身体健康,应该表示感谢。”

    “你不能那样一直保持健康,应该感到痛苦。”

    “你还活着,应该感到高兴。”

    “你将会死,为此而怒气冲冲。”

    “事情可能更糟,”她喊道。

    “它们也许好上千倍,”他情绪热烈地答道。

    “你只举出一件事情,”她抗议说,“你刚才说你能举出两件。”

    “别跟我说上帝的工作是神秘的,”约塞连不顾她的反对,连珠炮似地继续说道,“上帝没有什么特别神秘的地方。他根本没在工作。他在玩。要不就是他把我们全忘了。那就是你们这些人所说的上帝——一个土佬儿,一个笨手笨脚、笨头笨脑、自命不凡、粗野愚昧的乡巴佬。天啊,你对一个把像粘痰和龋齿这样的现象都必须包含在他神圣的造物体系之中的上帝能有多少尊敬呢?当他剥夺了老年人的大小便自控能力时,他那扭曲、邪恶、肮脏的大脑里究竟是怎么想的呢?他到底为什么要创造出疼痛来?”

    “疼痛?”沙伊斯科普夫中尉太太一下抓住这个词,露出得胜者的神态。“疼痛是个有用的病症,疼痛警告我们:身体有了危险。”
 


    “那么危险是谁创造出来的呢?”约塞连问道。他嘲笑说:“哦,他用疼痛警告我们,真是大慈大悲啊!他为什么不能用只门铃,或用他天上的一个唱诗班来通知我们呢?他也可以在每个人的额头正中间安一个红蓝霓虹灯装置嘛。这种事情任何一个地道的自动唱机制造商都能做得到。他为什么不能?”

    “人们额头中间装上霓虹灯管四处走动,那样子看起来肯定很丑。”

    “他们疼得扭动身体或被吗啡弄得呆头呆脑看起来就肯定漂亮吗?真是个制造大错误的不朽的罪人!你想想他有的是机会和权力去认认真真做事,再看看他搞的这个乱七八糟、丑陋不堪的局面,他的无能几乎让人吃惊。显然他从没有见到过工资单。唉,没有一个有自尊心的商人会雇用像他这样的笨蛋,哪怕雇他去做个发货员也不会。”

    沙伊斯科普夫中尉太太简直不相信自己的耳朵,脸色变得苍白,害怕地直向他做媚眼。“你最好别像那样谈论上帝,宝贝,”她用略带敌意的责备口气轻声警告他说,“他也许会惩罚你的。”

    “他难道惩罚得我还不够吗?”约塞连气呼呼地咕噜道,“嗨,我们不能让他做了错事就这么放过他。哦,不能,他给我们带来这么多苦难,我们不能让他逍遥法外。总有一天我会要他偿还的。我知道是哪一天。就是世界末日那天。对,那天我会离他很近,可以伸出手去抓住那个小乡巴佬的脖子,然后——”

    “住口!住口!”沙伊斯科普夫中尉太太突然尖叫起来,开始用她的两只拳头朝他的脑袋四周乱打一气。“你住口!”
 


    约塞连举起一只胳膊护着头,而她却在一阵狂怒中冲着他乱打一阵。过了片刻,他果断地抓住她的两只手腕,慢慢地使她坐回到床上去。“你到底出什么鬼这么激动不安?”他用后悔但又快活的口气疑惑不解地问她。“我以为你不信上帝。”

    “我是不信。”她抽泣着,突然放声大哭起来。“但是我不相信的上帝是个好上帝,是个公正的上帝,是个仁慈的上帝。他可不像你污蔑的那样是个卑鄙愚蠢的上帝。”

    约塞连笑了起来,松开她的双臂。“咱们两人之间应多一点宗教自由,”他彬彬有礼地建议道,“你不信你想信的上帝,我也不会信我想信的上帝。这样行了吧?”

    那是他能记得的过的最荒唐的感恩节。他的思绪又回到了前一年在医院里度过的十四天平静的与世隔离的生活。但即使那段田园生活也是以悲剧结束的:隔离期满时他的身体仍旧很好,于是他们再次告诉他,他得出院上前线。约塞连听到这个坏消息后,坐在床上喊起来:

    “我看什么东西都是两个图像!”

    病房里又是一片混乱。专家们从四面八方奔跑过来,把他围在中间进行仔细检查;他们围得那样紧,他都能感觉到从不同鼻孔里呼出的湿呼呼的气息喷到他身体的不同部位,怪难受的。他们用细微的光线来检查他的眼睛和耳朵,用橡皮槌和振动叉敲他的双腿和双脚,从他的血管里抽血,并随手拿起手边的东西,举到他视力所及之处让他看。

    这帮医生的头头举止庄重,细心体贴,颇有绅士风度。他在约塞连的正前方举起一只手指,问道:“你看见有几只手指?”

    “两只,”约塞连答道。

    “现在你看到几只?”医生伸出两只手指问道。

    “两只,”约塞连回答说。

    “那么现在几只?”医生问道,一只手指也没伸出来。

    “两只,”约塞连说。

    那个医生满脸堆笑。“啊,他没做假,”他兴高采烈他说道,“他真的看什么都是两个图像。”

    他们把约塞连放在担架车上,推到另外那个看东西有重影的士兵住的房间,并把病房里所有其他的人再隔离十四天。

    “我看什么东西都是两个图像!”当他们把约塞连推进病房时,那个看什么都是两个图像的士兵叫喊道。

    “我看什么东西都是两个图像!”约塞连用同样高的嗓门朝他喊道,同时偷偷地朝他眨眨眼。

    “有两道墙!有两道墙!”那个士兵嚷着,“把墙往后移一移。”

    “有两道墙!有两道墙!”约塞连也喊道,“把墙往后移一移。”

    其中一个医生假装把墙往后推去。“这样行了吗?”

    那个看什么东西都是两个图像的士兵无力地点了点头,又在床上睡下了。约塞连也无力地点了点头,以极其谦卑和钦佩的眼神注视着他这位室友。他知道在他面前的是位大师。他这位天才的室友显然是个值得学习和竭力仿效的人物。那天晚上,他那位天才的室友死掉了,约塞连断定自己跟着他已经走得够远的了。

    “我看什么东西只有一个图像啦!”他赶快喊道。

    又一组医生带着各种仪器噔噔噔地奔到他的病床旁边,来查看是否属实。

    “你看见几只手指?”带队医生伸出一只手指问道。

    “一只。”

    医生伸出两只手指。“现在你看见几只手指?”

    “一只。”

    医生伸出十只手指。“现在几只?”

    “一只。”

    带队医生诧异地转过脸望着其他医生。“他真的看什么都是一个图像!”他感叹道,“我们把他治得好多了。”

    “而且还很及时,”另一个医生评论说。这个医生后来与约塞连单独呆了一会。他与约塞连性格相似。他个头挺高,长得像只鱼雷似的,一嘴棕色胡子好久没有剃过了;衬衫口袋里装着一包香烟,靠在墙上漫不经心地一支接着一支地抽着。“有几个亲戚上这儿看你们来了。哦,别担心,”他笑着补充说,“不是你的亲戚。是那个死了的小伙子的母亲、父亲和兄弟。他们大老远地从纽约赶来看望一个快要死的士兵,而你则是我们手边现成的一个。”

    “你在说什么呀?”约塞连满腹狐疑地问道,“我可不是快要死的。”

    “你当然要死的。我们大家都要死的。你以为你还能往哪里跑?”

    “他们可不是来看我的,”约塞连反驳说,“他们来看他们的儿子。”

    “他们能看到什么人就只好看什么人了。对我们来说,反正是快要死的小伙子,好歹都一样。对一个科学家而言,所有快要死的小伙子一律平等。我给你提个建议,如果你让他们进来看你几分钟,我就不把你一直在撒谎说你肝有毛病的事告诉任何人。”

    约塞连退得离他更远点。“你知道那件事?”

    “我当然知道。请相信我们。”那医生和蔼地轻声笑了笑,然后又点燃了一支烟。“每次一有机会你就不断地拧那些护士的xx头,怎么能让人相信你肝有毛病呢?如果你想让人相信你有肝病,你得不沾女色才行。”

    “付那么大的代价仅仅为了活命。既然你知道我在装假,为什么不告发我?”

    “我干吗要告发你?”医生有点惊讶地问道,“我们大家都在一同做假。在求生的道路上,只要某个同伙也愿意帮我,我总是乐意帮他一把的。这些人走了这么远的路,我不愿让他们失望。我很同情老人。”

    “但是他们是来看他们的儿子的。”

    “他们来得太晚了。也许他们根本看不出你不是他们的儿子。”

    “说不准他们会哭起来呢。”

    “他们很可能会哭。那是他们来的原因之一。我在门外听着,要是哭得不可收拾了,我就来制止他们。”

    “这一切听起来都有点疯了。”约塞连沉思着。“但不管怎样,他们干吗要看着他们的儿子断气呢?”

    “我一直也没能琢磨出个所以然来,”医生承认说,“不过他们总是这样。哎,你说怎么样?你需要做的就是在那儿躺几分钟,装得像要死了似的。这个要求不太过分吧?”

    “好吧。”约塞连让步了。“但只能是几分钟,而且你保证等在门外。”他对这个角色产生了兴趣。“喂,我说,干吗不用绷带把我裹起来,那样效果不是更好吗?”

    “这听起来倒是个挺好的主意。”医生听了直鼓掌。

    他们在约塞连身上裹了一卷绷带。一帮护理员给两扇窗户都装上了棕褐色的窗帘,并放下窗帘,使房间里显得黑乎乎、阴沉沉的。约塞连建议放些花,医生马上派了一个护理员出去弄来两小束快要凋谢的花。花散发出刺鼻的、令人作呕的气味。当一切准备停当之后,他们让约塞连回到床上躺下来。然后他们让探访者进来了。

    这几位探访者带着歉意的眼神,蹑手蹑脚、战战兢兢地走进病房,就像是未经邀请闯入人家的不速之客一样。先进屋的是悲痛欲绝的母亲和父亲,然后是那位满面怒容的兄弟,他是个身材矮胖、虎背熊腰的水手。这对夫妇表情呆板地肩并肩走进病房,就像刚从一幅挂在墙上的既熟悉又神秘的结婚周年纪念银板照片上走下来似的。他俩身材矮小,形容枯槁但却颇有自尊心。他们虽穿着深色的旧衣服,但身体却似钢筋铁骨。那女人有一张椭圆形的长脸,呈红棕色,带着沉思的表情,一头粗黑的头发已经泛白,从头正中截然分开,简单地梳向脑后,披在后颈上,没有卷曲、波纹或带什么装饰。她既伤心而又心情沉重,满是皱纹的嘴唇紧紧地抿着。那位父亲直挺挺地站在那里,穿着一套配有垫肩的双排扣西装,西装太小,看起来有点滑稽。他个子不高,但粗壮结实,满是皱纹的脸上蓄着两撇漂亮的向上翘起的小胡子。他的两只眼睛淌着粘液,眼角布满皱纹。他窘迫地站在那儿,一双强壮的劳动者的手抓着他的黑毡软呢帽的帽檐,搁在西装翻领前,那样子看起来又尴尬又凄惨。贫穷和辛劳使他俩过早地衰老了。那位兄弟像是要找人打架似的。他那白色的圆帽傲慢地斜扣在头上,双手握成拳头,带着一种因受到伤害而产生的好斗神色怒视着病房中的一切。
 


    这三个人小心翼翼地朝前走来。他们紧挨在一起,像去参加葬礼似的,蹑手蹑脚,几乎步伐一致地一步一步地往前挪,直到走到床边才停下来,站在那儿低着头盯着约塞连。接下来是一阵令人厌恶、使人痛苦的沉默。这沉默像是要永远持续下去似的。最后,约塞连再也不能忍受了,便清了清嗓子。老头儿终于开口说话了。

    “他看起来挺糟糕,”他说。

    “他病得挺重,爸。”

    “吉乌塞普,”母亲喊道。她已经在一张椅子上坐了下来,青筋凸起的手指紧紧地抓着膝盖。

    “我叫约塞连,”约塞连说道。

    “他叫约塞连,妈。约塞连,你认不得我了吗?我是你哥哥约翰。

    你不认识我是谁了吗?”

    “我当然认得。你是我哥哥约翰。”

    “他真的认得出我呢!爸,他知道我是谁。约塞连,这是爸爸。跟爸爸说声好。”

    “你好,爸爸,”约塞连说。

    “你好,吉乌塞普。”

    “他叫约塞连,爸。”

    “他那样子太可怕了,我实在是很难过,”父亲说。

    “他病得挺重,爸。医生说他要死了。”

    “我不知道要不要信医生的话,”父亲说,“你知道那些家伙说话是多么不可信。”

    “吉乌塞普,”母亲又喊道,声音虽低,但却因为痛苦而变了调。

    “他叫约塞连,妈。她现在记性不大好了,在这儿他们待你怎么样,兄弟?他们待你还好吧?”

    “挺好,”约塞连告诉他说。

    “那就好。可别让这儿的任何人欺负你。哪怕你是个意大利人,你也同这里的任何人都一样。你还有你的权利嘛。”

    约塞连有些胆怯,便闭上了眼睛,这样他就不必再看着他兄弟约翰了。他开始感到恶心。

    “瞧,他现在这个样子多怕人,”父亲说。

    “吉乌塞普,”母亲喊道。

    “妈,他叫约塞连。”那兄弟不耐烦地打断她。“你难道记不住吗?”

    “没关系,”约塞连打断他说,“她想叫我吉乌塞普就让她叫吧。”

    “吉乌塞普,”她又叫了他一声。

    “别担心,约塞连,”兄弟安慰他说,“一切都会好起来的。”

    “别担心,妈,”约塞连说,“一切都会好起来的。”

    “你有神父吗?”兄弟想知道。

    “有的,”约塞连撒谎说,禁不住又一次畏缩起来。

    “那就好,”兄弟说,“只要你需要的东西都有就好。我们大老远从纽约赶来。原来还担心不能及时赶到呢。”

    “及时赶来干什么?”

    “在你死前见你一面呗。”

    “那又有什么区别?”

    “我们不想让你孤零零地死去。”

    “那又有什么区别?”

    “他一定是神志不清了,”兄弟说,“他总是翻来覆去地说同一句话。”

    “这事情真是滑稽,”老头儿说道,“我一直以为他的名字叫吉乌塞普,可现在我发现他的名字叫约塞连。真是太滑稽了。”

    “妈,使他高兴一点,”兄弟劝她说,”说点什么让他高兴高兴。”

    “吉乌塞普。”

    “不是吉乌塞普,妈。是约塞连。”

    “那有什么区别?”母亲用同样悲伤的调子,头也不抬地答道,“反正他就要死了。”

    她肿胀的双眼老泪纵横,开始哭起来,身体在椅子里缓慢地前后晃动着,两只手平躺在膝盖上,就像两只死去的飞蛾。约塞连担心她会大哭起来。父亲和兄弟也开始哭起来。约塞连突然想起来他们为什么都在哭,于是他也开始哭起来。这时候,一名约塞连从未见过的医生走进病房,很有礼貌地对来访者说他们该走了。父亲挺直身体,很正规地道了个别。

    “吉乌塞普,”他说。

    “约塞连,”儿子更正说。

    “约塞连,”父亲说。

    “吉乌塞普,”约塞连更正说。

    “你很快就要死了。”

    约塞连又开始哭起来。医生从房间的后部狠狠地朝他瞪了一眼,于是约塞连便止住了哭。

    父亲低下头神情庄重地接着说:“当你向天国里的那人汇报时,我想要你替我给他捎句话,告诉他让人年轻时就死掉是不对的。我是当真的。跟他说,要是人非死不可,得让他们老了再死。我要你把这话告诉他。我想他不一定知道这事不对,因为他应该是大慈大悲的,而这种事已经延续了好长好长时间了。行吗?”

    “别让上边的人欺负你,”那兄弟告诫他说,“哪怕你是意大利人,你也不比天堂里的任何人差。”

    “穿暖和些,”母亲说道,仿佛她知道天堂里的事情
