\chapter{多布斯}
 
    麦克沃特没有疯,麦克沃特执行任务去了。约塞连也执行了飞行任务,走路时仍然一瘸一拐的,又飞了两次之后,约塞连听说还要到博洛尼亚去执行一次飞行任务,感到生命受到了威胁,便在一个温暖的午后坚定地跛着脚走进多布斯的帐篷,把一个手指头放到嘴边,说了声“嘘!”

    “你干吗要这样?”基德-桑普森问道。他正在仔细地读着一本破旧的连环漫画册,一边用门牙剥开一只橘子的皮。“他还什么都没说呢。”

    约塞连把大拇指朝自己背后的帐篷出口处一指,对基德-桑音森说:“滚出去。”

    基德-桑普森理解地扬了扬他那淡黄的眉毛,顺从地起身往外走。他朝自己那垂到唇边的焦黄的小胡子吹了四声口哨,跨上那辆被撞得凹凸不平的绿色摩托车,向山里飞驰而去。这辆旧摩托车是他几个月前买的二手货。约塞连一直等到摩托车最后的微弱声响在远处完全消失掉。帐篷里的情况不大对劲,收拾得过于整洁了。多布斯抽着一支粗粗的雪茄,好奇地打量着他,既然约塞连已经拿定主意要大胆行事,他感到害怕得要命。

    “好吧,”他说,“我们去杀掉卡思卡特上校吧。我们俩一块干。”

    多布斯大惊失色,噌地一下从行军床上蹦了起来。“嘘!”他吼叫道,“杀死卡思卡特上校?你在说什么呀?”

    “你小声点,该死的,”约塞连咆哮着说,“全岛的人都听见了。

    你那枝枪还在吗?”

    “你是疯了还是怎么啦?”多布斯大声说,“我为什么要杀死卡思卡特上校呢?”

    “为什么?”约塞连满脸疑惑地瞪着多布斯。“为什么?这是你的主意,不是吗?不是你到医院去叫我来干的吗?”

    多布斯淡淡一笑,“那时候我只完成了五十八次飞行任务,”他美美地吐了一口雪茄烟,解释道,“可现在我行李都捆好啦,就等着回国了,我已经完成了我的六十次飞行任务了。”

    “那又怎么样?”约塞连反驳道,“他还会再增加飞行任务的次数的。”

    “也许这次他不会。”

    “他一直在增加次数。你他妈的怎么啦,多布斯?问问亨格利-乔,他捆好多少次行李了。”

    “我得再等一等,看看会发生什么事情,”多布斯执拗地坚持道,“我已经离开了战斗岗位,现在要是再搀和到这种事情当中去,那可是真疯了。”他轻轻弹去雪茄的烟灰。“不,要我说呀,”他劝道,“你先像我们这样完成你的六十次飞行任务,然后看看情况再决定。”

    约塞连克制着朝他眼睛啐一口唾沫的冲动。“我也许飞不完六十次就送命了,”他用干巴巴的悲观腔调哄骗多布斯说,“这儿到处都在传说,他又去主动请战,要求再派我们大队去轰炸博洛尼亚。”

    “这不过是谣传,”多布斯带着自命不凡的神情向他指出,“你不要听到什么谣传都相信。”

    “你别对我指手划脚好不好?”

    “你为什么不去和奥尔谈谈呢?”多布斯建议道,“上星期第二次飞到阿维尼翁执行任务时,奥尔又被击落到水里了。也许他很生气,正想干掉他呢。”

    “奥尔没有头脑,他才不会生气呢。”

    约塞连还在医院里时,奥尔又一次被击落到水里。他驾着受伤的飞机缓缓滑落到马赛港外明镜般清澈的碧波上。他的技术棒极了,机组的六个成员连一根毫毛也没伤着。海水还在飞机周围翻腾着蓝白相间的浪花时,飞机前后舱的应急出口便迅速打开,穿着松软的橙色飞行救生衣的机组人员尽可能快地爬了出来。他们的救生衣没能充气,软瘪瘪地垂挂在他们的脖子上,系在他们的腰间,丝毫不起作用。救生衣没能充气,是因为米洛从充气膛里取走了二氧化碳双管充气筒。他拿它们去做草莓和菠萝冰淇淋苏打,供应给军官食堂。在充气膛里,他贴上液印的纸条代替充气筒,上面印着“有益于M&M辛迪加联合体就是有益于国家。”奥尔是最后一个从下沉的飞机里蹦出来的。

    “你要是看见当时他那副样子就好了!”奈特中士向约塞连讲述事情经过时笑得震天响。“这是你这辈子见过的最他妈滑稽可笑的事。那些救生衣全部不管用了,就因为米洛偷走了二氧化碳,给你们这些在军官食堂就餐的家伙做冰淇淋苏打去了。不过结果证明,那还不算太糟。我们中间只有一个人不会游泳,我们把这家伙抬起来放到救生筏里。当我们还都站在飞机上时,奥尔就用绳子系着这只救生筏,把它贴着机身下降到海面上去了。那个古怪的小家伙干这种事情的确很在行。后来,另一只救生筏绳子松开漂走了。

 


    所以我们六个人最后只好挤在一只小筏上,胳膊肘碰胳膊肘,大腿紧挨大腿,谁也不能动弹一下,否则就会把你旁边的那个家伙挤到水里去。我们离开飞机大约只有二秒钟,飞机就沉下去了,把我们几个人孤零零地甩在救生筏上。我们随即打开救生衣充气膛的螺帽,看看里面他妈的出了什么毛病,这才发现米洛那些向我们宣称凡有益于他就有益于我们其余人的该死的纸条。这个狗杂种!他妈的,我们大伙全都在诅咒他,只有你那个伙计奥尔除外,他一直咧嘴笑着,好像他觉得有益于米洛的也可能真的有益于我们其余的人。

    “我发誓,你真应该看看他当时那副模样,他像个船长坐在救生筏边沿上,我们其余的人全都望着他,等着他告诉我们该怎么办。他每隔几秒钟就打摆子似地用手拍拍大腿说:‘现在没事了,没事了。’接着像个古怪的小疯子似的格格傻笑一阵后,他又说:‘现在没事了,没事了。’然后又像个古怪的小疯子似的格格傻笑一阵。

    他看上去活脱脱一个白痴。不过,亏得只顾看着他,我们在开头几分钟里才没有给吓垮掉。那个时候,大浪一个接一个朝我们的救生筏打过来,有时甚至把我们中的几个卷到海里,我们得赶忙爬回到筏里去,要不然下一个浪打过来就会把我们冲得更远。那真是滑稽透顶,我们就这么不断地掉下去又不断地爬上来。我们让那个不会游泳的家伙平躺在救生筏的中央,可即使在那个地方,他也差点被淹死,因为灌到救生筏里的水很深,不断地泼洒到他的脸上。嘿,太惊险了!

    “后来,奥尔动手打开救生筏的贮藏舱,滑稽事真正开始了。开头,他找到一盒巧克力,分发给我们大家,于是我们就坐在那儿一边吃又湿又咸的巧克力,一边让海浪一次次地把我们从救生筏上卷到水里去。接着,他找到一些固体牛肉汤料和几只铝杯子,他就给我们做牛肉汤喝。后来,他又找到些茶叶。真的,他沏了茶!我们屁股坐在水里,浑身湿透,他却请我们喝茶,你能想象出这种情景吗?当时我笑得太厉害了,一下子从救生筏上掉到水里去了。我们全都笑个不停,他却一本正经,除了每隔一会疯疯癫癫地咧开嘴格格傻笑一阵。真是个怪人!他找到什么用什么。他找到一些驱鲨剂,立刻全洒到海水里,他找到一些标识颜料,也马上扔到水里。

 


    接下来他找到一根钓鱼线和一块干鱼饵,顿时满脸放光,就好像当我们正要葬身大海,或者当德国鬼子从斯培西亚派船出来抓我们或者用机关枪扫射我们时,我们的海空救援艇及时赶到救出了我们似的。一转眼工夫,奥尔就把钓鱼线甩到水里钓起鱼来。他高兴得像只云雀。我问他:‘中尉,你指望钓到什么?’‘鳕鱼,’他告诉我。

    他的确指望能钓到鳕鱼。不过幸好他没有钓到,因为要是真的钓到了,他会把鳕鱼生吃了,还会迫着我们也生吃,因为他找到一本小书,那书上说生吃鳕鱼没关系。

    “接下来,他找到一把蓝色的小桨,小得和纸杯冰淇淋里的小勺一般大。真的,他就用这把桨划了起来。想靠这么根小木棍划动我们这条总共重九百磅的救生筏,你能想象得出来吗?再后来,他找到一个小小的罗盘和一张大大的防水地图,他把地图摊开在膝盖上,又把罗盘放在地图上。他坐在那里,背后拖着装有鱼饵的钓鱼线,膝盖上铺着地图,地图上压着罗盘。他使尽全身力气划着那把蓝色的小桨,好像他正全速划向马略卡岛。真他妈的!他就这样划了大约半个小时,直到救援艇来把我们接走。”

    对马略卡岛奈特中士知道得一清二楚,奥尔也一样,因为约塞连常常对他们谈起西班牙、瑞士和瑞典境内这样一些避难地的情况。美国飞行员只要飞到这些地方去,就会被拘留到战争结束,而且生活条件极其舒适奢侈。在拘留问题上,约塞连是中队里的头号权威。每回飞往意大利最北部执行任务时,他总是谋划着如何以紧急情况为借口飞到瑞士去。当然,他想去的地方是瑞典。瑞典人智商高。在那儿他可以脱得光溜溜的同那些低声细语、半推半就的漂亮女郎一块游泳,并且生下一大群快活散漫的小约塞连来。在瑞典,没有人会耻笑他的这些私生子。而且,他们一落地,国家就会担负起供养他们的责任,直到他们长大成人。但是,瑞典太远了,很难到达。约塞连只好等着飞越意大利境内的阿尔卑斯山时高射炮火把他飞机的一个引擎打掉,这样他就有理由飞往瑞士了。他甚至不想告诉他的驾驶员他要把飞机带到哪里去。约塞连常常想找一个他信得过的驾驶员合伙干。他们可以假称引擎受损,然后来个机腹着陆,毁掉说谎的证据。可是,他唯一真正信得过的驾驶员只有麦克沃特。那家伙无论走到哪儿都是一副乐呵呵的样子,仍然喜欢做低空俯冲来寻开心,擦着约塞连的帐篷飞过去;紧贴着海滩游泳者的头顶盘旋,飞机推进器喷出的强大气流在海里划出一道道黑浪,飞机过处,浪花飞溅,长达数秒钟。
 


    多布斯和亨格利-乔都不能考虑,奥尔也不行。当约塞连遭到多布斯的拒绝,心情绝望、一瘸一拐地走回到自己的帐篷时,奥尔又在摆弄那个炉子阀门了。这炉子是奥尔用一只铁壳油桶倒过头来改装而成的。他把炉子摆在地中央,水泥地面平坦光滑,是他铺修过的。他双腿跪在地上,正起劲地干着呢。约塞连竭力不去注意他,瘸着腿疲倦地走到自己的行军床前坐下来,吃力地发出一声长长的叹息。他前额上的汗珠变得冰凉冰凉的。多布斯使他感到沮丧,丹尼卡医生也使他感到沮丧。现在看到了奥尔,他似乎觉得厄运正在逼近,越发沮丧起来。在他的身体内部,各种各样的紧张感一起涌出来刺激着他,他的神经抽搐起来,一只手上的青筋开始突突直跳。

    奥尔转过脸打量着约塞连,两片湿漉漉的嘴唇咧开着;露出两排大龅牙。他把手伸到旁边他自己的床头柜里,取出一瓶温热的啤酒,撬开盖递给约塞连。约塞连啜饮完上面的啤酒泡沫,向后仰起脑袋。奥尔狡诈地望着他,不出声地咧嘴笑着。约塞连谨慎地盯着奥尔。奥尔窃笑了一阵之后,转过身蹲下去继续干活。约塞连紧张了起来。

    “你别摆弄了,”他双手紧握着啤酒瓶,用威胁的口吻请求道,“你别摆弄那炉子了。”

    奥尔平静地格格笑着说:“我快干完了。”

    “不,你没有,你正要开始干。”

    “这是阀门,看见了吗?就快全部装好了。”

    “你很快又要把它拆开。我知道你在干什么,你这混蛋。我已经看你这样干了三百次了。”

    奥尔高兴得浑身直抖动。“我要把这根汽油管漏油的地方补上,”他解释道,“我已经差不多全弄好了,只有一点点地方还渗油。”

    “我实在没法看下去,”约塞连干巴巴地说,“如果你想做一件大东西,那不成问题。可是这阀门是用这么多小零件拼凑起来的,它们那么小,那么无足轻重,我眼下可没有耐性看着你辛辛苦苦地摆弄这些该死的玩意。”

    “它们是小点,可这并不意味着它们无足轻重。”

    “这我不管。”

    “让我再干一回吧。”

    “等我不在这儿的时候你再干吧。你是个不知忧愁的白痴,你根本不理解我的感觉是什么滋味。就在你摆弄那些小玩意时,我出了一些事,这些事我根本无法向你解释。我发现我无法容忍你。我开始恨你。用不了多久,我就会认真考虑把这个瓶子砸到你的脑袋上,或者用那边那把猎刀戳穿你的脖子。你明白吗?”

    奥尔领悟地点点头。“现在我不会再把阀门拆开了。”他说着就动手拆阀门,他用手指费劲地捏着那个小小的装置,缓慢地、不知疲倦地、精益求精地干着。他俯着身子,脸紧贴着地面,一副专心致志、聚精会神的模样,好像他的脑子里什么杂念都没有。
 


    约塞连暗暗地诅咒着他,打定主意不再理睬他。“可你他妈的究竟为什么急着摆弄这炉子呢?”一转眼他又忍不住叫喊起来。“外面还热着呢。过一会儿我们还可能去游泳呢。你为寒冷操什么心呢?”

    “白天越来越短了,”奥尔不动声色地说,“趁着这会儿有空,我打算把这炉子给你装好。等我装好了,你就会有一个全中队最好的炉子。我现在正装着的这个供油控制器会保证这炉子整夜燃烧不灭,这些金属散热片会把整座帐篷烤得暖烘烘的。你睡觉前可以把钢盔盛满了水坐在炉子上,这样你醒来时就有热水洗脸。这不是很好吗?要是你想煮鸡蛋或者烧汤的话,你只要把锅坐在上面,拧大火苗就行了。”

    “你这是什么意思,给我?”约塞连追问道,“你会到哪里去?”

    奥尔忍不住心头一阵快活,矮小的身体突然哆嗦起来。“我不知道,”他大声说道。接着,从他那直打战的两排龅牙中间突然迸发出一串奇特的、颤抖的格格傻笑,好像一阵情感爆发。他满嘴唾沫,边笑边说,声音都变得含糊不清了。“要是他们不断地这样把我击落,我不知道我会到哪里去。”

    约塞连被感动了。“奥尔,你为什么不争取停飞呢?你是有理由的。”

    “我只剩下十八次飞行任务了。”

    “可你几乎每次都被击落。你每次飞上天不是降落到水面上就是强行着陆。”

    “噢,飞行任务我倒不在乎。我觉得它们非常好玩。你不领航飞行时应当试着跟我一块飞几回,就为开开心,嘿嘿。”奥尔满脸堆笑,斜眼瞅着约塞连。

    约塞连避开他的目光。“他们又叫我领航飞行了。”

    “那就等你不领航飞行的时候吧。要是你有头脑的话,你知道你该怎么办吗?你应该直接去找皮尔查德和雷恩,告诉他们说,你要和我一起飞行。”

    “每回飞行都跟你一起被击落吗?这有什么好玩的?”

    “就因为这个你才应该跟我一块飞呢,”奥尔坚持道,“我觉得,就水面降落或强行着陆这方面说,我大概算得上是这儿最优秀的飞行员了。对你来说,这将是很好的练习。”

    “练习这个做什么?”

    “万一你哪一次降落到水面上或者强行着陆的话,这不是很好的练习吗?嘿嘿嘿。”

    “你还能再给我一瓶啤酒吗?”约塞连愁眉不展地问。

    “你要把它砸到我的脑袋上吗?”

    这下约塞连乐了。“就像罗马那所公寓里的那个妓女吗?”

    奥尔淫荡地窃笑着,两个腮帮子高兴地鼓了起来,活像两只酸苹果。“你真的想知道她为什么拿鞋敲我的脑袋吗?”他揶揄道。

    “我已经知道了,”约塞连嘲笑道,“内特利的妓女告诉我的。”

    奥尔像个怪物似的咧嘴一笑。“不,她没告诉你。”

    约塞连为奥尔感到难过。奥尔是那么的矮小丑陋。要是他活下去,谁愿意保护他呢?谁愿意保护一个像奥尔这样热心而单纯的侏儒,使他免遭无赖、朋党以及阿普尔比那样的老牌运动员的欺辱呢?他们这些人全是目空一切、自命不凡、狂妄自大的家伙,一有机会就会把奥尔踩在脚底下。约塞连常常为奥尔担心。谁能替他抵挡憎恶和欺诈,抵挡野心勃勃的家伙和势利刻薄的贵妇人,抵挡谋取暴利者卑劣下流的侮辱,抵挡邻近专卖坏肉的客客气气的屠夫?奥尔是个无忧无虑轻信他人的傻瓜,一头浓密卷曲的杂色头发从中间一分为二。对那些家伙来说,对付他是再容易不过的了。他们会拿走他的钱,强xx他的妻子,冷酷地对待他的孩子。约塞连感到自己心底涌起一股同情的热流。

    奥尔是个古怪的小矮人,是个令人捉摸不透的可爱的侏儒。他心灵猥琐,却身怀无数种宝贵的技艺,这就使得他终生与低收入者为伍。他能够用烙铁把两块木板钉在一起,既不让木板裂缝,又不把钉子砸弯。他会钻孔眼。约塞连住院期间,他在帐篷里搞出不少名堂来。他先在帐篷外面的高台上建起一个油箱,然后在水泥地上连挫带凿,开出一条无可挑剔的槽沟。顺着这条沟,他把一根细长的汽油管贴着地面从外面的油箱一直引到炉子上。他用多余的炸弹零件给壁炉做了几个柴架,并在柴架上堆满了粗壮的次等圆木。
 


    他从一些三流杂志上剪下一些长着硕大Rx房的女人的照片,把它们镶在他用染色木条做成的镜框里,挂到壁炉架上面。奥尔会开油漆筒,会调配油漆,会稀释油漆,还会除掉油漆,他会劈木头,会用尺子测量东西。他知道怎么生火,怎么挖洞。他还有一项本事,那就是用罐头筒和水壶从食堂附近的水箱里运来足够他们俩用的水,他能够一连几小时聚精会神地做一项无足轻重的工作,既不急躁也不厌烦,像根树桩那样不知疲倦,也几乎像树桩那样不吭不响。对于野外生活,他具有非同寻常的知识。而且,他不怕狗,不怕猫,不怕甲虫,不怕飞蛾,还敢吃小鳕鱼、动物内脏之类的东西。

    约塞连烦闷地长叹一声,考虑起要去轰炸博洛尼亚的传闻来。

    奥尔正在拆卸的阀门大约有大拇指那么大小,除了外壳,里面一共有三十六个零件。奥尔小心地把这些零件按类别整整齐齐地排列在地面上。其中有许多零件非常细小,他不得不用两个指甲尖捏住它们,在这细致严密、有条不紊、单调乏味的工作进程中,他从不加快或是放慢速度,仿佛永远不知疲倦,永远不会停下来似的,唯一例外的是,他有时会斜眼瞥一下约塞连,那目光中饱含癫狂和恶作剧的神情。约塞连努力不去看奥尔。他数着那些零件,满以为这样就可以把奥尔从心里摆脱掉。他转过脸去,闭上眼睛,可结果更糟,因为这样一来,他只听到声音,听到那些细微清晰、持续不断、令人恼火的咔哒声以及奥尔的手接触那些轻巧的零件时发出的悉悉声。奥尔有节奏地喘着粗气,发出打鼾般的呼噜声,非常令人讨厌。

    约塞连握着拳头,眼睛盯着那把插在皮套里、挂在那个死掉的人的床上方的骨柄长猎刀。他脑袋里突然冒出拿这刀刺死奥尔的念头。

    这念头一出现;他的紧张情绪随即松弛下来。他觉得这个念头荒谬至极,便认真而专注地胡思乱想起来。他打量着奥尔的后脖颈,想找出他脊椎的大致部位,只要往那个部位很轻地戳上一刀,准能把他杀死。这样一来,他们俩之间许多令人痛苦的严重问题就都迎刃而解了。

    “痛不痛?”就在这个时候,奥尔仿佛出于自卫本能似地问了这么一句。

    约塞连紧盯着他。“什么痛不痛?”

    “你的腿呀。”奥尔发出一声神秘莫测的怪笑。“你还有点瘸。”

    “我想这只是出于习惯。”约塞连松了一口气,呼吸又通畅起来,“也许很快就改掉了。”

    奥尔在地上侧起身,又用一只膝盖撑着跪起来,把脸对着约塞连。他做出一副竭力回忆往事的神情,沉思般地拖长声调问:“你记得那天在罗马打我脑袋的那个妓女吗?”约塞连想起上一回受骗一事,非常恼火,不由得叫了一声,惹得奥尔格格地笑了起来。“我要拿这个妓女跟你做笔交易,你要是能回答我一个问题,我就告诉你那天她为什么拿鞋打我的脑袋。”

    “什么问题?”

    “你有没有跟内特利的女人睡过觉?”

    约塞连吃了一惊,不由得笑了起来。“我?没有。现在告诉我,她为什么拿鞋打你的脑袋。”

    “这不算问题,”奥尔得意洋洋地对他说,“这不过是随便聊聊。

    她装得好像你跟她睡过觉似的。”

    “我没有。她装出一副什么样呢?”

    “她装得好像不喜欢你。”

    “她谁也不喜欢。”

    “她喜欢布莱克上尉,”奥尔提醒他说。

    “那是因为他把她当贱货对待,用这法子谁都能把姑娘勾上手。”

    “她脚脖子上戴着一只只有奴隶才戴的镯子,上面刻着他的名字。”

    “是他让她戴上那玩艺的,他想拿这个气气内特利。”

    “她甚至把从内特利那儿得来的钱给了他一些,”“听着,你到底想向我打听什么?”

    “你有没有跟我的女人睡过觉?”

    “你的女人?谁妈的是你的女人?”

    “就是那个用鞋打我脑袋的妓女。”

    “我跟她睡过几次,”约塞连承认道,“她什么时候成了你的女人?你到底什么意思?”

    “她也不喜欢你。”

    “管她喜不喜欢我,我他妈的干吗要在乎,她喜欢我跟喜欢你的程度差不多。”

    “她有没有拿她的鞋子打过你的脑袋?”

    “奥尔,我累了。你为什么不能让我一个人呆一会呢?”

    “嘻嘻嘻。罗马那个干瘦干瘦的伯爵夫人和她那个干瘦干瘦的儿媳妇怎么样?”奥尔兴致越来越高,便淘气地缠着他问,“你有没有跟她们睡过觉?”

    “唉,我倒希望能跟她们睡觉,”约塞连老老实实地回答道。奥尔的这句话唤起了他的遐想。他习惯性地想象着自己用双手抚摸她们那小巧而又富于肉感的屁股和Rx房时的那种感觉,那真是叫人欲火中烧,神魂颠倒。

    “她们也不喜欢你,”奥尔评论道,“她们喜欢阿费,她们喜欢内特利,可是她们不喜欢你。女人似乎就是不喜欢你。依我看,她们认为你一去就没好事。”

    “女人全是疯子,”约塞连答道。他板着脸等待着奥尔发问,他早已知道奥尔接下来要问什么。

    “你的另一个姑娘怎么样?”奥尔装出一副好奇的沉思神情问,“就是那个胖胖的姑娘,那个秃头的姑娘。你知道,在西西里那一回,这个又胖又秃的姑娘戴着头巾,整夜浑身直冒汗,弄得我们全都跟着受罪。她也疯了吗?”

    “她也不喜欢我吗?”

    “你怎么能去搞一个没有长头发的姑娘呢?”

    “我怎么能知道她没长头发呢?”

    “我知道,”奥尔自夸道,“我一直知道。”

    “你知道她是秃子?”约塞连惊奇地叫起来。

    “不,我知道要是我漏装了一个零件,这个阀门就无法工作,”奥尔回答道。他高兴得红光满面,因为他又捉弄了约塞连一回。

    “你把滚到那边的那个小垫圈递给我好吗?它就在你脚旁边。”

    “不,不在。”

    “在这儿。”奥尔边说边用指甲尖捏起一个小得几乎看不见的东西,举到约塞连面前让他看。“现在我只好再从头开始啦。”

    “你再干的话,我就宰了你。我就在这儿宰了你。”

    “你为什么从来不跟我一块飞呢?”奥尔突然问道,第一次正视着约塞连的脸。“喂,这就是我想要你回答的问题。你为什么从来不跟我一块飞呢?”

    约塞连感到又愧又窘,尴尬地转过身去。“我告诉过你为什么。

    大部分时间里,他们都让我当领航轰炸员。”

    “这不是理由,”奥尔摇头说,“咱们第一次飞到阿维尼翁执行任务后,你去找过皮尔查德和雷恩,告诉他们,你决不想和我一共飞。这才是理由,不对吗?”

    约塞连感到浑身发烧。“不,我没去找过他们,”他抵赖说。

    “不,你找过,”奥尔平静地坚持道,“你请求他们不要派你到由我和多布斯或者赫普尔驾驶的飞机上去,因为你对我们的操纵技术没有信心。皮尔查德和雷恩说,他们不能给你破这个例,因为要是真的那样做了,对那些跟我们一起飞的人就太不公平了。”

    “那又怎么样?”约塞连说,“还不是没有什么区别嘛,对吧?”

    “可他们从来没有逼你跟我一起飞过。”奥尔双膝跪在地上又干起活来。他对约塞连说活时的神情既没有怨恨,也没有责备,却包含着一种含冤负屈的谦卑。他的这副神情叫人看上去越发感到难过,尽管他本人仍然咧嘴窃笑着,好像这种情况很滑稽似的。“你知道,你真的应该跟我一起飞。我是个很优秀的飞行员,我会照顾你的。也许,我会被击落好多次,但这不是我的惜,我飞机上的人从来没有受过伤。是的,长官——如果你有头脑的话,你知道你该怎么做吗?你该立刻去找皮尔查德和雷恩,告诉他们你要求跟我一起飞完你所有的飞行任务。”

    约塞连俯下身去,直盯着奥尔那张交织着各种矛盾情绪、令人费解的面孔。“你是想告诉我什么事吗?”

    “嘿嘿,嘿嘿,”奥尔回答道,“我想告诉你那个大块头姑娘那天为什么用她的鞋打我的脑袋。可你就是不让我说。”

    “告诉我吧。”

    “你愿意跟我一块飞吗?”

    约塞连大笑着摇摇头。“你只会再一次给击落到水里去的。”

    等到真的执行传闻中轰炸博洛尼亚的那次飞行任务时,奥尔的飞机果然又被击落到水里了。当时,天空乌云密布,电闪雷鸣。他驾着只剩下一个引擎的飞机歪歪扭扭、摇摇摆摆地扑通一声落到波涛滚滚风急浪高的海面上。他从飞机里钻出来晚了点,一个人独自上了一只救生筏。那只筏漂流而去,离其他人乘坐的救生筏越来越远。等到海空救援艇冒着狂风骤雨驶来营救他们时,奥尔的救生筏早已无影无踪了。获救人员回到中队时,夜幕已经降临,奥尔仍然没有消息。

    “别担心,”基德-桑普森安慰大家说。他身上仍然裹着救援艇救护人员给他披上的厚毯子和雨衣。“要是他没有在那场暴风雨中淹死的话,他很可能已经被救上来了。那场暴风雨没下多长时间。

    我敢说,他随时都会出现的。”

    约塞连走回自己的帐篷去,等待着奥尔随时出现。他生起炉火,好让自己暖和点,那炉子非常好使,炉火熊熊,烧得旺极了。奥尔终于把供油控制器修好了,要是想调大或者调小炉火,只消拧一下就行。外面正下着小雨,雨点淅淅沥沥地落在帐篷顶上,落在树上,落在地面上。约塞连用罐头筒给奥尔烧好了热汤预备着:可随着时间渐渐过去,他自己把汤全喝了。他又给奥尔煮了几个鸡蛋,可后来也让他自己吃了。接着,他又从应急干粮袋里拿出一整听切达干酪,吃了个精光。

    每当他为奥尔感到担心时,他就会想起奥尔什么事都做得来的本领。当想起奈特中士向他描述奥尔在救生筏上的那幅情景时,他不禁哑然失笑。奥尔把地图和罗盘放在自己的膝盖上,微笑着俯下身专心致志地研究着它们。他一边一块接一块地把湿透了的巧克力塞进自己那大咧着傻笑的嘴里,一边恪尽职守地在电闪雷鸣狂风暴雨之中使劲地划着那把丝毫不起作用的天蓝色的玩具船桨,身后还拖着根装有鱼饵的钓鱼线。约塞连对奥尔的生存能力毫不怀疑。如果用那很可笑的钓鱼线能钓到鱼的话,奥尔准能钓到鱼;如果奥尔想钓鳕鱼的话,那么,哪怕以前从来没有人在这些海域钓到过鳕鱼,奥尔也准能钓到一条鳕鱼。约塞连又煮了一罐头汤,然后趁热把它喝了。每次听到门外汽车门砰的一声响,约塞连都会露出一个饱含希望的微笑,期待着转身面对帐篷入口,倾听着脚步声。他知道,奥尔随时会走进帐篷的。他那双闪闪发光的大眼睛、大腮帮子和龅牙,全都会被雨浇得湿淋淋的;他的头上会戴着一顶黄色的油布雨帽,身上会穿着一件大好几号的宽松油布雨衣;

    他的手里会得意洋洋地举着一条他钓上来的硕大的死鳕鱼,用它来逗约塞连开心。那副样子看上去活像个快活的采牡蛎的新英格兰人,可笑极了。但是,他没有回来
