\chapter{基德·桑普森}
 
    待到飞博洛尼亚执行任务的时候,约塞连就连去目标上空盘旋一次的勇气都没有了。当最终发现自己坐在基德-桑普森飞机的机头,到了空中的时候,他便摁了一下喉式传声器的按钮,问道:

    “喂?飞机怎么啦?”

    基德-桑普森尖叫了一声。“是不是飞机出了故障?怎么回事儿?”

    基德-桑普森这一声尖叫,着实把约塞连吓得浑身冰凉。“是不是出啥事了?”他极恐怖地叫喊道,“我们要跳伞吗?”

    “我不知道!”基德-桑普森极痛苦地回了一句,激动得呜咽了起来。“有人说我们要跳伞!究竟是谁、是谁?”

    “是我约塞连,在机头!约塞连在机头!我听见你说出事了。难道你没说?”

    “我还以为是你说的哩。这会儿一切似乎都没问题。一切正常。”

    约塞连的心沉了下来。要是一切正常,他们便没了丝毫借口返回去,那么,事情更是糟糕透顶。他阴沉着脸,一时竟迟疑不决。

    “我听不见你说的话,”他说。

    “我是说一切正常。”

    太阳照耀在下面瓷青色的水面和其他几架飞机闪烁的边沿上,白色的光芒令人眼花镣乱。约塞连抓住连接内部通话系统转换开关盒的彩色电线,扯松了开来。

    “我还是听不见你说的话,”他说。

    他什么也没听见。他慢慢收拾起自己的图囊和三件防弹衣,爬回主舱。内特利端坐在副驾驶员的座位上,用了眼角余光瞟见他走上基德-桑普森身后的驾驶舱。内特利全身上下穿戴着重重的一大堆东西——耳机、帽子、喉式传声器、防弹衣和降落伞,看上去极虚弱,却显得异常地年轻腼腆。他朝约塞连懒洋洋地笑了笑。约塞连弓身凑近基德-桑普森的耳朵。

    “我还是听不见你说的话,”他于引擎均匀的嗡嗡声中叫喊道。

    基德-桑普森吃惊地回头扫了他一眼。基德-桑普森长了一副瘦削滑稽的面孔,配了两道弓形眉毛,一对稀稀落落的金黄色八字须。

    “什么?”他回过头喊道。

    “我还是听不见你说的话,”约塞连又说了一遍。

    “你说话还得大声点,”基德-桑普森说,“我还是听不见你说的话。”

    “我是说我还是听不见你说的话!”约塞连叫嚷道。

    “我也没办法,”基德-桑普森也冲着他高喊道,“我只能喊这么响了。”

    “我在对讲机里听不见你说的话,”约塞连愈发无可奈何,便大声咆哮道,“你必须返回去。”

    “就因为一只对讲机?”基德-桑普森表示怀疑地问道。

    “返回去,”约塞连说,“免得我砸了你的脑袋。”

    基德-桑普森望着内特利,以求得到道义上的支持,可内特利干脆就把目光收了回去。约塞连的军衔高于他们两个。基德-桑普森犹豫不决地又抵挡了片刻,然后洋洋得意地高呼了一声,便又急不可耐地屈从了。

    “这样对我来说也蛮好的,”他兴奋他说,于是撅了那对八字须,吹出一连串尖锐刺耳的唿哨。“是的,长官,这样对老基德-桑普森来说也蛮好的。”他又打了个唿哨,对着对讲机叫喊道,“注意听着,我的小山雀们。这是海军上将基德-桑普森在讲话。这是皇家海军骄傲的基德-桑普森上将在叫喊。是,长官。我们正在返航,弟兄们,上帝啊,我们正在返航!”

 


    内特利兴奋异常,一下子拽下了帽子和耳机,仿佛一个漂亮的小孩坐在高脚椅里,快活地前后轻摇了起来。奈特中士纵身从顶屋炮塔跳了下来,欣喜若狂,重重地捶打起每个人的后背。基德-桑普森驾驶飞机,划了一个漂亮的大圆弧,离开编队,直冲机场飞去。当约塞连把头戴式受话器接通了其中一个辅助通信转换开关盒的时候,飞机后部的那两个炮手竟一齐唱起了《库卡拉查舞曲》。

    待返回机场,他们却又突然蔫了。令人不安的沉默替代了狂喜。约塞连沉着脸且又极不自然地走下飞机,坐进了早就守在机场等候他们的那辆吉普车。车子返回驻地途中,穿越了阴森岑寂但是迷人的群山、大海和森林,一路上没人说一句话。当他们驶离近靠中队驻地的大道时,每一个人的心头依旧萦回着那种凄凉孤寂的感觉。约塞连最后一个走下车。片刻过后,在那一片老是令人心神不安的寂静——仿佛毒品一般,笼罩住那一顶顶空无一人的帐篷——中,只有约塞连和一阵和暖的微风在移动。中队一片死气沉沉,除丹尼卡医生——活像一只浑身哆嗦的红头美洲鹫,忧伤地栖息在医务室那扇关闭的门旁,四周泻下一片朦胧的阳光,把鼻子对了阳光使劲地抽吸,却全无效果——之外,没有丝毫人的气息。

 


    约塞连知道丹尼卡医生是不会随他一同去游泳的。丹尼卡医生再也不会下水游泳了;哪怕是在一两英寸深的水里,一个人也有可能因昏厥或轻度冠状动脉闭塞而淹死,让退浪给冲出海去,或是因了寒冷或用力过度而轻易染上脊髓灰质炎或导致脑膜炎球菌感染。

    博洛尼亚对其他人带来的威胁,更是让丹尼卡医生为自身的安全深深地担忧。入夜了,他听到了窃贼的响动。

    透过那片笼罩作战室入口的浅紫色暮蔼,约塞连看见一级准尉怀特-哈尔福特正极用心地盗用定量配给的威士忌酒,假冒了那些滴酒不沾者签名,且又边喝边快速地往一个个瓶子里灌,想抢在布莱克上尉记起这事后便懒洋洋地匆匆赶来盗了余下的酒之前,尽可能地多偷一些。

    吉普车又轻轻地起动了。基德-桑普森、内特利和其他人,在一阵无声的行动中,各自散开去了,融进了令人厌烦的黄色的寂静里。吉普车随着一阵喀喀的响声消失了。约塞连孑然一人处于沉重的原始寂寥之中,一切绿色的东西看去尽是黑的,而所有其他的一切则全部浸透了脓液的黄绿色。干燥朦胧的远处,微风吹过,刮得树叶飒飒作响。约塞连烦躁不安,既害怕又疲倦,两凹眼窝由于疲惫不堪而给人一种脏兮兮的感觉。他筋疲力尽地走进降落伞帐篷,里面搁着一张光滑的木制长桌。此刻,疑虑就像一只烦人的母狗在刨挖着一颗全然无愧的良心而让人毫无痛感。他把防弹衣和降落伞留了下来,再又返身出去,经过那辆运水车,前往情报室把图囊交还给布莱克上尉。布莱克上尉正坐在椅子里打盹儿,两条瘦长的腿跷在桌上,表面装出一副冷漠样,心里却是极好奇地探问约塞连的飞机为什么又返了回来。约塞连没搭理他,往桌上放下图囊,便走了出去。

    回到自己的帐篷,他便卸了降落伞背带和身上的衣服。奥尔在罗马,定于当天下午回来,因为他在离热亚那不远的海面上迫降,有了机会休假。内特利早就想打点好行装,准备接替奥尔。他实在是很欣喜:自己居然还活着,因而就急不可耐地想赶去罗马,继续毫无结果而又令人心碎地向那个妓女求婚。约塞连脱了个精光,在帆布床上坐下来歇息。一赤裸了身子,他便感觉好多了。只要身上穿了衣服,他从来就不曾有过舒服的感觉。稍过片刻,他又换上干净的短衬裤,穿上软帮鞋,肩披了一条土黄色浴巾,起身往海滩走去。

 


    沿中队驻地通向外面的那条路,约塞连绕过了森林里一处神秘的火炮掩体。有三个士兵驻守在那里,其中两个正躺在一圈沙袋上睡觉,还有一个正吃着一只紫石榴,一大口一大口地咬进不停嚼动的嘴里,再把咬碎的渣子吐进灌木丛里。每咬一口,红红的汁便从嘴里流淌了出来。约塞连蹑手蹑脚地往前走着,进了森林,不时爱惜地抚摸颤动着的光肚子,好像是让自己放心,这肚子还在原来的地方。他从肚脐眼处捻出了一块软麻布。突然他在路两侧的地上发现了不少雨后初生的蘑菇,一根根长有菌盖的指状菌柄钻出了黏湿的泥土,仿佛无生命的肉茎,他目光所及的地方,便长出了一大片,似乎它们正是在他的眼前冒出。到处是一大片一大片密密匝匝的蘑菇,就他目光所及,遍布了远处的林下灌木丛。他发现,它们的个头儿好像越来越大,数量似乎也越来越多。他觉得阴森森地恐惧,浑身一阵战栗,撒腿便跑,直到脚下的泥土消失,变成了干沙,那些蘑菇给抛在了后面,他才放慢了脚步。他忐忑不安地回头看了一眼,有些儿巴望着能见到那些又白又软的东西在后面盲目地爬着追赶他,或是突变成了蠕动的难以控制的一团,正悄悄地往上爬过树梢。

    海滩上空寂无人。唯一的声响也全都是极低沉的:溪流涨水的汩汩声,身后那高高的草丛和灌木林轻轻的呼吸声,还有那沉默无语半透明的波浪漠然的呜咽声。波浪总是很小,海水清澈透凉。约塞连把自己的东西留在了沙滩上,膛过齐膝深的海水,直到整个身子全都浸没在了水里。海的另一边,一片高低不平的暗色的狭长陆地笼罩在薄雾之中,隐隐约约。他懒洋洋地游到了浮台,扶住歇了一会儿,再又返身懒洋洋地游回到沙洲可以站立的地方。他好几次都是一头潜入碧绿的海水,直到觉得身体干净了,头脑又完全地清醒,便伸展了四肢趴在沙滩上睡觉,直睡到从博洛尼亚凯旋的机群差不多掠过了他的头顶。机群那许多台发动机一齐发出由弱而强的巨大的隆隆声,仿佛惊天动地的轰呜,闯进了他的梦乡。

    他醒了过来,眨眨眼,略觉头疼,睁开眼,见到的是一个乱腾腾的世界,一切倒是有条不紊。他惊愕地注视着眼前的奇观:十二支空军小队的飞机平稳地组成了精确的队形。这景象实在太是出乎意料,简直无法令人置信。没有一架飞机因载了伤员而猛冲在前。

    也没有一架飞机因受损而掉了队。空中也不见有冒出的遇难火焰。

    除他自己的飞机外,一架不少。顷刻间,他竟感到神经错乱,无法动弹。随即他便又清醒了过来,差不多因了这命运的嘲弄而落了泪。

    解释极简单:机群还没来得及轰炸,云层便掩住了目标,于是,得再飞博洛尼亚执行轰炸任务。

    他错了。压根就没有什么云层。博洛尼亚已遭了轰炸,飞博洛尼亚只是一次例行的飞行。那里也根本不见有什么高射炮火
