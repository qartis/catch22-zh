\chapter{布莱克上尉}
 
    科洛尼下士最初是从大队部打来的一个电话得知这一消息的。当时,他非常震惊,便轻手轻脚穿过情报室,走到布莱克上尉——他这会儿把平伸着的小腿搁在办公桌上,正打着盹儿——

    身边,用震惊的语调,低声把这消息告诉了他。

    布莱克上尉一下子来了精神。“博洛尼亚?”他兴奋得大叫起来。“太让我吃惊了。”他放声大笑。“博洛尼亚,嘿?”他又哈哈大笑了起来,惊喜地摇了摇头。“嗬,好家伙!要是那些狗杂种知道自己是飞博洛尼亚,真不知他们会是什么模佯,我巴不得马上就瞧瞧他们那一张张面容。哈,哈,哈!”

    自从梅杰少校击败他出任中队长那天以来,布莱克上尉这是第一次真正由衷地开怀大笑。当轰炸员们来到情报室,领取图囊时,他阴死阳活地站了起来,立在前部柜台的后面,为的是千方百计从中获取最大的乐趣。

    “没错,你们这些婊子养的,是博洛尼亚。”当全体轰炸员颇为怀疑地问他,他们是否真要飞博洛尼亚时,他便不厌其烦一遍又一遍地对他们这么说,“哈!哈!哈!试试你们的胆量吧,你们这些狗杂种。这次你们可是没有退路了。”

    布莱克上尉跟在全体轰炸员的最后面来到帐篷外。其他所有军官和士兵全都带着钢盔、降落伞和防弹衣,集聚在中队驻地中央四辆卡车——发动机正空转着——的周围。布莱克上尉饶有兴致地察看这些军官和士兵得知真相后的反应。这家伙个子虽大,却心胸狭窄,性情忧郁,脾气暴躁,又老是一副没精打采的模样。那张皱缩苍白的脸每隔三四天便修刮一次,大多数情况下,他似乎总在皮包骨的上嘴唇蓄两撇金红色的八字须。外面的场面倒是并没有让他失望。每张脸都因惊恐而阴沉了下来。布莱克上尉美美地打了个哈欠,擦了擦眼睛,擦去了最后一丝困意,于是,幸灾乐祸地纵声大笑起来。每当他告诉别人要试试胆量时,他总这么笑的。

    那天,杜鲁斯少校在佩鲁贾上空阵亡以后,布莱克上尉差点就被选中接任他的职位。自那以来,轰炸博洛尼亚不料竟成了布莱克上尉一生中最有收获的一件大事。当杜鲁斯少校阵亡的消息通过无线电传回中队驻地时,布莱克上尉内心一阵兴奋。先前,他从不曾真正考虑过这种可能性,不过,尽管如此,他马上便认识到,接替杜鲁斯少校担任中队长,他自己是合乎逻辑的必然人选。最初,他是中队的情报主任,也就是说,他比中队里任何别的人都要聪明。

 


    的确,他不属于战斗人员编制,而杜鲁斯少校生前得参加战斗,所有中队长通常也得作战;但,也正是这一点对他实在是另一个极有利的因素,因为他没有生命危险,只要祖国需要,无论多长时间,他都可以担任这一职位。布莱克上尉越琢磨,越觉得接任中队长似乎非他莫属了。只要立刻在最合适的地方说句合适的话,问题就可以解决了。他匆匆赶回自己的办公室,决定行动步骤。他在转椅里坐下,背往后一靠,两脚往桌上一跷,双目紧闭,开始想象:一旦当上中队长,一切该是多美啊。

    正当布莱克上尉想象着种种美景的时候,卡思卡特上校却在行动了。布莱克上尉断定,梅杰少校是智胜了他;其速度之快简直令他瞠目结舌。梅杰少校的中队长任命一宣布,布莱克上尉便大失所望,丝毫不掩饰自己内心的怨愤。对卡思卡特上校选用梅杰少校,与布莱克上尉共事的行政军官们都深表惊讶,而布莱克上尉则小声抱怨,这其中必定有什么蹊跷;同僚们对梅杰少校酷似亨利-方达这一点潜在的政治价值,作了种种猜测,而布莱克上尉则断定,梅杰少校其实就是亨利-方达;同僚们说梅杰少校这人颇有些古怪,而布莱克上尉则宣称他是共产党。

    “什么事都让他们做主了,”布莱克上尉表示反抗地声言道,“好吧,要是你们大伙乐意的话,尽管袖手旁观,由他们去,可我不愿意。我得想办法对付。从现在起,不管是哪个狗杂种来我的情报室,我都得让他签字效忠。不过,要是那个婊子养的梅杰少校来,即便他想签,我也决不会答应的。”

    几乎是一夜之间,这场光荣的宣誓效忠运动便轰轰烈烈地开展了起来。布莱克上尉发现自己竟成了运动先锋,欣喜若狂。他的确碰上了一个极妙的办法。所有参战官兵只有签字效忠后,才能从情报室领取图囊;第二道签字关过后,从降落伞室领取防弹衣和降落伞;再过了机动车辆军官鲍金顿中尉的第三道签字关后,这才获准从中队坐上其中一辆卡车赶往飞机场。每次转身,他们必须过一道签字效忠的关。无论是从财务军官处领取军饷,还是从军人服务社领取供给,或是找那些意大利理发师理发,他们都得签字效忠。

 


    在布莱克上尉看来,凡是支持他的这场光荣宣誓效忠运动的军官,都是竞争对手。于是,他便昼夜二十四小时密谋策划,始终保持一步领先。他要做报效国家第一人。每当其他军官在他的激励下,推行他们各自的签字效忠的方法,他便更进一步,让到情报室的每个杂种必须过两道签字效忠关,接着是三道,再又是四道;然后,他又推出宣誓效忠,之后,便让人一遍、两遍、三遍、四遍地同声齐唱《星条旗》歌。每次当他击败竞争对手,布莱克上尉便轻贱了他们,嗤笑他们不学他的招数。可每次当他们步他的后尘,他便又不安地退避一侧,绞尽脑汁想别的新计策,好再奚落他们一顿。

    不知不觉地,中队里的战斗人员发现自己竟受那些行政官员——原先是奉命来为他们服务的——操纵。他门整天受人欺侮,凌辱,骚扰,摆布,走了一个又来另一个。一旦他们表示反抗,布莱克上尉就答复他们说,只要是忠诚的人,是不会厌烦宣誓效忠必要的签字的,只要有人对宣誓效忠是否有效这一点提出质疑,他就回答,凡是确确实实效忠自己国家的人,只要由他经常敦促,是会很自豪地发誓自己将忠诚于祖国的。一旦有人问起这么做有何道德作用,他就回答说,《星条旗》是创作出的最伟大的音乐作品。一个人签字效忠的次数越多,他就越忠诚;对布莱克上尉来说,道理就是如此简单明了。他每天都让科洛尼下士签上百次名,这样,他就可以始终证明自己比任何别的人更加忠诚。

    “重要的是要让他们不停地宣誓,”他跟自己的追随者解释道,“至于他们是否心诚,这无关紧要。正因为如此,所以,他们也让小孩子们宣誓效忠,尽管孩子们连什么是‘宣誓’和‘效忠’都还一窍不通。”

 


    对皮尔查德上尉和雷恩上尉来说,这场光荣效忠宣誓运动实在是一桩又光荣又讨厌的事,因为这一来,每次安排机务人员执行作战任务,他们便无端地要费不少周折。中认上下全都忙着签名,宣誓,合唱。所有飞行任务得花上更多的时间才能执行。有效的紧急行动也就不可能了,然而,皮尔查德上尉和雷恩上尉都是极胆小的人,实在没胆量对布莱克上尉大声抗议。布莱克上尉呢,却天天严格认真地坚持由他首创的“不断重申”学说——意在遏止所有那些第一天签字第二天就不忠的官兵。就在皮尔查德上尉和雷恩上尉心中一片迷茫,为身陷困境而抓耳搔腮的当儿,布莱克上尉又给他们出了个主意。他带来了一个代表团,直截了当地跟他们说,必须让每一个飞行虽签字效忠后,方可准许他执行作战飞行任务。

    “当然,这都得由你们自己来决定,”布莱克上尉指出,“没人想强迫你们。可是,其他所有人都在让他们签字效忠。假如只有你们俩不怎么关心自己的国家,没让他们签字效忠的话,那么,这在联邦调查局看来,也必定有什么蹊跷的。要是你们俩甘愿得个恶名声,那是你们自己的事,跟别人全无关系。我们只是想尽力帮忙而已。”

    米洛没有被说服。他断然拒绝中止梅杰少校的饮食,即便梅杰少校是共产党人——对此,米洛心里亦颇有怀疑。米洛生来就反对所有破坏常规的革新。他有相当坚定的道德原则,断然拒绝加入这场光荣的效忠宣誓运动,直到后来,布莱克上尉带领他的代表团前来拜访他,请求他参加。

    “国防是每个人的天职,”米洛拒绝后,布莱克上尉说,“整个过程都是自愿的,米洛——别忘了这一点。假如他们不愿在皮尔查德和雷恩那里签字效忠,他们可以不必那么做。但,在你这里,假如他们不签,我们要你饿死他们。这就跟第二十二条军规一样。你明白吗?你总不至于违抗第二十二条军规吧?”

    丹尼卡医生却坚持自己的立场。

    “你凭什么断定梅杰少校就是共产党人?”

    “我们开始指控他以前,你从没听到他否认这一点,是不是?你也没有看见他在我们的效忠誓约上签过字。”

    “是你们不让他签。”

    “当然不能让他签,”布莱克上尉解释道,“否则,我们发起的这场运动也就前功尽弃了。你瞧,要是你不愿跟我们合作,你完全可以自便。可是,一旦米洛刚准备要饿死梅杰少校,而你却给他治疗,那么,我们其余的人这么竭尽全力又有什么意义呢?我只是不知道,对暗中破坏我们整个安全计划的人,大队部的上司们会想什么办法处置,他们很有可能会调你去太平洋。”

    丹尼卡医生立刻屈从了。“我这就去跟格斯和韦斯说,让他们按你的吩咐去做。”

    大队部的卡思卡特上校早就开始纳闷,究竟出了什么事情,“那个白痴布莱克,在大闹什么爱国主义,”科恩中校笑着说,“我想,既然是你提升梅杰少校当了中队长,你最好暂且跟他合作一段时间。”

    “那还不是你出的主意。”卡思卡特上校极恼火地责备他。“当初真不该听你的话。”

    “可我出的那个主意也是一条妙计,”科恩中尉反驳道,“那个多余的少校身为行政军官,却老是败坏你的名声,不就是我那条妙计把他给除掉了吗?不用担心,这一切大概马上就会走上正轨的。

    现在最好的办法是,给布莱克上尉去一封信,表示完全支待他,并希望他适可而止,免得到时闹得一塌糊涂。”科恩中校突然想出了个怪念头。“我很有点怀疑!那个白痴该不会把梅杰少校赶出他的活动房屋吧,你说呢?”

 


    “接下来我们要做的是,把那婊子养的梅杰少校赶出他的活动房屋。”布莱克上尉拿定了主意。“我还真巴不得把他的老婆孩子赶到树林子里去。可是我们做不到。他没有老婆孩子。所以,我们只得应付眼前的事,把他赶出去。谁负责这些帐篷?”

    “他。”

    “你们瞧见了?”布莱克上尉大声叫道,“所有一切都让他们给操纵了!哼,我可是不会容忍的。要是迫不得已,我会直接向德-科弗利少校本人汇报这事的。等他从罗马一回来,我就让米洛去跟他说这事。”

    布莱克上尉对德-科弗利少校的智慧、权力和正直深信不疑,即便他以前从未跟德-科弗利少校说过一句话,现在也还是没有胆量这么做。他委派了米洛替他去找德-科弗利少校谈话,自己则等待着这个高个子主任参谋回来,等不耐烦了,见人就大发脾气。德-科弗利少校威风凛凛,长一头白发,满脸皱纹,俨然一副救世主的神态,对他,布莱克上尉和中队其他所有官兵一向是怀有深深的敬畏之心的。少校最终从罗马回到了中队,伤了一只眼,用一只新的赛璐珞眼罩护着。他一下子就把布莱克上尉的整个光荣效忠宣誓运动砸了个稀巴烂。

    德-科弗利少校返回中队那天,极威严地走进食堂,正排队等候签字效忠的军官自成一道人墙,拦住了他的去路。此刻,米洛非常小心翼翼,没说一句话。食品柜台的尽端,早来的一群军官每人手上托了一盘饭菜,正面向国旗宣誓效忠,为的是获准在餐桌旁就座用餐。来的更早的一群军官呢,早就在餐桌旁坐了下来,这时正合唱《星条旗》国歌,为的是可以享用桌上的盐、胡椒粉,还有调味番茄酱。德-科弗利少校在门口停了下来,皱眉蹙额,一脸的困惑不满,仿佛是见到了什么怪事。喧嚷声这才慢慢平静了下来。德-科弗利少校端庄地往前走过去,面前的那道人墙像红海一样,往两侧分了开来。他目不斜视,威武地大步走向蒸汽消毒柜台,于是,用清晰圆润的声音——因年迈而显得粗哑,又因年高德劭、地位显赫而洪亮有力——说道:

    “给我拿吃的来,”斯纳克下士没有给德-科弗利少校吃的,倒是递给他一份效忠誓约让他签字。德-科弗利少校一见是这东西,不由得大为恼火,用力把它推至一旁,那只好眼睛令人无法理解地射出强烈的鄙视的怒火,那张布满皱纹、衰老的大脸盘因暴怒而越发阴沉可怕。

    “我说过,给我拿吃的来,”他大声命令道,嗓音十分刺耳,就像远处的霹雳,在寂静的帐篷里发出不祥的隆隆响声。

    斯纳克下士脸色刷白,浑身哆嗦起来。他向米洛投去恳求的目光,企求他的指点。过去了可怕的几秒钟,没有一丝声息。接着,米洛点了点头。

    “给他拿点吃的,”他说。

    斯纳克下士这才把吃的东西递给了德-科弗利少校。德-科弗利少校手托满满一盘饭菜,刚转身离开柜台,便又停住了脚步。他的目光落到了那一群群军官身上,军官们正默默地用恳求的目光注视着他。随即,他便摆出一副主持正义的战斗姿态,大声吼道:

    “给大伙拿吃的!”

    “给大伙拿吃的!”米洛如释重负,兴奋地应了一声。光荣的效忠宣誓运动就此宣告结束。

    布莱克上尉彻底失望了,他没料到,自己如此信赖并视作后盾、身居高位的上司竟然会从背后给他这么一刀。德-科弗利少校让他受尽了屈辱。

    “哦,我啥事儿都没有,”只要有人来向他表示同情,他便很愉快地回答道,“我们的任务已经完成了。我们的目的就是要让我们讨厌的人感到恐惧,让大家警惕梅杰少校的危险。我们的确达到了这个目的。既然我们压根就没想让他签字效忠,那么,要不要那些效忠誓约,其实已经是无关紧要了。”

    博洛尼亚大围攻没完没了,骇人听闻,又把中队里布莱克上尉讨厌的那些人一个个吓得胆战心惊。见了这一幕,布莱克上尉不由得怀恋起光荣效忠宣誓运动那段过去的美好时光。那时,他可是个举足轻重的风云人物,即便是像米洛-明德宾德、丹尼卡医生、皮尔查德和雷恩那样有权势的大人物,一见到他来就浑身哆嗦,对他俯首帖耳。为了向新来的人证明,自己确实曾一度是个叱咤风云的人物,他依旧保存着卡思卡特上校写给他的那封嘉奖信
