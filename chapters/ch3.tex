\chapter{哈弗迈耶}
 
    说实话,约塞连从医院回到中队驻地时,除了奥尔和约塞连帐篷里的那具尸体之外,没一个人在。那个死人实在是很讨厌,尽管约塞连从未见过他,但对他却是厌恶透顶。尸体整天搁在帐篷里,约塞连极其恼怒,三番五次跑中队办公室,向陶塞军士诉苦,可军士硬是否认有这么个死人存在。当然,约塞连也就不再去找他,自讨没趣了。于是,他便想了办法,直接上诉梅杰少校,但结果却是更让他沮丧。梅杰少校是中队长,瘦高的个儿,长相很有点像落难的亨利-方达。约塞连每次闯过陶塞军士,想跟他说说死人一事时,梅杰少校便从办公室的窗子里跳出去。跟死人合住一顶帐篷,太难为约塞连了。于是,他只得去麻烦奥尔,尽管这人亦极难相处。

    约塞连回中队的当天,奥尔正在修理炉子加油用的龙头。炉子是约塞连住院期间,奥尔自己动手做的。

    “你忙什么呢?”尽管他一进帐篷,便看得分明,约塞连依然很谨慎地问了一句。

    “这儿有个裂缝,”奥尔说,“我正想办法补呢。”

    “请你别再搞啦,”约塞连说,“搞得我都快烦死了。”

    “我小时候,”奥尔答道,“常常是每天从早到晚四处闲逛,嘴里还含着海棠果,一边一颗。”

    约塞连正取出野战背包里的梳妆用具,听罢,便随手把背包置于一旁,很是疑心地准备听他接着往下说。等过片刻。“为什么?”

    他终究等不及,便不知不觉地开口问道。

    奥尔很是得意,窃笑道:“因为海棠比七叶树果好吃。”

    奥尔跪在地上,不停地忙手中的活。他拆下龙头,极小心地摊开所有细小的零件,一一清点过后,便无休止地细心琢磨起每一个零件,仿佛先前从未见过什么与此有些许相仿的东西。接着,又聚起一个个零件,重新装配成完好的小龙头。如此,一遍又一遍,往复不已,依旧耐心之至,兴头十足,也不见有丝毫倦意。看来,一时半会儿,他是不会罢手的。约塞连在一旁看着他没完没了地折腾,心想假如他还不歇手,必定会逼得他无情地向他下毒手。他将目光移向挂在蚊帐横杆上的那柄猎刀,是那个死了的士兵在到达的当天挂在那里的,一旁还挂着他的那只空的手枪皮套,皮套里的枪就是让哈弗迈耶盗走的。

    “没有海棠果的时候,”奥尔接着说,“我就用七叶树果替代。这种果子跟海棠果差不多大小,其实,形状比海棠果漂亮,当然,形状如何,根本就无关紧要。”

    “你到处游荡,干吗嘴里要含海棠果?”约塞连又问了一遍。“刚才,我就是问这个。”

    “因为形状比七叶树果漂亮,”奥尔答道,“我才跟你说过。”

    “为什么,”约塞连以称许的口吻咒骂道,“你这眼冒邪气、整天只知道瞎捣鼓并且谁都不愿搭理的杂种,为什么到处转悠,嘴里还要含点什么东西?”

    “我可不是什么东西都含在嘴里的,”奥尔说,“我含的是海棠。

    弄不到海棠,我就含七叶树果。含在嘴里。”

    奥尔咯咯地笑了。约塞连决计住嘴,于是果真缄口,不再吭声了。奥尔等着。约塞连却更有耐心。

    “一边含一颗,”奥尔说。

    “为什么?”

    奥尔趁机反戈一击。“什么为什么?”

    约塞连没理会他,只是笑着摇了摇头。

    “这阀门真是挺有趣的,”奥尔自言自语道。

    “怎么啦?”约塞连问。

    “因为我想要——”

    约塞连明白了。“天哪!你干吗要——”

    “——圆圆的饱满的脸蛋。”

    “——圆圆的饱满的脸蛋?”约塞连问。

    “我想要圆圆的饱满的脸蛋。”奥尔又说了一遍。“还在我很小的时候,我就想有朝一日要一张圆圆的饱满的脸蛋。于是;我便下定决心,竭尽全力,脸蛋不圆鼓起来,誓不罢休。老天作证,我的确尽了力,总算达到了目的。我便是这么做的,嘴里从早到晚都含着海棠果。”他又咯咯地笑了起来。“一边一颗。”

    “你干吗想要圆圆的饱满的脸蛋?”

    “我想要的倒不是圆圆的饱满的脸蛋,”奥尔说,“是宽大的脸蛋。颜色我倒是不怎么在意,关键是,要宽要大。你常可以读到这样一些消息,说是有些家伙像发了疯似的,为了练手力,一天到晚握着橡皮球,东跑西遛。我自己呢,就跟那帮家伙一样,疯了似地卖劲。其实,我就是那号人,疯疯癫癫的。我也是经常手握着橡皮球,没早没晚地四处溜达。”

    “为什么?”

    “什么为什么?”

    “你为什么一天到晚东跑西窜,手里非捏着橡皮球不可?”

    “因为橡皮球——”奥尔说。

    “——比海棠漂亮?”

    奥尔摇了摇头,窃笑道:“我这么做,全是为了维护自己的好名声,免得让人撞见我东跑西窜时嘴里还含着海棠。手握了橡皮球,我就可以说,嘴里没含海棠呀。每当有人间我,为什么东跑西窜时嘴里非含了海棠不可,我就可以摊开双手,让他看清楚,我游逛时随身带着的是橡皮球,不是什么海棠,而且是在我手里,不是含在嘴里。这谎倒是编得挺好的,可别人信了没有,我从来就不知道,因为你跟别人说话时,嘴里含上两颗海棠,要想让人家听明白你的意思,实在不是很容易的。”

    这时、约塞连倒是的确发现,很难听清楚他在说些什么,他一时又说不准,奥尔是否用舌尖顶着他的一侧圆腮帮在跟他瞎说八道。

 


    约塞连打定主意,不再吐半个字儿。说了也白搭。他了解奥尔,知道要想让他亲口道出他喜欢阔脸蛋的真实原因,压根是不可能的。就像有人问过他,那天上午在罗马,那个妓女为什么用鞋子敲打他的头,而且是在内特利的妓女的小妹妹的房门外的窄小过道里,再说,那房门当时又是开着的。结果呢,问的人同样是白费了口舌。奥尔的那个妓女,身量颀长,体格健壮,披散一头长发,可可色的皮肤,极柔嫩处,密密地汇聚了一根根清晰可见的青筋。当时,她一边恶言辱骂,一边扬声尖叫,光着脚,一次次地高跳起来,不停地用细高的鞋跟敲打他的头顶。两个人全光着身,闹腾得极凶,结果,公寓里的房客都跑进过道看热闹,一对对男女全都赤条条地站在各自的房门口,除了一个老太婆和一个老头儿。老太婆系一条围裙,上身套了件针织套衫,在那儿叽里咕咯地责骂;可那老头儿呢,生来便是个浪荡的好色之徒,打从奥尔和妓女开始闹直至结束,他瞧得心花怒放,心里直痒痒,开心得咯咯地笑不停。那姑娘尖声叫嚣,奥尔则是一个劲地傻乐。她用鞋跟敲一下,奥尔便傻笑得更带劲,他越这样,她就越气。于是,跃得更高,猛击他的脑瓜,极丰腴的双乳不停地耸动,似强风中飘扬的三角旗,屁股和粗实的大腿左扭右摆,丰美迷人,极富性感,但令人畏葸。她拼命尖叫,奥尔还是一个劲地傻笑。于是,她又尖叫一声,对着奥尔的太阳穴狠狠一击,把他打昏了过去,终于终止了他的傻笑声。房客们用担架送他进了医院,他的头上给鞋跟扎了个不太深的窟窿眼儿,他得了轻度脑震荡,一时没上火线,尽管只有短短的十二天。

    这一切究竟是怎么回事,谁也无法弄个水落石出,就连咯咯直笑的老头儿和叽里咕喀责骂的老太婆,也无可奈何,尽管他俩照例应该了然这妓院上下发生的一切。妓院极大,仿佛走不到尽头,客房不计其数,皆分列于狭窄过道的两侧。过道由起居室往相反方向伸展。起居室极宽绰,所有的窗户皆上了窗帘,但室内仅安了一盏灯。那件事之后,每与奥尔相遇,那妓女便会高撩起裙子,露出白色弹力紧身短衬裤,再是满口脏话一番奚落,把个结结实实的圆肚凸起了冲着他,同时,又破口大骂轻侮的话,于是,见他嗤嗤地怯笑,躲及约塞连身后,就又嗓音粗哑了,呵呵大笑。当初,奥尔闭紧了门,在内特利妓女的小妹妹房里做了些什么,或是想做些什么,或是动手了却又没能做成什么,这究竟还是个不解之谜。那姑娘是无论如何不会向什么人道出真情的,不管是内特利的妓女,还是别的什么妓女,抑或内特利和约塞连。奥尔或许会说,但约塞连早已是定了主意,不愿再白费什么口舌。

    “你不是想知道我为什么喜欢饱满的圆脸蛋吗?”奥尔问道。

    约塞连还是缄口不语。

    “你记不记得,”奥尔说,“那次在罗马,那容不了你的娘们老是用鞋跟敲打我的头?你想不想知道她干吗这么做?”

    奥尔究竟做了些什么,惹那娘们发如此大的火,竟一连在他头上猛击了十五至二十分钟,却又没有令她气恼得抓住他的双脚倒提起来,摔他个脑袋开花。这实在是难以想象。论个儿呢,那娘们确实很高大,奥尔也确实很矮小。奥尔长一副龅牙,双目暴凸,极配了他那张鼓鼓的大圆脸蛋。他的身量比年轻的赫普尔还矮小。赫普尔住的那顶帐篷在铁道左侧的行政区,跟他同居的是亨格利-乔,每天晚上总会在睡梦里惊呼。
 


    这帐篷是亨格利-乔误搭人行政区的。行政区地处中队驻地的中心,两侧分别是堆了锈铁轨的壕沟和倾斜的黑色柏油路。路上每见有过往的年轻女子,体态丰盈,相貌却是丑极,咧开掉了牙的嘴,嘻嘻地傻笑。只要中队的弟兄们答应送她们到目的地,姑娘们是没一个不愿搭车的。于是,士兵们便可开车带她们离开那条大道,到杂草丛里野合。约塞连只要有机会,是绝对抓住不放的。不过,较之亨格利-乔,这样的机会在他是不常碰着的。亨格利-乔有本事搞来一辆吉普车,却不会开,因此,便求助于约塞连。中队士兵住的帐篷,搭在柏油路的另一侧,紧挨露天影剧场。影剧场是这些行将送命的兵士每日娱乐的处所,到了晚上,便在一方折叠式的银幕上放映愚蒙无知的军队厮杀的影片。约塞连回到中队的当天下午,影剧场便又迎来了另一个劳军联合组织的剧团。

    劳军联合组织的剧团,由P-P-佩克姆将军负责调遣。他已将指挥部迁移至罗马,与德里德尔将军钩心斗角,此外,别无什么更适宜的事可做。于佩克姆将军,办事必须绝对地爽利。他行动敏捷,举止文雅,工作一丝不苟。他知道赤道的周长,且总是把本意所指的“增长”,改写成“增进”。他是个卑鄙小人,这一点谁都没有德里德尔将军了解得清楚。近日,佩克姆将军下达了一道军令,要求地中海战区内的所有帐篷全都平行搭建,每顶帐篷的门必须极威风地面向美国国内的华盛顿纪念碑。但,德里德尔将军却为此大感恼怒。在他——一支作战部队的指挥官——看来,这命令实在是一派胡言。此外他联队里的帐篷该如何搭建,压根就轮不上佩克姆将军操什么心。于是,这两位指挥官便为了各自的权限,发生了激烈的争执。结果,因了前一等兵温特格林的缘故,德里德尔将军占了上风。温特格林是第二十七空军司令部邮件收发兵。他在处理信件时,把佩克姆将军的书信全部扔进了废纸篓,因为他觉着太冗长,这样,便定了争执的孰胜孰负。德里德尔将军的书信文体很少矫饰,意见的陈述也较质朴,颇合温特格林的口味,因此,他便竭诚遵照规章制度,快速把信件传送了上去。于是,因上方不曾收到佩克姆将军的函件,德里德尔将军便在这场纠纷中取胜了。

    佩克姆将军想竭力挽回失掉的声威,于是就不断地派遣出一个个劳军联合组织剧团,数量超出了以往任何一次,并授命卡吉尔上校,鼓励所有将士观看演出。
 


    然而,约塞连所在中队的所有官兵对此却全无兴趣。他们当中,倒有越来越多的人一天几次板着脸去找陶塞,询问遣送他们回国的命令是否已经下达。他们都已完成了五十次飞行任务。较之约塞连初进医院的时候,此刻完成五十次飞行任务的官兵人数早已上升,可他们依旧在等待。他们一个个焦心如焚,坐卧不安,犹如抑郁沮丧、窝囊透顶的年轻人,举止怪诞,走路作蟹行。他们等着设在意大利的第二十六空军司令部下达命令,遣送他们安全返回自己的家园。他们无所事事地等待着,焦心如焚,坐卧不安,一天几次神情严肃地上门找陶塞,探听遣送他们安全回国的命令是否已经下达。

    他们在进行一场竞赛,对此,他们谁都很清楚,因为他们全有过惨痛的经历,深知卡思卡特上校随时会再增加飞行次数。他们唯有待命,除此,别无其它更好的选择。唯独亨格利-乔每次完成飞行任务后,便有更称心的事可做。他做过噩梦,梦里常发出尖叫声,还跟赫普尔的猫屡屡发生拳斗,每回都赢。劳军联合组织每次来演出,他便带了照相机坐在前排,总想拍那黄头发女歌手的半身像,那演员穿一身饰有闪光装饰片的连衣裙,仿佛随时会让一双大丰乳给撑破。可那些照片从来就不见冲印出来。

    卡吉尔上校是佩克姆将军手下善解难题的高手,他体魄甚健,个性坚强。战前,他曾是一名极有魄力的销售经理,机警敏捷,敢作敢为。可他却是行径十分恶劣的销售经理,实在令人可怕,以致臭名远扬,反倒招徕了不少为逃税而急于亏损的公司,一家家争相雇用他。遍及整个文明世界,从巴特里公园到富尔顿大街,他便是众人眼里能于一夜之间创造逃税奇迹的可靠人选。他身价极高,因为失败常常也是来之不易。他得从上层开始一切,之后,便煞费苦心往下活动,在华盛顿的一些朋友颇有同感,在他们看来,亏蚀钱财实在不是简单的事,得花上几个月的时间,苦心经营,仔细地拟订错误的计划。错用一人,打乱一切程序,事事失算,忽视所有细节,处处漏洞百出,就在他以为马到功成的时候,政府竟赐他一汪湖,一片森林,或一片油田,于是,一切成了泡影。即便有这种种不利因素,人们可以绝对相信卡吉尔上校有能力使处于鼎盛期的企业倒闭。卡吉尔上校是白手起家的,因而,他的一事无成也就怪不得别人了。

    “弟兄们,”卡吉尔上校开始在约塞连所在的中队煽惑,一边留意说话时的每一处停顿。“你们都是美国军官。世界上没有其他军队的军官可以声言他们是美国军官。你们好好考虑考虑吧。”

    奈特中士想了想,于是极恭敬地告诉卡吉尔上校说,他正在给兵士们训话,军官们全在中队驻地的另一侧恭候他。卡吉尔上校很爽利地向他道了声谢,使得意扬扬地大步从士兵中穿越了过去。见自己服役二十九个月,依旧保持着当年天才般的无能,卡吉尔上校颇觉得意。
 


    “弟兄们,”他开始向军官们讲话,一边留意说话时的每一处停顿。“你们都是美国军官。世界上没有其他军队的军官可以声言他们是美国军官。你们好好考虑考虑吧。”他停顿片刻,让大家伙儿思量一番。“这些人是你们的客人!”突然,他高声叫道,“他们行走三千多英里,前来为你们演出。假如没人愿意去看他们的表演,那么,他们会怎么想?他们的士气又会如何呢?听着,弟兄们,你们去不去看演出,这跟我实在毫不相干,不过,今天想给你们拉手风琴的那个姑娘,早已到了做母亲的年龄。假如你们自己的母亲远行三千多英里的路,为一些并不想看她演出的士兵拉手风琴,你们会有何感想?那位早已到做母亲年龄的手风琴手,一旦她的孩子长大后得知自己的母亲受过这等遭遇,他内心会有什么感受?这答案,我们大家都很清楚。嗨,弟兄们,别误解我的意思。这当然全是自愿的。

    我这个上校是天底下最不愿意命令你们去观看劳军联合组织剧团这场演出的,不过,我要你们当中除有病非得住院不可的人无一例外地立刻去观看演出,尽情娱乐一番。这是军令!”

    约塞连确实感到身体很是不适,差不多又需住院治疗。完成三次作战任务后,他的病情更加严重,可是,丹尼卡医生愁闷地摇了摇头,怎么也不愿让他停飞。

    “你自以为苦恼?”丹尼卡医生痛心地训斥了他一番。“那我呢?

    当初学医,我可是吃了八年花生。这之后,我便在自己的诊所里靠鸡食为生。直到后来,业务渐渐好了起来,来看病的人多了,我才有能力平衡了收支。于是,就在诊所最终盈利的时候,他们征我服了兵役。我实在是不晓得你发什么牢骚。”

    丹尼卡医生是约塞连的朋友,却无论如何不肯在他能力所及的情况下帮约塞连一把。丹尼卡医生跟他讲了些飞行大队卡思卡特上校的事,说这家伙居然盼着做一名将军;还谈了联队德里德尔将军及其护士的有关情况;此外,再又介绍了第二十六空军司令部其余各位将军——他们再三主张,只要飞行四十次,就完成了任务。约塞连在一旁听得异常认真。

    “你何不乐观些,随遇而安呢?”丹尼卡医生郁郁不乐地劝慰约塞连。“瞧人家哈弗迈耶,多学着点儿。”

    约塞连听罢,便不寒而栗。哈弗迈耶是领队轰炸员,每次飞向轰炸目标时,从不采取规避动作。于是,跟他在同一编队飞行的所有飞行人员面临的危险陡增。

    “哈弗迈耶,你他妈的为什么老是不采取规避动作?”每次执行任务后,大伙便会气势汹汹地诘问哈弗迈耶。

    “嘿,你们这帮家伙就别缠着哈弗迈耶啦。”卡思卡特上校就会下命令。“他可是咱们最出色的轰炸手。”

    哈弗迈耶咧嘴一笑,点点头,于是,就告诉大伙儿说,每天晚上他是如何用猎刀把子弹改制成达姆弹,随后再用这些子弹打自己帐篷里的田鼠的。哈弗迈耶实在是他们最出色的轰炸手。然而,他从出发点一路直线飞往目标,甚至远远飞越目标,直到他亲眼见到投下的炸弹落地开花,猛地喷射出橘黄色的火焰,在滚滚烟幕下闪亮,炸成粉未状的瓦砾,似灰黑色的滚滚巨浪,涌向空中。哈弗迈耶透过普列克斯玻璃机头,全神贯注地盯着炸弹直落而下,这一来,让六架飞机上的飞行人员惊恐得直发愣,飞机稳稳地停留在空中,无疑成了敌人的活靶子。于是,下面的德国炮兵便获得了充裕的时间,调准瞄准具,瞄准目标,扣动扳机,拉火绳,或是掀按钮,抑或诉诸一切武器,一旦他们的确想置素不相识者于死地。

    哈弗迈耶是一名领队轰炸员,从未失过手。约塞连也是领队轰炸员,但被降了职,原因是他毫不在乎自己是否命中目标。他早就拿定了主意,或是永久生存,或是在求得永生中死去。他每次上天执行飞行任务,唯一的使命便是活着返回地面。
 


    先前,中队里的弟兄们极喜随约塞连后飞行。约塞连常自四面八方及各不同的高度,疾飞至目标上空,时而急上升,时而大角度俯冲,时而又大坡度盘旋——其他五架飞机上的飞行员竭尽了全力与他保持队形,继而,他仅用两三秒钟平飞,投下炸弹,于是,随发动机的一阵震耳欲聋的轰鸣声,再又急跃升飞。他急遽地从空中飞过,迂回穿行于密集的高炮火力之中,于是,六架飞机即刻在空中四散开来,似一个个祈祷者,每一架飞机便成了德国战斗机炮击的活靶子。然而,于约塞连,这实在是桩好事,因为他自己周围就不复见有德国战斗机,再者,他也不希望有什么飞机在自己飞机的近处爆炸。只是在远远甩掉德国人的“狂飚”战斗机之后,约塞连才会无精打采地把航空钢盔推至大汗淋漓的后脑勺,停止对把握操纵器的麦克沃特厉声叫喊着发号施令。此刻,麦克沃特唯一的疑惑,便是投下的炸弹不知落至了何方。

    “炸弹舱空了。”守在尾舱的奈特中士便会通报。

    “桥炸到没有?”麦克沃特会问道。

    “我看不见,长官,我在这尾舱颠得实在是厉害,没法看见。这会儿下面全是烟雾,根本就看不到。”

    “喂,阿费,炸弹有没有击中目标?”

    “哪个目标?”阿德瓦克上尉会反问道。胖墩墩的阿德瓦克上尉,喜抽烟斗,是约塞连的领航员,答话时,正置身机头,立于约塞连一侧,面前杂乱地堆着一张张由他设计的地图。“我想我们还没达到目标。我说得没错吧?”

    “约塞连,炸弹击中了目标没有?”

    “哪几枚炸弹?”约塞连反问道。他唯一关注的是高射炮火。

    “嗬,行了,”麦克沃特便会说,“算了吧。”

    约塞连毫不在乎自己是否击中目标,只要哈弗迈耶或是其他随便哪个领队轰炸员命中了目标,大伙儿便再也不必飞回去继续轰炸。有人时常对哈弗迈耶极恼火,恨不得揍他一拳。

    “我跟你们说过,别去打扰哈弗迈耶上尉。”卡思卡特上校忿忿地警告他们。“我早说过,他是我们最出色的轰炸手,难道你们忘了?”

    见上校出面斡旋,哈弗迈耶咧嘴一笑,又往嘴里塞了一颗花生薄脆糖。

    晚上打田鼠,在哈弗迈耶,已是得心应手了。用的武器便是从约塞连帐篷里那个死人处窃来的那枝枪,诱饵是一块糖。他坐等着田鼠来啃糖块,一边在黑夜里细察;另一只手的一根手指套住一根绳尾端打成的圈,绳就拉在蚊帐架和头顶上方那只非磨砂灯泡的开关线之间。绳绷得极紧,似班卓琴的琴弦,轻轻一拉,电灯便随一声吧嗒亮了开来,炫目的灯光照得浑身哆嗦的田鼠两眼昏花。目睹着这小田鼠惊吓得动也不动,骨碌碌地转动恐惧的眼睛,紧张万分地拼命搜寻来犯之敌,哈弗迈耶总会咯咯地欢笑不止。待到田鼠的目光和他的目光相碰,他便纵声狂笑,同时扣动扳机,于是,一声巨响回荡,毛茸茸的躯壳给击成腥臭的肉酱,飞溅得帐篷里到处都是。

    一天深夜,哈弗迈耶朝一只田鼠开了一枪,枪声一响,亨格利-乔便光脚冲了出来,直奔哈弗迈耶的帐篷,一边尖声叫嚷,一边手持四五口径手枪把一颗颗子弹射了进去,同时,从壕沟的一侧猛冲下去,又从另一侧猛冲了上来,随即便突然消失在一条狭长掩壕里。这样的掩壕,自米洛-明德宾德轰炸中队驻进后的次日上午,竟似变魔术一般,眨眼间现于每一顶帐篷的旁边。这事就发生在博洛尼亚大会战期间的一天黎明前夕。当天夜晚,处处见有默默无言的死人,恰似一个个活幽灵。亨格利-乔当时也因忧心忡忡而近乎精神错乱,因为他又完成了飞行任务,一时不再会上天。待弟兄们从阴湿的掩壕底把他捞上来时,他正断断续续地说着胡话,一会儿蛇,一会儿耗子,一会儿又是蜘蛛。其他人打着手电往下照,想看个分明,然而,掩壕里除几英寸已变臭的雨水之外,便什么也见不到。

    “你们瞧见了吧?”哈弗迈耶高声叫道,“我早跟你们说过,他疯了,难道你们忘了?”
