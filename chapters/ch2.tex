\chapter{克莱文杰}
 
    从某种意义上来说,刑事调查部的那名工作人员倒是挺走运的,因为医院外面,依旧是硝烟弥漫。人人都成了疯子,却又被授予种种勋章,作为嘉奖。在世界各地,士兵们正在各轰炸前线捐躯,有人告诉他们,这是为了他们的祖国。但,似乎没人在意,更不用说那些正献出自己年轻生命的士兵了。目下是见不到有什么结局的。唯一可望的,倒是约塞连自己的结局。要不是为了那个爱国的得克萨斯人——下颌大得像漏斗,头发凌乱不堪,脸部永远挂着的笨拙的笑容,极似高顶宽边黑呢帽的帽檐——约塞连是本可以留在医院的,直到世界未日。那个得克萨斯人希望病房里的每一个人都快快乐乐,唯独约塞连和邓巴除外。他病得实在是很厉害。

    得克萨斯人不想让约塞连好过,尽管如此,约塞连亦是不可能快乐起来的。因为医院外面,还是不见有什么逗人发笑的事情。唯一在进行的,便是战争。除约塞连和邓巴之外,似乎没人注意到这一点。每当约塞连想提醒人们的时候,他们便赶紧躲开他,觉得他是个疯子。就连克莱文杰,本该很了解他的,这次却是一改往常的善解人意。就在约塞连躲进医院之前,他俩曾见过最后一面,当时,克莱文杰便对他说他是个疯子。

    克莱文杰圆睁怒目地盯着他,两手紧抓住桌子,高声忿詈:“你是个疯子!”

    “克莱文杰,你究竟要别人如何才是?”邓巴在军官俱乐部的喧闹声里,提高嗓门,极不耐烦地回敬了一句。

    “我可不是在开玩笑,”克莱文杰毫不退让。

    “他们是想把我杀了,”约塞连镇定地对他说。

    “没人想杀你,”克莱文杰高声叫道。

    “那他们干吗向我开枪?”约塞连问。

    “他们谁都不放过,见谁便开枪,”克莱文杰回答说,“他们想杀尽所有的人。”

    “那又有什么不同?”

    克莱文杰早已失去了控制,激动得把半个身体从椅子上抬了起来,两眼噙着泪水,嘴唇苍白,直打哆嗦。为了维护自己坚信的原则,他总免不了要跟人大吵一番,可是,每回吵到最后,他总是气急败坏,不住地眨眼,强忍住伤心泪,以示自己对信念的坚定不移。克莱文杰对许多原则信守不渝。他才是实实在在地失去了理智。

    “他们是谁?”他想弄个清楚。“确切点说,你觉得是谁想谋害你?”

    “他们中的每一个人,”约塞连告诉他说。

    “哪些人中的每一个人?”

    “你看呢?”

    “这我可说不上来。”

    “那你又怎么晓得他们不想杀我呢?”

    “因为……”克莱文杰语无伦次,随即又沮丧至极,缄口不语。

    克莱文杰确实自以为有理,但约塞连亦有他自己的证据,因为他每次执行空中轰炸任务,总会遭到陌生人的炮火袭击,这实在是毫无趣味的。假如说那种事无甚趣味,那其他许多事情更是没什么乐趣可言了。比如说,像流浪汉似地宿营皮亚诺萨岛上的帐篷,背靠崇山峻岭,面对蓝色大海——纵使风平浪静,却能于瞬息间吞噬水中的痉挛者,三天后,再把他冲回海岸,人就此一了百了,遍体青紫浮肿,且有海水慢慢地流出冰冷的鼻孔。

    他宿营的帐篷,依偎一片稀落晦暗的森林——于他和邓巴的中队之间自成一道屏障。紧靠帐篷一侧,是一条废弃的铁路壕沟,沟里铺设一根输送管,往机场的燃料卡车上运送航空汽油。多亏了与他同居的奥尔,他才有幸住进这间全中队最舒适的帐篷。约塞连每次从医院疗养回来或是从罗马休假返回营地,总会惊喜地发现,奥尔趁他不在时,又添了些新的生活设施——自来水,烧木柴的壁炉,水泥地板。帐篷是由约塞连择定地点,然后与奥尔合作搭建的。

 


    奥尔个头极矮,成天笑嘻嘻的,胸佩空军飞行徽章,一头浓密的褐色卷发,由正中向两边分开。他负责出谋策划。约塞连较他身高肩宽,强壮迅捷,因而,大部分粗活均由他承当。帐篷仅住他们两人,尽管很大,足以容纳六人。每当炎夏来临,奥尔便卷起帐篷侧帘,透些许清风,纵然,却是怎么也驱散不了帐篷内的暑气。

    约塞连的紧邻是哈弗迈耶。此人嗜食花生薄脆糖,独居一顶双人帐篷,每晚用四五口径手枪的大子弹射杀小田鼠。枪是从约塞连帐篷里那个死人身上窃得的。哈弗迈耶另一侧的邻居是麦克沃特,早先跟克莱文杰同住,但是约塞连出院时,克莱文杰尚未回来,麦克沃特便让内特利住进了自己的帐篷。眼下,内特利正在罗马,追求自己深恋着的那个妓女,可那妓女却是成日一副睡不醒的面容,早已深恶了自己的营生,对内特利亦生了厌倦。麦克沃特很疯狂。

    他是个飞行员,竟时常放大了胆开着飞机,从极低的高度掠过约塞连的帐篷,只是想看看约塞连会被吓成啥样。有时,他又极爱让飞机低飞,发出震耳欲聋的轰鸣声,掠过由空油筒浮载的木筏,再飞过洁白海滩处的沙洲,海滩那儿正有士兵赤裸着下海游泳呢。跟一个疯子合住一顶帐篷,实在不是件易事,但内特利并不在意。他自己也是个疯子,只要哪天有空,便会赶去帮忙建造军官俱乐部——

    于此,约塞连可是没曾插过手的。

    其实,许多军官俱乐部营建时,约塞连都不曾帮什么忙,不过,皮亚诺萨岛上的这个俱乐部,倒是最令他得意。这实在是为了他的果断坚毅而竖起的一幢坚实牢固、构造复杂的纪念碑式建筑。俱乐部竣工以前,约塞连从未上工地搭把手,之后,他倒是常去。俱乐部用木瓦盖的屋顶,外观极漂亮,尽管大而无当,他见了,满心欢喜。

    说实话,这幢建筑的确很壮观。每当举目凝望时,约塞连内心总升腾起一股极强的成就感,尽管他意识到自己从未为此流过点滴汗水。

 


    上一回,他和克莱文杰曾相互谩骂对方是疯子,当时,他们有四人在场,一起围坐在军官俱乐部里的一张桌子旁。他们坐在后面,紧挨那张双骰子赌台,阿普尔比一上这赌台,总会想办法赢钱。

    阿普尔比精于掷骰子,就如他擅长打乒乓一样,而他擅长打乒乓,就如他善于应付其他任何事情一样。阿普尔比每做一件事,都做得相当出色。阿普尔比是个衣阿华年轻人,长一头金发,信奉上帝、母爱和美国人的生活方式,尽管他对这一切从来都不曾做过什么周至的思虑。熟稔他的人,对他都颇有好感。

    “我恨那个狗娘养的,”约塞连怒吼道。

    同克莱文杰吵架,是早几分钟的事。当时,约塞连想找一挺机关枪,但结果没有找到。那天晚上极是热闹。酒吧间熙熙攘攘,双骰子赌台和乒乓台上压根没见空闲的时候,煞是一派繁忙的气象。

    约塞连想用机枪扫射的那帮人,正在酒吧间里劲头十足地吟唱那些百听不厌的古老的感伤歌曲。他没有用机关枪向他们射击,倒是用脚跟狠狠地踩了一下正朝他滚来的那只乒乓球,这球是从两名打球的军官之一的球拍上掉落下来的。

    “约塞连这家伙,”那两个军官摇了摇头笑道,随后便从架上的盒里又取了一只球。

    “约塞连这家伙,”约塞连回了他们一句。

    “约塞连,”内特利向他低声警告。

    “你们懂我的意思?”克莱文杰问。

    听到约塞连学舌,那两个军官又笑道:“约塞连这家伙。”这回,声音更响。

    “约塞连这家伙,”约塞连又照着说了一句。

    “约塞连,你行行好,”内特利恳求道。

    “你们懂我的意思?”克莱文杰问,“他有反社会的敌对心理。”

    “唉呀,你给我闭嘴吧,”邓巴对克莱文杰说。邓巴喜欢克莱文杰,原因是,克莱文杰常惹他恼火,仿佛让时间走慢了些。

    “阿普尔比根本没上这儿来,”克莱文杰洋洋得意地对约塞连说。

    “谁在说阿普尔比?”约塞连想弄个清楚。

    “卡思卡特上校也没来。”

    “谁又在说卡思卡特上校?”

    “那你究竟恨哪个狗娘养的?”

    “哪个狗娘养的在这儿?”

    “我不想跟你吵。”克莱文杰下定了决心。“你自己都不清楚恨谁。”

    “谁想毒死我,我就恨谁,”约塞连告诉他说。

    “没人想毒死你。”

    “他们在我吃的东西里下过两次毒,是不是有这回事?一次是弗拉拉战役,一次是博洛尼亚围攻大战役,他们是不是这么干过?”

    “他们在每个人的食物里都下过毒,”克莱文杰解释道。

    “那又有啥不同?”

    “那根本不是什么毒药!”克莱文杰很激动地大叫道。他愈发慌乱,也就愈发加重了自己说话的语调。

    约塞连耐了性子,微笑着给克莱文杰做解释,就他的记忆所及,有人一直想谋害他。有人喜欢他,也有人不喜欢他;不喜欢他的那些人便恨他,想尽办法害他。他们恨他,就因为他是亚述人。但是,他对克菜文杰说,他们别想碰他一下,因为他的躯体纯洁,灵魂健全,体壮如牛。他们别想碰他一下,因为他是泰山,曼德雷克,霹雳火戈登。他是比尔-莎士比亚。他是该隐,尤利西斯,漂泊的荷兰水手。他是所多玛的罗得,忧伤的黛特,树林里夜莺群中的斯威尼。他是神奇人物Z——247,他是——

    “疯子!”克莱文杰打断他的话,锐声叫喊,“你是个十足的疯子!”

    “——与众不同,我的的确确是个非同寻常、长了三头六臂的了不起的人物。我是个真正的奇人。”

    “超人?”克莱文杰嚷道,“超人?”

    “奇人,”约塞连纠正道。

    “嘿,伙计们,别争啦。”内特利很是尴尬地恳求他俩。“大伙儿都瞧着咱们哩。”

    “你是个疯子!”克莱文杰大叫,激动得热泪盈眶。”你心理变态,想做耶和华。”

    “我想人人都是拿但业。”

    克莱文杰突然中止了自己的慷慨陈词,面露猜疑状。“谁是拿但业?”

    “拿但业是谁?”约塞连故作无知地问道。

    克莱文杰知道是圈套,极乖觉地避了过去。“你觉得人人都是耶和华。说实话,你跟拉斯柯尔尼科夫没什么不同。”

    “谁?”

    “——没错,拉斯柯尔尼科夫,他——”

    “拉斯柯尔尼科夫!”

    “——他——我说的是实话一他以为自己杀了个老太婆,是正当合法的。”

    “我跟他没什么不同。”

    “——是这样的,杀了人,再替自己开脱,千真万确——用斧头砍死!我可以用事实证明,让你心服口服。”克莱文杰喘吁吁地一一列数了约塞连的种种症状:无缘无故地把周围所有的人视作疯子;

    一见陌生人,便顿生杀机,想用机枪扫射;好怀旧,却又时常颠倒过去的黑白;凭空猜疑别人憎恨他,一直合谋着想害他。
 


    但约塞连知道自己没错,因为正如他曾给克莱文杰解释的那样,他很清楚自己从来就没错过。他目光所及,处处是疯子,而在这疯子充塞的世界里,唯有像他自己这样明智而有教养的年轻人,方能明察事理。他必须如此,因为他明白他的生命危在旦夕。

    约塞连出院归队时,不管遇见谁,总要警惕地审视一番。米洛亦离开中队,去了士麦那,忙着收获无花果。尽管米洛不在,但食堂照常运转,医院和中队驻地之间,蜿蜒了一条崎岖的道路,恰似断裂的吊袜带。约塞连人还坐在救护车的驾驶室里,沿那条路颠簸前行时,便闻到了羔羊肉的扑鼻香味,顿生津液,食欲大起。午餐吃的是烤肉,一块块又大又香的肉用炙叉串着搁在木炭上,烤得咝咝直响。这肉烤前需在一种用秘方配制的卤汁里浸泡七十二小时,而秘方是米洛从黎凡特的一个刁滑奸商那里窃取来的。食用烤肉时,需拌上伊朗大米和芦笋尖帕尔马干酪,接着上的便是樱桃甜食,再来是一杯杯热气腾腾的用新磨的咖啡豆煮出来的咖啡,里面还掺了本尼迪克特甜酒和白兰地。午餐分成若干份,由熟练的意大利侍者端上铺着织花台布的餐桌。这些侍者,由德-科弗利少校从欧洲大陆诱拐得来后,交送给米洛。

    约塞连在食堂里拼命大吃,直到觉得肚子快要胀破,方才心满意足,一动不动地瘫靠在坐椅上,嘴里还含着薄薄的一层残菜渣。

    交米洛的食堂里,中队所有的军官时常品尝珍馐美味,除此之外,谁也不曾如此畅快地大饱口福。约塞连思忖片刻,或许还真划得来呢。可是,他接着打了嗝,想了起来:他们一直想杀他。于是,他猛冲出食堂,跑着去找丹尼卡医生,请求免除自己的作战任务,把他遣送回家。他找到了丹尼卡,医生正坐在自己帐篷外的一只高凳上晒太阳。

    “完成五十次飞行任务,”丹尼卡医生摇着头跟他说,“上校要求飞满五十次。”

    “可我才飞了四十四次!”

    丹尼卡医生却无动于衷。这家伙长得像只鸟,老是愁眉苦脸的模样。那张脸酷似一柄刮刀,上宽下尖,修刮得光溜溜的,极像一只刷洗干净的耗子。

    “完成五十次飞行任务,”他还是摇了摇头,又说了一遍。“上校要求飞满五十次。”

 
上一页	《第二十二条军规》	下一页
 
line
  书坊首页	努努书坊 版权所有
