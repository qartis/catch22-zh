\chapter{勇敢的米洛}
 
    约塞连平生头一遭下跪求人了。他双膝跪在内特利面前,求他不要主动要求执行七十次以上的战斗飞行任务,可内特利怎么也不肯听他的话。一级准尉怀特-哈尔福特果然在医院里死于肺炎,内特利己经申请接替他去完成飞行任务。

    “我非得多飞几次不可,”内特利强词夺理地坚持道,脸上浮现出一丝狡诈的微笑。“不然他们就要送我回国了。”

    “那又怎么样?”

    “只有当我能带她跟我一块回去时,我才会愿意回国。”

    “她对你就这么重要吗?”

    内特利沮丧地点点头,“我也许永远见不到她了。”

    “那你就停飞,”约塞连怂恿道,“你已经完成了你的飞行任务,你又不需要飞行津贴。如果替布莱克上尉干活你都能受得了的话,你又何必申请接替一级准尉怀特-哈尔福特的职务呢?”

    内特利摇了摇头。他又是害臊又是悔恨,脸色沉了下来。“他们不会让我停飞的。我找科思中校谈过,他告诉我说,要么多飞几次,要么送我回国。”

    约塞连粗野地骂了一句。“这简直卑鄙到了极点。”

    “我觉得我不在乎。我已经飞了七十次了,还没受过伤呢。我想我还能够再多飞几次。”

    “在我找人谈谈之前,你什么事都不要干。”约塞连拿定了主意,便去找米洛帮忙。米洛随即向卡思卡特上校请求帮助,要求分配给他更多的战斗任务。

    米洛一直在为自己赢得一项又一项的荣誉,他曾经无所畏惧地冒着危险和责难,以很好的价钱把石油和滚珠轴承卖给德国,不仅赚了一大笔钱,而且还帮着维持住了交战双方的力量均势。他在炮火下谈笑风生,沉着镇定。为了全力以赴做本职以外的工作,他拼命抬高食堂的伙食价格,弄得全体官兵为了填饱肚子不得不拿出全部薪水支付给他。他们的另一个选择——当然,是有另一个选择的,因为米洛不喜欢强迫别人,言谈之中一向主张自由选择——

 


    就是挨饿。当他的提价攻势遭到敌对势力的抵制时,他坚守阵地寸步不让,丝毫没有顾忌到自身的安危和名声,并且果敢地援引供求法则作为自卫武器。当有的地方有人说不行时,他会勉勉强强地退却,但即使在撤退当中,也敢于捍卫自由人所具有的历史性的权利,即为了获得维持生命的必需品,人们必须付出他们应付的钱款。

    米洛掠夺自己的同胞时,曾经被当场抓获过。作为这种掠夺的结果,他的股份总额到达了前所未有的高度。他说话一向算数。有一回,一个来自明尼苏达州的骨瘦如柴的少校撇着嘴唇向米洛发难,要求退出联营机构,抽回自己的那份股金,因为米洛口口声声说每个人在联营机构里都有股份。面对他的挑战,米洛顺手拿起手边的一张纸条,在上面写上“一股”两个字,鄙夷地递了过去,从而赢得了几乎所有认识他的人的羡慕和钦佩。米洛的荣耀目前正处在顶峰。对于他的战斗业绩,卡思卡特上校既清楚又敬佩,所以,当米洛来到大队部,毕恭毕敬地提出一个荒谬绝伦的请求,要求给他分派更多的危险任务时,卡思卡特上校不禁大吃一惊。

    “你想多执行几次战斗任务吗?”卡思卡特上校气呼呼地问,“这究竟是为了什么?”

    米洛恭顺地低下头,故作拘谨地回答道:“我想尽我的一份职责,长官。我们的国家在打仗,我想和其他人一样,为保卫祖国而战斗。”

    “可是,米洛,你正在尽你的职责呢,”卡思卡特上校快活地哈哈大笑起来。“我想不出还有哪一个人为部队做的事比你做的多。

    是谁让他们吃上裹着巧克力的棉花糖的?”

    米洛伤心地慢慢摇了摇头。“可是,在战时仅仅做一名优秀的司务长是不够的,卡思卡特上校。”

    “当然是够的,米洛,我不知道你这是怎么啦?”

    “当然是不够的,上校。”米洛颇有几分坚决地表示异议。他恰到好处地抬起充满谄媚的双眼,意味深长地与卡思卡特上校对视了一下。“有些人开始说闲话了。”

    “噢,就为这个?把他们的名字写给我,米洛,把他们的名字写给我,每逢大队有危险的飞行任务时,我就派他们去,我会做到这一点的。”

 


    “不,上校,我想他们是对的。”米洛说着又低下了头,“我是作为飞行员被派到海外来的,我应该完成更多的战斗飞行任务,而在食堂管理的工作上,我应该少花点时间。”

    卡思卡特上校虽然很吃惊,但还是愿意帮助他。“好吧,米洛,如果你真的这样认为,我敢肯定,无论你要求什么,我们都会作出安排的。你来海外有多长时间了?”

    “十一个月了,长官。”

    “你执行过多少次飞行任务了?”

    “五次。”

    “五次?”卡思卡特上校问。

    “五次,长官。”

    “五次,是吗?”卡思卡特上校沉思地摸了摸自己的面颊。“这不算太好,对吗?”

    “不算太好?”米洛用刺耳的声音反问道,同时又抬眼扫视了他一下。

    卡思卡特上校心里一阵慌乱。“不不,相反,这非常好,米洛,”他连忙改口说道,“这确实不错。”

    “不,上校。”米洛懒洋洋地、愁眉苦脸地长叹一声。“这不算太好,你这么说真是太宽宏大量了。”

    “但这确实不错,米洛,的的确确不惜,想想你另外的那些宝贵贡献吧。你是说五次吗?就五次吗?”

    “就五次,长官。”

    “就五次。”卡思卡特上校弄不清楚米洛究竟是怎么想的,更不知道自己是不是已经被米洛给耍弄了。一时间,他变得非常沮丧。

    “五次就非常好了,米洛。”他热情洋溢地发着议论,似乎看到了一线希望。“平均起来算,你差不多每两个月执行一次战斗飞行任务。

    我敢说,你的飞行总次数没有把你袭击我们的那一次包括进去。”

    “不,长官,包括进去了。”

    “包括进去了了?”卡思卡特上校略显困惑地问,“执行那一次任务时,你实际上没有飞行,对吗?如果我没记错的话,你是和我一起呆在指挥塔台上的,不是吗?”

    “但那是我的飞行任务,”米洛分辩道,“那是由我组织的,使用的也是我的飞机和给养,我策划并监督了执行那次任务的全过程。”

 


    “噢,当然喽,米洛,当然喽。我不和你争论。我不过是在核对一下数字,以便弄清楚你是不是把你所执行的飞行任务都包括进去了,你把你跟我们签约去轰炸奥尔维那托大桥的那一次也包括进去了吗?”

    “噢,不,长官,我认为不应当包括进去。因为当时我在奥尔维那托指挥防空炮火。”

    “我看不出这有什么区别,米洛。这仍然是你的飞行任务,而且我必须指出,这次任务你完成得极为出色。我们没有炸掉大桥,可我们的炸弹散布面非常漂亮。我记得佩克姆将军曾经提到过这件事。不,米洛,我坚持认为你应当把轰炸奥尔维那托也算作你的一次飞行任务。”

    “如果你坚持认为的话,好吧,长官。”

    “我坚持认为,米洛。现在,让我们算算看——你总共执行了六次飞行任务,这真是好极了,米洛,的确好极了。就在一两分钟之内,你的飞行次数就增加了百分之二十。这确实不错,米洛,确实不错。”

    “别的许多人已经执行了七十次飞行任务了,”米洛指出。

    “但他们从来没有做出过裹了巧克力的棉花糖,不是吗?米洛,你的贡献已经超过你应尽的职责了。”

    “但他们正在获得各种各样的荣誉和机会,”米洛急红了脸,坚持道,眼泪似乎马上就要掉下来了。“长官,我想参加进来,和其他人一样飞行作战。这就是我今天为什么来这儿的原因,我也想得几枚勋章。”

    “是啊,米洛,那当然。我们都想把更多的时间花在参加战斗上,可是,像你和我这样的人,服役的方式是跟别人不同的,你看看我的记录吧。”卡思卡特上校不以为然地笑了笑,“我敢说,没有几个人知道,米洛,我本人总共只执行过四次飞行任务。没人知道吧?”

    “没人知道,长官,”米洛回答道,“一般人只知道你仅仅执行过两次飞行任务,而且其中一次是阿费驾机送你去那不勒斯买黑市冰箱,当时你们一不当心飞进了敌人的领空。”

    卡思卡特上校窘得面红耳赤,再也不愿意争论下去了。“好吧,米洛,对于你执行飞行任务的愿望,我是非常赞赏的。如果这对你真的这么重要的话,我会叫梅杰少校把其余的六十四次飞行任务派给你,这样你也就可以飞满七十次了。”

    “谢谢你,上校,谢谢你,长官。你不知道这意味着什么。”

    “别说了,米洛。这意味着什么,我知道得一清二楚。”

    “不,上校,我认为你并不知道这意味着什么,”米洛直率地反驳说,“马上就得有个人来替我管理联营机构。这项工作非常复杂,而且,我又随时可能被击落下来。”

    听到这话,卡思卡特上校顿时容光焕发,两只手开始贪婪地、急不可耐地搓来搓去。“你知道,米洛,我想科恩中校和我将会很愿意从你手里接管联营机构,”他不假思索地建议道,就像闻到了什么美味佳肴似的舔着嘴唇。“我们俩做红色梨形番茄黑市买卖的经验会很有帮助的。我们从哪儿开始交接呢?”
 


    米洛露出一副和蔼而又直率的表情,目不转睛地望着卡思卡特上校。“谢谢你,长官,你真是太好了。我们就从佩克姆将军的无盐饮食和德里德尔将军的脱脂饮食开始吧。”

    “让我拿支铅笔。下一项是什么?”

    “雪松。”

    “雪松?”

    “来自黎巴嫩的雪松。”

    “来自黎巴嫩的?”

    “我们从黎巴嫩弄来雪松,打算把它们运到奥斯陆的木材加工厂去加工成木瓦,再卖给科德角的营造商。货到付款。下一项是豌豆。”

    “豌豆?”

    “它们在公海上呢。我们现在有好几船豌豆正从亚特兰大运往荷兰,全在公海上呢。我们要拿它们抵付山慈姑的货款。那些山慈姑已经运往日内瓦去抵付必须运往维也纳的乳酪的货款,M-I-F。”

    “M-I-F-?”

    “就是货款预付。哈布斯堡王室不可靠。”

    “米洛。”

    “接下来是弗林特仓库里的电镀锌。不要忘记,弗林特的四卡车电镀锌必须在十八号中午以前空运到大马士革的冶炼厂,以离岸价格结算。月底前十天内,再把百分之二的电镀锌运到加尔各答去。接下来是一架满载大麻的梅塞施米特战斗机预定飞往贝尔格莱德,我们将用它们去交换装了一架半C-47型运输机的去核椰枣,这些椰枣是我们从喀土穆运过来硬塞给他们的。接下来的一项是把葡萄牙鳗鱼倒卖回里斯本,再用这钱去支付我们从马马罗内克倒卖回来的埃及棉花的货款。另一项是尽量从西班牙多弄些桔子来。Naranjas一向是用现款支付的,”“Naranjas?”

    “他们在西班牙就是这样叫桔子的,这些都是西班牙桔子。还有——噢,对了,别忘了辟尔唐人。”

    “辟尔唐人?”

    “是的,辟尔唐人。美国国立博物馆眼下出不起我们开出的第二个辟尔唐人化石的价钱,他们正眼巴巴地盼着哪位富有的、受人爱戴的施主早点呜呼哀哉——”

    “米洛。”

    “我们能运过去多少欧芹,法国人就想收购多少,我想我们还是尽量多运,因为我们需要用法郎去兑换里拉和芬尼,以便买下被倒卖回来的椰枣。我们还订购了一大批秘鲁轻质木材,将按比例分配给联营机构下属的每一个军人食堂。”

    “轻质木材?军人食堂要这些轻质木材干什么?”

    “眼下这种优等轻质木材不容易搞到,上校。我认为放过这个购买机会是很不明智的。”

    “是的,我也认为不明智,”卡思卡特上校模棱两可地附和道,脸上浮现出晕船人的神情。“我想,价钱挺公道吧。”

    “价钱嘛,”米洛说,“说来叫人生气——实在是太贵了:但因为我们是从我们自己的一个子公司购买的,我们还是乐意付钱的。下一项是照管好兽皮。”

    “蜂房。”

    “兽皮。”

    “兽皮?”

    “兽皮。在布宜诺斯艾利斯。必须把它们制成皮革,”“制成皮革?”

    “在纽芬兰制成皮革,然后在开春冰消雪化之前用船把它们运到赫尔辛基去,N-M-IF。开春冰消雪化之前所有运往芬兰的货物都是N-M-I-F。”

    “货款不预付吗?”卡思卡特上校猜道。

    “不错,上校。你有天才,长官。下一项是软木塞。”

    “软木塞?”

    “必须把它们运往纽约,还有要运往图卢兹的鞋子,要运往暹罗的火腿,从威尔士运来的钉子,从新奥尔良运来的柑橘。”

    “米洛。”

    “还有我们存放在纽卡斯尔的煤,长官。”

    卡思卡特上校举起双手。“别说了,米洛!”他大叫道,眼泪都快要掉下来了。“说也没有用。你就和我一样——是不可缺少的!”他把铅笔推到一边,怒不可遏地站起身来”“米洛,你不能去执行那六十四次飞行任务,一次都不行。要是你出了什么事,整个系统就算全完了。”

    米洛平静地点了点头。他感到心满意足洋洋自得。“长官,你是禁止我再去执行任何一次飞行任务咯?”

    “米洛,我禁止你再去执行任何一次飞行任务,”卡思卡特上校用严厉的、毫无商量余地的长官口吻说道。

    “但是,这不公平,长官,”米洛说,“我的作战记录怎么办?其他人可是正在获得荣誉、勋章和名声呢。为什么我应当吃这个亏,难道就因为我把司务长的工作干得很好吗?”

    “是的,米洛,这是不公平。但是我想不出怎么才能解决这个问题。”

    “也许我们可以找个人替我执行飞行任务。”

    “对呀,也许我们可以找个人替你执行飞行任务,”卡思卡待上校建议道,“找宾夕法尼亚州或西弗吉尼亚州罢工的矿工怎么样?”

    米洛摇摇头。“训练他们要花太多的时间,为什么不找中队里的人呢,长官?我毕竟是在为他们干这一切事情。他们应当乐意为我干点事情,作为对我的报答。”
 


    “对呀,为什么不找中队里的人呢,米洛?”卡思卡特上校叫道,“不管怎么说,你是在为他们干这一切事情,他们应当乐意为你干点事情,作为对你的报答。”

    “这才是公平交易。”

    “这才是公平交易。”

    “他们可以轮流干,长官。”

    “他们可以轮流替你执行飞行任务,米洛。”

    “功劳算在谁的帐上呢?”

    “功劳当然算在你的帐上,米洛。如果谁在执行你的飞行任务时得了勋章,那勋章就归你。”““如果他送了命,那么死的是谁呢?”

    “死的当然是他咯。这毕竟是公平交易嘛。这样就只剩下一件事了。”

    “你必须增加飞行任务的次数。”

    “也许,我必须再次增加飞行任务的次数,可我拿不准他们是不是愿意执行。就因为我把飞行次数增加到七十次,他们到现在还气得要命呢。要是我能让某一个常备军官再多飞几次,其余的人也许就会跟着飞了。”

    “内特利愿意多执行几次飞行任务,长官,”米洛说,“刚刚有人私下里对我泄露说,为了想留在海外,跟一个他所爱的姑娘呆在一起,他什么都愿意干。”

    “对呀,内特利愿意再多飞几次!”卡思卡特上校宣布说。他把双手往一块啪的一拍,以庆贺自己的胜利。“是的,内特利愿意多飞几次。这一回,我可真的要把飞行次数一下子增加到八十次了,这下子准把德里德尔将军的眼珠子气得鼓出来。这也是让约塞连那个下流畜生重新参战的好办法,也许这一次就送了他的命呢。”

    “约塞连?”米洛那张单纯朴实的脸上闪过一层忧虑的阴影。他若有所思地挠了挠他那红褐色的胡子尖。

    “是啊,是约塞连。我听说他到处宣扬他已经完成了他的飞行任务,说什么战争对他来说已经结束了。哼,也许他已经完成了他的飞行任务,可是他还没有完成你的飞行任务呢,是吧,哈!哈!这一回他可要大吃一惊啦!”

    “长官,约塞连是我的一个朋友,”米洛反对道,“我可不愿意承担使他重新参战的罪责。我欠约塞连一大笔人情。我们有没有什么办法可以使他成为一个例外呢?”

    “噢,不,米洛。”卡思卡特上校故作严肃地啧啧了几声。这个建议使他大为震惊。“我们绝不应该偏心眼。我们应该对所有的人一视同仁。”

    “我倒是甘愿为约塞连献出一切的。”米洛继续固执地替约塞连说情。“可是既然我并不拥有一切,我也就没法为他献出一切,对吧?所以,他只好跟其他人一样去冒冒险了,对吗?”

    “这才是公平交易,米洛。”

    “是的,长官,这才是公平交易。”米洛表示同意。“约塞连并不比别人出色,他没有权利享受任何特权,对吗?”

    “对的,米洛。这才是公平交易。”

    卡思卡特上校当天傍晚就宣布把飞行次数增加到八十次。第二天拂晓,警报突然响了起来,空勤人员没来得及等到早饭做好就被赶上卡车,以最快的速度运到简令下达室,接着又运到机场。因此,约塞连根本没有时间逃避战斗任务,更没有时间再次去跟多布斯密谋暗杀卡思卡特上校。机场上,咔哒咔哒的加油车把汽油灌压进飞机油箱,匆匆忙忙的军械士费劲地尽可能快地把一颗颗重这一千磅的爆破炸弹吊起装入飞机炸弹舱。人人忙着跑来跑去。加油车一加完油,引擎马上发动起来,准备起飞。

    情报部门报告说,就在那天早上,德国人打算把停泊在斯培西亚干船坞里的一艘报废的意大利巡洋舰拖到港湾入口处的水道上炸沉,以使盟军部队攻占该市后无法使用深水港湾设备。这一回,军方的情报倒是准确的。当美国人从西边飞过来时,那艘巡洋舰正好给拖到了港湾水道中间。他们轮番俯冲,每回都直接击中了目标,最后把它炸得七零八落。于是他们一个个全都洋洋得意,为他们的飞行大队感到无比自豪。就在这时,他们突然发现自己陷入了高射炮火力网的包围之中。下面的陆地上层峦叠障,看上去像一个巨大无比的马蹄。炮火呼啸着从这块马蹄形陆地的每一个隐蔽处飞向空中。就连哈弗迈耶也使出浑身解数做起最狂野的规避动作来了,因为他看到自己必须飞很长一段距离才能逃出火力网。多布斯驾机在之字形编队中飞行时,应该往右转时他却突然往左急转,结果他的飞机一下子撞到了旁边的飞机上,把那架飞机的尾翼给撞掉了。他自己飞机的一侧机翼也从根部折断,飞机像一块大石头似的落了下去,一转眼就不见了。没看见火,没看见烟,甚至没听见哪怕最轻微的不祥之声。剩下的那一侧机翼像只水泥搅拌器似的笨重地旋转着,与此同时,飞机正头朝下直直地向下栽去,速度越来越快,最后猛然撞到水面上,激起了一圈圈泡沫,仿佛深蓝色的海面上突然绽开一朵雪白的睡莲。随着飞机的下沉,无数果绿色的水泡向海面喷涌而去。几秒钟之后,飞机便无影无踪了。没有看见降落伞。此时,在刚才被撞的另一架飞机里,内特利也送了命
TODO
