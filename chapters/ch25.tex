\chapter{随军牧师}
 
    很久以前随军牧师便开始在心里起了疑惑,世间的一切究竟是怎么回事?到底有没有上帝,他怎么能肯定呢,身为美国军队中的一名浸礼教牧师,即便在最顺利的情况下,处境就够艰难的了;若再没了信仰,那境况就几乎无法容忍了。

    那些大嗓门的人总让他感到害怕。像卡思卡特上校那样无所畏惧、敢做敢为的人总让他感到自己孤立无助,形单影只。在军中,无论他走到哪里,他总像个局外人似的。官兵们在在他面前总不及在别的官兵面前那么自在;连其他的牧师对他也不如他们彼此之间那么友好。在一个以成功为唯一美德的世界里,他自认自己是个失败者。一名教士应当镇定自若,且能随机应变。他痛苦地认识到,自己缺乏教士应具备的这种基本素质,而其他教派的那些同僚就因为具有这两点而干得相当出色。他生就没有胜过别人的本领。他认为自己丑陋不堪,没有一天不想立即回家去与妻子团聚。

    其实,牧师的长相几乎是英俊的。他有一张讨人喜爱而又显得十分敏感的脸,像沙岩一样苍白、脆弱。他的思想相当开放。

    也许,他真的是华盛顿-欧文。也许在一些信件上他一直都签的是华盛顿-欧文的姓名,尽管对此他一无所知。他知道,在医学史上,这种记忆错误是很常见的。他也明白,要想真正将什么事情都弄清楚是办不到的,甚至连为什么办不到也是无法知晓的。他清楚地记得——或者说他有印象清楚地记得——他见到约塞连时的那种感觉;他觉得在他第一次看到约塞连躺在医院里的病床上之前,就已经在什么地方见过他。他记得,大约两周以后当约塞连再次出现在他的帐篷,要求免除他的战斗任务时,他产生了同样的不安的感觉。当然,在此之前牧师已的确在什么地方见过他,就是在那间临时的、非正规的病房里。那个病房里的每个病人看上去都为怠工而来,只有一名不幸的病人除外。那人浑身上下敷着石膏,绑着绷带。一天人们发现他就这么死了,嘴里还含着温度计。但是在牧师的印象中,在此之前他就在某个更为重大、更为神秘的场合见过约塞连。那次有意义的会面是在某个遥远的、为时间的烟尘所淹没的、甚至是在纯属超现实的时代里发生的;而那次,他也曾同样命中注定地承认:他没有办法,没有任何办法可帮助约塞连。

    这样的疑虑一刻不停地折磨着牧师那瘦削、多病的躯体。世上有没有哪怕是一种真正的信仰,或者人死后究竟有没有灵魂?有多少天使能够在一根大头针的针尖上跳舞?上帝在创造万物之前的那段漫长岁月里究竟在忙活些啥?如果没有其他的什么人需要防范,那有何必要在该隐的前额打上个保护的印记呢?亚当和夏娃真的生过女儿吗?这些就是一直不断地折磨着他的重大而又复杂的本体论问题,然而,在他看来,这些问题从来就不及善良和礼貌等问题来得重要。那些怀疑论者在认识论方面进退维谷的困境让他急得冒汗,他不能接受对一些问题的解释,可又不情愿将问题视为无法解释而不予理会。他从来都是处在痛苦之中,可又一直心怀希望。

 


    那天约塞连坐在他的帐篷里,手里捧着一瓶热乎乎的可口可乐。这可乐是牧师为了安慰他才给他的。牧师犹豫不决地问道:

    “你有没有过这样的感觉:你明明知道你是第一次碰到某一情形,但你却感到你过去好像经历过它?”约塞连敷衍地点了点头。牧师的呼吸由于急切的期待而变得急促起来,因为他准备让自己的意志与约塞连的联合起来,同心协力,最终揭开像巨大的黑幕一样笼罩在人类生存之上的永恒奥秘。

    约塞连摇了摇头,接着解释说,所谓dejavu不过是两根共同活动的感觉神经中枢——他们通常是同时起作用的——在瞬间产生的极细微的时间差。他的话牧师几乎没听进去。他感到很失望,但他不愿相信约塞连的话,因为他曾得到过一个征兆,一个秘密而又不可思议的幻觉,那就是约塞连仍然缺乏勇气,不敢将真话说出来。无疑,在牧师所揭示的事情中有着令人敬畏的含义,这就是:它要么是一种神赐的顿悟,要么是一种幻觉;他本人不是得到了神灵的垂青就是丧失了理智。这两种可能使他内心充满了同样的恐惧和沮丧。这既不是dejavu,也不是presquevu或jamaisvu。很可能还有他从未听说过的其他幻觉,其中之一可以简单明了地解释他亲眼看见并亲身经历过的令人困惑的种种现象。也有这些可能:

    可能他以往以为会发生的事情压根就没发生过;可能他患了记忆方面而不是感觉方面的毛病;可能他从来也没真正认为他亲眼见过现在他自认为过去一度曾以为自己见过的东西;可能对于他曾一度以为是的东西,他现在的印象只不过是幻党中的幻觉;可能他只是想象自己曾经在想象中看见过一个赤身裸体的男人坐在公墓里的一棵树上。

 


    显然,牧师现在已意识到自己并不特别适合干目前的这份工作。他常常考虑,如果他到部队的某一其他部门去服役,比如说去步兵或野战炮兵部队当一名列兵,或者甚至去当一名伞兵,是不是会比现在开心点。他没有真正的朋友。在没遇到约塞连之前,在飞行大队的任何一个人面前他都会感到不自在,即使同约塞连相处,他也感到局促不安。约塞连常常表现得十分粗鲁,并不时爆发出一些反抗行为,这常使得他感到紧张不安,并伴有一种说不出来的心情,既开心又惶恐。当牧师同约塞连和邓巴一起呆在军官俱乐部里,甚至同内特利和麦克沃特呆在一起时他才感到安全。同他们在一起,他便无需再与其他人坐在一起了;他该坐在哪儿的问题也就解决了,他用不着再同那些他不喜欢的军官坐在一起了。平时,每当他走近这些军官时,他们无一例外地用过分的热情来欢迎他的到来,然后又非常不自在地等着他离去。他使得那么多的人不舒服。大伙都对他非常友好,但没有一个人真心待他。人人都同他说话,但没有一人同他说过真心话。约塞连和邓巴要随和得多,同他俩在一起,牧师几乎没有什么不自在的感觉。那天晚上,当卡思卡特上校又一次想把牧师从军官俱乐部撵出去时,他俩甚至还保护了他。当时约塞连气势汹汹地站了起来要进行干预,内特利想阻止他,就大叫了一声“约塞连!”卡思卡特上校一听到约塞连的名字,脸色顿时煞白,而且让大家感到吃惊的是,他吓得六神无主,一个劲地往后退,最后竟撞到了德里德尔将军的身上。将军气恼地用胳臂肘将他推开,并命令他立即回到牧师面前,叫他从今天开始每晚都到军官俱乐部来。

    牧师要想保持他在军官俱乐部的地位是很难的,就同他想记往下一餐他该在大队的十个食堂的哪一个食堂就餐一样难。要不是如今他在军官俱乐部里从他的那些新伙伴那里找到了乐趣,他倒很愿意被人从那儿撵出来。晚上如果牧师不去军官俱乐部,那他也就没地方可去了。他时常坐在约塞连和邓巴的桌旁消磨时光,羞怯、沉默地微笑着,除非别人同他说话,否则他便一言不发。他的面前总是放着一杯浓浓的甜酒,可他几乎一口也不尝,只是不熟练地、别别扭扭、装模作样地玩弄着一只用玉米芯做成的烟斗,偶尔也往里面塞些烟丝,抽上几口。他喜欢听内特利讲话,因为内特利酒后说出的那些伤感的、又苦又乐的话在很大程度上反映出了牧师本人那充满了浪漫情调的孤寂惆怅,并且总能引发起牧师对妻儿的思念,使他的心情如潮水一样久久不得平静。内特利的坦率和幼稚让牧师感到有趣,他频频地朝着内特利点头表示理解和赞同,以鼓励他继续说下去。内特利还没有冒失到会向人夸耀自己的女朋友是个妓女的程度,牧师之所以会知道这事主要是由于布莱克上尉的缘故。每当布莱克上尉懒洋洋地从他们的桌旁经过时,他总要先使劲朝牧师眨眨眼,然后就转向内特利,就他的女友将他嘲弄一番,说出来的话既下流又伤人。牧师对布莱克上尉的这种做法很是不满,因此就产生了一个按捺不住的念头,那就是希望他倒大霉。

 


    似乎没有人,甚至连内特利也不例外,真正意识到他,艾尔伯特-泰勒-塔普曼牧师,不光是个牧师,而且也是个活生生的人。

    没人意识到他还有个漂亮迷人、充满激情的妻子——让他爱得几乎发狂,三个蓝眼睛的小孩,他们的相貌显得陌生,因为他已记不太清他们的模样了。将来有一天当他们长大了的时候,他们会将他视为一个怪物。他的职业会给他们在社会上带来种种尴尬,为此他们可能永远不会原谅他。为什么就没人明白他实际上并不是个怪物,而是一个正常、孤独的成年人,竭力想过一种正常、孤独的成年人的生活?假如他们刺他一下,难道他就不会出血吗?如果有人呵他痒,难道他就不会笑?看来他们从来就没想过,他,同他们一样,有眼、有手、有器官、有形体、有感觉、有感情。和他们一样,他也会被同样的武器所伤,因同样的微风而感到温暖和寒冷,并以同样的食物充饥,虽然在这一点上他被迫做出让步,每一顿都得去不同的食堂用餐。只有一个人似乎意识到了牧师是有感情的,这个人就是惠特科姆下士,而他所做的一切只是想方设法去伤害这些感情,因为正是他越过了他的上司去找卡思卡特上校,建议向阵亡或负伤士兵的家属寄发慰问通函。

    在这个世界上,唯一能让他感到踏实的就是他的妻子。如果就让他与妻儿们在一起过一辈子,那他也就满足了。牧师的妻子是个文静的小个子女人,和蔼可亲,年纪刚过三十,皮肤黝黑,富有魅力。她的腰身纤细,眼睛里流露出沉着和机灵;牙齿雪白,又尖又小,再配上一张孩子似的脸蛋,显得既生气勃勃又娇小可爱。牧师常常忘记自己孩子的长相,每次拿出孩子们的照片,总觉得好像是第一次见到他们的面孔。牧师就像这样爱着他的妻儿,这种爱简直强烈得不可遏制,以致他总想放弃强打精神的努力,就此瘫倒在地,像个被人遗弃的残废人那样放声大哭。围绕着他的家人,他产生了许多病态的怪念头,产生了许多悲惨、可怕的预感,不是想到他们得了重病就是认为他们遭到了可怕的意外。这些东西每天都在无情地折磨着他。他的思维也受到了这些念头的侵扰,尽想着他的妻儿可能得了诸如恶性骨癌和白血病之类的可怕疾病。每周他至少有二三次会看见他那刚出生不久的儿子夭折了,因为他从未教过妻子如何止住动脉出血。他还曾泪流满面、眼睁睁地一声不响地目睹了全家人在墙基插座旁一个接一个地触电而亡的情景,因为他从未告诉过妻子人体是会导电的。几乎每天夜里他都会看到,家里的热水锅炉发生了爆炸,他家那两层木结构的楼房燃烧了起来,他的妻儿四人统统被烧死;他还看到了一件恐怖、惨不忍睹、令人震惊的惨祸的全部细节:他可怜的爱妻那一向整洁而又娇弱的躯体竟被一个喝醉了酒的白痴司机撞到了市场大楼的砖墙上,压成了黏糊糊的一滩肉酱;他还看到,他那被吓得歇斯底里地哭个不休的五岁女儿被一个长一头雪白头发、面目慈祥的中年男子领着离开了那可怖的事故现场;那男人驱车把她带到一个废弃的采沙场,一到那里他就一次接一次地对他的女儿进行奸污,最后把她给杀害了;帮他照管孩子的岳母,从电话里得知了他妻子的惨祸,当即就发了心脏病,倒在地上死掉了。于是,他那两个年幼的孩子就在家里慢慢地饿死了。牧师的妻子是个和蔼可亲、总能给人以安慰并善于体贴的女人。牧师渴望能再一次触摸到她那匀称的胳臂上的肌肤,抚摸到她那乌黑、光滑的秀发,听到她那亲切、充满了安慰的嗓音。她是一个比他坚强得多的人。他每周一次,有时两次给她去一封内容简单而又干巴巴的短信,而内心里他成天想着要给她去许许多多封情真意切的情书,在那些数不清的信纸上热切地、无拘无束地向她表达自己的真情,告诉他自己是如何谦卑地崇拜她,需要她,还要极其详细地对她讲明人工呼吸的实施方法。他还想滔滔不绝地向她倾诉他对自己的怜悯以及自己所感受到的无法忍受的孤独和绝望,同时要嘱咐她千万不要将硼酸或阿司匹林等物放在孩子们够得着的地方,或者提醒她在过马路的时候一定要看红绿灯。他不想让她担心。牧师的妻子是个具有直觉、性格温柔、富有同情心并且生性敏感的女人。他成天做白日梦似地想着同妻子团聚的情景,而这种想象总是无可避免地以历历在目的做爱动作而告结束。
 


    让牧师最感虚伪的就是主持葬礼。如果说那天树上出现的鬼怪是上帝显灵,借以指责他对神明的亵渎和他在行使自己的职责时内心所感到的那种洋洋自得,那么,对此他一点都不会感到震惊。面对死亡这一可怕而又神秘的事件,却要装出一脸的庄严,故作悲伤之态,还要装得像神灵似的对人身后的情况有所知晓,这乃是罪过中的罪过。他清晰地回忆起——或者似乎相信自己回忆起——那天在公墓的情景。他至今仍能看见梅杰少校和丹比少校像两根残破的石柱似地肃立在他的两旁;看见与那天同样数目的士兵,以及他们那天所站立的位置;还看见了那四个拿着铲子对周围的一切都无动于衷的人,还有那令人厌恶的棺材和那个用红褐色的泥土松松垮垮地堆起来的、显得得意洋洋的巨大坟头,以及那广漠无垠、寂然无声、深不可测并令人感到压抑的天空。那天的天空出奇地空旷与蔚蓝,就这种场合来说,它几乎是带有一种恶意。

    他将会永远记住这些情景,因为它们是自他有生以来降临到他身上的最不寻常的事件的重要组成部分。这事件也许是一种奇迹,也许是一种病态的胡思乱想——就是那天出现在树上的那个裸体男子的幻象。他该怎么解释这个幻象呢?它既不是曾经见过的东西,又不是从未见过的东西,也不是几乎能见着的东西;无论是“曾经相识”,还是“似曾相识”或是“从不相识”,这些说法都不够圆满,不足以将它概括进去。那么它是鬼吗?是死人的灵魂?是天国的天使还是来自地狱的小鬼?或者这整个怪诞的事件只是他那病态的想象臆造出来的?难道他的思维发生了病变,或者是他的大脑朽烂了?树上竟然会有一个裸体的男人——实际上有二个,因为第一个人出现不久就跟来了第二个,那人唇上留着棕色的小胡子,从头到脚严严实实地裹在一件不祥的黑衣服里;只见他贴着树枝,像行宗教仪式似地向前弯下腰,将一只茶色的高脚酒杯递给前者,让他喝里面的东西。发生这种事的可能性以前从未在牧师的脑子里出现过。

    牧师是一个有真诚助人之心的人,只是他从来也没法帮助任何人,甚至连约塞连的这件事他也没帮上忙。当时他最终下定了挺而走险的决心,决定偷偷地去找一下梅杰少校,问问他卡思卡特上校飞行大队里的队员是否真的如约塞连所说的那样,当真会被逼着接受比别人更多的战斗飞行任务。牧师之所以会决定采取这一大胆、冲动的行动,是因为在此之前他又同惠特科姆下士吵了一架。这以后,他就着水壶里的温水草草吞下了一块银河和鲁丝宝贝牌夹心巧克力,权且用这些东西充当了一顿毫无乐趣可言的午餐。
 


    餐毕,他便步行去找梅杰少校,这样他离开时就不会让惠特科姆下士看见。他悄无声息地溜进了树林,直到他刚离开的林间空地里的那两顶帐篷看不见了才敢出声。这之后他跳进了一条被废弃的铁路壕沟,因为在那里面走路步子要踏实些。他顺着那些陈旧的枕木匆匆走着,心里越来越感到怒火难平。那天上午他接二连三地受到卡思卡特上校、科恩中校和惠特科姆下士的欺侮和羞辱。他必须让自己受到一些尊敬!不一会,他那瘦弱的胸脯就因透不过气来而上下起伏不已。他尽可能快地朝前走着,就差没跑起来,因为他担心一旦他慢了下来,他的决心可能会动摇。不久,他看见一个身穿制服的人在生锈的铁轨之间向他走来。他立即从沟边爬了出来,俯身钻进一片稠密的矮树丛中隐藏起来,而后他发现了一条蜿蜒的小道直通向阴暗的森林深处,于是他便沿着这条狭窄、簇叶丛生且布满了青苔的小路,朝着他既定的方向快步走去。这一段路走起来要艰难得多,但他仍抱着与先前一样的不顾一切的坚强的决心,跌跌撞撞地一个劲地向前走着。许多坚硬的树枝挡在他的去路上,将他那毫无遮护的双手扎得生痛,直至路两旁的灌木和高大的蕨类植物变得稀疏起来。透过逐渐稀疏的低矮灌木可清楚地看到有座草绿色军用活动房子架在煤渣堆上,牧师东倒西歪地从它旁边走过,继而又经过了一顶帐篷,外面有一只银灰色的猫在晒太阳。后来他又经过了另一座架在煤渣堆上的活动房子,最后闯进了约塞连所在中队的驻扎的那块空地。此时他的嘴唇上渗出了咸咸的汗珠。他没有停下,径直穿过空地来到了中队的文书室。一名瘦瘦的、弓腰曲背的参谋军士迎上前来招呼他。这个军士长着高高的颧骨,留着一头长长的淡黄色头发。他彬彬有礼地告诉牧师,说他尽管进去好了,因为梅杰少校不在里面。

    牧师向他微微点了点头以示谢意,接着就沿着夹在一排排办公桌和打字机之间的通道,独自朝后面用帆布隔出的那间办公室走去。他跃过了那条呈三角形的过道,发现自己已经来到一间空空的办公室里。那扇活板门已在他身后关上。他艰难地喘着气,浑身大汗淋漓。办公室仍然是空空的。他觉得他听见有人在窃窃私语。

    十分钟过去了。他板着面孔不悦地朝四下打量着。他一直紧闭着嘴巴,一副毫不气馁的样子;后来他突然想起那位参谋军士刚才说的话:他尽管进去好了,因为梅杰少校不在里面,这时,他的面部表情一下子软了下来。原来这些士兵在搞恶作剧!牧师惊恐万状地从墙边缩了回来,辛酸的泪水一下子涌进了他的眼眶。他那颤抖的嘴唇里迸发出一声哀哀的呜咽。梅杰少校在别处,而另一间屋子里的士兵却把他当成了恶意嘲弄的对象。他几乎能看见他们像一群贪婪的杂食野兽一样,扬扬得意地躲在帆布墙的另一面,只等他重一露面他们就要带着粗野的欢笑和嘲讽无情地朝着他猛扑过去。

    牧师为自己的轻信而暗暗地在心里咒骂自己。惊恐中,他真希望能找到一样东西,如一副面具,或一副墨镜和一撮假胡子什么的,好让自己化装一下;或者他要是像卡思卡特上校那样有一个低沉有力的嗓子和一对宽厚的、肌肉发达的、长着二头肌的肩膀就好了,那样的话他就能毫无惧色地踱出门来,以咄咄逼人的权威和充分的自信,将这几个迫害他的恶毒家伙彻底击败,让他们一个个都吓破胆,全都魂飞魄散、后悔不迭地悄悄溜走。然而他缺乏勇气去面对他们。此时通向外面的唯一出路就是窗子。这条路倒是很清静,于是牧师从梅杰少校办公室的窗口跳了出去,迅速绕过帐篷的一角,纵身跳进铁路的壕沟躲了起来。

    他低低地弓着身子急急忙忙地溜着,故意挂着一脸怪模怪样的笑容,装出一副若无其事、和蔼可亲的样子,生怕会被什么人撞见。每当见对面有人向他走来,他就立即离开壕沟钻进树林,然后便发疯似地跑过树木横生的树林,就像后面有人在追他似的,他的双颊因羞愤而火辣辣的。他好像听见从四面八方传来了一阵阵震耳的嘲弄他的狂笑声,还隐约瞥见在灌木丛的深处和高高挂在头顶上方的茂密的树叶中有许多张邪恶的醉脸,正冲着他假笑。他感到肺部像在被刀刺一样,阵阵发痛,于是只得放慢速度,一瘸一拐地走了起来。他疾步向前走着,渐渐脚步蹒跚起来,最后实在走不动了,一下子瘫坐在了一棵满是树瘤的苹果树上。当他跌跌撞撞向下倒去时,为了不让自己摔倒,他伸开两只胳臂抱住了树身,可不料脑袋却重重地撞在了树干上。此时他满耳朵听到的只有他自己的刺耳并夹杂着呜咽的喘息声。几分钟过去了,可感觉却像是过了几小时,这时他才意识到这阵将他整个人淹没了的震耳欲聋的声音原来是他自己发出来的。他胸部的疼痛逐渐减退。不久,他感到有力气站起来了。他竖起耳朵仔细地听了听。林子里静悄悄的,没有一点声音。既没有魔鬼般的笑声,也没有人在追赶他。此时他感到极度的疲惫、伤心,并且浑身脏兮兮的,因而无法感到宽慰。他用麻木和颤抖的手指将皱巴巴的衣服弄平,以极大的自制力走完了剩下的那段通往林间空地的路。一路上牧师不时痛苦地想到心脏病发作的危险。

    惠特科姆下士的吉普车仍旧停在空地上。牧师踮起脚尖偷偷地绕到惠特科姆下士的帐篷后面,却不愿从前面的入口处经过,以免被下士看见,受到他的羞辱。在如释重负地吁了一口长气之后,他赶紧溜进了自己的帐篷,可一进门却发现惠特科姆下士弯曲了两腿躺在他的吊床上,一双沾满了泥巴的鞋子就搁在牧师的毯子上。下士嘴里吃着牧师的条形糖块,脸上挂着一种轻蔑的神情,正在用大姆指翻弄着牧师的一本《圣经》。

    “你上哪去了?”下士粗鲁地、毫无兴趣地质问道,连头都没抬一下。

    牧师的脸红了起来,立即躲躲闪闪地将脸避开。“我到树林散步去了。”

    “好吧,”惠特科姆下士抢白道,“别相信我。可你就等着吧,看我会干出些什么事来。”他在牧师的糖块上咬了一大口,一副饥饿的样子,然后含着满嘴的糖继续说道,“你不在的时候有人来拜访你了,是梅杰少校。”

    牧师吃惊地猛然转过身来,叫道:“梅杰少校?梅杰少校来过?”

    “我们现在说的不就是这个人吗,难道不对?”

    “他上哪去了?”

    “他跳进了铁路壕沟,像只受了惊吓的兔子似的跑了,”惠特科姆下士窃笑道,“真是个怪物。”

    “他有没有说他来干什么的?”

    “他说他有件要紧事需要你帮忙。”

    牧师大吃一惊。“梅杰少校是这么说的吗?”

    “不是说的,”惠特科姆下士以苛求精确的口气更正道,“他是写在一封给你的私信上的,信还封了口。他把信留在了你的桌上。”

    牧师朝那张他用来当办公桌的桥牌桌上扫了一眼,桌上只有一只令人讨厌的桔红色梨形番茄。这只番茄是他今天早上从卡思卡特上校那儿得来的。他已经把它给忘了,而此时它仍旧躺在桌子上,就像一个不可磨灭的血红色的象征物,象征着他的愚蠢与无能。“信在哪儿呀?”

    “我把它拆了,读完后就扔了。”惠特科姆下士砰地一声将《圣经》合了起来,紧接着又从床上跳了下来。“怎么啦,你不信我的话?”说完便走出了帐篷。可他紧接着又折了进来,差点和牧师撞个满怀,因为牧师正跟在他的后面往外奔,打算再回去找梅杰少校。

    “你不知道怎样将职责委托给别人,”惠特科姆下士阴沉着脸对他说,“这是你的另一个毛病。”

    牧师知错地点了点头,匆匆地从他的身边走了过去,也来不及向他表示歉意。此时他能感觉到命运之手正在老练而又专横地摆弄着他。现在他意识到了,这天梅杰少校已经两次在壕沟里迎面向他跑来。而牧师也两次窜进林子,非常愚蠢地将这次注定的会面给推迟了。他尽可能快地沿着碎木横陈、宽窄不一的铁道枕木往回奔,心里因强烈的自责而无法平静。灌进鞋袜的小砂砾将他的脚趾磨得生痛。这种强烈的不适使他那张苍白而又劳累的脸不自觉地皱了起来。八月初的这个下午变得越来越闷热。从他的住地到约塞连的中队将近一英里。等他到达那里时,牧师身上那件浅褐色的夏季制服衬衫早已被汗水给浸透了。他气吁吁地又一次冲进了中队文书室的帐篷,不料却遭到了前次碰到的那位心地奸诈、说话和气、瘦脸上架着一副圆圆的眼镜的参谋军士的断然阻拦。他要求牧师呆在外面,因为梅杰少校在里面,并告诉他在梅杰少校出来之前不能让他进去。牧师用迷惑不解的眼光看着他。为什么这个军士这么恨他?他的嘴唇苍白,不住地颤抖着。他感到渴得难受。这些人究竟是怎么回事?这一切难道还不够可悲吗?参谋军士伸出一只手,牢牢地抓住牧师。

    “对不起,长官,”他用低沉、彬彬有礼的忧郁语调抱歉地说,“可这是梅杰少校的命令。他不想见任何人。”

    “他想见我,”牧师恳求道,“我刚才来这儿的时候他去我的帐篷找我了。”

    “梅杰少校去你那儿了?”

    “是的,他去过。请你进去问问他。”

    “恐怕我不能进去,长官。他也不想见到我。或许你可以留张纸条给他。”

    “我不想留条子。难道他就不能破个例吗?”

    “只在极特殊的情况下才这样。上一次他离开帐篷是为了参加一位士兵的葬礼。而最近他在完全被迫的情况下才在办公室里接见了一个人。一个叫约塞连的轰炸员逼着——”

    “约塞连?”这一新的巧合使牧师兴奋得满脸放光。这难道是正在形成中的另一个奇迹吗?“可我现在想和他谈的正是这个人的事呀!他们有没有谈到约塞连究竟该执行多少次飞行任务?”

    “谈了,长官。他们那次谈的正是这件事。约塞连上尉已经执行过五十一次战斗飞行任务,他请求梅杰少校允许他停飞,这样他就用不着再多飞四次了。当时卡思卡特上校还只要求飞满五十五次。”

    “梅杰少校是怎么说的?”

    “梅杰少校告诉他这件事他无能为力。”

    牧师的脸沉了下来。“梅杰少校是这么说的吗?”

    “是的,长官。实际上他还建议约塞连去找你帮忙。长官,您真的不想留张条子下来吗?我这儿有现成的铅笔和纸。”

    牧师摇了摇头,失望地咬着他那干得发硬的嘴唇走了出去。天色尚早,可却发生了一大堆的事。树林里的空气较前凉爽了些。他的嗓子又干又痛。他慢吞吞地走着,一边沮丧地自问还能有什么样的不幸降临到他的身上。就在这时,一个疯疯癫癫的人似从天而降,突然从树林里的一片桑树丛后面出现在他的面前,吓得牧师放声尖叫起来。

    牧师的叫喊声把这位高个子、面无血色的陌生人吓得直往后退,嘴里不住地尖叫着:“不要伤害我!”

    “你是谁?”牧师朝他喊道。

    “求你不要伤害我!”那人也在喊。

    “我是个随军牧师!”

    “那你为什么想伤害我?”

    “我没想伤害你!”牧师有点恼怒地坚持道,尽管他像生了根似地站在原地一动也不动。“告诉我你是谁,想要我为你做点什么。”

    “我只想知道一级准尉怀特-哈尔福特是不是已经得肺炎死了,”那人喊叫着回答,“我想知道的就是这事。我就住在这儿,我的名字叫弗卢姆。我是这个中队的人,可我住在这儿的林子里。你随便向谁打听都行。”

    牧师将眼前这位怪模怪样、畏畏缩缩的人仔细打量了一番,慢慢恢复了镇静。这人破破烂烂的衬衣领上缀着一对锈烂了的上尉须章。他的一个鼻孔下长着一个带毛的黑痣,嘴唇上的胡须浓密、粗硬,那颜色和杨树皮差不多。

    “既然你是这个中队的人,干吗要住在树林里?”牧师好奇地问。

    “我是没办法,才住在这树林里的,”上尉气冲冲地答道,好像牧师应该知道似的。他慢慢直起身来,虽然他比牧师高出一个头还多,但他仍然不放心地盯着牧师。“难道你没听人说起过我?一级准尉怀特-哈尔福特曾经发誓,说等哪天夜里我睡熟了的时候,他要割断我的喉咙。所以,只要他还活着,我就不敢睡在中队里。”

    牧师怀疑地听着他的难以置信的解释。“可这是不可信的,”牧师答道,“否则那就是预谋杀人了。你为什么不把这件事报告给梅杰少校?”

    “我向梅杰少校报告过,”上尉伤心他说,“可梅杰少校说要是我再向他提起这件事,他就割断我的喉咙。”这人胆怯地仔细打量着牧师。“你是不是也要割断我的喉咙?”

    “哦,不,不,不会的,”牧师安慰道,“当然不会。你真的住在树林里吗?”

    上尉点了点头。牧师盯着他的脸,这张脸因疲惫和营养不良而显得粗糙不堪,面色灰白。此时他的心情很复杂,既可怜同时也很尊敬这个人。上尉的身体在皱巴巴的衣服下瘦得皮包骨头,衣服就像一堆乱糟糟的麻袋片似的挂在他的身上。他浑身上下沾满了一撮撮的干草,头发急需剪理,眼睛下方布满了大大的黑圈圈。上尉这副受尽磨难、衣衫褴褛的模样让牧师感动得几乎要哭出来。想到这个可怜人每天都不得不忍受许多非人的折磨,牧师内心充满了敬意和同情。他压低嗓门十分谦恭地问:

    “谁替你洗衣服呢?”

    上尉噘起嘴很认真地说:“我让路那头一个农户家的女人给我洗。我把衣服放在我的活动房子里,每天溜进去一两次,拿条干净手帕,或换身内衣。”

    “到冬天你准备怎么办?”

    “哦,我想到那个时候我可以回中队了,”上尉满怀信心地答道,那口气有点像个殉道者。“一级准尉怀特-哈尔福特一直都在对大家保证,说他很快就会得肺炎死掉。我想我只要有耐心就行了,等到天气稍稍冷点,潮湿点就行了。”他迷惑不解地凝视着牧师,又道,“这事难道你一点都不知道?难道你没听到大伙全在谈论我吗?”

    “我想我从来没听见过任何人提起过你。”

    “哦,那我就真的弄不明白了,”上尉忿忿地说,但又设法装出乐观的样子继续说,“瞧,现在己是九月,所以我也不会等得太久了。下次要是有哪位小伙子问起我,你就告诉他,说只要一级准尉怀特-哈尔福特得肺炎一死,我就立即回去卖力地干我那宣传报道的老行当。你愿意替我告诉他们吗?就说只要冬天一到,一级准尉怀特-哈尔福特得肺炎一死,我就立刻回中队,行吗?”

    牧师神情庄重地将这些预言一样的话印在了脑子里,更加出神地琢磨着话里的深奥含义。“你是靠吃浆果、草药和草根来维持生命的吗?”牧师又问。

    “不,当然不,”上尉惊讶地答道,“我从后门溜进食堂,在厨房里吃饭。米洛总拿三明治和牛奶给我吃。”

    “下雨时你怎么办呢?”

    上尉坦白地答道:“被淋湿呗。”

    “你睡哪儿呢?”

    上尉一下子弯下身子,抱成一团蹲了下来,开始一步步地向后退。“你也想割我的喉咙?”

    “啊,不会,”牧师喊道,“我向你发誓。”

    “你就是想割我的喉咙!”上尉坚持说。

    “我向你保证,”牧师恳求他说,但已经来不及了,因为这个难看的多毛幽灵已经不见了。他利索地钻进了由乱叶、光线和阴影组成的奇怪世界——那里花朵盛开、五彩斑斓并且支离破碎——中,消失得无影无踪。牧师甚至开始怀疑这人究竟有没有出现过。发生了如此多的怪事,他都不敢确定哪些是怪事,哪些是真事。他想尽快查清林子里这个疯子的情况,看看是不是真的有个弗卢姆上尉。然而,他很不乐意地想起,他的当务之急是要消除惠特科姆下士对自己的不满,因为他太疏忽,没有将足够的职责托付给下士。

    他迈着沉重的脚步,无精打采地沿着弯弯曲曲的小路穿过了树林,一路上他口渴难耐,感到累得几乎走不动了。一想到惠特科姆下上,他就懊悔不已。他满心希望当他到达林间空地时,惠特科姆下士不在那里,这一来他就可以无拘无束地脱去衣服,好好把胳臂、胸脯和肩膀洗一洗,然后喝点水,舒舒服服地躺下,也许还能睡上几分钟。谁知他命中注定要重新经受一次失望和震惊,因为当他到达住地时惠特科姆下士已经成了惠特科姆中士了。惠特科姆正光着膀子坐在牧师的椅子上,用牧师的针线把崭新的中士臂章往衬衫袖子上缝。卡思卡特上校提升了惠特科姆下士,同时命令牧师立即去见他,就那些信件的事和他谈一谈。

    “啊,不,”牧师呻吟道,惊得目瞪口呆地倒在自己的吊床上。他的保温水壶是空的。此时他实在心慌意乱,因而想不起来他那只盛了水的李斯特口袋就挂在外面两顶帐篷之间的阴凉处。“我真不能相信竟会有这种事。我真不能相信竟会有人当真认为我一直在伪造华盛顿-欧文的签名。”

    “不是为那些信,”惠特科姆下士更正道,显然,他正在得意地欣赏着牧师的那副懊丧神情。“他见你是为了同你谈谈有关给伤亡人员家属的慰问信的事情。”

    “为了那些信?”牧师吃惊地问。

    “正是。”惠特科姆下士幸灾乐祸地看着他。“他准备把你好好臭骂一通,因为你不准我将那些信发出去。我提醒他说那些信都将附上他的亲笔签名,他十分赞赏这个主意,你真该看到他当时的那副神情。就为这,他提升了我。他绝对相信,这些信会让他的大名登上《星期六晚邮报》。”

    牧师更加迷惑起来。“可是他怎么知道我们正好在考虑这个主意?”

    “我去他的办公室告诉他的。”

    “你干了什么?”牧师尖叫着质问,同时以一种不常有的愤怒一下子从床上跳了起来冲到下士面前。“你是说你真的未经我的允许就越过我去找上校了?”

    惠特科姆下士带着轻蔑的满意神情厚颜无耻地咧开嘴笑了起来。“对了,牧师,”他回答说,“你要是知道好歹,就最好别追究这事,连想都别想。”他恶意挑衅地不慌不忙地大笑了起来。“要是卡思卡特上校发现你为了我把这个主意告诉了他而想报复我,他会不高兴的。你懂吗,牧师?”惠特科姆下士继续说,一面轻蔑地啪嗒一声将牧师的黑线咬断了,然后开始扣衬衫纽扣。“那个蠢家伙真的认为这是他所听到过的最好的主意之一。”

    “这甚至可能让我的名字上《星期六晚邮报》呢,”卡思卡特上校在他的办公室里微笑着自夸地说,一边乐不可支地昂首阔步地来回走着,一边责备牧师。“你真没什么头脑,竟然看不到这个主意的妙处。你有个像惠特科姆下士这样的好部下,牧师。我希望你有足够的头脑,能看到这一点。”

    “是惠特科姆中士了,”牧师冲动地纠正道,但随即又克制住了自己。

    卡思卡特上校瞪了他一眼。“我是说惠特科姆中士,”他答道,“我希望你就听别人一次吧,不要老找人家的茬儿。你不想一辈子就当个上尉吧,是不是?”

    “什么,长官?”

    “咳,要是你一直这样下去,我真不知道你能有什么样的出息。

    惠特科姆下士认为你们这帮人在一千九百四十四年里头脑里从来就没有装进过一点点新思想,我也很乐意赞同他的看法。那个惠特科姆下士真是个聪明的小伙子。行了,一切都会改变的。”卡思卡特上校带着一种不容置疑的神情在办公桌前坐下,动手在自己的记事簿上清理出一大块空白来,然后用手指在里面敲了敲。“从明天开始,”他说,“我要求你同惠特科姆下士一道,替我给大队里的每一位阵亡、受伤或被俘人员的直系亲属发一封慰问信。我要求信写得恳切些。我还要求信里要多写些有关个人的详情,这样人家就不会怀疑你们写的都是我的真心话了。你明白吗?”

    牧师冲动地跨上前去表示抗议。“可是长官,这不可能!”他脱口而出,“我们并不是对所有的人都很了解。”

    “那又有什么关系呢?”卡思卡特上校质问他,然后又友好地微笑道,“惠特科姆下士给我拿来了一封最常用的通函,它足以能应付任何情况。听着:‘亲爱的太太/先生/小姐或者先生和夫人:当我获悉您的丈夫/儿子/父亲或兄弟阵亡/负伤或据报告在战场失踪时,任何语言都无法表达我内心所经受的深切的痛苦。’等等。我认为这样的开场白精确地概括了我的全部感受。听着,要是你觉得干不了,那就最好让惠特科姆下士来负责这事。”卡思卡特上校突然拿下烟嘴,两手拿住它的两端,就好像它是一根条纹玛瑞和象牙做的马鞭一样。“这是你的一个毛病,牧师。惠特科姆下士告诉我,你不知道怎样将职责委托给旁人。他还说你这人没有一点创新精神。

    我说的这些你不反对吧,对不对?”

    “对,长官。”牧师摇了摇头,心里感到沮丧,觉得自己很可鄙,这是因为他不知道怎样将职责委托给旁人,没有创新精神,也因为他实在想斗胆跟上校作对。他脑子里乱成一团麻。屋外士兵们正在进行飞碟射击,每次枪响都让他的神经受到一次刺激。他无法适应这些枪声。他的周围是若干蒲式耳的红色梨形番茄,他几乎相信自己很久以前在某个类似的场合,也曾站在卡思卡特上校的办公室里,四周围也是这么多蒲式耳的红色梨形番茄。又是“曾经相识的幻觉”。这场景看起来很熟悉,可同时看上去又是那么遥远。他感到自己的衣服满是污垢,且旧得不成样,因而心里怕得要命,生怕身上会散发出怪味。

    “你对什么事情都太认真了,牧师,”卡思卡特上校用成年人的客观口吻直率地说,“这是你的另外一个毛病。你老是把脸拉得长长的,让人丧气。你就让我看你笑一回吧,笑呀,牧师。你若现在就能捧腹大笑,我就给你整整一蒲式耳的红色梨形番茄。”他等了一两秒钟,两眼盯着牧师,然后得胜地哈哈大笑着说,“瞧,牧师,我没说错吧。你不会朝着我捧腹大笑,不是吗?”

    “不会,长官,”牧师低声下气地承认道,一面费力地、慢吞吞地咽了口唾沫。“现在笑不出来,我很渴。”

    “那你就弄点什么喝喝吧。科恩中校的办公桌里有些波旁烈性威士忌酒。你该试试在哪天晚上同我们一道去军官俱乐部转转,给自己找点乐。不妨也试着醉上那么一回。我希望你不要因为自己是个专职的神职人员,就觉得应该高我们大伙一等。”

    “啊,没有,长官。”牧师窘迫地向他保证。“事实上,我前几天晚上天天都上军官俱乐部的。”

    “要知道,你只不过是个上尉。”卡思卡特上校没理会牧师的话,继续说道,“你尽可以当你的神职人员,但你仍然只是个上尉。”

    “是的,长官。我明白。”

    “那就好。你先前不笑也好。我好歹用不着送你红色梨形番茄了。惠特科姆下士告诉我,说你今天早上在这里的时候拿走了一个番茄。”

    “今天早上?可是,长官!那是你送给我的。”

    卡思卡特上校歪着脑袋,显出怀疑的样子。“我又没说它不是我送你的,我说了吗?我只是说你拿了一个。我不明白,如果你真的没偷,干吗要那么心虚?我给了你番茄吗?”

    “是的,长官。我发誓您给了。”

    “那我只好相信你的话了。可尽管如此,我还是想象不出其中的理由,我为什么要给你一个番茄。”卡思卡特上校带着一种显示长官资格的神态,将一个圆形的玻璃镇纸从他的办公桌的右边移到了左边,然后又拿起了一技削尖的铅笔。“好了,牧师,要是你没事了,我可还有许多重要的工作要处理呢。等惠特科姆下士发出几十封慰问信后,你就来告诉我,那时我们就可以同《星期六晚邮报》的编辑们联系了。”他突然来了灵感,满脸放光他说,“嗨!我想我可以再次自愿要求派我们大队去袭击阿维尼翁。那样可以加速事情的发展。”

    “去袭击阿维尼翁?”牧师的心差点停止了跳动,浑身先是感到一阵刺痛,接着便汗毛直竖。

    “没错,”上校劲头十足地解释道,“我们大队越早有人伤亡,这事就进展得越迅速。要是可能,我希望能在圣诞节这一期里刊登出来。我估计这一期的发行量要大些。”

    让牧师感到惊恐不已的是,上校当真拎起了电话筒,主动要求派遣他的大队去袭击阿维尼翁,并且就在当天晚上他又竭力想把牧师从军官俱乐部撵出去。就在牧师被撵出前的一刹那,约塞连醉醺醺地站了起来,先是将椅子掀翻,然后便打出了复仇性的一击。

    他的这一举动使得内特利大叫起他的名字来,同时使得卡思卡特上校脸色发白,小心翼翼地向后退去,可不料却不偏不斜正好重重地踩到了德里德尔将军,后者厌恶地将他从自己那被踩得青肿的脚上推开,并命令他向前走,将牧师重新赶回军官俱乐部。这一切把卡思卡特上校弄得心烦意乱。先是约塞连!这个令人胆寒的名字像丧钟似的再度清清楚楚地响了起来,接着自己又把德里德尔将军的脚给踩肿了;再就是卡思卡特上校在牧师身上找到的另一个毛病:无法预料德里德尔将军每次见到牧师都会有些什么样的反应。卡思卡特上校永远也不会忘记德里德尔将军在军官俱乐部第一次见到牧师的那个晚上。那天将军抬起他那红润、热汗淋淋、满是醉意的脸,透过烟卷散发出的黄色烟幕,目光沉重地盯着独自躲在墙边的牧师。

    “我真是太吃惊了!”德里德尔将军一认出那人是个牧师,就皱起他那蓬松吓人的灰眉毛,声音沙哑地喊了起来。“那边的那个人不是牧师吗?一个侍奉上帝的人竟开始出没在这样一个地方,和一群肮脏的醉鬼和赌徒混在一起,这可真是件大好事。”

    卡思卡特上校一本正经地抿紧嘴唇,起身站了起来。“您的看法我十分赞同,长官,”他语气尖刻地附和道,话音里流露出明显的不满。“我真不明白如今这些牧师都是怎么回事。”

    “他们变得越来越好了,他们就是这么回事,”德里德尔将军强调地咆哮道。

    卡思卡特上校尴尬地哽住了,但马上又乖巧地恢复了常态。

    “是的,长官。他们变得越来越好了。我刚才恰恰也是这样想的,长官。”

    “这里正是牧师应该呆的地方。趁官兵们出来喝酒、赌博时同他们混在一起,这样就可以了解他们,得到他们的信任。除此之外,他究竟还有什么别的法子让他们相信上帝呢?”

    “我命令他到这里来的时候,恰恰也是这样想的,长官,”卡思卡特上校小心谨慎地说。接着他走过去亲热地用胳臂搂住牧师的肩,同他一起走到一个角落,压低嗓门,用冷冰冰的口气命令他从现在起每晚到军官俱乐部来履行他的职责,以便在军官们喝酒、赌博的时候同他们混在一起,这样就可以了解他们,赢得他们的信任。

    牧师同意了,真的每晚都去军官俱乐部履行他的职责,与那些想避开他的人混在一起,直到那天晚上在乒乓球桌旁爆发了那场凶狠的斗殴。一级准尉怀将-哈尔福特在没人招惹他的情况下突然来了个急转身,猛地一拳,正好砸在穆达士上校的鼻子上,将他打得一屁股坐在地上。德里德尔将军见了,突然放声大笑起来,笑了一阵后,突然察觉牧师就站在近旁,神情古怪、呆若木鸡地看着他,一副痛苦而又惊讶的样子。德里德尔将军一见到牧师就立即僵住了。他怒火中烧,狠狠地看了牧师片刻。他一下子便没了情绪,于是转过身去,迈着那两条短短的罗圈腿,像水手一样左右摇摆着,极不高兴地朝酒吧柜台走去。卡思卡特上校胆战心惊地一路小跑着跟在他的后面,一面徒劳地左顾右盼,想从科恩中校那里寻得一点帮助。

    “这倒是件好事,”德里德尔将军冲着酒吧柜台咆哮道,粗壮的手牢牢地抓着那只喝空了的小酒杯。“这真是件好事,一个侍奉上帝的人竟然开始出没在这样一个地方,和一群肮脏的醉鬼和赌徒混在一起。”

    卡思卡特上校松了一口气。“是的,长官,”他得意地大声说,“这的确是件好事。”

    “那你他妈的干吗不管?”

    “什么,长官?”卡思卡特上校问,惊愕地看着将军。

    “你以为让你的牧师每晚都混在这里会给你脸上增光吗?我他妈每次来,他都在这里。”

    “您说得对,长官,绝对正确,”卡思卡特上校附和道,“这根本不会为我增光。我这就处理这事,现在就处理。”

    “难道不是你命令他来这里的?”

    “不是我,长官。是科恩中校。我也准备严厉处分他。”

    “要不是因为他是个牧师,”德里德尔将军嘟哝着说,“我就叫人把他给毙了。”

    “他不是牧师,长官,”卡思卡特上校帮忙似地提醒说。

    “他不是?既然他不是牧师,那他为什么在领子上挂十字架的符号?”

    “他没在领子上挂十字架,长官。他挂的是银叶。他是个中校。”

    “你有一个中校军衔的随军牧师?”德里德尔将军吃惊地问。

    “啊,不是的,长官。我的随军牧师只是个上尉。”

    “既然他只是上尉,那他干吗要在领子上挂银叶?”

    “他没在领子上挂银叶,长官。他挂的是十字架。”

    “给我立即滚开,你这个狗杂种。”德里德尔将军骂了起来。“否则我叫人把你拖出去毙了!”

    “是,长官。”

    卡思卡特上校咽了口唾沫,从德里德尔将军身边走开,将牧师赶出了军官俱乐部。两个月后,当牧师试图说服卡思卡特上校撤销把飞行任务增加到六十次的那道命令时,结果几乎是一模一样,这次努力也宣告彻底失败。要不是他对妻子的思念以及对上帝的智慧和公正所抱有的终生信赖,他简直就要绝望了。他怀着强烈的感情爱着妻子,思念着妻子,其间既夹杂着强烈的肉欲,也含有高尚的热情。在他眼里,上帝是永生的,他无所不能,无所不知,并且十分仁慈;他为世间万物所共有,且被拟人化了;他说的是英语,属盎格鲁一撤克逊族人种,并且对美国人格外垂青。不过,他现在对上帝的这些看法已开始有所动摇了。有许多事物都在考验他的信仰。没错,是有一本《圣经》,可《圣经》只不过是一本书,而《荒凉山庄》、《金银岛》、《伊坦-弗洛美》和《最后的莫希干人》也都是书呀。有一次他无意中听到邓巴问人家,创世之谜是由一群无知无识、连下雨是怎么回事都不明白的人解答出来的,这看起来真的有可能吗?那万能的上帝,以他那无穷的智慧,真的害怕六千年以前的人会建成一座直通天国的巨塔吗?那天国究竟在哪里?在上面?

    还是在下面?在一个有限的但不断扩展着的宇宙中是没有上、下之分的。在这个宇宙中,就连那个巨大、炽热、耀眼、无比壮丽的太阳也处于逐渐衰亡之中,它的衰亡最终也会毁灭地球。那些奇迹是根本没有的;人们的祈祷也没有任何回应。灾难,无论是降临到正直者还是堕落者的头上,都是一样的残酷无情。最近,他接连遇见了一些神秘现象——几周前,在为那个可怜的中士举行的葬礼上,树上出现了那个裸体男人;而就在那天下午,预言家似的弗卢姆又作出了这么一个含义隐晦、令人不安但同时又令人振奋的许诺:告诉他们,冬天一到,我就会回来——要不是为了这些,他这样一个有良知和个性的牧师,早就会听从理智,放弃祖先们传下来的对上帝的信仰,并且当真会辞去职务和放弃军衔,去当一名步兵或野战炮兵,甚至去伞兵部队当一名下士,一切悉听命运的安排
