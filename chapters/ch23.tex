\chapter{内特利的老头}
 
    中队里唯一真正见到过米洛的红香蕉的人就是阿费。当香蕉熟了,并通过正常的黑市渠道开始流入意大利时,他从一个在军需部供职的颇有权势的兄弟会的弟兄那儿拿了两只。内特利花了好多个星期去找他那个妓女,却都徒劳无功,令人泄气,那天晚上终于找到了,并答应给她和她的两个女朋友每人三十块美金,把她们哄骗回了军官公寓。那天晚上,阿费和约塞连一起呆在军官公寓里。

    “每人三十块美金?”阿费慢悠悠地似问非问地评论说,一面不相信地又是摸又是拍这三个身材高大而匀称的姑娘,那样子就像一个吝啬的行家。“像这样的姑娘出三十块美金可不少啊。再说,我这一生从没有为这种人花过钱。”

    “我不要你付钱,”内特利急忙向他保证说,“她们的钱全由我来付。我只要你们两个家伙把另外两个姑娘带走。你们就不能帮我一下?”

    阿费自鸣得意地笑了笑,他那肌肉松软的圆脑袋摇得像货郎鼓一般。“没有人需要为好心的老阿费付这种钱。无论何时我想要,我就能弄到。只不过这会儿我没有情绪。”

    “你干吗不付三个人的钱,让另外两个人走呢?”约塞连建议说。

    “因为那样我的那位就会因我让她为了钱而干活跟我生气,”内特利回答说,一面焦急地看着他的姑娘。那姑娘正不耐烦地盯着他,嘴里咕咕哝哝地开始抱怨起来。“她说如果我真的喜欢她,就该把她送走,而同另外两个人中间的一个上床。”

    “我有一个更好的主意。”阿费吹嘘起来。“我们为什么不把她们三人留在这儿,一直留到宵禁开始,然后我们威胁说要把她们赶到大街上去被人抓起来,除非她们把她们的钱都给我们。我们甚至可以威胁说要把她们从窗户里推下去。

    “阿费!”内特利吓得目瞪口呆。

    “我只不过是想帮你,”阿费羞怯地说。阿费总是千方百计想帮助内特利,因为内特利的父亲又有钱又有名,战争结束后完全能够帮助他。“哎呀,”他牢骚满腹地为自己辩护说,“以前在学校里我们总是那样做的。我记得有一天我们把两个这样笨头笨脑的女中学生从市区骗到了联谊会馆,让她们跟所有想和她们睡觉的会友上床,我们威胁说要打电话给她们的父母,说她们在和我们睡觉。我们把她俩困在床上足足有十多个小时。当她们开始抱怨时,我们甚至还打她们几下耳光。后来,我们把她们的五分、一角的硬币和口香糖拿走后,把她们赶了出去。老兄,我们过去在那个联谊会馆里玩得很痛快。”他平静地回忆着,他那肥胖的双颊因怀念起往事而焕发出快乐、红润的光泽。“我们过去把任何人都排斥在外,甚至互相排斥。”

    但是此刻阿费对内特利毫无帮助,因为内特利如此深深迷恋上的姑娘变得郁郁不乐,越来越气,并以威胁的口气开始骂他。幸运的是,亨格利-乔就在这时闯了进来。于是一切问题又解决了,只是邓巴醉醺醺地、摇摇晃晃地迟进来一会儿,一下搂住了另一个咯咯笑着的姑娘。现在是四男三女,七个人把阿费留在公寓里,爬进了一辆出租马车。马车还停在路边时,姑娘们就要求先付给她们钱。内特利向约塞连借了二十美金,向邓巴借了三十五美金,向亨格利-乔借了十六美金,然后潇洒地一挥手付给了她们九十美金。

 


    姑娘们这才变得友好起来,大声对马车夫说了个地址,马车夫便赶着马得得地载着他们穿过半个城市,来到一个他们以前从未光顾过的地段,在一幢坐落于一条漆黑的大街上的古老而高大的楼房前停了下来。姑娘们领着他们爬过四段又陡又长、踩上去嘎嘎作响的木楼梯,穿过一个门廊,走进她们自己的富丽堂皇的公寓套房。

    这里神奇般地不断涌出越来越多的身体柔软、一丝不挂的年轻姑娘。公寓里有个邪恶、淫荡的丑老头儿,他那刻薄的笑声常惹内特利生气;那里还有个整天咯咯叫唤着的循规蹈矩的老太婆,她穿着烟灰色羊毛衫,对那里发生的所有伤风败俗的事情都看不惯,并竭尽全力要把公寓收拾干净。

    这个令人惊愕的地方是块肥沃、富饶而沸腾的宝地,这里到处可见女人的乳头和肚脐。起初,在那间灯光昏暗的黄褐色的起居室里只有他们的三个姑娘。那间起居室坐落在三条阴暗的走廊的交界处,这三条走廊从不同的方向通往这间离奇古怪、不可思议的妓院深处的幽室。姑娘们立即开始脱衣,有时还停下来得意地炫耀她们那些花花绿绿的内衣,还一刻不停地同那个憔悴、放荡的老头打情骂俏。那老头一头长长的白发乱蓬蓬的,穿着一件白衬衫,没扣扣子,一副邋遢相。他坐在一张几乎放在房间正中的上了霉的蓝色扶手椅里,与妓女们嘀嘀咕咕地说着下流话;他笑嘻嘻地但又带着嘲讽的神态,礼节性地向内特利和他的同伴们表示欢迎。接着,那老太婆伤心地低着她那颗好找茬的脑袋,磕磕绊绊地出去给亨格利-乔叫一个姑娘来,然而却带回来两个Rx房高耸的美人儿,一个已经脱了衣服,另一个只穿着一件透明的粉红色短衬衣,就这一点衣服,她坐下时也扭动着身体把它脱掉了。又有三个一丝不挂的姑娘从另外一个方向荡过来,她们停下聊起来,然后又来了两个。接着又有四个姑娘穿过这间起居室,她们结成懒洋洋的一伙,正在谈着什么,其中三个人光着脚,另一个穿着一双好像不是她自己的银色舞鞋,没结鞋带,走起路来东摇西摆,怪吓人的。后来,又有一个只穿着三角裤的姑娘来到这间房间并坐了下来。这样,在短短几分钟内那里就来了一大群人,一共十一人,除一人外,全都光着身子。

 


    到处是闲逛着的赤裸裸的人体,大多数都很丰满,亨格利-乔的魂都不在了。他惊讶地站在那儿,一动不动,任凭姑娘们从容轻松地走进来,舒舒服服地坐下来。后来,他突然尖叫一声,像脱了弦的箭一般冲向门口,想回士兵公寓去取他的照相机,可半路上又想到即使他离开片刻,这个可爱的、刺激的、丰富多彩的异教徒的天堂便会从他这儿被掠走,不复再有,这使他感到害怕,脊骨一阵冰凉,于是狂叫一声,停住了脚步。他在门口停了下来,唾沫飞溅,脸上和脖子上的筋脉剧烈地动着。那老头坐在那张发了霉的蓝色扶手椅里,就像坐在宝座上耽于享乐的魔王,两条细长的腿上裹着一条偷来的美军军用毛毯御寒,带着胜利的喜悦望着亨格利-乔。

    他不出声地笑着,两只凹陷而机警的眼睛闪烁着因熟知一切而玩世不恭、放荡不羁的神情。他一直在喝酒。一看见这个邪恶、堕落、没有爱国心的老头,内特利就恨得毛发倒竖。那老头年纪够大的了,使内特利想到自己的父亲,他不停地开着低毁美国的玩笑。

    “美国,”他说,“将会被打败。而意大利将会赢得胜利。”

    “美国是世界上最强大、最繁荣的国家,”内特利激情满怀、庄严肃穆地对他说,“而且美国的军人是无与伦比的。”

    “的确如此。”那老头欣然表示同意,口气中带着少许以嘲讽别人为乐趣的意味。“但另一方面,意大利是世界上最不繁荣的国家。

 


    意大利士兵也许是最差劲的。但正是因为如此,我的国家在这场战争中打得如此出色,而你的国家却打得那么差劲。”

    内特利先是感到意外,捧腹大笑起来,接着脸红耳赤地为自己的失礼表示歉意。“对不起,我刚才嘲笑了你,”他真诚地说,接着又用尊敬、屈尊俯就的语调继续说,“但意大利过去被德国人占领,现在又正被我们占领。你不会说这是打得出色吧,是吗?”

    “不过,我当然要这么说,”那老头快乐地说,“德国人正在被赶出去,而我们还在这儿。几年以后你们也会走的,而我们仍然在这儿。你瞧,意大利确实是一个十分贫穷、弱小的国家,然而正是这一点使我们这么强大。意大利士兵不再死亡了,可美国和德国的士兵正在死亡。我把这叫做打得极其出色。是的,我确信意大利将会在这场战争中幸存下来,并将在你自己的国家被摧毁之后永远存在下去。”

    内特利简直难以相信自己的耳朵。他以前从未听到过这样令人吃惊的恶毒的言词。他的直觉使他感到纳闷,为什么联邦调查局的人不来把这个背叛祖国的老东西抓起来。“美国是不会被摧毁的!”他慷慨激昂地喊道。

    “永远不会吗?”那老头轻声激了他一句。

    “这个……”内特利结结巴巴地说。

    那老头压抑住一种更深沉、更强烈的喜悦放声大笑起来。他仍然温和地刺激他说:“罗马被摧毁了,希腊被摧毁了,波斯被摧毁了,西班牙被摧毁了。所有的大国都被摧毁了。为什么你的国家不会被摧毁,你实实在在认为你自己的国家还会存在多长时间?永远?请记住地球本身在大约二千五百万年之后也注定要被太阳毁灭的。”

    内特利不安地扭动着身体。“这个,永远是个很长的时间,我想。”

    “一百万年?”那个喜欢嘲弄人的老头带着强烈的虐待狂的热情坚持说,“五十万年?青蛙几乎有五亿年的历史了。你真的十分有把握地说,美国尽管强大而繁荣,拥有无以伦比的士兵,拥有世界上最高的生活标准,会存在得像——青蛙那么久吗?”

    内特利真想揍他那张嘲笑人的脸。他环顾四周,想找人帮他反驳这个狡猾、邪恶的老头的那些该受谴责的诽谤,以捍卫他的国家的未来。他很失望。约塞连和邓巴在一个较远的角落里正忙着同四五个嬉皮笑脸的姑娘寻欢作乐,已经喝了六瓶葡萄酒。亨格利-乔早就沿着一条神秘的过道荡走了,他像个贪得无厌的暴君,两只瘦弱的膀子不停地舞动着,尽可能多地把臀部最大的年轻妓女拥在身前,和她们一起挤睡在一张双人床上。
 


    内特利感到进退两难,不知所措。他自己的姑娘伸开四肢样子难看地躺在一张又厚又软的沙发上,露出一副懒散无聊的表情。内特利感到烦恼不安,因为她对他态度冷淡,无动于衷。她第一次看见他是在士兵公寓的客厅里他们许多人在一起玩二十一点小赌博的时候,但她没有理他,自那时起,她对他一直是若即若离,提不起精神,这一点他记得如此清楚,如此甜蜜而又如此伤心。她的嘴张着,成一个完美无缺的0字形,只有天晓得她那双呆滞、蒙胧的眼睛用如此残忍、冷漠的眼神在凝视着什么。那老头静静地等待着,脸上带着一种既轻蔑又同情的洞察一切的微笑望着他。一个满头金发、身体柔软成曲线形、肌肤呈蜂蜜色、长着两条漂亮的腿的姑娘坐在那老头的椅子扶手上,尽情地炫耀着她的姿色,一面无精打采地、卖弄风情地撩摸着他那骨瘦如柴、苍白而放荡的脸。见到一个这么老的人还如此淫荡好色,内特利真是又气又恨。他心情沉重地转过身,心想他干吗不带着他自己的姑娘睡觉去。

    这个肮脏、贪婪、魔鬼似的老头之所以使他想到他的父亲,是因为他们两人毫无相同之处。内特利的父亲是个衣着得体、举止优雅的白发绅士,而这老头却是个举止粗鲁的游手好闲之徒;内特利的父亲是个冷静、善于思考、有责任心的人,而这老头却是个用情不专、放浪形骸的老色鬼;内特利的父亲言行谨慎、有教养,而这老头却是个粗野的乡巴佬;内特利的父亲自尊自爱、学识渊博,而这老头却寡廉鲜耻、愚昧无知;内特利的父亲蓄着高贵的白胡子,而这老头一根胡子也没有;内特利的父亲——和内特利遇到过的所有其他人的父亲——都很高贵、聪明、受人尊敬,而这老头却实实在在令人憎恶。内特利又同他辩论起来,决心痛斥他的无耻逻辑和含沙射影的诽谤,雄心勃勃地要报一箭之仇,以吸引那个讨厌他、对他无动于衷而他却如此强烈地爱恋着的姑娘的注意,从而永远赢得她的爱慕。

    “这个,坦率地说,我不知道美国将存在多久,”他无所畏惧地说,“我想如果世界本身有一天将被毁灭的话,那我们也不可能永远存在下去。但是我确实知道我们将会赢得胜利,并活很长、很长时间。”

    “多长时间?”那个喜欢诽谤别人的老头嘲讽地问道,一脸居心叵测的得意神情。“甚至不如青蛙活得久吗?”

    “比你或者我活得长久得多。”内特利笨拙地脱口而出。

    “喔,原来如此!考虑到你是那么有勇无谋,而我已经这么一大把年纪,那就不会太长久啦。”

    “你多大年纪?”内特利问,不禁对这个老头产生了兴趣,被他迷住了。

    “一百零六岁。”那老头看见内特利满脸懊恼,开心地抿着嘴轻声笑起来。“我看得出你也不相信这一点。”

    “我不相信你跟我说的一切,”内特利回答说,脸上露出羞怯和怒气平息后的微笑。“我唯一相信的就是美国将会赢得战争的胜利。”

    “你太看重胜利了,”那个肮脏而邪恶的老头嘲笑说,“真正的诀窍在于输掉几场战争,在于知道哪几场战争可以输掉。几个世纪以来,意大利一直在战争中打败仗,然而你瞧我们干得多出色。法国打赢了战争,然而却不断处于危机之中。德国打输了但却繁荣起来。意大利在埃塞俄比亚打了胜仗,但立即陷入严重的困境。胜利给我们制造了许多辉煌的假象,使我们丧失了理智,于是便引发了一场我们没有机会获胜的世界大战。可是既然我们又要输了,所有的事情就开始向好的方面转化。假如我们成功地被打败了,我们就一定会成功。”

    内特利目瞪口呆地看着他,脸上露出未加掩饰的迷惑神情。

    “现在我真的不明白你在说什么。你说话像个疯子。”

    “但我像个正常人一样生活。墨索里尼执政时,我是个法西斯分子;现在他被赶下了台,我就成了一名反法西斯分子。当德国人在这儿保护我们反对美国人时,我是狂热的亲德派,而现在美国人在这儿保护我们抵抗德国人,我就成了狂热的亲美派。我可以向你保证,我义愤填膺的年轻朋友”——看见内特利变得更加惊慌失措、张口结舌,老头儿那双机警、轻蔑的眼睛里闪耀出更加得意的光芒——“你和你的国家在意大利不会有比我更忠实的支持者了——但这仅仅是在你们驻守意大利期间。”

    “但是,”内特利不相信地大声喊道,“你是个叛徒!是个趋炎附势的小人!是个不知廉耻、肆无忌惮的机会主义者!”

    “我已经一百零七岁了,”那老头温和地提醒他说。

    “你难道没有任何信条?”

    “当然没有。”

    “没有道德标准?”

    “哦,我是个很有道德的人。”那个恶棍似的老头半是讽刺半是认真地向他保证说,一边说一边摸着一个丰满的、脸上长着两个漂亮酒窝的黑发妓女的光屁股。那妓女勾魂摄魄地在他椅子的另一边扶手上舒展开了身体。他沾沾自喜地坐在两个裸体女郎中间,像个乞丐王似的一手搂着一个,挖苦地咧着嘴向内特利笑着。

    “我难以相信,”内特利怨恨地说,硬着头皮竭力不去看他与那两个姑娘搂搂抱抱的样子。“我只是难以相信。”

    “但这一切全是真的。德国人进城的时候,我像个朝气蓬勃的女芭蕾舞演员在大街上翩翩起舞,一边喊着:‘嗨,希特勒!’我把嗓子都喊哑了。我甚至还挥舞着一面纳粹小旗,那是我趁她母亲不注意,从一个漂亮的小姑娘手里抢来的。当德国人离开城市时,我拿着一瓶上等白兰地,提着一筐鲜花跑出去欢迎美国人。当然,白兰地是我自己喝的,花是用来撒向我们的解放者的。在第一辆车子上直挺挺地坐着一个自命不凡的老少校,我用一朵红玫瑰不偏不倚地砸在他的眼睛上。多么美妙的一击!你要是看见他往后躲的样子就好啦。”

    内特利吃惊地站了起来,直喘粗气,脸色发白。“是——德-科弗利少校!”他叫喊起来。

    “你认识他?”那老头乐滋滋地问道,“真是太巧了!”

    内特利吃惊不小,没有听见他的话。“那么你就是那个打伤——德-科弗利少校的人!”他又气又怕地喊道,“你怎么能做这样的事情?”

    那个魔鬼似的老头泰然自若。“你的意思是说,我怎么能忍住不砸他?你真该看到那个傲慢、讨厌的老家伙,他那么严厉地坐在车子里,大脑袋挺得笔直,愚蠢的脸上一本正经的样子,就像上帝亲临似的。他是个多么诱人的靶子啊!我用一枝美国红玫瑰打中了他的眼睛。我认为这是最合适不过的。你说呢?”
 


    “那件事做得糟透了!”内特利大声指责他说,“那是一件恶意的犯罪事件!——德-科弗利少校是我们中队的主任参谋!”

    “是吗?”那个顽固不化的老头戏弄他说,一边神态严肃地捏着他那个尖下巴,装出一副懊悔的样子。“如果是那样的话,你必须为我的公正而称赞我。当德国人开进来的时候,我用一小枝火绒草差点把一个强壮的年轻中尉扎死。”

    这个可恶的老头竟不能明白自己犯下了多大的罪过,这使得内特利惊愕不已,手足无措。“你难道不知道自己干了些什么?”他言词激烈地叱责他。“——德-科弗利少校是个品德高尚的大好人,大家都钦佩他。”

    “他是个老傻瓜,他实在没有权力做得像个年轻的傻瓜似的。

    他现在在哪儿?死了?”

    内特利带着忧郁、敬畏的神情轻声回答说:“没人知道。他好像失踪了。”

    “你明白了吧?想一想吧,一个像他这样年龄的人,为了什么国家之类的荒唐事情,竟拿自己所剩不多的生命去冒险。”

    内特利马上竭力反对。“为自己的国家用生命去冒险没什么荒唐的!”他郑重地说。

    “是吗?”那老头问,“国家是什么?国家是四周用界线围着的一块土地。通常是非自然的。英国人为英国而死,美国人为美国而死,德国人为德国而死,俄国人为俄国而死。现在有五六十个国家在打这场战争。当然,这么多国家不可能都值得人们为了它们去死。”

    “任何值得人为它而生的东西,”内特利说,“都值得人为它而死。”

    “而任何值得人为它去死的东西,”那个亵渎神灵的老头回答说,“肯定值得人为它而生。你知道,你是个如此单纯、天真的年轻人,我简直为你感到惋惜。你多大啦,二十五?二十六?”

    “十九,”内特利说,“到一月份我就二十岁了。”

    “但愿你活下去。”那老头摇了摇头,有那么一会儿,他像那个满腹牢骚、事事看不惯的老太婆一样眉头紧锁,像是生气又像是沉思。“如果你不提防着点,他们会杀了你。我现在能看得出来你不打算提防。你为什么不理智些,努力做得更像我这样、你也可能活到一百零七岁呢。”

    “因为我宁愿站着死,不愿跪着生,”内特利带着崇高的信念得意洋洋地反驳说,“我想你以前听说过这句俗话吧。”

    “是的,我当然听说过,”那个阴险的老头沉思地说,脸上又堆起了微笑。“然而恐怕你把这句俗话说颠倒了,宁愿站着生,不愿跪着死。那句俗话是这么说的。”

    “你肯定吗?”内特利有点糊涂地问,“好像我那样说更讲得通。”

    “不,我这么说更讲得通。去问你朋友。”

    内特利转过身去问他的朋友,却发现他们都走了。约塞连和邓巴都不见踪影。那老头看着内特利又尴尬又吃惊的样子,发出轻蔑而快乐的狂笑。内特利羞愧得沉下了脸。他孤力无援地犹豫了片刻,接着快速转过身,匆匆逃进最近的那条走廊去寻找约塞连和邓巴,希望及时找到他们,把那老头同——德-科弗利少校之间发生的那场出人意料的冲突告诉他们,把他们带回来给他解围。所有的走廊里的门都关上了。也没有哪道门下有灯光。夜已经很深了。内特利绝望了,便不再寻找了。最后他意识到,除了去找他爱恋着的姑娘,和她在什么地方躺下来,跟她亲热,向她献殷勤,与她共同安排他们的未来,他没有什么事情可做了;但是当地回到起居室来找她的时候,她已上床睡觉去了。他无事可做,只好去同那个讨厌的老头继续谈刚才未谈完的话题。可那老头却从扶手椅里站起身来、用开玩笑似的客套说夜已深,他得告辞了,让内特利和两个睡眼蒙胧的姑娘呆在那里。那两个姑娘也说不出他自己的妓女进了哪个房间,她俩百般挑逗他,想让他对她俩感兴趣,但却是白费力气,于是她们过了一会儿也上床睡觉去了,留下他一人在起居室里的那张凹凸不平的小沙发上睡着了。

    内特利是个敏感、富有、漂亮的小伙子,生着一头乌黑的头发,两只眼睛流露出信任他人的眼神。他第二天一大早在沙发上醒来时,脖子感到酸疼,昏昏沉沉地不知自己身在何处。他性格温和、文质彬彬。他快二十岁了,不知道心灵创伤、紧张、仇恨或神经机能病是怎么回事,在约塞连看来,这恰恰证明他实实在在疯得有多么厉害。他在童年虽常受到责骂,但却是愉快的。他与他的兄弟姐妹们相处得很好,他不恨他的父母,因为他们俩待他很好。

    内特利从小受到的家教是要憎恶像阿费和米洛那样的人。他母亲把像阿费那样的人描绘成拼命向上爬的野心家,他父亲把像米洛那样的人说成是投机倒把犯,但他们从不让他接近那些人,因此他从来也没有学会怎样去恨。就他所能记得的,他的家曾在费城、纽约、缅因、棕榈滩、南安普敦、伦敦、多维尔、巴黎和法国南部呆过,无论在哪儿,他家里总是高朋满座,客人都是绅士淑女,没有一个拼命向上爬的野心家或投机倒把犯。内特利的母亲出身新英格兰地区的桑顿家族,是美国革命的后代。他的父亲却是个私生子。

    “永远记住,”他母亲过去常常提醒他说,“你是内特利家的人。

    你不是范德比尔特家的人,他家是靠当一个地位卑微的拖船船长发财的,也不是洛克菲勒家的人,他家的财富是通过肆无忌惮地进行原油投机积累起来的;你也不是雷诺兹或杜克家族的人,他们的收入是靠欺骗公众、推销致癌的树脂和柏油制品获得的;你当然也不是阿斯托家的人,我相信,他家还在出租房屋。你是内特利家的一员,而内特利家从来没有为了钱而什么事都干。”

    “你妈的意思是,孩子,”有一次他父亲和蔼可亲地插话说,那种措辞优雅、简洁的天才内特利佩服得五体投地,“旧时的富翁要比新富翁好,新兴的暴发户永远不会像新近的破落户那样受人尊敬。这么说对吗,亲爱的?”

    内特利的父亲不断提出那种贤明而通晓世事的忠告。他热情奔放,脸色红润得像加过热的香甜的红葡萄酒一样。虽然内特利不喜欢香甜的红葡萄酒,但他却很喜欢他父亲。战争爆发后,内特利一家决定他应该参军,因为他太年轻了,不能从事外交工作,同时还因为他父亲根据权威人士的消息说,俄国将会在几个星期或几个月内垮台,而希特勒、邱吉尔、罗斯福、墨索里尼、甘地、佛朗哥、庇隆和日本天皇将签署一个和平协议,他们从此将幸福地生活在一起。内特利参加陆军航空队是他父亲的主意,在那儿他可以作为飞行员安全地接受训练,而在此期间俄国人有条件地投降了,停战的具体条款也制定好了。此外,在航空队里当一名军官,他接触到的只会是有教养的绅士。

    事与愿违,他却发觉自己和约塞连、邓巴和亨格利-乔等人在罗马一家妓院里鬼混,而且他深深地爱上了妓院里一个对他态度冷漠的姑娘。他独自一人在起居室里睡了一夜后,第二天早上他终于和她同床共枕了,但几乎立刻就被她那任性的小妹妹打断了好事。那小姑娘没敲门便闯了进来,妒忌地扑到床上,这样内特利也可以搂着她。内特利的妓女吼叫着跳了起来,怒气冲冲地使劲揍她,抓着她的头发把她拎了起来。这个十二岁的小姑娘眼巴巴地望着内特利,像只拔了毛的小鸡,或者说像根剥了皮的嫩树枝。她那稚嫩的身体早熟地模仿着那些比她年龄大的女人的样子,使所有人感到难堪,因此她总是被赶走,穿上衣服,到外面大街上去和其他孩子在新鲜的空气里玩。这姐妹俩此刻正粗野地对骂,互相吐唾沫,发出一阵震耳欲聋的喧闹声,引来一大群喜欢热闹的旁观者挤进这间房间。内特利气恼地放弃了做爱的念头。他叫他的妓女穿上衣服,带着她下楼去吃早饭。那个小妹妹跟在后面。当他们三人在附近一家露天咖啡馆里体面地吃早餐时,内特利觉得自己就像是个神气的一家之主。但是等到他们开始往回走的时候,内特利的妓女已经感到厌烦了,于是她决定和其他两个姑娘上街去卖淫,不想再同他在一起了。内特利和那个小妹妹温顺地远远跟在后面,那个野心勃勃的小姑娘想学几手拉客的技巧,内特利则是情场失意而出来散散心。当那几个姑娘被一辆军用汽车里的士兵拦住并带走后,他俩都变得垂头丧气。

    内特利回到咖啡馆,给那个小妹妹买了一份巧克力冰淇淋,等她情绪好了些之后,带着她回到公寓里。约塞连和邓巴已在起居室里,还有精疲力竭的亨格利-乔,他那憔悴的脸上还带着快乐、麻木、得意洋洋的微笑。那天早晨他就这样笑着从妻妾成群的后宫里跌跌撞撞地走出来,全身骨头像散了架似的,那个淫荡、堕落的老头看到亨格利-乔破裂的嘴唇和青一块紫一块的眼睛,心里乐滋滋的。他热情地跟内特利打招呼。他仍然穿着前一天晚上那件皱巴巴的衣服。他那种衣衫褴褛、面容猥琐的模样使内特利心烦意乱。无论何时他来公寓,他总希望那个荒淫无耻的老头能穿上一件干净的布鲁克斯兄弟公司做的衬衫,刮过脸,梳过头,穿着一件花呢夹克衫,蓄两撇干净利落的白八字胡,这样,内特利每次看到他并想到自己父亲时,就不会有那种说不清的羞愧感了
