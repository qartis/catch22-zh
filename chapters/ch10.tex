\chapter{温特格林}
 
    克莱文杰死了。那是他哲学的根本性缺点。一日下午,十八架飞机从帕尔马执行完每周一次的例行飞行任务返回,在离厄尔巴岛海岸的海面上空下降,穿过一片金灿灿的云彩;其中的十六架从云端钻了出来,另外还有一架却不见了踪影,没见在空中,也没见在平静的绿玉色的海面上,更没见丝毫残骸。一架架直升飞机在那片云彩上盘旋,直到了太阳西落。夜里,那片云消散了去,次日上午便不再有克莱文杰了。

    克莱文杰和飞机的失踪,实在是令人愕然,其程度绝不亚于洛厄里基地的那次大阴谋——一座兵营的六十四个人在某个发饷日突然下落不明,从此就再没有一点消息。约塞连始终认为,那六十四个士兵不过是一致决定在同一天集体开小差而已。直到克莱文杰被神奇地夺去了性命,他方才改变了这种观点。说实在的,那次看似集体擅离神圣职守的开小差,当初确实很让约塞连大受鼓舞,他竟兴冲冲地跑出去把这振奋人心的消息告诉了前一等兵温特格林。

    “这有啥让你那么兴奋?”前一等兵温特格林惹人厌恶地嗤笑道,一面把一只沾满泥土的军鞋踏在铁锹上,铁板着脸,没精打采地倚靠在一个极深的方坑坑壁上。像这样的坑他在四围挖了不少,这可是他的军事特长。

    前一等兵温特格林实在是个卑鄙的小流氓,做事总喜欢我行我素,屡教不改。他每回开小差给捉住了,就被判在规定的时间内挖填若干长宽深均为六英尺的土坑。每次刑期一满,他便又开小差。前一等兵温特格林以一个真正的爱国者坚定的献身精神,心甘情愿地接受了这份挖填土坑的活计。

    “这工作还是蛮不错的,”他常常很达观他说,“我想总得有人去做。”

    他是个极聪明的人,深知战争期间在科罗拉多州挖土坑,实在算不得是一桩十分触楣头的差事。由于土坑的需求量不大,因此,他便可以不慌不忙地挖,然后再不慌不忙地填埋,这样,他也就很少有劳累过度的时候。尽管如此,他每受一次军法审判,便被降为列兵。这样丢失军阶,很让他感到深切的痛惜。

    “做个一等兵也不赖,”他颇是恋旧地回忆道,“过去我有地位——你明白我的意思吗?——我经常出入于上流社会。”他的脸阴沉了下来,显得极是无可奈何。“不过,这一切对我来说都已成了过去,”他很肯定他说,“下次我再开小差,就只是个列兵了,我很清楚,到时候情况跟现在可是大不一样了。”挖土坑实在是无甚出息的。“这工作甚至还不是固定的。每次刑期结束,我就没法再干这活。要是我还想回来挖土坑,那就得再开小差。可我又不能老这么做。有一条军规,也就是第二十二条军规。假如我下次再开小差,就该去坐班房了。我不清楚等着我的会是什么样的下场。要是我一不留神,我最后甚至可能去海外服役。”他不希望一辈子挖土坑,不过,只要战争还在进行,挖土坑就是战争期间的一部分工作,他也就不会对此有什么反感。“这可是责任问题,”他说,“我们每个人都有自己应尽的职责。我的职责就是不停地挖土坑,而且我做得相当出色,刚刚获得品行优良奖章的提名。你的职责就是在航空军校鬼混,希望战争结束之后再毕业。我只希望他们跟我一样尽到自己的职责。要是我也不得不去海外并替他们尽义务,那不就太不公平了,是不是?”

 


    一天,前一等兵温特格林在挖一个土坑时,捣破了一根水管,险些被淹死。待让人从坑里捞上来时,他已差不多人事不知。事后,谣传水管流出的是石油,结果,一级准尉怀特-哈尔福特被逐出了基地。不多久,只要是能弄来铁锹的,全都跑到外面,发了疯似地采掘石油。到处尘土飞扬。那场面差不多跟七个月后的一天早晨皮亚诺萨岛上的情形一模一样:头天晚上,米洛动用自己的M&M辛迪加联合体收集到的每一架飞机,轰炸了中队营地、机场、炸弹临时堆集处和修理机库。所有死里逃生的官兵全都聚到外面,在硬地上挖了一个个又大又深的掩体,然后在顶部搁上从机场修理机库窃取的装甲板和从别人帐篷侧帘偷来的一方块一方块千疮百孔的防水帆布。有关石油的谣传刚起,一级准尉怀特-哈尔福特便被调离科罗拉多州,最后来到皮亚诺萨岛落脚,接替库姆斯少尉——一天,他以宾客的身份随机外出飞行,只是想察看一下战况,不料,在弗拉拉上空竟跟克拉夫特一同遇难。每每忆起克拉夫特,约塞连总是很内疚。他之所以负疚,是因为克拉夫特是在他作第二轮轰炸时牺牲的,还因为克拉夫特在那次辉煌的阿的平叛乱中无辜受了牵连。那次叛乱是在波多黎各——他们飞往国外的第一段行程——

    发起的,十天后,在皮亚诺萨岛告终。当时,阿普尔比一到岛上,便出自责任心,大步跨进中队办公室,报告说约塞连拒不服用阿的平药片。中队办公室的那个军士赶忙请他坐下。

    “谢谢你,军士,我想我会坐的,”阿普尔比说,“我大概得等多长时间?今天我还有不少事情要做,这样,到明天一大早我就可以做好充分准备,一旦他们需要,我就能马上投入战斗。”

    “长官?”

    “你说啥,军士?”

    “你刚才问什么?”

    “我大概得等多长时间才能进去见少校?”

    “只要等他出去吃午饭,”陶塞军士回答说,“到时你可以马上进去。”

    “可到时他就不在里边了。是不是?”

    “是的,长官。梅杰少校要等吃完午饭才回办公室。”

    “我知道了。”阿普尔比口头上作了决定,可心里依旧没个数。

    “那么我想我还是午饭后再来一趟吧。”

    阿普尔比转身离开中队办公室,内心却很困惑。他刚走到外面,便觉得自己看见一个长得颇有些像亨利-方达的高个子黑皮肤军官从中队办公室的窗户里跳了出来,接着拐过弯,飞奔而去,便不见了踪影。阿普尔比收住脚步,紧闭了双眼。令人焦急不安的疑惑袭上他的心头。他怀疑自己是否得了疟疾,或许更糟糕,因服了过量的阿的平药片而引发了什么后遗症。当初,他服用的阿的平药片,超出了规定剂量的三倍,因为他想做一名出色的飞行员,强过其他任何人三倍。他依旧紧闭着双眼,这当儿,陶塞军士突然在他的肩上轻轻拍了拍,跟他说,梅杰少校才出去,要是他愿意,他现在就可以进去。阿普尔比这才又恢复了信心。

    “谢谢你,军士。他会马上回来吗?”

    “他一吃完午饭就回来。等他回来,你就得马上出去,在前面等他,直到他离开办公室去吃晚饭。梅杰少校在办公室的时候,是向来不在办公室见任何人的。

    “军士,你刚才说什么来着?”

    “我是说,梅杰少校在办公室的时候,是向来不在办公室见任何人的。”

    阿普尔比目不转睛地直盯着陶塞军士,试着用坚定的语调,说:“军士,是不是就因为我刚来中队,而你在海外混了很长时间,就想法子作弄我?”

    “哦,不,长官,”军士很恭敬地答道,“我只是奉命行事而已。等你见了梅杰少校,可以当面问他。”

    “我正想问他呢,军士。我什么时候能见到他?”

    “你永远见不到他。”

    阿普尔比因受了羞辱而满脸通红。军士给他递过一本拍纸簿,他便在上面写下了自己的报告,汇报约塞连和阿的平药片一事,随后就赶紧离去,同时又纳闷了起来:或许钓塞连还不是唯一的一个有幸穿上军官制服的疯子。

    等卡思卡特上校把飞行次数增加到五十五次的时候,陶塞军士早就开始怀疑,或许每一个穿制服的军人都是疯子。陶塞军士身材瘦削,一头漂亮的金发淡得差不多没了颜色,双颊凹陷,一副牙齿酷似又白又大的果浆软糖。他负责中队的事务,可他不觉得有什么称心。跟亨格利-乔一样的那些人始终用苛责仇恨的目光怒视他,而阿普尔比呢,如今已是一名顶呱呱的飞行员,又是一名打球从不失分的乒乓球选手,一心一意地要报复陶塞军士,更是对他无礼、陶塞军士负责中队的事务,是因为中队里再也没有别的什么人挑这个担子。无论是对战争,还是对升官发财,他全无兴趣。他感兴趣的是陶瓷碎片和赫波怀特式家具。

 


    对约塞连帐篷里的那个死人,陶塞军士已经习惯性地接受了——这差不多连他自己都没意识到——约塞连本人的说法——

    确实把他看做是约塞连帐篷里的一个死人。其实呢,压根就不是那回事。那家伙只是个替补飞行员,还没来得及正式报到,就在前线送了命。当初,他曾在作战室停留过,询问去中队办公室的路,结果,即刻被送往前线作战,因为那时那么多人都已完成了规定的三十五次飞行任务,而皮尔查德上尉和雷恩上尉又正巧为无法调集大队部明确的机组成员人数犯难。由于他从来没有正式被列入中队的编制,所以,也就永远无法把他正式除名。陶塞军士意识到,有关那个可怜虫的各种公文越来越多,永远会引起没完没了的冲击波。

    那个可怜虫名叫马德。对痛恨暴力和浪费的陶塞军士来说,他们用飞机送马德一路越过大洋,却不过是让他在到达后还不到两小时就在奥尔维那托上空被炸个粉身碎骨,这似乎是莫大的浪费,实在令人痛心疾首。没人想得起来他是谁,也回忆不出他长个啥模样,皮尔查德上尉和雷恩上尉就更不用提了。他俩只记得有个新来的军官出现在作战室,恰好赶上时间送死。每当有人提起约塞连帐篷里的死人那件事,他俩总是很显得尴尬,满脸通红。本该见过马德的那仅有的几个人,是他同机的机组成员,也都跟他一起被炸了个粉身碎骨。

    不过,约塞连倒是确切知道马德是谁。马德只是个无名小卒,从来不曾有过什么机遇,因为人们知道有关所有无名小卒的事情只有一点——他们从来没什么机遇。他们非死不可。送了命的马德,是地地道道的无名小卒,尽管他的遗物依旧杂乱地堆放在约塞连帐篷里的那张帆布床上,差不多跟三个月前他从未到过帐篷的那天留下那些东西时一模一样——所有那些东西在不到两个时辰之后便都沾染上了死气,就跟博洛尼亚大围攻发动后的第二个星期出现的情形完全一样。当时,四处弥漫硫磺气味的烟雾,潮湿的空气中散发着霉臭的死亡气味,所有即将执行轰炸飞行任务的官兵都已沾染上了这股死气。
 


    一旦卡思卡特上校主动要求让自己的大队去炸毁博洛尼亚的弹药库——驻扎意大利大陆的重型轰炸机由于飞行高度过高,没能把它们摧毁,那就不再有丝毫可能逃避这次轰炸飞行任务了。每延迟一天,便不断加剧大队全体官兵的恐惧感和沮丧情绪。那种萦绕不散又难以抗拒的死亡意识,随持续不断的雨,渐渐地弥散开去,就像是某种具有腐蚀作用的慢性病,侵蚀一般地渗透了每个人痛苦的面容。每个人身上都有一股甲醛味。无处可以求助,即便去医务室也无济于事。科恩中校下令关闭了医务室,所以,再也没有人能上那儿看门诊了。科恩中校所以这么做,是因为好不容易碰上的那个晴天,中队竟神秘地流行起了腹泻,大伙全都跑到医务室就诊,结果,迫使轰炸任务再次延期。暂停门诊,又封了医务室的门,丹尼卡医生每逢雨的间隙,便高坐在一只高凳上,以愁肠百结的不偏不倚的态度,默默感受着阴森森弥散开来的恐怖气氛,仿佛一只悒悒不乐的红头美洲鹫,栖息在医务室封闭的门上的那块不祥的手写牌子的下端。这牌子是布莱克上尉当初开玩笑钉上去的,丹尼卡医生始终没把它取下来,因为这在他实在不是什么玩笑。牌子四边用黑色炭笔画了一圈,上面写道:“另行通知以前,医务室暂停门诊。家有丧事。”

    恐怖往四处扩散,钻进了邓巴的中队。某日黄昏,邓巴很好奇地把头探进自己中队医务室的门,对着斯塔布斯医生模糊的身影——他正坐在幽暗处,面前摆了一瓶威士忌和一只盛满饮用水的钟形玻璃瓶——说起了话来。

    “你没事吧?”他关切地问道。

    “糟糕透顶,”斯塔布斯医生回答说。

    “你在这里干吗?”

    “坐坐而已。”

    “我还以为不再有门诊了呢。”

    “是没有门诊了。”

    “那你干吗还坐在这里?”

    “我还能坐哪里?去那该死的军官俱乐部,跟卡思卡特上校和科恩中校坐一块儿?你知道我在这里干什么?”

    “坐呗。”

    “我说的是在中队里,不是在帐篷里。别再他妈的自作聪明了。

    你可知道医生在中队里的职责?”

    “其他中队的医务室都给封了门,”邓巴说。

    “不管谁病了,只要走进我的门,我就会禁止他飞行,”斯塔布斯医生郑重他说,“我才不在乎他们说什么呢。”

    “你是不能禁止任何人飞行的,”邓巴提醒道,“难道你不知道那命令?”

    “我会给病人打上一针,让他彻彻底底躺倒下来,停止飞行。”

    斯塔布斯医生想到这情景,不由得带着嘲讽的兴味笑了起来。“他们以为只要他们一下命令,就可以让门诊彻底停止。那些狗杂种。

    哎哟!又下雨了。”雨又开始下了,先是落在树林里,再是落在泥潭里,然后便是轻轻地落到了帐篷的顶上,仿佛一阵抚慰的柔声细语。“所有一切都是潮呼呼的,”斯塔布斯医生极厌恶他说,“就连厕所和小便池都在泛滥,以此表示抗议。这讨厌的世界整个就像是一处藏尸处,臭气熏天。”

    当他停止了说话,四周静得似乎没了边际。夜幕落了下来。弥散着一种极度的孤独。

    “把灯打开,”邓巴建议道。

    “没电。我也懒得启动自己那台发电机。以前,我救别人的命,常常从中得到极大的快感。现在,我实在不知道救人性命究竟还有什么意义,既然他们反正都得死。”

    “哦,意义到底还是有的,”邓巴肯定地对他说。

    “是吗?有什么意义?”

    “意义就在于,尽你的可能让他们多活一些时间。”

    “你说的不错,但是,既然他们反正都得死,那又有什么意义呢?”

    “诀窍就是别考虑这个问题。”

    “别谈什么诀窍了。救人性命究竟有什么意义?”

    邓巴默默沉思片刻。“谁知道呢?”

    邓巴不知道。轰炸博洛尼亚一事,本该让邓巴欣喜万分,因为时间一分钟一分钟走得慢悠悠的,几个小时拖得像几个世纪那么长。然而,他反倒感到痛苦,因为他知道自己即将送命。

    “你真的还想要些可待因吗?”斯塔布斯医生问道。

    “是替我朋友约塞连要的。他确信自己马上会送命的。”

    “约塞连?究竟谁是约塞连?约塞连,到底是什么名字?前天晚上,在军官俱乐部喝醉了酒跟科恩中校打架的那个家伙,是不是他?”

    “没错,就是他。他是亚述人。”

    “那个发了疯的狗杂种。”

    “他倒是没那么疯,”邓巴说,“他发誓不飞博洛尼亚。”

    “我正是这个意思,”斯塔布斯医生说道,“那发了疯的狗杂种,或许只有他一个人才是清醒的。”
