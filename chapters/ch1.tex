\chapter{得克萨斯人}
 
这可是实实在在的一见钟情。

初次相见,约塞连便狂热地恋上了随军牧师。

约塞连因肝痛住在医院,不过,他这肝痛还不是黄疸病的征兆,正因为如此,医生们才是伤透了脑筋。如果它转成黄疸病,他们就有办法对症下药;如果它没有转成黄疸病而且症状又消失了,那么他们就可以让他出院。可是他这肝痛老是拖着,怎么也变不了黄疸病,实在让他们不知所措。

每人早晨,总有三个男医生来查病房,他们个个精力充沛,满脸一本正经,尽管眼力不好,一开口却总是滔滔不绝。随同他们一起来的是同样精力充沛、不苟言笑的达克特护士。讨厌约塞连的病房护士当中就有她一个。他们看了看挂在约塞连病床床脚的病况记录卡,不耐烦地问了问肝痛的情况。听他说一切还是老样子,他们似乎很是恼怒。

“还没有通大便?”那位上校军医问道。

见他摇了摇头,三个医生互换了一下眼色。

“再给他服一粒药。”

达克特护士用笔记下医嘱,然后他们四人便朝下一张病床走去。没有一个病房护士喜欢约塞连。其实,约塞连的肝早就不疼了,不过他什么也没说,而那些医生也从来不曾起过疑心。他们只是猜疑他早就通了大便,却不愿告诉任何人。

约塞连住在医院里什么都不缺。伙食还算不错,每次用餐都有人送到他的病床上,而且还能吃到额外配给的鲜肉。下午天气酷热的时候,他和其他病号还能喝到冰果汁或是冰巧克力牛奶。除了医生和护士,从来就没有人来打扰过他。每天上午,他得花点时间检查信件,之后他便无所事事,整日闲躺在病床上消磨时光,倒亦心安理得。在医院里他过得相当舒但,而且要这么住下去也挺容易,因为他的体温一直在华氏一百零一度。跟邓巴相比,他可是快活极了。邓巴为了拿那份人家端到他病床前的餐点,不得不一而再再而三地将自己摔成个狗吃屎。




约塞连打定主意要留在医院,不再上前线打仗,自此以后,他便去信告知所有熟人,说自己住进了医院,不过从未提及个中缘由。有一天,他心生妙计,写信给每一个熟人,告知他要执行一项相当危险的飞行任务。“他们在征募志愿人员。任务很危险,但总得有人去干、等我一完成任务回来,就给你去信。”但是从那以后,他再也没有给谁写过一封信。

依照规定,病房里的每个军官病员都得检查所有士兵病员的信件,士兵病员只能呆在自己的病房里。检查信件实在枯燥得很。

得知士兵的生活只不过比军官略多些许趣味而已,约塞连很觉失望。第一天下来,他便兴味索然了。于是,他就别出心裁地发明了种种把戏,给这乏味单调的差事添些色彩。有一天,他宣布要“处决”信里所有的修饰语,这一来,凡经他审查过的每一封信里的副词和形容词便统统消失了。第二天,他又向冠词开战。第三天,他的创意达到了更高点,把信里的一切全给删了,只留下冠词。他觉得玩这种游戏引起了更多力学上的线性内张力,差不多能使每一封信的要旨更为普遍化。没隔多久,他又涂掉了落款部分,正文则一字不动。有一次,他删去了整整一封信的内容,只保留了上款“亲爱的玛丽”,并在信笺下方写上:“我苦苦地思念着你。美国随军牧师A-T-塔普曼。”A-T-塔普曼是飞行大队随军牧师的姓名。

当他再也想不出什么点子在这些信上面搞鬼时,他便开始攻击信封上的姓名和地址,随手漫不经心地一挥,就抹去了所有的住宅和街道名称,好比让一座座大都市消失,仿佛他是上帝一般。第二十二条军规规定,审查官必须在自己检查过的每一封信上署上自己的姓名。大多数信约塞连看都没看过。凡是没看过的信,他就签上自己的姓名;要是看过了的,他则写上:“华盛顿-欧文”。后来这名字写烦了,他便改用“欧文-华盛顿”。审查信件一事引起了严重反响,在某些养尊处优的高层将领中间激起了一阵焦虑情绪。




结果,刑事调查部派了一名工作人员装作病人,住进病房。军官们都知道他是刑事调查部的人,因为他老是打听一个名叫欧文或是华盛顿的军官,而且第一天下来,他就不愿审查信件了。他觉得那些信实在是太枯燥无味。

约塞连这次住的病房挺不错,是他和邓巴住过的最好的病房之一。这次跟他们同病房的有一名战斗机上尉飞行员,二十四岁,蓄着稀稀拉拉的金黄色八字须。

这家伙曾在隆冬时节执行飞行任务时被击中,飞机坠入亚得里亚海,但他竟安然无事,连感冒也没染上。时下已是夏天,他没让人从飞机上给击落,反倒说是得了流行性感冒。约塞连右侧病床的主人是一名身患疟疾而吓得半死的上尉,这家伙屁股上被蚊子叮了一口,此刻正脉脉含情地趴在床上。约塞连对面是邓巴,中间隔着通道。紧挨邓巴的是一名炮兵上尉,现在约塞连再也不跟他下棋了。这家伙棋下得极好,每回跟他对弈总是趣味无穷,然而,正因为趣味无穷,反让人有被愚弄的感觉,所以约塞连后来就不再跟他下棋了。再过去便是那个来自得克萨斯州颇有教养的得克萨斯人,看上去很像电影里的明星,他颇有爱国心地认为,较之于无产者——

流浪汉、娼妓、罪犯、堕落分子、无神论者和粗鄙下流的人,有产者,亦即上等人,理应获得更多的选票。

那天他们送得克萨斯人进病房时,约塞连正在删改信件。那一天天气酷热,不过宁静无事。暑热沉沉地罩住屋顶,闷得屋里透不出一丝声响。邓巴又是纹丝不动地仰躺在床上,两眼似洋娃娃的眼睛一般,直愣愣地盯着天花板。他正竭尽全力想延长自己的寿命,而办法就是培养自己的耐烦功夫。见邓巴为了延长自己的寿命竟如此卖力,约塞连还以为他已经咽气了呢。得克萨斯人被安置在病房中央的一张床上。没隔多久,他便开始直抒高见。

邓巴霍地坐起身,“让你说中了,”他激奋得叫了起来。“确实是少了样什么东西,我一直很清楚少了样什么东西,这下我知道少了什么。”他使劲一拳击在手心里。“就是缺少了爱国精神,”他断言道。




“你说得没错,”约塞连也冲他高喊道,“你说得没错,你说得没错、你说得没错。热狗、布鲁克林道奇队、妈妈的苹果派。为了挣得这些东西,我们每个人都在不停地拼死拼活,可有谁甘愿替上等人效力?又有谁甘愿替上等人多拉几张选票而卖命?没有爱国精神,就这么回事儿。也毫无爱国心。”

约塞连左侧床上的准尉却是无动于衷。“哪个在胡说八道?”他不耐烦地问了一句,随即翻过身去,继续睡他的觉。

得克萨斯人倒是显得性情温和、豪爽,着实招人喜爱。然而三天过后,就再也没人能容忍他了。

他总惹得人心烦意乱,浑身不自在,心生厌恶,所以大家全都躲着他,除了那个全身素裹的士兵以外,因为他根本没办法动弹,全身上下都裹着石膏和纱布,双腿双臂已全无用处。他是趁黑夜没人注意时被偷偷抬进病房的。直到第二天早晨醒来,大伙儿才发现病房里多了他这么个人,他的外观实在古怪得很:双腿双臂全都被垂直地吊了起来,并且用铅陀悬空固定,只见黑沉沉的铅舵稳稳地挂在他的上方。他的左右胳膊肘内侧绷带上各缝入了一条装有拉链的口子,纯净的液体从一只明净的瓶里由此流进他的体内。在他腹股沟处的石膏上安了一节固定的锌管,再接上一根细长的橡皮软管,将肾排泄物点滴不漏地排入地板上一只干净的封口瓶内。等到地板上的瓶子满了,从胳膊肘内侧往体内输液体的瓶子空了,这两只瓶子就会立刻被调换,液体便重新流入他的体内。这个让白石膏白纱布缠满身的士兵,浑身上下唯有一处是他们看得到的,那就是嘴巴上那个皮开肉绽的黑洞。

那个士兵被安顿在紧挨着得克萨斯人的一张病床上。从早到晚,得克萨斯人都会侧身坐在自己的床上,兴致勃勃又满腔怜悯地跟那士兵说个没完没了。尽管那个士兵从不搭腔,他也毫不在意。

病房里每天测量两次体温。每天一早及傍晚,护士克拉默就会端了满满一瓶体温计来到病房,沿着病房两侧走一圈,挨个儿给病员分发体温计。轮到那个浑身雪白的士兵时,她也有自己的绝招——把体温计塞进他嘴巴上的洞里,让它稳稳地搁在洞口的下沿。发完体温计,她便回到第一张病床,取出病人口中的体温计,记下体温,然后再走向下一张床,依次再绕病房一周。一天下午,她分发完体温计后,再次来到那个浑身裹着石膏和纱布的士兵病榻前,取出他的体温计查看时,发现他竟死了。

“杀人犯,”邓巴轻声说道。

得克萨斯人抬头看着他,疑惑地咧嘴笑了笑。

“凶手,”约塞连说。

“你们俩在说什么?”得克萨斯人问道,显得紧张不安。

“是你谋杀了他,”邓巴说。

“是你把他杀死的,”约塞连说。

得克萨斯人的身子往后一缩。“你们俩准是疯了,我连碰也没碰过他。”

“是你谋杀了他,”邓巴说。

“我听说是你杀死他的,”约塞连说。

“你杀了他,就因为他是黑人,”邓巴说。

“你们俩准是疯了,”得克萨斯人大声叫道,“这儿是不准黑人住的,他们有专门安置黑人的地方。”

“是那个中士偷偷送他进来的,”邓巴说。

“是那个共产党中士,”约塞连说。

“看来,这事你们俩早就知道了。”

约塞连左侧的那个准尉对那个士兵意外死亡的事却无动于衷。他对什么事部很冷漠,只要不惹到他头上,他绝不会开口说一句话。

约塞连遇见随军牧师的前一天,餐厅的一只炉子爆炸,烧着了厨房的一侧,一股强烈的热浪迅速弥漫这个地方,甚至在约塞连的病房——离火灾现场差不多有三百英尺远,病员也能听到大火呼呼的咆哮声,以及燃烧着的木材发出的刺耳的爆裂声。滚滚浓烟快速涌过病房映着橘红光亮的窗户。大约过了一刻钟,空难消防车赶到现场救火。经过半个小时紧张急速的行动,消防队员开始控制住火势。突然,空中传来了一阵熟悉的单调的嗡嗡声,原来是一群执行完任务后返航的轰炸机。消防队员只得收起水龙带,火速返回机场,以防有飞机坠毁起火。轰炸机全都安全降落,最后一架飞机一着地,消防队员便立刻掉转车头,火速驶过山坡,赶回医院继续灭火。当他们赶回医院,大火己熄。火是自己灭的,而且灭得很彻底,甚至没留下一处要用水浇泼的余烬。消防队员自是很失望,无所事事,只好喝口温咖啡,四处转悠,想法子勾引护士。



失火的第二天,随军牧师来到医院,当时,约塞连正忙着删改信件,只保留了其中卿卿我我的甜言蜜语。牧师在两张病床间的一张椅子上坐了下来,问约塞连感觉如何。他的身体微微倾向一侧,衬衫上别着的一枚上尉领章是约塞连所能见到的唯一能证明他官衔的标志,至于他是什么人,约塞连一无所知,于是便想当然地认为,他不是医生就是疯子。

“哦,感觉挺好,”约塞连答道,“只是肝有些疼,所以我猜想自己应该也不是很正常吧,不过,不管怎么说,我必须承认,我感觉确实很不错。”

“这就好,”牧师说。

“是啊,”约塞连说,“没错,感觉好就行了。”

“我本来想早点来的,”牧师说,“可是最近我的身体一直不怎么好。”

“那实在是太不幸了,”约塞连说。

“我只是得了伤风,”牧师马上补充道。

“我一直在发烧,烧到华氏一百零一度。”约塞连也连忙补上一句。

“那真糟糕,”牧师说。

“是啊!”约塞连表示同意。“没错,是太糟了。”

牧师有些坐立不安。片刻后,他问道:“有什么事需要我帮忙?”

“没有,没有,”约塞连叹息道,“我想医生们尽了全力。”

“不,不。”牧师有些脸红了。“我不是这个意思。我是指香烟啦……书啦……或者……玩具什么的。”

“不,不,”约塞连说,“谢谢你。我想我要的东西都有了,缺的只是健康。”

“真是太糟糕了。”

“是啊,”约塞连说,“没错,是太糟了。”

牧师又动了一下身子,左顾右盼了好几回,然后抬头凝视天花板,接着又垂目盯着地上出神。最后,他深吸了一口气。

“内特利上尉托我向你问好,”他说。

约塞连听说内特利上尉也是他的朋友,心里很是过意不去。看来,他俩的谈话终究有了一个基础。“你认识内特利上尉?”他遗憾地问道。

“认识,我跟他很熟,”“他有些疯疯癫癫的,对不对?”

牧师笑了笑,笑得很尴尬。“这我倒是不怎么清楚,我想我跟他还没那么熟。”

“你尽可相信我的话,”约塞连说,“他的确有些疯疯癫癫的。”

接着是片刻的沉默,牧师仔细考虑了一番,之后,突然打破沉默,问了个突兀的问题:“你就是约塞连上尉?”

“内特利一开始就很不如意,因为他的家庭背景很好。”

“请原谅,”牧师胆法地追问道,“我或许犯了个大错。你就是约塞连上尉?”

“没错,”约塞连坦诚他说,“我就是约塞连上尉。”

“二五六中队的?”

“是二五六中队的,”约塞连答道,“我不知道这儿还有别的什么人也叫约塞连上尉。据我所知,我是唯一的约塞连上尉,不过这只是就我自己所知道而言的。”

“我明白了,”牧师说,显得有些不怎么高兴。

“如果你想替我们中队写一首象征主义诗的话,”约塞连指出,“那就是二的八次方。”~一-“不,”牧师低声道,“我没打算给你们中队写什么象征主义诗。”

约塞连猛地挺直身子。他发现了牧师衬衫领子的另一边有一枚小小的银十字架。他惊愕不已,因为以前他从未跟一位随军牧师这么面对面谈过话。

“原来你是一位随军牧师,”他兴奋得大声叫了起来,“我不知道你是随军牧师。”

“呃,没错,我是牧师,”牧师答道,“难道你真的不知道?”

“是啊,我真的不知道你是随军牧师。”约塞连目不转睛地看着牧师,咧大了嘴,一副入迷的样子。“我以前还真没见过随军牧师呢。”

牧师又红了脸,垂目注视着自己的双手。他约摸有三十二岁,个子瘦小,黄褐色头发,一双棕色的眼睛看来缺乏自信。他那狭长的脸很苍白,面颊两侧的瘦削处满是昔日长青春痘所留下的瘢痕。

约塞连很想帮他忙。

“要我帮什么忙吗?”倒是牧师先开口问了起来。

约塞连摇了摇头,还是咧着嘴笑。“不用,很抱歉,我想要的东西都有了,我在这儿过得很舒服。说实在的,我也没什么病。”

“那很好嘛。”牧师话一出口就觉得懊悔,连忙把指节塞进嘴里,惶惶然地傻笑起来,可是约塞连依旧缄口不语,甚是令他失望。

“我还得去探望飞行大队的其他人,”末了,他语带歉意地说,“我会再来看你的,也许明天吧。”

“请你一定要来,”约塞连说。

“只要你真想见我,我就来,”牧师低下头,很是羞怯地说,“我晓得好多人见了我都很不自在。”

约塞连充满深情他说:“我真的想见你,你不会让我感到不自在的。”

牧师甚是感激地绽开了笑容,随即垂目细细看了看一直捏在手里的一张纸条。他不出声地挨次数着病房里的床位,接着,将信将疑地把注意力集中到了邓巴身上。

“请问一下,”他低声道,“那位是邓巴中尉吗?”

“没错,”约塞连高声回答,“那位就是邓巴中尉。”

“谢谢你,”牧师轻声说,“多谢了。我必须跟他谈谈,我必须跟飞行大队所有住院的官兵聊一聊。”

“住其他病房的也要吗?”约塞连问。

“是的。”

“去其他病房你可得要留神啊,神父,”约塞连提醒他说,“那儿关的可全是精神病病人,尽是些疯子。”

“你不必叫我神父,”牧师解释道,“我是个再洗礼派教徒。”

“刚才提到其他那些病房的事,我可是说真的,”约塞连神情严肃地接着说下去,“宪兵是不会保护你的,因为他们才是疯到了极点。我本应该亲自陪你一块儿去,但是我不敢。精神病可是接触传染的。我们住的这一间是全医院唯一没有精神病病人的病房,除了我们这些人之外,人人都是疯子。这样说来,全世界或许只有这间病房没住精神病病人。”

牧师立刻站了起来,悄悄离开约塞连的病床,随即微笑着点了点头,要他放心,并答应一定谨慎行事。“我该去看望邓巴中尉了,”他说。可是他又有点悔恨地舍不得离去。最后,他问了一句:“邓巴中尉人怎么样?”

“没话说,”约塞连满有把握他说,“实实在在是个好人,令人钦佩。他可是全世界最有奉献精神的一个人。”

“我不是这个意思,”牧师说罢,又低声问道,“他病得厉害吗?”

“不,不厉害。说实在的,他压根儿就没什么病。”

“那就好。”牧师松了口气,如释重负。

“是啊,”约塞连说,“没错,是很好。”

牧师见过邓巴后,便起身离开了病房。他刚走,邓巴就对约塞连说:“随军牧师你看见没有?随军牧师。”

“他真可爱是不是!”约塞连接口道,“也许他们该投他三票。”

“他们是谁?”邓巴有些疑惑地问道。

病房尽头有一个小小的空间,用绿色三合板隔了起来,里面搁了张床铺,主人则是位中年上校,始终板着一张脸。他老是在床上忙个不歇。有个女人每天都来探望他,这女人看来很温柔,长得很甜,一头银灰色卷发。她不是护士,不是陆军妇女队队员,也不是红十字会的女职员,但是每天下午,她必定来皮亚诺萨岛上的这所医院报到。每次来,她都穿一身色彩柔和淡雅且又时髦考究的夏装,一双半高跟白皮鞋,腿上穿的尼龙长袜始终笔直。这位上校在通讯司令部供职,昼夜忙碌不停地把内地传送来的一连串电文记录到一本本用纱布做成的正方形记录簿上,每记满一本,他便细心封好,放入床头柜上一只有盖的白桶内。上校风度不凡,嘴巴宽大,两颊凹陷,双眼深迭,目光阴郁,似发了霉一般,脸色灰蒙蒙的。每次咳起嗽来,他总是小心翼翼地压低声音,心里亦不由自主地厌恶起来,遂用记录簿慢慢轻拍自己的嘴唇。

上校老是被一大群专家围绕着。为了确诊他的病情,这些专家正在进行特别研究。他们用光照他的眼睛,检测他的视力,用针扎他的神经,看他是否有感觉。这些专家中有泌尿学家、淋巴学家、内分泌学家、心理学家、皮肤学家、病理学家、囊肿学家,而他们的任务就是研究上校身上各个与自己学科相关的系统。此外,还有一位哈佛大学动物学系的鲸类学家,此人是个秃顶,一脸迂腐,曾因IBM公司一台机器的阳极出了毛病,被人无情地劫持到这支卫生队来,陪伴这位垂死的上校,试着想跟他探讨《白鲸》这部小说。

上校接受了全面检查。他身上的每一个器官都上了麻醉药,动过刀,涂过药粉,清洗干净,接着又让人摆弄着照了相,同时亦被挪动过,取出后再放回原先的部位。那个衣着整洁、身材修长挺秀气的女人则常坐在床边抚摸着他,而她微笑时的神情都带着一种端庄的忧伤。上校身材瘦长,有些驼背,起身走路时,弯腰曲背得更是厉害,身体屈成一个拱形。他挪步时异常小心翼翼,一步步缓慢前移,此外他的两眼下还有很深的黑眼圈。那女人说话很轻,甚至比上校的咳嗽声还要轻,大伙儿谁亦不曾听见她的说话声。

不出十天,得克萨斯人便把所有病员清理出了病房。最先离开病房的是那位炮兵上尉,随后,大批病员相继迁出。邓巴、约塞连和驾驶战斗机的上尉飞行员是同一天上午逃出病房的。邓巴的晕眩症状消失了,上尉飞行员擤了擤鼻涕,约塞连则跟医生们说,他的肝早就不痛了。这病好得还真快,就连那位准尉也逃之夭夭了。十天之内,得克萨斯人就把病房里所有的病员赶回了各自的岗位,只有刑事调查部的那名工作人员留了下来——他从上尉飞行员那儿染上了感冒,后来竟转成了肺炎。
