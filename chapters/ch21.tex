\chapter{德里德尔将军}
 
    卡思卡特上校不再想有关牧师的任何事情,而是陷入了一个使他不寒而栗的新问题:约塞连!

    约塞连!只要一提到这个令人讨厌、憎恶的名字就会使他血液冰凉、呼吸困难而直喘粗气。牧师第一次提到约塞连这个名字时就像在他的记忆深处敲响了一面预示不祥之兆的锣。门栓咋咯一声,门关上了,他头脑中所有有关队伍中那个裸露着身体的军官的记忆立刻涌现出来,使他感到羞辱,那些刺痛他的细节像令人痛苦、窒息的潮水一样劈头盖脸朝他袭来。他浑身冒汗、发抖。这个不吉祥的、不大可能的巧合如此狰狞可怖,除了是最骇人听闻的不祥之兆外,实在没有什么别的解释。那天,那个一丝不挂地站在队伍中从德里德尔将军手里接受优异飞行十字勋章的军官也叫——约塞连!现在他刚刚下达命令,要他的飞行大队的官兵飞行六十次,可又有一个叫约塞连的人威胁说要同这道命令过不去。卡思卡特上校满腹忧愁,不知这会不会是同一个约塞连。

    他带着一副难以忍受的痛苦神情吃力地站起来,开始在办公室里来回走动。他觉得自己的面前是个神秘人物。他闷闷不乐地承认,对他而言,队伍中有个一丝不挂的军官的确是件丢人现眼的事。就像原先制定好的轰炸线在空袭博洛尼亚之前被篡改,还有轰炸弗拉拉的大桥的任务被拖延了七天一样使他丢丑。好在弗拉拉的大桥最后终于被炸毁了,这也算是他的一个荣耀,他想起来心里乐滋滋的。不过,第二次转回去轰炸时损失了一架飞机,这又是桩丢脸的事,想到这他又很泄气;由于一个投弹手胆怯而不得不两次飞抵目标,这给他丢了脸,然而他却请求并获准为那个投弹手颁发了勋章,这又使他感到十分荣耀。他突然想到,那个投弹手的名字也叫约塞连,因此一时惊愕得说不出话来。现在有三个约塞连!他那双淌着粘液的眼睛因吃惊而胀得鼓鼓的,他惊慌失措地赶忙转过身去看看身后在发生什么事情。几分钟前,他的生活中根本没有什么约塞连,而现在他们就像妖精似的越变越多。他努力使自己保持平静。约塞连不过是个普通的名字,也许实际上并没有三个约塞连而只有两个约塞连,甚至可能只有一个约塞连——然而那没有什么区别!上校仍然处于严重的危险之中。直觉警告他,他正接近一个巨大的,不可测知的宇宙顶点。一想到约塞连,不管他最终会是谁,将注定要成为他的克星,他那宽厚、肥胖、高大的身躯从头到脚像筛糠似的颤抖起来。

    卡思卡特上校并不迷信,但他确实相信预兆,于是他在办公桌后坐了下来,在他的活页记事本上做了个秘密的记号,便立即开始研究有关约塞连的这一整个可疑的事件。他用粗重、果断的笔迹写下了提示,在提示后面醒目地画上一连串密码似的标点符号以示强调,然后在整个内容下面画上两道横线,结果便是如下:

    约塞连!!!(?)!

    上校写完后靠向椅背,对自己感到非常满意,因为他刚才采取了迅速的行动来应付这一显露凶兆的危机。约塞连——看见这个名字他就发抖。这个名字里竟有那么多的S字母。它一定具有颠覆性,就像颠覆这个词本身一样。它也像煽动和阴险这两个词,像社会主义者、多疑、法西斯分子和共产主义者这些词。这是一个可僧的、令人厌恶的外国人的名字,一个引不起别人信任的名字。

    它一点也不像卡思卡特、佩克姆和德里德尔这些干净、利落、诚实的美国名字。

    卡思卡特上校慢慢地站起来、又开始在办公室里踱起步来。他几乎是无意识地从一筐红色梨形番茄的上面拿起一只,狠狠地咬了一大口。他立刻扭曲了脸,把剩下的番茄扔进了废纸篓。上校并不喜欢吃红色梨形番茄,即使是他自己的也不喜欢,而这些番茄并不是他自己的。这些番茄是科恩中校从遍布皮亚诺萨岛的各个市场上以不同的名义买来的,然后在半夜里把它们搬到上校在山上的农舍里,第二天早晨再运到大队司令部来卖给米洛,由米洛付给卡思卡特上校和科恩中校一些佣金。卡思卡特上校时常怀疑他们这样倒卖番茄是否合法,但科恩中校说这事合法,于是他尽力不常去考虑这件事。他也无法知道他在山上的房子是否合法,因为那也是由科恩中校一手安排的。卡思卡特上校对他是否买下了那房子的产权或者只是租用、是从谁手中买下的、付了多少钱等,一概不知。科恩中校是律师,如果科恩中校跟他说欺骗、敲诈、盗用现金、贪污、偷漏所得税和黑市投机是合法的,卡思卡特上校也只能同意。

    关于他在山上的那所房子,卡思卡特上校所知道的一切就是他有这么一所房子,而且讨厌它,他每隔一周去那儿呆上两三天。

 


    为的是保持一种假象,即他山上的那所潮湿、漏风的石头墙农舍是个寻欢作乐的金碧宫殿,但实际上没有什么比呆在那儿更让他厌烦的了。各地的军官俱乐部里都充斥着模糊不清但熟悉的话语,大家谈论着那些放荡不羁但又见不得人的狂饮乱嫖之事,谈论与那些最漂亮、最惹人、最容易被撩动、也最容易满足的意大利名妓、电影明星、模特儿和伯爵夫人幽会的销魂之夜:但从未有过这样的令人销魂的幽会之夜或见不得人的狂饮乱嫖之事。假如德里德尔将军或佩克姆将军哪怕有一次表示过有兴趣同他一起参加这些狂欢,这些事情也许有可能发生、但他们两人谁也没有表示过。因此,上校当然不会浪费时间与精力去同漂亮女人寻欢作乐,除非那样做对他有什么好处。

    上校害怕在农场的房子里度过那些阴湿、寂寞的夜晚和沉闷、单调的白昼。他回到飞行大队后有更多的兴趣,可以对所有他不害怕的人吹胡子瞪眼睛。但是,正如科恩中校时常提醒他的那样,假如他从不去住,那么在山上拥有一所农舍就没有多大魅力。他每次开车去他的农舍时都是一副顾影自怜的样子;他在吉普车里带着一支猎枪,用它打鸟,打红色梨形番茄,以此来消磨那单调无聊的时光。那儿确实种了一些红色梨形番茄,一行行歪七扭八的,无人照看,摘起来也太麻烦。

    对有些下级军官,卡思卡特上校仍然认为有必要表示一点敬意,尽管他不愿意也没有把握是不是非得把——德-科弗利少校包括在内,但他还是把他包括进去了。对他来说,——德-科弗利少校是个极为神秘的人物,就像他本人对梅杰少校和其他所有曾注意过他的人来说也很神秘一样。对于——德-科弗利少校,卡思卡特上校不知道该持什么态度,是尊敬呢还是蔑视。尽管——德-科弗利少校比卡思卡特上校要年长许多,但他只不过是个少校。不过,许许多多其他的人如此尊敬、敬畏甚至害怕——德-科弗利少校,因此卡思卡特上校觉得他们也许都知道些什么事情——德-科弗利少校是个不吉利的、不可思议的人物,他使卡思卡特上校常常坐立不安,就连科恩中校也得提防他;每个人都害怕他,但谁也不知道为什么。甚至没有一个人知道——德-科弗利少校的教名是什么,因为从来没有人敢冒冒失失地去问他。卡思卡特上校得知——

 


    德-科弗利少校外出了,他不在,上校很高兴,可他又想到——德-科弗利少校也许在什么地方阴谋反对他,于是他又希望德-科弗利少校回到他所属的中队,那样他就处于监视之中了。

    过了一会儿,卡思卡特上校的两只脚由于来回走动过多而疼痛起来。他重又在办公桌后坐下,下决心对整个军事形势作一周密而系统的估计。他摆出一副善于处理事务的人具有的那种做事井然有序的样子,找出一大本白色的拍纸簿,在纸正中划了一道竖线,在靠近竖线的上方划了一道横线,将整页纸分成两个宽度相等的空白栏。他休息了一会儿,对一些关键问题作了考虑。然后他伏在桌子上,用拘谨而过分讲究的笔迹在左边一栏的顶端写上:“耻辱!!!”在右边一栏的顶端写上:“荣誉!!!”他再次靠向椅背,带着赞赏的目光从客观的角度来检查他画的图。在慎重地考虑了几秒钟后,他小心翼翼地舔了舔铅笔尖,在“耻辱!!!”一栏下写了起来,每写完一项都要停下来仔细考虑一下,其内容如下:

    弗拉拉

    博洛尼亚(轰炸期间轰炸线在地图上被篡改了)

    双向飞碟射击场

    队伍中有个赤裸着身体的军官(轰炸阿维尼翁之后)

    然后他补充写上:

    食物中毒(轰炸博洛尼亚期间)

    再写上:

    呻吟声(下达轰炸阿维尼翁简令时的流行病)

    然后又加上:

    牧师(每晚在军官俱乐部里逗留)

    尽管他不喜欢牧师,但他还是决定对牧师宽宏大量,于是在“荣誉!!!”一栏下写上:

    牧师(每晚在军官俱乐部里逗留)

    这样,关于牧师的两条记录就互相抵消了。在弗拉拉和队伍中有个赤裸着身体的军官(轰炸阿维尼翁之后)这两条旁边,他又写上:

    约塞连!

    在博洛尼亚(轰炸期间轰炸线在地图上被篡改了),食物中毒(轰炸博洛尼亚期间)和呻吟声(下达轰炸阿维尼翁简令时的流行病)这三条旁边,他果断地打上了醒目粗大的?

    那些打上了“?”的条目是他想立刻进行调查的事件,为的是确定约塞连是否参与了这些事件。

    突然,他写字的手臂开始发抖,无法再写下去。他惊恐地站起来,感到手脚迟钝、极不灵活,于是急忙冲到敞开着的窗户旁,大口地呼吸着新鲜空气。他的目光落在了双向飞碟射击场上。他一阵昏眩,痛苦地尖叫了一声,两只狂乱、通红的眼睛疯狂地在办公室的墙壁上扫来扫去,仿佛墙上挤满了许许多多的约塞连。

    没有人爱他。虽然佩克姆将军喜欢他,但德里德尔将军恨他。

    不过,他不能肯定佩克姆将军喜欢他,因为佩克姆将军的副官卡吉尔上校无疑有自己的野心,他可能一有机会就在佩克姆将军面前说他的坏话。他断定,除了他自己之外,唯一的一名好上校是一位死了的上校。在上校中,他唯一信赖的是穆达士上校,但即便穆达士上校也是靠他岳父提携的。虽然他的大队被米洛的飞机轰炸一事也许是他的一个奇耻大辱,但米洛无疑是他的骄做。米洛通过向大家透露部队联营企业同敌军的交易取得了巨额纯利润而最终平息了所有的抗议。而且,他还使所有的人相信,从私营企业的立场出发,轰炸自己的人和飞机的的确确是一个值得称赞并十分有利可图的打击。上校对米洛不十分有把握,因为其他上校正竭力想把他引诱走。此外,那个讨厌的一级准尉大个怀特-哈尔福特还在卡思卡特上校的飞行大队里。据那个又讨厌又懒惰的布莱克上尉说,一级准尉大个怀特-哈尔福特实际上是应对博洛尼亚大围攻期间轰炸线被篡改之事负责的人。卡思卡特上校之所以喜欢一级准尉大个怀待-哈尔福特,是因为每次一级准尉大个怀特-哈尔福特喝醉了酒而且看见穆达士上校也在场,他就不停地对着那个讨厌的穆达士上校的鼻子狠揍。他希望一级准尉大个怀特-哈尔福特也会开始朝科恩中校的胖脸上狠揍。科恩中校是个讨厌的、自作聪明的家伙。第二十六空军司令部里有人对他怀恨在心,把他写的每份报告都签上辱骂、训斥的批示退回来。科恩中校买通了司令部里一个名叫温特格林的精明的邮件管理员,竭力想搞清楚那人是谁。他不得不承认,第二次转回去轰炸弗拉拉时损失了一架飞机对他不会有什么好处,另一架飞机在云层中失踪也同样不会对他有益——

    这件事他甚至忘了写下来。他带着渴望的神情极力想记起约塞连是否同那架在云层里的飞机一起失踪,但他很快就意识到,如果约塞连还在这儿吵吵闹闹,说只要再飞五次就完成了这些讨厌的飞行任务的话,那他就不可能同那架在云层中的飞机一起失踪。
 


    卡思卡特上校理智地想了想,如果约塞连反对飞六十次,那么六十次的飞行任务对那些官兵来说也许是太多了。然而他随后又想到,强迫他的部下去执行比别人更多的飞行任务被认为是他取得的最明显的实绩了。正如科恩中校常常说的那样,战争中只知道执行命令的飞行大队长比比皆是,因此要突出自己独一无二的领导才能,必需采取某种富有戏剧性的姿态,比如要求自己的大队去执行比其他任何轰炸机大队都要多的战斗飞行任务。当然,将军中似乎没有一位反对他的做法,但就他所能察觉到的,他们对此也没有什么特别深的印象,这使他觉得也许六十次战斗飞行任务还远远不够,他应该立即把飞行次数提到七十、八十、一百,甚至二百、三百,或者六千次!

    毫无疑问,他在像佩克姆将军那样文雅、和蔼的人手下工作要比在像德里德尔将军那样粗鲁、迟钝的人手下工作处境会好得多,因为尽管佩克姆将军从未丝毫表示过他赏识或喜欢他,但佩克姆将军有眼力,有天赋,受过名牌大学的教育,能充分了解他的价值,赏识他的能力。卡思卡特上校敏锐的洞察力足以使他认识到,在像他自己和佩克姆将军这样阅历丰富而又十分自信的人之间从不需要明确地表示对对方的承认,他们生来就互相了解,离得很远就能互相产生好感。他们属于同一类人,这就足够了,他知道提升只是个时机问题,他得小心谨慎地等待。不过他又注意到佩克姆将军从未特别看中他,也从不煞费苦心地给卡思卡特上校留下满腹警句和学识的印象、就像将军对他周围的人,甚至士兵一样。要么是卡思卡特上校的心思没有传到佩克姆将军那儿,要么佩克姆将军就不是那个他假装出来的才智横溢、辨别力强、文质彬彬、具有远见卓识的人;而德里德尔将军倒的的确确是个敏锐、可爱、才华横溢、阅历丰富的人,在他的手下他的处境肯定会好得多:突然,卡思卡特上校对众人是否支持他一无所知,于是他用拳头打起铃来,叫科恩中校速到他的办公室来,向他保证,每一个人都爱他,约塞连只是他在想象中虚构出来的人物,他上校本人在为成为将军而进行的英勇、辉煌的战役中正取得惊人的进展。

    事实上,卡思卡特上校根本没有机会成为将军。一方面是因为有个叫温特格林的前一等兵,他也想当将军,于是对任何可能给卡思卡特上校带来声誉的信函,无论是卡思卡特上校本人写的,还是别人写给卡思卡特上校的或是有关卡思卡特上校的:他一概加以歪曲、销毁、拒投或者写错投递地址;另一方面是因为已经有了一个将军用,即德里德尔将军,他知道佩克姆将军在觊觎他的位子但又不知道如何阻止他。

    联队司令德里德尔将军五十岁刚出头,他粗率迟钝、身材矮胖、胸部圆得像水桶似的。他的鼻子又短又阔、红乎乎的,肥胖、苍白、凸起的眼睑像咸肥肉似的一圈圈围着他那对灰色的小眼睛。他有个护士和女婿跟着他。没有喝醉酒时,他习惯于长时间沉默不语。德里德尔将军为把部队的工作搞好浪费了太多的时间,现在已为时太晚了。新的权力联盟已经形成,而祖他排除在外,他简直不知如何去应付。稍不留神,他那张冷峻、阴沉的脸就会因失败和挫折而露出闷闷不乐、心事重重的神色。德里德尔将军以酒浇愁。他的情绪变得反复无常、难以捉摸。“战争就是地狱。”他无论是喝醉了还是清醒时常常这样说,而且他心里也真的是这么想的,然而这并不妨碍他靠战争谋得高官厚禄,也不妨碍他把女婿拉进军队同他在一起,尽管翁婿两人常常争吵。
 


    “那个杂种,”无论谁在军官俱乐部里那张曲线形柜台前碰巧站在他旁边,他都会这样轻蔑地咕哝一句,向他抱怨自己的女婿。

    “他能有这一切全亏了我。他是靠了我发迹的,这个狗娘养的混帐东西!他还嫩着呢,还不能独自混出个样子来。”

    “他以为他什么都知道。”在柜台的另一头,穆达士上校总会用气愤的语气向他周围的人反驳他的岳父。“他不接受批评,也不愿听别人的忠告。”

    “他所能做的一切就是给别人提忠告,”德里德尔将军总会粗声粗气地哼着鼻子说,“要不是我,他现在还只是个下士。”

    德里德尔将军总是由穆达士上校和他的护士两人陪着。那护士可是个美人儿,见过她的人都认为她与人们见过的任何漂亮女人比都毫不逊色。德里德尔将军的护士身材小巧,圆圆的脸上生着一对快乐的蓝眼睛,丰满的双颊上有两个小酒窝,一头金色的卷发下边向上卷起,梳得整整齐齐。她逢人便露出微笑,却从不开口说话,除非有人跟她说话才应酬几句。她胸脯丰满,皮肤雪白。她的媚力是难以抗拒的,男人们总是目不转睛地侧着身子慢慢地从她身旁走开。她丰满娇艳、甜美温顺、沉默寡言,弄得所有的人,除了德里德尔将军之外,都如痴如醉。

    “你该看看她光着身子是什么样子,”德里德尔将军用沙哑的嗓门津津有味地笑着说,而此时他的护士就站在他的肩旁得意地微笑着。“在联队我的房间里,有她的一件用紫红色丝绸做的制服,那衣服太小,她的两个乳头鼓得老高,像两只大樱桃似的。是米洛给我弄来的衣料。那制服小得里面连短裤和胸罩都不能穿。有几个晚上穆达士在这儿时,我让她穿上那制服,撩得他魂不守舍。”德里德尔将军放开沙哑的嗓子哈哈大笑。“要是你能看见她每次挪动身体时她那件衣裳里面的情景才妙呢。她把他弄得神魂颠倒。只要我抓到他向她或其他别的女人伸一伸手,我就立刻把这个好色的杂种一下子降为列兵,让他当一年炊事兵。”
 


    “他让她在我身边转悠,就是想把我撩得魂不守舍,”穆达士上校在柜台的另一头愤愤不平地指责说,“在联队里,她有一件用紫红色丝绸做的制服,那衣服太小,她的两个乳头鼓得老高,像两只大樱桃似的。那制服小得里面连短裤和胸罩都不能穿。要是你能听见她每次挪动身体时那绸衣服发出的沙沙声就好啦。要是我对她或其他别的姑娘有什么非礼的举动,他就会把我一下子降为列兵,让我当一年炊事兵。她撩得我神魂颠倒。”

    “自从我们到海外以来,他还没有和女人上过床呢。”德里德尔将军吐露了秘密。一想到这个恶毒的主意,他就像个性虐待狂似的大笑起来,他那四四方方、满头灰白头发的脑袋也随着笑声直晃悠。“我之所以不让他呆在我看不见的地方,这就是其中一个原因,这样他就不能去找女人。你能想象出这个可怜的狗娘养的有多难过吗?”

    “自从我们到海外以来,我还没有和女人上过床呢,”穆达士上校眼泪汪汪地抱怨说,“你能想象出我有多难过吗?”

    德里德尔将军生气的时候,对任何人都会像对穆达士上校那样寸步不让。他不喜欢装假、圆滑、做作。作为职业军人,他的信条是,始终如一,简单明了。他认为接受他命令的年轻军人应该心甘情愿地为了这位向他们发布命令的老军人的理想、抱负和特有的风格献出自己的生命。对他而言,他手下的军官和士兵都只是军人。他所要求的就是他们做好自己的工作,除此之外,他们可以随心所欲,想干什么就干什么。只要愿意,他们可以像卡思卡特上校那样强迫他们的部下执行六十次飞行任务;只要乐意,他们也可以像约塞连那样一丝不挂地站在队列里,尽管当时一看到这一情景,德里德尔将军那花岗岩似的下巴一下子张了开来。他专横而傲慢地大步沿着队伍走过去,想看清楚队伍中是不是真的有个人浑身一丝不挂,只穿了双皮鞋立正站在那儿,等着他颁发勋章。德里德尔将军一句话也没说。卡思卡特上校发现约塞连时,差点昏过去。

    科恩中校快步走到他身后,一把抓住他的一只手臂。接着是一阵静得出奇的沉默。温暖的海风不停地从海滨吹来,一头黑毛驴拉着一辆装满了脏草的旧马车在大路上辘辘驶过来,赶车的农夫头戴一顶帽檐低垂的帽子,身穿一套褪了色的棕褐色工作服,他对右边那一小块场地上正在举行的正式军事仪式毫不在意。最后,德里德尔将军说话了。“回到汽车里去,”他转过头对跟在他身后的护士厉声说道。护士带着微笑蹦蹦颠颠地朝将军的那辆深褐色军用汽车走去。汽车停在约二十码之外那块长方形空地的边上。德里德尔将军带着严厉的表情静静地等着,直到他听见车门砰的一声关上后才问道:“这个人叫什么名字?”

    穆达士上校查看了一下名册。“这个人叫约塞连,爹。他获得了一枚优异飞行十字勋章。”

    “唉;真该死,”德里德尔将军嘟哝着说,由于觉得有趣,他那血红色的石板似的脸上露出了温和的神色。“你为什么不穿衣服,约塞连?”

    “我不想穿。”

    “你说不想穿是什么意思?你究竟为什么不想穿?”

    “我只是不想穿,长官。”

    “他为什么不穿衣服?”德里德尔将军回过头来问卡思卡特上校。

    “他在跟你说话,”科恩中校从后面贴着卡思卡特上校的肩膀小声对他说道,一边用胳膊肘猛地捅了一下他的背。

    “他为什么不穿衣服?”卡思卡特上校带着极度痛苦的表情问科恩中校,一面轻揉着刚才被科恩中校捅过的地方。

    “他为什么不穿衣服?”科恩中校问皮尔查德上尉和雷恩上尉。

    “他的飞机里有个士兵上周在阿维尼翁上空被打死了,溅得他浑身上下都是血,”雷恩上尉回答说,“他发誓再也不穿军装了。”

    “他的飞机里有个士兵上周在阿维尼翁上空被打死了,溅得他浑身上下都是血,”科恩中校直接向德里德尔将军报告说,“他的制服还在洗衣房里。”

    “他的其他制服呢?”

    “也都在洗衣房里。”

    “他的内衣呢?”德里德尔将军问道。

    “他的所有内衣也都在洗衣房里,”科恩中校答道。

    “这些话我听起来好像是一大堆胡说八道,”德里德尔将军断言道。

    “是一大堆胡说八道,长官,”约塞连说。

    “请别担心,长官,”卡思卡特上校向德里德尔将军保证说,一边狠狠地瞪了约塞连一眼。“我亲口向您保证,这个人会受到严厉的惩罚的。”

    “我干吗要在乎他会不会受到惩罚?”德里德尔将军又惊奇又气愤地回他一句。“他刚刚得到一枚勋章。如果他愿意不穿衣服接受勋章,那又关你什么屁事?”

    “这正是我的意思,长官!”卡思卡特上校以毫不含糊的热情附和道,一边说一边用潮湿的白手帕擦额头的汗水。“但是,长官,如果按照佩克姆将军最近发布的关于在战区应着合适军装的备忘录的精神,您还会那么说吗?”

    “佩克姆?”德里德尔将军的脸色阴沉了下来。

    “是的,长官,长官,”卡思卡特上校奉承他说,“佩克姆将军甚至建议我们让官兵穿着军礼服去作战,这样,他们被击落时会给敌军留下一个好印象。”

    “佩克姆?”德里德尔将军重复了一遍,仍旧迷惑不解地斜视着他。“佩克姆与这事到底有什么关系?”

    科恩中校又用胳膊肘使劲捣了一下卡思卡特上校的背。

    “绝对没有关系,长官!”卡思卡特上校利落地答道,背上疼得要命,只好缩着身子,轻轻地揉着科恩中校刚才又捣过的地方。“正是因为这个原因,我才决定在没有机会同您商量之前,绝对不采取任何行动。我们完全不必理会它,行吗,长官?”

    德里德尔将军完全不理会他,轻蔑而带着恶意地转过身去,把装在盒子里的勋章递给了约塞连。

    “把我那个姑娘从车里叫回来。”他怒气冲冲地命令穆达士上校,然后沉着脸低着头呆在原地,等着他的护士来到他的身边。

    “立刻命令办公室取消我刚刚下达的我部官兵在执行战斗任务时必须戴领带的那条命令,”卡思卡特上校急切地从嘴边小声对科恩中校说。

    “我跟你说不要下这道命令吧,”科恩中校窃笑道,“可你就是不愿听我的。”

    “嘘——!”卡思卡特上校警告他说,“该死的,科恩,你捣我的背干吗?”

    科恩中校又窃笑起来。

    德里德尔将军无论去哪里,他的护士总跟着他,甚至在下达轰炸阿维尼翁任务时跟着他进了简令下达室。那天,她带着傻乎乎的微笑站在讲台旁边,她身着上红下绿的制服站在德里德尔将军身旁,就像肥沃的绿洲里盛开的一朵鲜花。约塞连看着她,疯狂地爱上了她。他情绪低沉,内心感到空虚、麻木。他坐在那里,一面听着丹比少校用单调沉闷的男低音以教训人的口气描绘在阿维尼翁等着他们的密集的高射炮火,一面垂涎欲滴地盯着她那丰满的红嘴唇和长着酒窝的脸。一想到他也许再也见不到这个可爱的女人了,而他现在无限深情地爱上了她,但还没有和她说过一句话,他突然万分绝望地呻吟起来。当他凝神看着她时,由于伤心、害怕和渴望,他浑身颤抖、疼痛。她是那么美丽。他崇拜她脚下的那块土地。他用黏糊糊的舌头舔了舔他那干枯的嘴唇,又痛苦地哼起来,这次哼得声音比较响,吸引了他周围那些穿着深褐色工作服、系着白色降落伞带、坐在一排排粗糙的木条凳上的人。他们用吃惊、搜寻的目光向他这边张望着。

    内特利惊慌地匆忙转向他。“怎么啦?”他低声问,“怎么回事?”

    约塞连没听见他说话。他情欲难熬,内心烦乱,又很遗憾,变得痴迷不醒。德里德尔将军的护士只是稍有些丰满。约塞连的头脑里充满了奇想:她那闪闪发光的金发、他未曾握过的纤纤素手、那领口敞开着的粉红色衬衫里面圆滚滚的、他从未摸过的妙龄女郎的Rx房,还有她那光滑的草绿色华达呢紧身军短裤下肚皮和大腿交汇处晃动着的、成熟的三角形腹肌。他贪婪地陶醉于她,从她的头一直到她那涂了颜色的脚趾。他决不想失去她。“哎哎哎哎哎哎哟。”他又哼起来。这次,整屋子的人都被他那颤抖着拉长了的呻吟声惊动了。一股吃惊、不安的感觉袭向讲台上的军官们,甚至正在给大家对表的丹比少校也一时分了神。他正在数秒,几乎得重新开始。内特利顺着约塞连被钉住了似的目光一直看到长长的木板礼堂那头,直到他看见德里德尔将军的护士。当他猜到了是什么在折磨着约塞连时,他吓得浑身发抖,脸色苍白。

    “别哼了,行吗?”内特利压低嗓门小声警告他说。

    “哎哎哎哎哎哎哎哎哎哎哟。”约塞连第四次哼了起来,这次声音大得所有的人都能听得清清楚楚。

    “你疯了吗?”内特利使劲用嘘声说,“你会有麻烦的。”

    “哎哎哎哎哎哎哎哎哎哎哟。”邓巴从房间的另一头附和着约塞连。

    内特利听出是邓巴的声音。现在局面已经失去了控制,他转过身去,轻轻地哼了一声:“哎哎哟。”

    “哎哎哎哎哎哎哎哎哎哎哟。”邓巴附和地哼起来。

    “哎哎哎哎哎哎哎哎哎哎哟。”当内特利意识到自己刚才哼了一声时,便恼怒地大声呻吟起来。

    “哎哎哎哎哎哎哎哎哎哎哟。”邓巴又回应他哼起来。

    “哎哎哎哎哎哎哎哎哎哎哟。”一个新的声音从屋子的另一端加入进来,内特利的毛发都竖了起来。

    约塞连和邓巴两人都附和着哼起来,而内特利却缩起了身子,徒劳地向四下打量,想找个洞,带着约塞连一起藏起来。有几个人在强忍住笑。一阵想捣蛋的冲动支配了内特利,当没有人哼哼时,他就故意哼一声。又一个新的声音附和起来。这种不服从上司的做法趣味无穷。内特利趁无人呻吟的间隙又故意挤出一声哼哼。又有一个新的声音响应了他。屋子里一片喧闹,不可收拾,像精神病院似的。有的人怪声尖叫,有的人用脚在地上拖,有的人把东西丢到地上——铅笔、计算器、地图盒,以及敲得丁当作响的防空钢帽。一些未发哼声的人此刻公开地咯咯笑起来。假如不是德里德尔将军亲自出来平息这场喧闹,谁也说不准这自发的呻吟造反行动会闹到什么地步。德里德尔将军坚决地走到讲台中央,走到丹比少校的正前方。丹比少校低着他那颗认真严肃、不屈不挠的头,仍全神贯注地看着表念着:“——二十五秒——二十——十五——”德里德尔将军那张宽大、通红、盛气凌人的脸上露出困惑不解的神色和令人生畏的决心。

    “别闹了,弟兄们,”他简要地命令道。他的眼睛里闪烁着不赞同的眼光,他那四四方方的下巴显得很坚定。“我领导着一支战斗部队,”他语气严厉地对他们说,这时屋子里已变得一片肃静,坐在凳子上的人都吓得直哆嗦。“只要我还是司令,这个大队里就不准再有人呻吟。听明白了吗?”

    所有的人都明白了,唯有丹比少校除外,因为他还在聚精会神地看着他手腕上的表,大声倒数着秒数。“——四——三——

    二——时间到!”丹比少校喊道,说完带着完成任务后的喜悦心情抬起头,却发现没有人在听他的,因此他还得再数一遍。“哎哎哎哎哟。”他失望地哼了一声。

    “怎么回事?”德里德尔将军难以相信地吼了起来,他勃然大怒,杀气腾腾,一下子转过身看着丹比少校,而少校却被吓得慌了神,踉踉跄跄地倒退了几步,开始发抖,冒冷汗。“这个人是谁?”

    “丹比少——少校,长官,”卡思卡特上校结结巴巴地回答说,“我的大队作战参谋。”

    “把他拉出去枪毙,”德里德尔将军命令道。

    “长——长官?”

    “我说把他拉出去枪毙。你听不见吗?”

    “遵命,长官!”卡思卡特上校强忍住自己的感情,口气干脆地答道,然后迅速转向他的司机和气象员。“把丹比少校拉出去枪毙。”

    “长——长官?”他的司机和气象员结结巴巴地问。

    “我说把丹比少校拉出去枪毙,”卡思卡特上校厉声说道,“难道你们听不见吗?”

    两个年轻的中尉机械地点点头,但都不愿意动手,两人不知所措,有气无力地你看看我,我看看你,等着对方先动手把丹比少校拉出去枪毙。他俩以前谁也没有把丹比少校拉出去枪毙过。他俩犹豫不决地从不同方向慢慢挪向丹比少校。丹比少校吓得脸色苍白。

    突然,他两腿一软,向下倒去,两个年轻的中尉冲上前去,一人架住一只胳膊抓住他,使他不致倒在地上。现在他们既然已经抓住了丹比少校,其余的事似乎就很容易了,但是他们没有枪。丹比少校开始哭起来。卡思卡特上校真想跑到他的身边安慰他几句,但又不想在德里德尔将军面前显得婆婆妈妈的。他想到阿普尔比和哈弗迈耶在执行任务时总带着四五口径的自动步枪,于是便开始用目光在一排排的军官中寻找他们。

    丹比少校一哭,刚才还在一旁犹豫不决的穆达士上校再也控制不住自己了,他带着一副自我牺牲的神色苦巴巴地、缺乏信心地向德里德尔将军走过去。“我认为你最好等一分钟,爹,”他犹犹豫豫地建议说,“我认为你不能枪毙他。”

    他的插话使德里德尔将军勃然大怒。“到底是谁说我不能枪毙他的?”他兴师问罪地怒喝道,声音大得使整个建筑都嘎嘎作响。穆达士上校尴尬得满脸通红,俯身贴近他的耳朵小声说着什么。“我究竟为什么不能枪毙他?”德里德尔将军吼道。穆达士上校又小声说了几句。“你是说我不能想枪毙谁就枪毙谁?”德里德尔将军用不妥协的愤怒口气问道。但当穆达士上校继续小声说下去时,德里德尔将军竖起了耳朵,来了兴趣。“那是真的吗?”他问道,满腹怒气也由于好奇消了许多。

    “是的,爹。恐怕是的。”

    “我想,你以为你他妈的精明绝顶,是吧?”德里德尔将军突然痛斥起穆达士上校来。

    穆达士上校的脸又涨得绯红。“不是,爹,这不是——”

    “好吧,把那个违抗上司的狗狼养的放掉,”德里德尔将军厉声说,一边恶狠狠地从他女婿那边转过身来,怒气冲冲地对着卡思卡特上校的司机和卡思卡特上校的气象员吼道:“但是要把他赶出这所房子,让他呆在外面。让咱们继续下达这个该死的简令吧,要不战争就要结束了。我从未见过这么多无能鼠辈。”

    卡思卡特上校机械地向德里德尔将军点了点头,急忙向他手下打了个手势,让他们把丹比少校推到屋外去。然而,当丹比少校被推出去后,却没有人来继续下达简令。大家面面相觑,又吃惊又不知如何是好。德里德尔将军见到大家都愣着不动,气得脸色发紫。卡思卡特上校也不知该怎么办。他刚要开始大声哼哼,这时科恩中校走上前来,帮他控制住了局面。卡思卡特上校噙住泪水,万分欣慰地舒了一口气,感激的心情几乎不知如何表达。

    “现在,弟兄们,我们来对表。”科恩中校以敏捷、威严的神态迅速发号施令起来,两只眼睛讨好地朝着德里德尔将军那个方向骨碌碌转个不停。“我们将对一次表,只对一次,如果一次对不好,德里德尔将军和我将要查一查是什么原因。明白了吗?”他的两眼又转向德里德尔将军,想弄清楚他的这番话是否给将军留下了印象。

    “现在把你们的表拨到九点十八分。”

    科恩中校十分顺利地给大家对好了表,然后信心十足地继续下去。他把当天的指令交待给了大家,又把天气情况说了一下,显得灵活、事事精通但却华而不实。他发觉他正给德里德尔将军留下极好的印象,因此他每隔几秒钟就傻笑着瞟一眼德里德尔将军,从他那儿得到越来越大的鼓舞。他来了劲头,神气活现地整了整衣冠,昂首阔步地在讲台上走来走去,虚荣心十足。他把当天的指令又给大家交待了一遍,然后巧妙地转入鼓舞士气的战前动员,大谈轰炸阿维尼翁大桥对于赢得这场战争是如何重要以及执行任务的每一个人都应该把热爱祖国放在热爱生命之上。他把这番激励士气的宏论讲完后,又把当天的指令给大家说了一遍,强调了进攻的角度,随后又说了一下天气情况。科恩中校觉得自己拥有至高无上的权威。他已经成了大人物了。

    卡思卡特上校慢慢明白过来,当他悟出了个中原因时,他气得目瞪口呆。他妒忌地望着科恩中校继续推行他的鬼计,他的脸拉得越来越长。当德里德尔将军走到他身边时,他简直不敢听他要说什么。将军用整个屋子里的人都能听见的耳语问他:

    “那个人是谁?”

    卡思卡特上校作了回答,心里有一种淡淡的不祥的预兆。接着,德里德尔将军把手握成杯状放在嘴上对他小声说了些什么,使长思卡特上校的脸上放出无比喜悦的光芒。科恩中校看见后,高兴得难以自制,浑身直抖。他是不是刚才被德里德尔将军在战场上提升为上校了?他无法忍受这种悬念。他专横地把手一挥,结束了下达简令,满怀期望地转过身去,准备接受德里德尔将军的热烈祝贺——将军已经迈着大步,头也不回地向屋外走去,身后尾随着他的护士和穆达士上校。科恩中校看见这种情景,失望得一阵晕眩,但只是很短的一刻。他看见了卡思卡特上校还咧开嘴笑着,笔直地站在那儿出神,于是他兴高采烈地跑过去拉住他的膀子。

    “他说了我些什么?”他满怀自豪而又幸福的期望心情激动地问道,“德里德尔将军说了些什么?”

    “他想知道你是谁?”

    “我知道这个。我知道这个。但他说了我些什么?他说了些什么?”

    “你使他恶心。”
