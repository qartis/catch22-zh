\chapter{地下室}

 
    听到内特利阵亡的消息,牧师差点死过去。塔普曼牧师当时正坐在自己的帐篷里,戴着老花镜认认真真地处理着日常文件。突然,电话铃响了,机场上的人向他通报了半空中的飞机相撞事件。

    他顿时感到心如刀割。他的手哆哆嗦嗦地放下电话,另一只手也抖动起来。这真是一场无法想象的灾难。十二个人阵亡——多么令人恐怖,多么令人毛骨悚然!他越想越心惊胆战。他不由自主地祈祷上帝保佑约塞连、内特利、亨格利-乔以及他的其他朋友不在阵亡之列。祈祷完毕,他又懊悔地责备自己,因为祈求他们平安就等于祈求别的他根本不认识的年轻人战死。祈祷也太晚了,可他偏偏只会祈祷。他的心怦怦直跳,那心跳声好像是从外面什么地方传来的。他知道,往后他只要坐上牙科医生的手术椅,只要看到外科手术器械,只要目睹汽车事故,或者只要夜里听见喊声,他的心都会像现在这样怦怦乱跳,并会产生现在这种马上就要死去的可怕感觉。往后他只要看见有人打架斗殴,就要担心自己会被吓昏过去,会在人行道上碰破脑袋,或者会因心脏病发作而毙命,或者突发脑溢血。他不知道自己还能不能见到妻子和三个孩子。他不知道自己应该不应该再去见妻子,因为布莱克上尉对他的劝诱使他在心里对所有女性的贞操和品德产生了强烈的怀疑。他觉得许多别的男人能够给予他妻子更多的性满足。现在,当他考虑死亡问题时,他总是想到他的妻子,而当他想到他的妻子时,他又总是担心会失去她。

    过了一两分钟,牧师觉得自己有力气站起来了,于是便起身心情沉重地、慢慢吞吞地走到隔壁帐篷去找惠特科姆中士。他俩坐上惠特科姆中士的吉普车。为了不让放在膝盖上的双手颤抖,牧师使劲把它们握成拳头。他咬紧牙关,竭力不去听惠特科姆中士兴致勃勃、喋喋不休地对这次灾难性事件大发议论。十二个人阵亡意味着又要准备十二封由卡思卡特上校签名的吊唁通函。这些信件邮寄给阵亡者亲属时可以捆成一捆。这件事使惠特科姆中士产生了一线希望,也许复活节之前可以在《星期六晚邮报》上发表一篇有关卡思卡特上校的文章。

    大地笼罩在深深的寂静之中,似乎那些唯一能打破寂静的人全都被一种不可抗拒的、残忍无情的魔力降服住了。牧师油然生出一股敬畏之感。他还从来没有见到过如此阴森可怕的寂静场面。大约两百名精疲力竭、形容枯槁、无精打采的军人手里拎着降落伞袋,沮丧地、一动不动地围在简令下达室外面。他们面无表情,一个个呆若木鸡,目光死死地盯着不同的方向。他们似乎不愿意离去,也不能够移动了。牧师朝他们走过去时,清清楚楚地听到了自己轻微的脚步声。他的眼睛急切而慌乱地在无声无息呆呆站立着的人群中搜寻着。他终于看见了约塞连,心中不禁一阵狂喜。紧接着,他就注意到约塞连满是灰尘的脸上明显地流露着疲惫、迷惘和深深的绝望,他不禁感到惊恐万分,慢慢地张开了嘴。他立刻就明白了,可又痛苦地不敢承认事实:内特利已经死了。他一脸苦相,轻轻地摇着头,像是在抗议,又像是在哀求。这个消息好似一记重量的拳头,打得他手脚发麻。他不由得抽泣起来。他感到双腿瘫软,好像马上就要倒下去。内特利已经死了。他满心希望是自己弄错了,可是这一线希望也破灭了,因为他突然第一次意识到,周围许多人正用几乎听不见的嗓音低低地但清晰地反复念着内特利的名字。内特利已经死了:这个小伙子战死了。牧师从喉咙里发出一阵呜咽声,他的下巴开始颤抖,他的眼中充满泪水,他放声哭了起来。

 


    他踮起脚尖朝约塞连走过去,想站到他身边去哀悼内特利,分担他无言的悲伤。就在这时,一只手粗暴地抓住了他的胳膊,有人粗声粗气地问道:

    “是塔普曼牧师吗?”

    他吃惊地转过身去,看见面前站着一个又矮又胖、气势汹汹的上校。这个人脑袋很大,面色红润,留着两撇小胡子。他以前从来没有见过此人,“是我,有什么事?”牧师的胳膊被这个人的手指捏得很痛,他使劲地扭动着胳膊,可就是挣脱不出来。

    “跟我们走。”

    牧师惊慌地向后退缩着。“去哪儿?为什么、你们到底是什么人?”

    “你最好跟我们走一趟,神父,”站在牧师另一边的一个身材瘦削、长着一张鹰脸的少校用恭敬而悲伤的语调拖着腔说道,“我们是政府派来的。我们要问你几个问题。”

    “什么样的问题?出了什么事?”

    “你是不是塔普曼牧师?”胖上校质问道。

    “就是他,”惠特科姆中士回答道。

    “跟他们走吧,”布莱克上尉仇视而轻蔑地冷笑一声,冲着牧师大叫起来。“你要是想不吃苦头,就上车吧。”

    几只手不容分说就把牧师拖走了。他想向约塞连呼救,可约塞连离得太远,似乎不会听见。附近的一些军人如梦初醒,开始好奇地打量着他。牧师感到脸上火辣辣的,羞愧地转过脸低下头去。他乖乖地被人领进一辆指挥车里,坐到了后座上那个脸盘又大又红的胖上校和那个虚情假意、萎靡不振的瘦少校之间。刚坐下时,他以为他们要给他戴手铐,便自动地向他们一人伸出一只手腕。前排座位上已经坐着一个军官。一个脖上挂着哨子、头上戴白色钢盔的高个宪兵坐到了方向盘的后面。车门关上了,汽车东倒西歪地开出机场,在崎岖不平的柏油马路上飞驰着。直到这时,牧师才敢抬起眼睛来。

    “你们要把我带到哪里去?”他心虚胆怯地轻声发问,眼睛依然盯着别处。他突然想到,他们是要把飞机空中相撞事件和内特利的阵亡归罪于他,“我做了什么事?”

    “你就不会闭上嘴,让我们向你提问题吗?”上校问。

    “别这样对他说话,”少校说,“没有必要那么粗鲁。”

    “那么叫他闭上嘴,让我们来提问题。”

    “神父,请你闭上嘴,让我们来提问题,”少校同情地劝道,“这样对你更好些。”

    “没有必要叫我神父,”牧师说,”我不是天主教徒。”

    “我也不是,神父,”少校说,“可我恰巧是个非常虔诚的人,我喜欢把所有神职人员都叫做神父。”

    “他甚至不相信散兵坑里有无神论者,”上校嘲弄地说。他随随便便地用胳膊肘戳了戳牧师的肋骨。“说下去,牧师。告诉他,在散兵坑里有无神论者吗?”

    “我不知道,长官,”牧师回答道,“我从来没有到过散兵坑。”

    坐在前排的那个军官猛地转过头来,露出一副找茬吵架的嘴脸。“你不是也从来没有到过天堂吗?可你知道有个天堂,不对吗?”

    “对吗?”上校说。

    “这是你犯下的一项严重罪行,神父,”少校说。

    “什么罪行?”

    “我们还不知道,”上校说,“但我们会调查出来的。而且我们确信,你的罪行是非常严重的。”

    在大队司令部门前,汽车拐下了马路。轮胎发出吱吱扭扭的声响,车速稍微减慢了一点。汽车绕过停车场,开到司令部大楼后面停了下来。三个军官把牧师带下了车。他们排成单行,领着牧师沿一道颤悠悠的木制楼梯往下一直走到地下室,把他带到一间潮湿阴暗的房间里。房间的水泥天花板非常低矮,石头墙裸露着,各个墙角里全都布满了蜘蛛网。一只蜈蚣嗖的一下窜过地板,钻到一根水管下面去了。他们叫牧师坐到一张硬邦邦的靠背椅上,椅子前面是一张小桌子,上面什么也没有摆。

    “你不要客气,牧师。”上校一边亲切地招呼着牧师,一边打开一盏耀眼的聚光灯,把光线直射到牧师的脸上。他又把一套指节铜套和一盒木制火柴放到桌子上。“我们要给你放松放松。”

    牧师不相信地瞪起眼睛。他的牙齿格格打战,四肢瘫软无力。

    他感到无能为力。他知道,他们可以想怎么处治他就怎么处治他。

    这几个残忍的家伙可以就在地下室里活活打死他,没有人会插手救他,没有任何人。也许,那位虔诚、富有同情心的瘦长脸少校是例外,可这位少校正在把一个水龙头打开;让水响亮地滴到水池里。

    接着,他走回到桌前,把一根长长的、沉甸甸的橡皮管放到指节铜套旁。

    “现在一切就绪了,牧师,”少校鼓励说,“只要你没有罪,你就一点用不着害怕。你这么害怕是为什么呢?你没有罪,对吗?”

    “他肯定有罪,”上校说,“罪大着呢。”

    “我犯的是什么罪呀?”牧师哀求道,他越来越感到困惑不解,弄不清该向这几个人中的哪一个求情。那第三个军官没有佩戴肩章,这会儿默不作声地溜到了一旁。“我干了什么啦?”

    “这正是我们打算弄清楚的,”上校回答说。他把一本拍纸薄和一枝铅笔从桌子的另一边推到牧师跟前。“给我们写下你的名字,好吗?用你自己的笔迹。”

    “用我自己的笔迹?”

    “对。随便写在纸上的什么地方。”牧师写完后,上校把拍纸簿拿了回去,从一个文件夹里取出一页纸,把拍纸簿与这页纸并排放好。“瞧见了吗?”他对走到他身旁的少校说。少校正从他的身后严肃地凝视着这两样东西。

    “它们不一样,是吗?”少校承认道。

    “我告诉过你是他干的。”

    “我干什么啦?”牧师问。

    “牧师,这件事太使我感到震惊了,”少校用极为悲哀的语调指责道。

    “什么呀?”

    “我没法告诉你我对你多么的失望。”

    “因为什么呀?”牧师更加慌乱地追问道,“我干了什么事情?”

    “就因为这个,”少校一边回答,一边带着失望、厌恶的神情把牧师方才在上面签过名的拍纸簿扔到桌子上。“这不是你的笔迹。”

    牧师惊奇得直眨眼睛。“这当然是我的笔迹。”

    “不,这不是,牧师,你又在说谎了。”

    “但这是我刚刚写的呀!”牧师恼怒地叫道,“你们看着我写的。”

    “就是这个问题,”少校愤怒地回答道,“我看着你写的。你不能否认这确实是你写的。一个人在自己的笔迹这件事上都说谎,那他在什么事上都敢说谎。”

    “但是,谁在我自己的笔迹这件事上说谎了?”牧师质问道。他心里猛地升腾起一股怒火,一时间竟忘了害怕。“你们是疯了还是怎么啦?你们两个都在讲些什么呀?”

    “我们叫你用你自己的笔迹写下你的名字,可你并没有这么做。”

    “我当然这样做了。如果不是用我自己的笔迹,那么我是用谁的笔迹?”

    “用别的什么人的笔迹。”

    “谁的?”

    “这正是我们打算弄清楚的,”上校威胁说。

    “说吧,牧师。”

    牧师望望这个人,又看看那个人。他越来越疑惧重重,越来越歇斯底里。“那笔迹是我的,”他情绪激昂地坚持道,“如果那不是我的笔迹,那我的笔迹在哪里?”

    “就在这里,”上校回答道。他神情傲慢地把一份缩印邮递邮件的影印件扔在桌上。那上面除了“亲爱的玛莉”这个称呼外,所有的字迹都被涂抹掉了。军邮检查官在信上写着:“我苦苦地思念着你。
 


    美国随军牧师A-T-塔普曼。”上校看到牧师变得面红耳赤,便嘲弄地笑了起来。“怎么样,牧师?你知道这是谁写的吗?”

    牧师已经认出了约塞连的笔迹。过了好长时间,他才回答道:

    “不知道。”

    “可你是认字的,对吧?”上校不依不饶地继续挖苦他。“写信的人签上了自己的姓名。”

    “那是我的姓名。”

    “那么是你写的喽。这就是所要证明的。”

    “但我没有写。这也不是我的笔迹。”

    “这么说,你又一次用别人的笔迹签上了你自己的名字,”上校耸耸肩反驳道,“就是这个意思。”

    “天哪,这简直荒谬透顶!”牧师再也忍耐不下去了,大声叫喊起来,他怒气冲冲地跳了起来,两只拳头握得紧紧的。“我再也不能容忍下去了!你们听见了吗?十二个人刚刚阵亡,我没有时间来回答这些愚蠢的问题。你们没有权利把我扣留在这地方。我可是再也不能容忍下去了。”

    上校一声不吭地朝着牧师的胸部使劲一推,把牧师推倒在椅子上。牧师突然感到浑身软弱无力,又一次心慌意乱起来。少校捡起那根长长的橡皮管,恐吓地在自己摊开的手掌上轻轻抽打着。上校拿起那盒火柴,从里面抽出一根,把它对着火柴盒划火的那面,准备划火。他双眼怒视着牧师,看他还敢做出什么反抗的表示。牧师面容苍白,几乎僵在椅子上不能动弹。聚光灯的强烈光线终于逼得他扭过脸去,水龙头的滴水声越来越响,弄得他心烦意乱,不堪忍受。他真希望他们告诉他,他们究竟需要什么,这样他就知道他应该坦白交待些什么。上校对第三个军官做了个手势,那人便缓步从墙边走到桌子跟前,在离牧师仅仅几英寸的地方坐了下来。牧师紧张不安地等待着。那人的脸上毫无表情,目光阴森逼人。

    “把灯关掉吧,”他回过头去平静地低声说,“这灯光太刺眼了。”

    牧师对他感激地微微一笑,“谢谢你,长官。还有那个滴水的龙头,请关上它吧。”

    “别管那滴水声,”那军官说,“我并不讨厌它。”他往上扯了扯裤腿,好像怕弄皱了那两条整齐的裤缝似的。“牧师,”他随随便便地问,“你是属于哪个教派的?”

    “我属于再浸礼教派,长官。”

    “这是个相当可疑的教派,不是吗?”

    “可疑?”牧师疑惑不解地问,“为什么,长官?”

    “噢,我对这个教派一点都不了解。你不得不承认这一点,对吧?难道这还不使它显得可疑吗?”

    “我不知道,长官,”牧师像个外交官似的心神不定、结结巴巴地回答道。这个人没佩戴肩章,这一点使他觉得很为难,他甚至拿不准自己应该不应该称他为“长官”。他是谁?他有什么权力审问他呢?

    “牧师,我曾经学过拉丁文。在向你提出下一个问题之前我要先让你知道这一点,我认为只有这样做才是公正的。‘再浸礼教徒’这个词是否仅仅意味着你不是浸礼教徒?”

    “我,不,长官,它的含义更广些。”

    “你是浸礼教徒吗?”

    “不是,长官。”

    “那么你不是个浸礼教徒,不对吗?”

    “长官?”

    “我真不明白,你为什么要在这一点上跟我争论不休。你已经承认了这一点。听着,牧师,说你不是浸礼教徒并不等于真正告诉了我们你究竟是什么人,对吗?你可以是任何教派的教徒,任何人。”他把身体微微向前倾斜,摆出一副精明、深沉的样子。“你甚至可能是,”他接着说,“华盛顿-欧文,难道你不是吗?”

    “华盛顿-欧文?”牧师吃惊地重复着。

    “承认吧,华盛顿,”胖上校烦躁地插话道,“你究竟为什么不全部交待出来呢?我们知道是你偷了那个红色梨形番茄。”

    牧师一下子给吓蒙了。过了一会,他才松了一口气,神经质地格格笑了起来。“哦,原来是这样!”他叫道,“现在我开始明白了。我并没有偷那个红色梨形番茄,长官,是卡思卡特上校送给我的。你们要是不相信我,可以去问问他。”

    房间另一头的一扇门打开了,卡思卡特上校走进了地下室。他好像是从壁橱里钻出来的。

    “你好,上校。他声称那个红色梨形番茄是你送给他的,上校,你送了吗?”

    “我为什么要送给他一个红色梨形番茄呢?”卡思卡特上校反问道。

    “谢谢你,上校,这就够了。”

    “愿意效劳,上校,”卡思卡特上校回答道,说完便退出了地下室,并随手在身后关上了门。

    “怎么样,牧师,现在你还有什么可说的?”

    “就是他送给我的!”牧师色厉内荏地低声抗议道,“就是他送给我的!”

    “你是在指责一个上级军官说谎吗,牧师?”

    “为什么一个上级军官会送给你一个番茄,牧师?”

    “这就是你想把它送给惠特科姆中士的原因,是吗,牧师?就因为这个番茄是偷来的?”

    “不,不,不,”牧师抗议道。他痛苦地想,他们为什么不能理解呢?“我把番茄送给惠特科姆中士,是因为我不想要它。”

    “如果你不想要它,为什么要从卡思卡特上校那儿把它偷来呢?”

    “我不是从卡思卡特上校那儿偷来的!”

    “如果你没有偷,那你为什么显出这么一副有罪的模样?”

    “我没有罪。”

    “如果你没有罪,那我们为什么要审问你?”

    “天哪,我不知道。”牧师呻吟了一声。他把放在膝盖上的手指互相捏来捏去,极其痛苦地晃动着低垂的脑袋。“我不知道。”

    “他以为我们有工夫跟他磨蹭。”少校气愤地哼了一声。

    “牧师,”没佩戴肩章的军官从打开的文件夹里取出一张黄色打印纸,口气更加从容地继续说道,“我这儿有一张卡思卡特上校亲笔签名的证词,证词中声明是你从他那儿偷走了那个番茄。”他把这张纸正面朝下放到文件夹的一边,又从另一边拿起另一张纸。

    “我这儿还有一份经过公证的惠特科姆中士的宣誓证词。他在证词中说,他当时看到你急着把番茄塞给他的那副样子,就知道那番茄来路不正。”
 


    “我向上帝发誓,我没有偷那个番茄,长官,”牧师苦恼地恳求道,眼泪都快要掉下来了。“我郑重地向你起誓,那个番茄不是偷来的。”

    “牧师,你信仰上帝吗?”

    “是的,长官,我当然信仰上帝。”

    “这就很奇怪了,牧师。”那军官说着从公文夹里抽出一张黄色打印纸。“因为我这儿还有一份卡思卡特上校的声明,他发誓说你拒绝跟他合作,不愿意在每次飞行任务之前在简令下达室里主持祈祷仪式。”

    牧师愣了一下,接着便回忆起来了。他很快地点点头。“哦,这并不完全是事实,长官,”他急切地解释道,“当卡思卡特上校认识到士兵和军官是在向同一个上帝祈祷时,他自己放弃了这一打算。”

    “他自己干了什么?”那军官不相信地叫道。

    “简直是一派胡言!”红脸上校斥责道。他威严而气恼地从牧师身边转身走开。

    “他难道以为我们会相信他这套谎言吗?”少校表示怀疑地喊道。

    没佩戴肩章的军官尖刻地窃笑着。“牧师,你是不是把事情编得太离奇了?”他宽容而冷漠地笑了笑问道。

    “但是,长官,这是事实,长官!我发誓这是事实。”

    “我看不出这跟是不是事实有什么关系,”那军官无动于衷地回答道,又伸手到旁边去拿那个打开着的装满文件的文件夹。“牧师,你在回答我的问题时说过你是信仰上帝的吗?我记不得了。”

    “是的,长官,我的确这样说过,长官。我的确是信仰上帝的。”

    “那么,这就的确是非常奇怪的了,牧师,因为我这儿还有一份卡思卡特上校的宣誓证词,那上面说你曾经对他说过,无神论不违犯法律。你记得你的确对什么人说过这样的话吗?”

    牧师毫不犹豫地点点头。这一回他觉得自己很有把握。“是的,长官,我的确这么说过。我这么说是因为这是事实。无神论并不违犯法律。”
 


    “但是,你仍然没有理由这么说,牧师,对吗?”那军官皱着眉刻薄地责备道。他又从文件夹里抽出一份经过公证的打印文件。“我这儿还有一份惠特科姆中士的宣誓证词,上面说他计划给在战斗中阵亡或负伤的军人的亲属邮寄由卡思卡特上校签名的慰问信,你却表示反对。这是真的吗?”

    “是的,长官,我的确表示过反对,”牧师回答道,“我为自己这么做而感到自豪。这些信是虚伪的,是骗人的。它们的唯一目的是往卡思卡特上校脸上贴金。”

    “可这又有什么关系呢?”那军官回答道,“它们仍然能给那些收到信的亲属带去一些安慰和问候,不是吗?牧师,我实在无法理解你的思维方式。”

    牧师一时间给难住了,一句话也回答不上来。他垂下脑袋,觉得自己张口结舌,傻里傻气。

    那个面色红润的矮胖上校精神抖擞地朝前迈了几步。他突然有了一个想法。“我们为什么不能把他这该死的脑壳敲开呢?”他跃跃欲试地向其他人建议道。

    “对,我们可以把他这该死的脑壳敲开,不是吗?”长着一张鹰脸的少校表示同意。“他不过是个再浸礼教徒罢了。”

    “不,我们必须首先确定他有罪,”没佩戴肩章的军官懒洋洋地摆了摆手告诫道。他轻轻站立起来,走到桌子的另一边,双手平展地按在桌面上,脸正对着牧师。他的表情阴沉、严厉、狠毒,令人望而生畏。“牧师,”他专横严厉地宣布道,“我们正式指控你假冒华盛顿-欧文之名,未经许可恣意检查官兵们的信件。你是有罪还是无罪?”

    “无罪,长官,”牧师用发干的舌头舔了舔发干的嘴唇,忐忑不安地把坐在椅子边沿上的身体往前探了探。

    “有罪,”上校说。

    “有罪,”少校说。

    “那就是有罪。”没佩戴肩章的军官说。他在文件夹里的一页纸上写了个字。“牧师,”他抬起头来继续说,“我们还要指控你犯了目前我们尚未了解的罪行和违法行为。你是有罪还是无罪?”

    “我不知道,长官。如果你们不告诉我究竟是什么罪行和违法行为,那叫我怎么说呢?”

    “如果我们不知道,我们怎么能告诉你呢?”

    “有罪,”上校断然他说。

    “他肯定有罪。”少校表示同意。“如果那是他的罪行和违法行为的活,那他肯定就是犯罪了。”

    “那就是有罪,”没佩戴肩章的军官拖着长腔说道,他往房间的另一侧走去。“他就全交给你了,上校。”

    “谢谢你,”上校称赞他说,“这件事你干得很出色。”他转过身来对着牧师。“好吧,牧师,一切都完了,走吧。”

    牧师没听明白他的话。“你要我干什么?”

    “走吧,滚吧,我叫你快滚!”上校咆哮起来,生气地朝肩后扬了扬大拇指。“你他妈的快从这儿滚出去!”

    牧师被上校挑衅的言辞和语气吓得目瞪口呆。他感到惊奇,感到困惑不解,他们居然要放他走,这使他大为懊恼。“你们不是打算惩治我吗?”他既惊奇又不满地问道。

    “对极了,我们是打算惩治你的。但是,在我们决定如何惩治你及什么时候惩治你之前,我们当然不会让你跟着我们团团转的。所以,走吧,滚吧。”

    牧师试探地站起身,往外走了几步。“我可以走了?”

    “暂时可以走。但是不许有任何离开这个岛的企图。我们记下了你的号码,牧师。你记住,你一天二十四小时全都处在我们的监视之下。”

    牧师不敢相信他们会真的放他走。他提心吊胆地往出口走去,随时准备被某人专横的声音喝令回去,或者要么肩膀要么脑袋挨上一记重击,倒在半道上爬不起来。他们没做任何事情来阻拦他。
 


    他在阴暗潮湿、密不透风的走廊里摸索着走到楼梯口。当他踉踉跄跄地爬到楼梯顶部,呼吸到新鲜空气时,已经是气喘吁吁了。一经脱离险境,他立刻义愤填膺。他这一天所遭遇的暴行气得他怒不可遏,他这辈子还从来没有这样愤怒过。他旋风般冲过宽敞的、回声不断的门厅,胸中怒火燃烧,怨恨难平。他再也不能忍受下去了,他对自己说,他实在无法忍受下去了。当他走到大楼门口时,看到科恩中校独自快步跑上宽阔的台阶,心中不禁感到一阵高兴。他先深深吸了一口气给自己鼓劲,然后勇敢地走上前去拦住科恩中校。

    “中校,我再也忍受不下去了,”他斩钉截铁地宣布道。可是科恩中校匆匆跑上台阶,根本没有注意到他,这使他大为沮丧。“科恩中校!”

    他的这位上级军官这才停住脚步,转过他那矮胖难看的身体,慢吞吞地走下台阶。“什么事,牧师?”

    “科恩中校,我想和你谈谈今天早上的飞机相撞事件。这件事发生得太可怕了,太可怕了!”

    科恩中校沉默了片刻,露出一丝讥笑,饶有兴致地打量着牧师。“是的,牧师,的确很可怕,”他终于说道,“我不知道我们应该怎样呈文向上级报告才不至于给我们自己丢脸。”

    “我不是这个意思,”牧师态度坚决、毫无顾忌地反驳道,“这十二个人当中有一些已经完成了他们的七十次飞行任务。”

    科恩中校笑了。“要是他们都是些新来的,这次事件就不那么可怕了吗?”他挖苦他说。

    牧师又一次给问住了。不道德的推理似乎时时处处都在刁难他。当他再次开口说话时,他不像方才那样充满自信了,他的嗓音颤抖起来。“长官,要求我们大队的官兵执行八十次飞行任务的做法是完全错误的。别的大队的官兵只要执行五十到五十五次就可以回国了。”

    “我们会考虑这个问题的,”科恩中校厌烦他说。他抬腿打算离去。“再见,随军牧师。”

    “这是什么意思,长官?”牧师嗓音尖厉地追问道。

    科恩中校从台阶上倒退一步,脸上显得很不高兴。“这意思就是我们会考虑的,随军牧师,”他嘲讽而鄙夷地回答道,“难道你是要我们不加考虑就干事情吗?”

    “不,长官,我没有这样想,但你们一直都在考虑这个问题,不是吗?”

    “是的,随军牧师,我们一直在考虑这个问题。但是,为了使你开心,我们会对这个问题多加考虑的。如果我们作出新的决定,我们将会首先通知你的。”科恩中校又转过身去,匆匆跑上台阶。

    “科恩中校!”牧师的喊声又一次使科恩中校停住脚步。他慢慢转过脸来对着牧师,眉头紧锁,显得极不耐烦。牧师内心非常紧张,他滔滔不绝地一口气说下去。“长官,请你允许我把这一事件报告给德里德尔将军。我要向联队司令部提出我的抗议。”

    科恩中校猛地鼓起他那黑乎乎的胖下巴,好不容易才抑制住一阵大笑。过了一会他才回答。“这很好,随军牧师,”他竭力装出一副一本正经的样子,带着捉弄人寻开心的口气回答说,“我允许你向德里德尔将军报告。”

    “谢谢你,长官。我认为我对德里德尔将军还是有一定影响的。

    我觉得事先把这一点告诉你才算公平。”

    “你能事先告诉我,真是太好了,随军牧师。不过你在联队司令部是找不到德里德尔将军的。我也觉得事先把这一点告诉你才算公平。”科恩中校先是歹毒地咧嘴笑笑,随后得意地哈哈大笑起来。

    “德里德尔将军调走了,随军牧师。佩克姆将军调进来了。我们有了一位新的联队指挥官。”

    牧师愣住了。“佩克姆将军!”

    “是的,牧师,你对他也有影响吗?”

    “怎么会?我根本不认识佩克姆将军,”牧师沮丧地反驳道。

    科恩中校又笑了。“这就太糟了,牧师,因为卡思卡特上校跟他关系很熟。”科恩中校幸灾乐祸地格格笑了好一阵,然后突然止住了。“顺便说一句,牧师,”他用手指头戳了一下牧师的胸口,冷冷地告诫道,“你和斯塔布斯医生两个人的一切都完蛋了。我们知道得很清楚,今天是他派你来这儿发牢骚的。”

    “斯塔布斯医生?”牧师困惑不解地摇摇头。“我没见过斯塔布斯医生,中校。是三个陌生的军官未经军方批准把我带到这儿的地下室来的。他们审问并侮辱了我。”

    科恩中校又戳了戳牧师的胸口。“你知道得很清楚,斯塔布斯医生一直在告诉他那个中队的人不要执行七十次以上的飞行任务。”他发出刺耳的大笑。“不过,牧师,他们必须执行七十次以上的飞行任务,因为我们正在把斯塔布斯医生调往太平洋战区。好吧,再见,随军牧师,再见。”
